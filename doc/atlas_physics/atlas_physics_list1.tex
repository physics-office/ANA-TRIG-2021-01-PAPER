\newpage
%-------------------------------------------------------------------------------
\section{\File{atlasjournal.sty}}

Turn on including these definitions with the option \Option{journal=true} and off with the option \Option{journal=false}.

\begin{xtabular}{ll}
\verb|\AcPA| & \AcPA \\
\verb|\ARevNS| & \ARevNS \\
\verb|\CPC| & \CPC \\
\verb|\EPJ| & \EPJ \\
\verb|\EPJC| & \EPJC \\
\verb|\FortP| & \FortP \\
\verb|\IJMP| & \IJMP \\
\verb|\JETP| & \JETP \\
\verb|\JETPL| & \JETPL \\
\verb|\JaFi| & \JaFi \\
\verb|\JHEP| & \JHEP \\
\verb|\JMP| & \JMP \\
\verb|\MPL| & \MPL \\
\verb|\NCim| & \NCim \\
\verb|\NIM| & \NIM \\
\verb|\NIMA| & \NIMA \\
\verb|\NP| & \NP \\
\verb|\NPB| & \NPB \\
\verb|\PL| & \PL \\
\verb|\PLB| & \PLB \\
\verb|\PR| & \PR \\
\verb|\PRC| & \PRC \\
\verb|\PRD| & \PRD \\
\verb|\PRL| & \PRL \\
\verb|\PRep| & \PRep \\
\verb|\RMP| & \RMP \\
\verb|\ZfP| & \ZfP \\
\verb|\collab| & \collab \\
\end{xtabular}



\newpage
%-------------------------------------------------------------------------------
\section{\File{atlasmisc.sty}}

Turn on including these definitions with the option \Option{misc=true} and off with the option \Option{misc=false}.

%------------------------------------------------------------------------------
% Miscellaneous useful definitions for ATLAS documents.
% Include with misc option in atlasphysics.sty.
%
% Note that this file can be overwritten when atlaslatex is updated.
%
% This package assumes that \ATPackageUseVersion has been defined.
% This is done by either atlasdoc.cls or atlaspackage.sty.
%
% Copyright (C) 2002-2024 CERN for the benefit of the ATLAS collaboration.
%------------------------------------------------------------------------------

%------------------------------------------------------------------------------
% Useful things for proton-proton physics
%------------------------------------------------------------------------------

\newcommand*{\pT}{\ensuremath{p_{\text{T}}}\xspace}
\newcommand*{\pt}{\ensuremath{p_{\text{T}}}\xspace}
% \newcommand*{\pTX}[1]{\ensuremath{p_{\text{T},#1}}\xspace}
% \pT macro with optional superscript and extra subscript.
\NewDocumentCommand {\pTX} { o o } {%
  \IfNoValueTF {#1} {%
    \IfNoValueTF {#2} {%
      \ensuremath{p_{\text{T}}}\xspace%
    }{%
      \ensuremath{p_{\text{T},#2}}\xspace%
    }%
  }{%
    \IfNoValueTF {#2} {%
      \ensuremath{p_{\text{T}}^{#1}}\xspace%
    }{%
      \ensuremath{p_{\text{T},#2}^{#1}}\xspace%
    }%
  }%
}
\newcommand*{\ET}{\ensuremath{E_{\text{T}}}\xspace}
\newcommand*{\eT}{\ensuremath{E_{\text{T}}}\xspace}
\newcommand*{\et}{\ensuremath{E_{\text{T}}}\xspace}
\newcommand*{\HT}{\ensuremath{H_{\text{T}}}\xspace}
\newcommand*{\pTsq}{\ensuremath{p_{\text{T}}^{2}}\xspace}
%\newcommand*{\ptsq}{\ensuremath{p_{\text{T}}^{2}}\xspace}
\newcommand*{\MET}{\ensuremath{E_{\text{T}}^{\text{miss}}}\xspace}
\newcommand*{\met}{\ensuremath{E_{\text{T}}^{\text{miss}}}\xspace}
\newcommand*{\sumET}{\ensuremath{\sum \ET}\xspace}
\newcommand*{\EjetRec}{\ensuremath{E_{\text{rec}}}\xspace}
\newcommand*{\PjetRec}{\ensuremath{p_{\text{rec}}}\xspace}
\newcommand*{\EjetTru}{\ensuremath{E_{\text{true}}}\xspace}
\newcommand*{\PjetTru}{\ensuremath{p_{\text{true}}}\xspace}
\newcommand*{\EjetDM}{\ensuremath{E_{\text{DM}}}\xspace}
\newcommand*{\Rcone}{\ensuremath{R_{\text{cone}}}\xspace}
\newcommand*{\abseta}{\ensuremath{|\eta|}\xspace}
\newcommand*{\Ecm}{\ensuremath{E_{\text{cm}}}\xspace}
\newcommand*{\rts}{\ensuremath{\sqrt{s}}\xspace}
\newcommand*{\sqs}{\ensuremath{\sqrt{s}}\xspace}
\newcommand*{\Nevt}{\ensuremath{N_{\mathrm{evt}}}\xspace}
\newcommand*{\zvtx}{\ensuremath{z_{\mathrm{vtx}}}\xspace}
\newcommand*{\dzero}{\ensuremath{d_{0}}\xspace}
\newcommand*{\zzsth}{\ensuremath{z_{0} \sin(\theta)}\xspace}

% LHC standard terms
\newcommand*{\RunOne}{Run~1\xspace}
\newcommand*{\RunTwo}{Run~2\xspace}
\newcommand*{\RunThr}{Run~3\xspace}

% ATLAS jets standard terms
\newcommand*{\kt}{\ensuremath{k_{t}}\xspace}
\newcommand*{\antikt}{anti-\kt}
\newcommand*{\Antikt}{Anti-\kt}
% Different hyphenation for pile-up in UK and US English
\iflanguage{USenglish}{%
    \newcommand*{\pileup}{pileup\xspace}
    \newcommand*{\Pileup}{Pileup\xspace}
}{%
    \newcommand*{\pileup}{pile-up\xspace}
    \newcommand*{\Pileup}{Pile-up\xspace}
}

% b-tagging standard terms
\newcommand*{\btag}{\ensuremath{b\text{-tagging}}\xspace}
\newcommand*{\btagged}{\ensuremath{b\text{-tagged}}\xspace}
\newcommand*{\bquark}{\ensuremath{b\text{-quark}}\xspace}
\newcommand*{\bquarks}{\ensuremath{b\text{-quarks}}\xspace}
\newcommand*{\bjet}{\ensuremath{b\text{-jet}}\xspace}
\newcommand*{\bjets}{\ensuremath{b\text{-jets}}\xspace}


%------------------------------------------------------------------------------
% Masses
%------------------------------------------------------------------------------
\newcommand*{\mh}{\ensuremath{m_{h}}\xspace}
\newcommand*{\mW}{\ensuremath{m_{W}}\xspace}
\newcommand*{\mZ}{\ensuremath{m_{Z}}\xspace}
\newcommand*{\mH}{\ensuremath{m_{H}}\xspace}
% \newcommand*{\mA}{\ensuremath{m_{A}}\xspace}

%\def\mass#1{\ensuremath{m_{#1#1}}}%  "\mass{\mu}" produces "msub{mumu}".
\newcommand{\twomass}[2]{\ensuremath{m_{#1#2}}\xspace}

%------------------------------------------------------------------------------
% Monte Carlo generators
%------------------------------------------------------------------------------
% Set up generic command for MC with optional version number
%%% e.g. \PYTHIA[8] -> Pythia 8
\NewDocumentCommand {\AtlasMC} { o m } {%
  \IfNoValueTF {#1} {%
    \textsc{#2}\xspace%
  }{%
    \textsc{#2}\,#1\xspace%
  }%
}
\NewDocumentCommand {\ACERMC} { o } {\AtlasMC[#1]{AcerMC}}
\NewDocumentCommand {\ALPGEN} { o } {\AtlasMC[#1]{Alpgen}}
\NewDocumentCommand {\AMCatNLO} { o } {\AtlasMC[#1]{aMC@NLO}}
\NewDocumentCommand {\BLACKHAT} { o } {\AtlasMC[#1]{BlackHat}}
\NewDocumentCommand {\CALCHEP} { o } {\AtlasMC[#1]{CalcHEP}}
\NewDocumentCommand {\COLLIER} { o } {\AtlasMC[#1]{Collier}}
\NewDocumentCommand {\COMPHEP} { o } {\AtlasMC[#1]{CompHEP}}
\NewDocumentCommand {\EVTGEN} { o } {\AtlasMC[#1]{EvtGen}}
\NewDocumentCommand {\FEYNRULES} { o } {\AtlasMC[#1]{FeynRules}}
\NewDocumentCommand {\GGTOVV} { o } {\AtlasMC[#1]{gg2VV}}
\NewDocumentCommand {\GOSAM} { o } {\AtlasMC[#1]{GoSam}}
\NewDocumentCommand {\HATHOR} { o } {\AtlasMC[#1]{Hathor}}
\NewDocumentCommand {\HEJ} { o } {\AtlasMC[#1]{HEJ}}
\NewDocumentCommand {\Herwig} { o } {\AtlasMC[#1]{Herwig}}
\NewDocumentCommand {\HERWIG} { o } {\AtlasMC[#1]{Herwig}}
\NewDocumentCommand {\HERWIGpp} { o } {\AtlasMC[#1]{Herwig++}}
\NewDocumentCommand {\HRES} { o } {\AtlasMC[#1]{Hres}}
\NewDocumentCommand {\JIMMY} { o } {\AtlasMC[#1]{Jimmy}}
\NewDocumentCommand {\MADSPIN} { o } {\AtlasMC[#1]{MadSpin}}
\NewDocumentCommand {\MADGRAPH} { o } {\AtlasMC[#1]{MadGraph}}
\NewDocumentCommand {\MGNLO} { o } {\AtlasMC[#1]{MadGraph5\_aMC@NLO}} %short-hand
\NewDocumentCommand {\MCatNLO} { o } {\AtlasMC[#1]{MC@NLO}}
\NewDocumentCommand {\MCFM} { o } {\AtlasMC[#1]{MCFM}}
\NewDocumentCommand {\METOP} { o } {\AtlasMC[#1]{METOP}}
\NewDocumentCommand {\OPENLOOPS} { o } {\AtlasMC[#1]{OpenLoops}}
\NewDocumentCommand {\POWHEG} { o } {\AtlasMC[#1]{Powheg}}
\NewDocumentCommand {\POWHEGBOX} { o } {\AtlasMC[#1]{Powheg\,Box}}
\NewDocumentCommand {\POWHEGBOXRES} { o } {\AtlasMC[#1]{Powheg\,Box\,Res}}
\NewDocumentCommand {\PHOTOS} { o } {\AtlasMC[#1]{Photos}}
\NewDocumentCommand {\PHOTOSpp} { o } {\AtlasMC[#1]{Photos++}}
\NewDocumentCommand {\PROPHECY} { o } {\AtlasMC[#1]{Prophecy4f}}
\NewDocumentCommand {\PROTOS} { o } {\AtlasMC[#1]{Protos}}
\NewDocumentCommand {\Pythia} { o } {\AtlasMC[#1]{Pythia}}
\NewDocumentCommand {\PYTHIA} { o } {\AtlasMC[#1]{Pythia}}
\NewDocumentCommand {\RECOLA} { o } {\AtlasMC[#1]{Recola}}
\NewDocumentCommand {\Sherpa} { o } {\AtlasMC[#1]{Sherpa}}
\NewDocumentCommand {\SHERPA} { o } {\AtlasMC[#1]{Sherpa}}
\NewDocumentCommand {\TOPpp} { o } {\AtlasMC[#1]{Top++}}
\NewDocumentCommand {\VBFNLO} { o } {\AtlasMC[#1]{VBFNLO}}

% common ME+PS combinations
\NewDocumentCommand {\MGNLOHER} { o } {\AtlasMC[#1]{MadGraph5\_aMC@NLO+Herwig}}
\NewDocumentCommand {\MGNLOPY} { o } {\AtlasMC[#1]{MadGraph5\_aMC@NLO+Pythia}}
\NewDocumentCommand {\MGHER} { o } {\AtlasMC[#1]{MadGraph5+Herwig}}
\NewDocumentCommand {\MGPY} { o } {\AtlasMC[#1]{MadGraph5+Pythia}}
\NewDocumentCommand {\POWHER} { o } {\AtlasMC[#1]{Powheg+Herwig}}
\NewDocumentCommand {\POWPY} { o } {\AtlasMC[#1]{Powheg+Pythia}}
\NewDocumentCommand {\SHERPABH} { o } {\AtlasMC[#1]{Sherpa+BlackHat}}
\NewDocumentCommand {\SHERPAOL} { o } {\AtlasMC[#1]{Sherpa+OpenLoops}}


%%% Set up generic command for common PDFs with optional 
%%% version number/perturbative order (but no space)
%%% e.g. \NNPDF[3.0nnlo] -> NNPDF3.0nnlo
\NewDocumentCommand {\AtlasPDF} { o m } {%
  \IfNoValueTF {#1} {%
    \textsc{#2}\xspace%
  }{%
    \textsc{#2#1}\xspace%
  }%
}
\NewDocumentCommand {\ABM} { o } {\AtlasPDF[#1]{ABM}}
\NewDocumentCommand {\ABKM} { o } {\AtlasPDF[#1]{ABKM}}
\NewDocumentCommand {\CT} { o } {\AtlasPDF[#1]{CT}}
\NewDocumentCommand {\CTEQ} { o } {\AtlasPDF[#1]{CTEQ}}
\NewDocumentCommand {\GJR} { o } {\AtlasPDF[#1]{GJR}}
\NewDocumentCommand {\HERAPDF} { o } {\AtlasPDF[#1]{HERAPDF}}
\NewDocumentCommand {\LUXQED} { o } {\AtlasPDF[#1]{LUXqed}}
\NewDocumentCommand {\MSTW} { o } {\AtlasPDF[#1]{MSTW2008}}
\NewDocumentCommand {\MMHT} { o } {\AtlasPDF[#1]{MMHT2014}}
\NewDocumentCommand {\MSHT} { o } {\AtlasPDF[#1]{MSHT2020}}
\NewDocumentCommand {\NNPDF} { o } {\AtlasPDF[#1]{NNPDF}}
\NewDocumentCommand {\PDFforLHC} { o } {\AtlasPDF[#1]{PDF4LHC}}

%%% Event generator extras
\newcommand*{\AUET}{AUET2\xspace}
\newcommand*{\AZNLO}{AZNLO\xspace}
\newcommand*{\FXFX}{\textsc{FxFx}\xspace}
\newcommand*{\GEANT}{\textsc{Geant4}\xspace}
\newcommand*{\MENLOPS}{\textsc{MENLOPS}\xspace}
\newcommand*{\MEPSatLO}{\textsc{MEPS@LO}\xspace}
\newcommand*{\MEPSatNLO}{\textsc{MEPS@NLO}\xspace}
\newcommand*{\MINLO}{\textsc{MiNLO}\xspace}
\newcommand*{\Monash}{Monash\xspace}
\newcommand*{\Perugia}{Perugia\xspace}
\newcommand*{\Prospino}{Prospino\xspace}
\newcommand*{\UEEE}{UE-EE-5\xspace}

\newcommand*{\LO}{\ensuremath{\text{LO}}\xspace}
\newcommand*{\NLO}{\ensuremath{\text{NLO}}\xspace}
\newcommand*{\NLL}{\ensuremath{\text{NLL}}\xspace}
\newcommand*{\NNLO}{\ensuremath{\text{N}}\NLO}
\newcommand*{\muF}{\ensuremath{\mu_\mathrm{f}}\xspace}
\newcommand*{\muQ}{\ensuremath{\mu_\mathrm{q}}\xspace}
\newcommand*{\muR}{\ensuremath{\mu_\mathrm{r}}\xspace}
\newcommand*{\hdamp}{\ensuremath{h_\mathrm{damp}}\xspace}
\newcommand*{\NLOEWvirt}{NLO~\ensuremath{\text{EW}_\text{virt}}\xspace}

%------------------------------------------------------------------------------
% Useful symbols for use in or out of math mode
%------------------------------------------------------------------------------
\newcommand*{\ra}{\ensuremath{\rightarrow}\xspace}
\newcommand*{\la}{\ensuremath{\leftarrow}\xspace}
\newcommand*{\rarrow}{\ensuremath{\rightarrow}\xspace}
\newcommand*{\larrow}{\ensuremath{\leftarrow}\xspace}
%\let\rarrow=\ra
%\let\larrow=\la
\newcommand*{\lapprox}{\ensuremath{\sim\kern-1em\raise 0.65ex\hbox{\(<\)}}\xspace}%  Or use \lsim
\newcommand*{\rapprox}{\ensuremath{\sim\kern-1em\raise 0.65ex\hbox{\(>\)}}\xspace}%  and \rsim.
\newcommand*{\gam}{\ensuremath{\gamma}\xspace}
\newcommand*{\stat}{\mbox{\(\;\)(stat.)}\xspace}
\newcommand*{\syst}{\mbox{\(\;\)(syst.)}\xspace}
\newcommand*{\radlength}{\ensuremath{X_0}}
\newcommand*{\StoB}{\ensuremath{S/B}\xspace}

% Different differential symbols in American and British English
\iflanguage{USenglish}{%
  \providecommand*{\dif}{\ensuremath{d}}
}{%
  \providecommand*{\dif}{\ensuremath{\mathrm{d}}}
}

%------------------------------------------------------------------------------
% Commands for separate statistical and systematic errors and asymmetric errors.
%------------------------------------------------------------------------------
% Include options to steer rounding.
% Backward incompatible changes were made in version 15 of ATLAS LaTeX.
% Need a macro to make superscript and subscript the same width.
% This is needed as + and - can have different widths (pdflatex problem).
% The idea for the way to do this came from:
% https://tex.stackexchange.com/questions/291291/control-horizontal-alignment-of-sub-superscripts
\ExplSyntaxOn
\ifthenelse{\ATPackageUseVersion > 14}{%
  \cs_new_protected:Npn \supsub:nn #1#2
  {
    \hbox_set:Nn \l_tmpa_box { \ensuremath { \c_math_superscript_token { #1 } } }
    \hbox_set:Nn \l_tmpb_box { \ensuremath { \c_math_subscript_token { #2 } } }
    % \box_show:N \l_tmpa_box
    % \box_show:N \l_tmpb_box
    \dim_compare:nNnTF { \box_wd:N \l_tmpa_box } > { \box_wd:N \l_tmpb_box }
      { \dim_set:Nn \l_tmpa_dim { \box_wd:N \l_tmpa_box } }
      { \dim_set:Nn \l_tmpa_dim { \box_wd:N \l_tmpb_box } }
    \ensuremath {
      \c_math_superscript_token { \mathmakebox[\l_tmpa_dim][r]{ #1 } }
      \c_math_subscript_token   { \mathmakebox[\l_tmpa_dim][r]{ #2 } }
    }
    %\c_math_subscript_token { #1 } \c_math_superscript_token { #2 }
  }
  \cs_generate_variant:Nn \supsub:nn { xx }

  \@ifpackageloaded{siunitx}{%
    \PackageInfo{atlasmisc}{Using~version~15~error~macros.}
    \@ifpackagelater{siunitx}{2021/04/09}{%
      \PackageInfo{atlasmisc}{Using~siunitx~version~3~options}
      \newcommand*{\AT@siunitx_version}{3}
    }{%
      \PackageInfo{atlasmisc}{Using~siunitx~version~2~options}
      \newcommand*{\AT@siunitx_version}{2}
    }

    % Keeping plus sign option depends on siunitx version.
    % Arguments are rounding digits, rounding type, superscript, subscript.
    \tl_new:N \atl_si_options_tl
    \NewDocumentCommand {\numpmerr} { o o m m } {
      \ifthenelse{\AT@siunitx_version > 2}{%
        \IfNoValueTF {#1} {
          \tl_set:Nn \atl_si_options_tl { print-implicit-plus }
        }{
          \IfNoValueTF {#2} {
            \tl_set:Nn \atl_si_options_tl { round-precision=#1, print-implicit-plus }
          }{
            \tl_set:Nn \atl_si_options_tl { round-mode=#2, round-precision=#1, print-implicit-plus }
          }
        }
      }{
        \IfNoValueTF {#1} {
          \tl_set:Nn \atl_si_options_tl { retain-explicit-plus }
        }{
          \IfNoValueTF {#2} {
            \tl_set:Nn \atl_si_options_tl { round-precision=#1, retain-explicit-plus }
          }{
            \tl_set:Nn \atl_si_options_tl { round-mode=#2, round-precision=#1, retain-explicit-plus }
          }
        }
      }
      \ifthenelse{\boolean{APHY@numpmcorr}}{
        % \tl_show:N \atl_si_options_tl
        \supsub:xx
          { \exp_not:N \num [ \tl_use:N \atl_si_options_tl ] {#3} }
          { \exp_not:N \num [ \tl_use:N \atl_si_options_tl ] {#4} }
      }{
        \tl_set:Nx \l_tmpa_tl { \exp_not:N \num [ \tl_use:N \atl_si_options_tl ] {#3} }
        \tl_set:Nx \l_tmpb_tl { \exp_not:N \num [ \tl_use:N \atl_si_options_tl ] {#4} }
        \ensuremath{
          \c_math_superscript_token { \tl_use:N \l_tmpa_tl } 
          \c_math_subscript_token   { \tl_use:N \l_tmpb_tl }
        }
      }
    }
  }{}
}{}
\ExplSyntaxOff

% Version 15 onwards commmands.
\ifthenelse{\ATPackageUseVersion > 14}{%
  \@ifpackageloaded{siunitx}{%
    \PackageInfo{atlasmisc}{Using version 15 error macros.}
    \newcommand*{\valuesep}{\,}
    \sisetup{number-unit-product=\valuesep}
    % Basic error plus text.
    \NewDocumentCommand {\numerrt} { o m m } {%
      \begingroup
        \IfValueT {#1} {\sisetup{#1}}%
        \ensuremath{\pm \num{#2}\valuesep\text{(#3)}}%
      \endgroup
    }
    % Numbers with rounding.
    \NewDocumentCommand {\numR} { o m } {%
      \IfNoValueTF {#1} {%
        \num{#2}
      }{%
        \num[round-precision=#1]{#2}
      }
    }
    \NewDocumentCommand {\numRF} { o m } {%
      \IfNoValueTF {#1} {%
        \num[round-mode=figures]{#2}
      }{%
        \num[round-mode=figures, round-precision=#1]{#2}
      }
    }
    \NewDocumentCommand {\numRP} { o m } {%
      \IfNoValueTF {#1} {%
        \num[round-mode=places]{#2}
      }{%
        \num[round-mode=places, round-precision=#1]{#2}
      }
    }

    % Asymmetric errors.
    \NewDocumentCommand {\numpmerrx} { o m m } {%
      \begingroup
        \IfValueT {#1} {\sisetup{#1}}%
        \numpmerr{#2}{#3}%
      \endgroup
    }
    \NewDocumentCommand {\numpmerrt} { o m m m } {%
      \begingroup
        \IfValueT {#1} {\sisetup{#1}}%
        \numpmerr{#2}{#3}\valuesep\text{(#4)}%
      \endgroup
    }
    \NewDocumentCommand {\qtypmerr} { o o m m m m } {%
      \begingroup
        \IfNoValueTF {#1} {%
          \ensuremath{\num{#3}\valuesep\numpmerr{#4}{#5}\valuesep\unit{#6}}%
        }{
          \IfNoValueTF {#2} {%
            \ensuremath{\num[round-precision=#1]{#3}\valuesep\numpmerr[#1]{#4}{#5}\valuesep\unit{#6}}%
          }{%
            \ensuremath{\num[round-precision=#1, round-mode=#2]{#3}\valuesep\numpmerr[#1][#2]{#4}{#5}\valuesep\unit{#6}}%
          }
        }
      \endgroup
    }
    \NewDocumentCommand {\qtypmerrx} { o m m m m } {%
      \begingroup
        \IfValueT {#1} {\sisetup{#1}}%
        \ensuremath{\num{#2}\valuesep\numpmerr{#3}{#4}\valuesep\unit{#5}}%
      \endgroup
    }
    \NewDocumentCommand {\qtyerrt} { o m m m m } {%
      \begingroup
        \IfValueT {#1} {\sisetup{#1}}%
        \ensuremath{\num{#2}%
        \valuesep\numerrt{#3}{#4}%
        \valuesep\unit{#5}}%
      \endgroup
    }
    \NewDocumentCommand {\qtyerrtt} { o m m m m m m } {%
      \begingroup
        \IfValueT {#1} {\sisetup{#1}}%
        \ensuremath{\num{#2}%
        \valuesep\numerrt{#3}{#4}%
        \valuesep\numerrt{#5}{#6}%
        \valuesep\unit{#7}}%
      \endgroup
    }
    \NewDocumentCommand {\qtypmerrt} { o m m m m m } {%
      \begingroup
        \IfValueT {#1} {\sisetup{#1}}%
        \ensuremath{\num{#2}%
        \valuesep\numpmerrt{#3}{#4}{#5}%
        \valuesep\unit{#6}}%
      \endgroup
    }
    \NewDocumentCommand {\qtypmerrtt} { o m m m m m m m m } {%
      \begingroup
        \IfValueT {#1} {\sisetup{#1}}%
        \ensuremath{\num{#2}%
        \valuesep\numpmerrt{#3}{#4}{#5}%
        \valuesep\numpmerrt{#6}{#7}{#8}%
        \valuesep\unit{#9}}%
      \endgroup
    }
    \NewDocumentCommand {\qtyerrs} { o m m m m } {%
      \begingroup
        \IfValueT {#1} {\sisetup{#1}}%
        \ensuremath{\num{#2}%
        \valuesep\numerrt{#3}{stat.}%
        \valuesep\numerrt{#4}{sys.}%
        \valuesep\unit{#5}}%
      \endgroup
    }
    \NewDocumentCommand {\qtypmerrs} { o m m m m m m } {%
      \begingroup
        \IfValueT {#1} {\sisetup{#1}}%
        \ensuremath{\num{#2}
        \valuesep\numpmerrt{#3}{#4}{stat.}
        \valuesep\numpmerrt{#5}{#6}{sys.}%
        \valuesep\unit{#7}}%
      \endgroup
    }
  }{
    \PackageInfo{atlasmisc}{numpmerr macros not defined, as siunitx has not been loaded.}
  }
}{%
  % Version 14 and earlier commmands.
  \PackageInfo{atlasmisc}{Using version 14 error macros.}
  % Macro to make superscript and subscript the same width.
  \providecommand{\@firstoftwo}[3]{#1}
  \providecommand{\@secondoftwo}[3]{#2}
  \newcommand{\makesupsub}[2]{%
    \sbox\z@{$^{\@firstoftwo#2}$}%
    \sbox\tw@{$_{\@secondoftwo#2}$}%
    \dimen@=\wd\z@
    \ifdim\wd\tw@>\wd\z@ \dimen@=\wd\tw@\fi
    {\mkern0mu}%
      ^{\mathmakebox[\dimen@][r]{\@firstoftwo#2}}%
      _{\mathmakebox[\dimen@][r]{\@secondoftwo#2}}%
  }
  \DeclareRobustCommand{\supsub}[2]{{%
    \mathpalette\makesupsub{{#1}{#2}}%
  }}
  % Useful macros for rounding.
  % Keeping plus sign option depends on siunitx version.
  \newcommand*{\numR}[2]{\num[round-precision=#2]{#1}}
  \newcommand*{\numRF}[2]{\num[round-mode=figures,round-precision=#2]{#1}}
  \newcommand*{\numRP}[2]{\num[round-mode=places, round-precision=#2]{#1}}
  \@ifpackageloaded{siunitx}{%
    \@ifpackagelater{siunitx}{2021/04/09}{%
      \PackageInfo{atlasmisc}{Using siunitx version 3 options}
      \newcommand{\numpmerr}[3]{\ensuremath{^{\num[round-precision=#3, print-implicit-plus]{#1}}_{\num[round-precision=#3, print-implicit-plus]{#2}}}}
      \newcommand{\numpmRF}[3]{\ensuremath{^{\num[round-mode=figures, round-precision=#3, print-implicit-plus]{#1}}_{\num[round-mode=figures, round-precision=#3, print-implicit-plus]{#2}}}}
      \newcommand{\numpmRP}[3]{\ensuremath{^{\num[round-mode=places, round-precision=#3, print-implicit-plus]{#1}}_{\num[round-mode=places, round-precision=#3, print-implicit-plus]{#2}}}}
      % Arguments are rounding digits, rounding type, superscript, subscript.
      \NewDocumentCommand {\numpmerrx} { o o m m } {%
        \IfNoValueTF {#1} {%
          \ensuremath{\supsub{\num[print-implicit-plus]{#3}}{\num[print-implicit-plus]{#4}}}
        }{%
          \IfNoValueTF {#2} {%
            \ensuremath{\supsub{\num[round-precision=#1, print-implicit-plus]{#3}}{\num[round-precision=#1, print-implicit-plus]{#4}}}
          }{%
            \ensuremath{\supsub{\num[round-mode=#2, round-precision=#1, print-implicit-plus]{#3}}{\num[round-mode=#2, round-precision=#1, print-implicit-plus]{#4}}}
          }
        }
      }
    }{
      \PackageInfo{atlasmisc}{Using siunitx version 2 options}
      \newcommand{\numpmerr}[3]{\ensuremath{^{\num[round-precision=#3, retain-explicit-plus]{#1}}_{\num[round-precision=#3, retain-explicit-plus]{#2}}}}
      \newcommand{\numpmRF}[3]{\ensuremath{^{\num[round-mode=figures, round-precision=#3, retain-explicit-plus]{#1}}_{\num[round-mode=figures,round-precision=#3, retain-explicit-plus]{#2}}}}
      \newcommand{\numpmRP}[3]{\ensuremath{^{\num[round-mode=places, round-precision=#3, retain-explicit-plus]{#1}}_{\num[round-mode=places, round-precision=#3, retain-explicit-plus]{#2}}}}
      % Arguments are rounding digits, rounding type, superscript, subscript.
      \NewDocumentCommand {\numpmerrx} { o o m m } {%
        \IfNoValueTF {#1} {%
          \ensuremath{\supsub{\num[retain-explicit-plus]{#3}}{\num[retain-explicit-plus]{#4}}}
        }{%
          \IfNoValueTF {#2} {%
            \ensuremath{\supsub{\num[round-precision=#1, retain-explicit-plus]{#3}}{\num[round-precision=#1, retain-explicit-plus]{#4}}}
          }{%
            \ensuremath{\supsub{\num[round-mode=#2, round-precision=#1, retain-explicit-plus]{#3}}{\num[round-mode=#2, round-precision=#1, retain-explicit-plus]{#4}}}
          }
        }
      }
    }
  }{
    \PackageInfo{atlasmisc}{numpmerr macros not defined, as siunitx has not been loaded.}
  }
}

%------------------------------------------------------------------------------
% QCD
%------------------------------------------------------------------------------
\newcommand*{\alphas}{\ensuremath{\alpha_{\text{s}}}\xspace}
\newcommand*{\NF}{\ensuremath{N_{\text{F}}}\xspace}
\newcommand*{\NC}{\ensuremath{N_{\text{C}}}\xspace}
\newcommand*{\CF}{\ensuremath{C_{\text{F}}}\xspace}
\newcommand*{\CA}{\ensuremath{C_{\text{A}}}\xspace}
\newcommand*{\TF}{\ensuremath{T_{\text{F}}}\xspace}
\newcommand*{\Lms}{\ensuremath{\Lambda_{\overline{\text{MS}}}}\xspace}
\newcommand*{\Lmsfive}{\ensuremath{\Lambda^{(5)}_{\overline{\text{MS}}}}\xspace}
\newcommand*{\kperp}{\ensuremath{k_{\perp}}\xspace}

%------------------------------------------------------------------------------
% CKM matrix
%------------------------------------------------------------------------------
\newcommand*{\Vcb}{\ensuremath{\vert V_{cb} \vert}\xspace}
\newcommand*{\Vub}{\ensuremath{\vert V_{ub} \vert}\xspace}
\newcommand*{\Vtd}{\ensuremath{\vert V_{td} \vert}\xspace}
\newcommand*{\Vts}{\ensuremath{\vert V_{ts} \vert}\xspace}
\newcommand*{\Vtb}{\ensuremath{\vert V_{tb} \vert}\xspace}
\newcommand*{\Vcs}{\ensuremath{\vert V_{cs} \vert}\xspace}
\newcommand*{\Vud}{\ensuremath{\vert V_{ud} \vert}\xspace}
\newcommand*{\Vus}{\ensuremath{\vert V_{us} \vert}\xspace}
\newcommand*{\Vcd}{\ensuremath{\vert V_{cd} \vert}\xspace}

%------------------------------------------------------------------------------
% Figure width
%------------------------------------------------------------------------------
\newlength{\figwidth}
\setlength{\figwidth}{\textwidth}
\addtolength{\figwidth}{-2.0cm}

%------------------------------------------------------------------------------
% The decay symbol, to be used in \eqalign.                       
% It works like: \[\eqalign{a\ra &b+c\cr &\dk &e+f\cr &&\dk g+h}\]
%                                                                 
%                 a  -->  b + c                                   
%                         |                                       
%                         |                                       
%                         +----> e + f                            
%                                |                                
%                                |                                
%                                +----> g + h                     
%------------------------------------------------------------------------------
\newdimen\dkwidth
\def\dk{%
   \dkwidth=\baselineskip
   {\def\to{\rightarrow}%  allows "\rightarrowfill" to work.
   \kern 3pt
   \hbox{%
      \raise 3pt
      \hbox{%
         \vrule height 0.8\dkwidth width 0.7pt depth0pt
      }
      \kern-0.4pt%
      \hbox to 1.5\dkwidth{%
         \rightarrowfill
      }
   \kern0.6em
   }}
}

%------------------------------------------------------------------------------
% Redefine \eqalign to allow more than one column;
% very useful for multiple decays as defined above.
%------------------------------------------------------------------------------
%\unlock
\def\eqalign#1{%
   \,
   \vcenter{%
      \openup\jot\m@th
      \ialign{%
         \strut\hfil\(\displaystyle{##}\)&&\(%
         \displaystyle{{}##}\)\hfil\crcr#1\crcr%
      }
   }
   \,
}

%------------------------------------------------------------------------------
% Hours:minutes macro
%------------------------------------------------------------------------------
%\newcount\hrs\newcount\minu\newcount\temptime
%\def\hm{\hrs=\time \divide\hrs by 60 \minu=\time\temptime=\hrs
%\multiply\temptime by 60%
%\advance\minu by -\temptime
%\ifnum\minu<10 \let\zerofill=0\else \let\zerofill=\relax\fi
% \the\hrs:\zerofill\the\minu}



\noindent A length \Macro{figwidth} is defined that is \qty{2}{\cm} smaller than \Macro{textwidth}.

\noindent Monte Carlo generators and PDFs have an optional argument
that allows you to include the version, e.g.\
\verb|\PYTHIA[8]| to produce \PYTHIA[8] or\\
\verb|\PYTHIA[(v8.160)]| to produce \PYTHIA[(v8.160)].

\noindent A macro \Macro{pTX} is included that allows you to add an extra subscript and/or a superscript to \pT.
e.g.\\
\verb|\pTX| to produce \pTX;\\
\verb|\pTX[2]| to produce \pTX[2];\\
\verb|\pTX[][\text{had}]| to produce \pTX[][\text{had}];\\
\verb|\pTX[3][\text{had}]| to produce \pTX[3][\text{had}].

\noindent A generic macro \verb|\twomass| is defined, so that for example
\verb|\twomass{\mu}{\mu}| produces \twomass{\mu}{\mu} and \verb|\twomass{\mu}{e}| produces \twomass{\mu}{e}.

\noindent Several macros that help format numbers with rounding are provided:
e.g.\\
\verb|\numR[2]{3.145159}|  \(\rightarrow\) \(\numR[2]{3.14159}\)\\
\verb|\numRF[3]{3.145159}| \(\rightarrow\) \(\numRF[3]{3.14159}\)\\
\verb|\numRP[3]{3.145159}| \(\rightarrow\) \(\numRP[3]{3.14159}\)\\
\verb|\numpmerr[2]{+1.234}{-3.456}|          \(\rightarrow\) \(\numpmerr[2]{+1.234}{-0.3456}\)\\[0.5ex]
\verb|\numpmerr[2][figures]{+1.234}{-3.456}| \(\rightarrow\) \(\numpmerr[2][figures]{+1.234}{-3.456}\)\\[0.5ex]
\verb|\numpmerr[2][places]{-1.234}{+3.456}|  \(\rightarrow\) \(\numpmerr[2][places]{-1.234}{+3.456}\)\\[0.5ex]
\verb|\numpmerr[2][figures]{+1.234}{-3.456}| \(\rightarrow\) \(\numpmerr[2][figures]{+1.234}{-3.456}\)\\[0.5ex]
Note that \Macro{numR} and \Macro{numpmerr} only perform rounding if \Option{round-mode} is set to something
via \Macro{sisetup} (which is not the case here).
See the \Package{siunitx} package for more details.
For details on the macros can be found in the \enquote{Guide to formatting tables}~\cite{atlas-tables}.

% The macro \Macro{supsub} ensures that superscripts and subscripts have the same width.
% This is useful for asymmetric errors if the widths of \enquote{\(+\)} and \enquote{\(-\)}
% are not the same.

A macro \verb|\dk| is defined which makes it easier to write down decay chains.
For example
\begin{verbatim}
\[\eqalign{a \to & b+c\\
   & \dk & e+f \\
   && \dk g+h}
\]
\end{verbatim}
produces
\[\eqalign{a \to & b+c\cr
   & \dk & e+f \cr
   && \dk g+h}
\]
Note that \Macro{eqalign} is also redefined in this package so that \Macro{dk} works.

The following macro names have been changed:\\
\verb|\ptsq| \(\to\) \verb|\pTsq|.


\newpage
%-------------------------------------------------------------------------------
\section{\File{atlasxref.sty}}

Turn on including these definitions with the option \Option{xref=true} and off with the option \Option{xref=false}.

\begin{xtabular}{ll}
\end{xtabular}


\noindent The following macros with arguments are also defined:
\begin{xtabular}{ll}
\verb|\App{1}|  & \App{1}\\
\verb|\Eqn{1}|  & \Eqn{1}\\
\verb|\Fig{1}|  & \Fig{1}\\
\verb|\Refn{1}|  & \Refn{1}\\
\verb|\Sect{1}| & \Sect{1}\\
\verb|\Tab{1}|  & \Tab{1}\\
\verb|\Apps{1}{4}| & \Apps{1}{4} \\
\verb|\Eqns{1}{4}| & \Eqns{1}{4} \\
\verb|\Figs{1}{4}| & \Figs{1}{4} \\
\verb|\Refns{1}{4}| & \Refns{1}{4} \\
\verb|\Sects{1}{4}| & \Sects{1}{4} \\
\verb|\Tabs{1}{4}| & \Tabs{1}{4} \\
\verb|\Apprange{1}{4}| & \Apprange{1}{4} \\
\verb|\Eqnrange{1}{4}| & \Eqnrange{1}{4} \\
\verb|\Figrange{1}{4}| & \Figrange{1}{4} \\
\verb|\Refrange{1}{4}| & \Refrange{1}{4} \\
\verb|\Sectrange{1}{4}| & \Sectrange{1}{4} \\
\verb|\Tabrange{1}{4}| & \Tabrange{1}{4}
\end{xtabular}

The idea is that you can adapt these definitions according to your own preferences (or those of a journal).
Note that the macros \Macro{Ref} and \Macro{Refs} were renamed to \Macro{Refn} and \Macro{Refns}
in \Package{atlaslatex} 08-00-00, as \Macro{Ref} is now defined in the \Package{hyperref} package.


\newpage
%-------------------------------------------------------------------------------
\section{\File{atlasbsm.sty}}

Turn on including these definitions with the option \Option{BSM} and off with the option \Option{BSM=false}.

The macro \Macro{susy} simply puts a tilde (\(\tilde{\ }\)) over its argument,
e.g.\ \verb|\susy{q}| produces \susy{q}.

For \susy{q}, \susy{t}, \susy{b}, \slepton, \sel, \smu and
\stau, L and R states are defined; for stop, sbottom and stau also the
light (1) and heavy (2) states.
There are four neutralinos and two charginos defined,
the index number unfortunately needs to be written out completely.
For the charginos the last letter(s) indicate(s) the charge:
\enquote{p} for \(+\), \enquote{m} for \(-\), and \enquote{pm} for \(\pm\).

% Note that this file can be overwritten when atlaslatex is updated.
%
% Copyright (C) 2002-2024 CERN for the benefit of the ATLAS collaboration.
%
% +--------------------------------------------------------------------+
% |                                                                    |
% |  New particle stuff                                                |
% |                                                                    |
% +--------------------------------------------------------------------+
%
\newcommand*{\Azero}{\ensuremath{A^0}\xspace}
\newcommand*{\hzero}{\ensuremath{h^0}\xspace}
\newcommand*{\Hzero}{\ensuremath{H^0}\xspace}
\newcommand*{\Hboson}{\ensuremath{H}\xspace}
\newcommand*{\Hplus}{\ensuremath{H^+}\xspace}
\newcommand*{\Hminus}{\ensuremath{H^-}\xspace}
\newcommand*{\Hpm}{\ensuremath{H^{\pm}}\xspace}
\newcommand*{\Hmp}{\ensuremath{H^{\mp}}\xspace}
\newcommand*{\susy}[1]{\ensuremath{\tilde{#1}}\xspace}
\newcommand*{\ggino}{\ensuremath{\mathchoice%
      {\displaystyle\raise.4ex\hbox{\(\displaystyle\tilde\chi\)}}%
         {\textstyle\raise.4ex\hbox{\(\textstyle\tilde\chi\)}}%
       {\scriptstyle\raise.3ex\hbox{\(\scriptstyle\tilde\chi\)}}%
 {\scriptscriptstyle\raise.3ex\hbox{\(\scriptscriptstyle\tilde\chi\)}}}\xspace}

\newcommand*{\chinop}{\ensuremath{\mathchoice%
      {\displaystyle\raise.4ex\hbox{\(\displaystyle\tilde\chi^+\)}}%
         {\textstyle\raise.4ex\hbox{\(\textstyle\tilde\chi^+\)}}%
       {\scriptstyle\raise.3ex\hbox{\(\scriptstyle\tilde\chi^+\)}}%
 {\scriptscriptstyle\raise.3ex\hbox{\(\scriptscriptstyle\tilde\chi^+\)}}}\xspace}
\newcommand*{\chinom}{\ensuremath{\mathchoice%
      {\displaystyle\raise.4ex\hbox{\(\displaystyle\tilde\chi^-\)}}%
         {\textstyle\raise.4ex\hbox{\(\textstyle\tilde\chi^-\)}}%
       {\scriptstyle\raise.3ex\hbox{\(\scriptstyle\tilde\chi^-\)}}%
 {\scriptscriptstyle\raise.3ex\hbox{\(\scriptscriptstyle\tilde\chi^-\)}}}\xspace}
\newcommand*{\chinopm}{\ensuremath{\mathchoice%
      {\displaystyle\raise.4ex\hbox{\(\displaystyle\tilde\chi^\pm\)}}%
         {\textstyle\raise.4ex\hbox{\(\textstyle\tilde\chi^\pm\)}}%
       {\scriptstyle\raise.3ex\hbox{\(\scriptstyle\tilde\chi^\pm\)}}%
 {\scriptscriptstyle\raise.3ex\hbox{\(\scriptscriptstyle\tilde\chi^\pm\)}}}\xspace}
\newcommand*{\chinomp}{\ensuremath{\mathchoice%
      {\displaystyle\raise.4ex\hbox{\(\displaystyle\tilde\chi^\mp\)}}%
         {\textstyle\raise.4ex\hbox{\(\textstyle\tilde\chi^\mp\)}}%
       {\scriptstyle\raise.3ex\hbox{\(\scriptstyle\tilde\chi^\mp\)}}%
 {\scriptscriptstyle\raise.3ex\hbox{\(\scriptscriptstyle\tilde\chi^\mp\)}}}\xspace}

\newcommand*{\chinoonep}{\ensuremath{\mathchoice%
      {\displaystyle\raise.4ex\hbox{\(\displaystyle\tilde\chi^+_1\)}}%
         {\textstyle\raise.4ex\hbox{\(\textstyle\tilde\chi^+_1\)}}%
       {\scriptstyle\raise.3ex\hbox{\(\scriptstyle\tilde\chi^+_1\)}}%
 {\scriptscriptstyle\raise.3ex\hbox{\(\scriptscriptstyle\tilde\chi^+_1\)}}}\xspace}
\newcommand*{\chinoonem}{\ensuremath{\mathchoice%
      {\displaystyle\raise.4ex\hbox{\(\displaystyle\tilde\chi^-_1\)}}%
         {\textstyle\raise.4ex\hbox{\(\textstyle\tilde\chi^-_1\)}}%
       {\scriptstyle\raise.3ex\hbox{\(\scriptstyle\tilde\chi^-_1\)}}%
 {\scriptscriptstyle\raise.3ex\hbox{\(\scriptscriptstyle\tilde\chi^-_1\)}}}\xspace}
\newcommand*{\chinoonepm}{\ensuremath{\mathchoice%
      {\displaystyle\raise.4ex\hbox{\(\displaystyle\tilde\chi^\pm_1\)}}%
         {\textstyle\raise.4ex\hbox{\(\textstyle\tilde\chi^\pm_1\)}}%
       {\scriptstyle\raise.3ex\hbox{\(\scriptstyle\tilde\chi^\pm_1\)}}%
 {\scriptscriptstyle\raise.3ex\hbox{\(\scriptscriptstyle\tilde\chi^\pm_1\)}}}\xspace}

\newcommand*{\chinotwop}{\ensuremath{\mathchoice%
      {\displaystyle\raise.4ex\hbox{\(\displaystyle\tilde\chi^+_2\)}}%
         {\textstyle\raise.4ex\hbox{\(\textstyle\tilde\chi^+_2\)}}%
       {\scriptstyle\raise.3ex\hbox{\(\scriptstyle\tilde\chi^+_2\)}}%
 {\scriptscriptstyle\raise.3ex\hbox{\(\scriptscriptstyle\tilde\chi^+_2\)}}}\xspace}
\newcommand*{\chinotwom}{\ensuremath{\mathchoice%
      {\displaystyle\raise.4ex\hbox{\(\displaystyle\tilde\chi^-_2\)}}%
         {\textstyle\raise.4ex\hbox{\(\textstyle\tilde\chi^-_2\)}}%
       {\scriptstyle\raise.3ex\hbox{\(\scriptstyle\tilde\chi^-_2\)}}%
 {\scriptscriptstyle\raise.3ex\hbox{\(\scriptscriptstyle\tilde\chi^-_2\)}}}\xspace}
\newcommand*{\chinotwopm}{\ensuremath{\mathchoice%
      {\displaystyle\raise.4ex\hbox{\(\displaystyle\tilde\chi^\pm_2\)}}%
         {\textstyle\raise.4ex\hbox{\(\textstyle\tilde\chi^\pm_2\)}}%
       {\scriptstyle\raise.3ex\hbox{\(\scriptstyle\tilde\chi^\pm_2\)}}%
 {\scriptscriptstyle\raise.3ex\hbox{\(\scriptscriptstyle\tilde\chi^\pm_2\)}}}\xspace}

\newcommand*{\nino}{\ensuremath{\mathchoice%
      {\displaystyle\raise.4ex\hbox{\(\displaystyle\tilde\chi^0\)}}%
         {\textstyle\raise.4ex\hbox{\(\textstyle\tilde\chi^0\)}}%
       {\scriptstyle\raise.3ex\hbox{\(\scriptstyle\tilde\chi^0\)}}%
 {\scriptscriptstyle\raise.3ex\hbox{\(\scriptscriptstyle\tilde\chi^0\)}}}\xspace}

\newcommand*{\ninoone}{\ensuremath{\mathchoice%
      {\displaystyle\raise.4ex\hbox{\(\displaystyle\tilde\chi^0_1\)}}%
         {\textstyle\raise.4ex\hbox{\(\textstyle\tilde\chi^0_1\)}}%
       {\scriptstyle\raise.3ex\hbox{\(\scriptstyle\tilde\chi^0_1\)}}%
 {\scriptscriptstyle\raise.3ex\hbox{\(\scriptscriptstyle\tilde\chi^0_1\)}}}\xspace}
\newcommand*{\ninotwo}{\ensuremath{\mathchoice%
      {\displaystyle\raise.4ex\hbox{\(\displaystyle\tilde\chi^0_2\)}}%
         {\textstyle\raise.4ex\hbox{\(\textstyle\tilde\chi^0_2\)}}%
       {\scriptstyle\raise.3ex\hbox{\(\scriptstyle\tilde\chi^0_2\)}}%
 {\scriptscriptstyle\raise.3ex\hbox{\(\scriptscriptstyle\tilde\chi^0_2\)}}}\xspace}
\newcommand*{\ninothree}{\ensuremath{\mathchoice%
      {\displaystyle\raise.4ex\hbox{\(\displaystyle\tilde\chi^0_3\)}}%
         {\textstyle\raise.4ex\hbox{\(\textstyle\tilde\chi^0_3\)}}%
       {\scriptstyle\raise.3ex\hbox{\(\scriptstyle\tilde\chi^0_3\)}}%
 {\scriptscriptstyle\raise.3ex\hbox{\(\scriptscriptstyle\tilde\chi^0_3\)}}}\xspace}
\newcommand*{\ninofour}{\ensuremath{\mathchoice%
      {\displaystyle\raise.4ex\hbox{\(\displaystyle\tilde\chi^0_4\)}}%
         {\textstyle\raise.4ex\hbox{\(\textstyle\tilde\chi^0_4\)}}%
       {\scriptstyle\raise.3ex\hbox{\(\scriptstyle\tilde\chi^0_4\)}}%
 {\scriptscriptstyle\raise.3ex\hbox{\(\scriptscriptstyle\tilde\chi^0_4\)}}}\xspace}

\newcommand*{\gravino}{\ensuremath{\tilde{G}}\xspace}
\newcommand*{\Zprime}{\ensuremath{Z^\prime}\xspace}
\newcommand*{\Zstar}{\ensuremath{Z^{*}}\xspace}
\newcommand*{\squark}{\ensuremath{\tilde{q}}\xspace}
\newcommand*{\squarkL}{\ensuremath{\tilde{q}_{\mathrm{L}}}\xspace}
\newcommand*{\squarkR}{\ensuremath{\tilde{q}_{\mathrm{R}}}\xspace}
\newcommand*{\gluino}{\ensuremath{\tilde{g}}\xspace}
\renewcommand*{\stop}{\ensuremath{\tilde{t}}\xspace}
\newcommand*{\stopone}{\ensuremath{\tilde{t}_1}\xspace}
\newcommand*{\stoptwo}{\ensuremath{\tilde{t}_2}\xspace}
\newcommand*{\stopL}{\ensuremath{\tilde{t}_{\mathrm{L}}}\xspace}
\newcommand*{\stopR}{\ensuremath{\tilde{t}_{\mathrm{R}}}\xspace}
\newcommand*{\sbottom}{\ensuremath{\tilde{b}}\xspace}
\newcommand*{\sbottomone}{\ensuremath{\tilde{b}_1}\xspace}
\newcommand*{\sbottomtwo}{\ensuremath{\tilde{b}_2}\xspace}
\newcommand*{\sbottomL}{\ensuremath{\tilde{b}_{\mathrm{L}}}\xspace}
\newcommand*{\sbottomR}{\ensuremath{\tilde{b}_{\mathrm{R}}}\xspace}
\newcommand*{\slepton}{\ensuremath{\tilde{\ell}}\xspace}
\newcommand*{\sleptonL}{\ensuremath{\tilde{\ell}_{\mathrm{L}}}\xspace}
\newcommand*{\sleptonR}{\ensuremath{\tilde{\ell}_{\mathrm{R}}}\xspace}
\newcommand*{\sel}{\ensuremath{\tilde{e}}\xspace}
\newcommand*{\selL}{\ensuremath{\tilde{e}_{\mathrm{L}}}\xspace}
\newcommand*{\selR}{\ensuremath{\tilde{e}_{\mathrm{R}}}\xspace}
\newcommand*{\smu}{\ensuremath{\tilde{\mu}}\xspace}
\newcommand*{\smuL}{\ensuremath{\tilde{\mu}_{\mathrm{L}}}\xspace}
\newcommand*{\smuR}{\ensuremath{\tilde{\mu}_{\mathrm{R}}}\xspace}
\newcommand*{\stau}{\ensuremath{\tilde{\tau}}\xspace}
\newcommand*{\stauL}{\ensuremath{\tilde{\tau}_{\mathrm{L}}}\xspace}
\newcommand*{\stauR}{\ensuremath{\tilde{\tau}_{\mathrm{R}}}\xspace}
\newcommand*{\stauone}{\ensuremath{\tilde{\tau}_1}\xspace}
\newcommand*{\stautwo}{\ensuremath{\tilde{\tau}_2}\xspace}
\newcommand*{\snu}{\ensuremath{\tilde{\nu}}\xspace}



\newpage
%-------------------------------------------------------------------------------
\section{\File{atlasheavyion.sty}}

Turn on including these definitions with the option \Option{hion=true} and off with the option \Option{hion=false}.
The heavy ion definitions use the package \Package{mhchem} to help with the formatting of chemical elements.
This package is included by \File{atlasheavyion.sty}.

\begin{xtabular}{ll}
\verb|\NucNuc| & \NucNuc \\
\verb|\nn| & \nn \\
\verb|\pn| & \pn \\
\verb|\np| & \np \\
\verb|\PbPb| & \PbPb \\
\verb|\AuAu| & \AuAu \\
\verb|\CuCu| & \CuCu \\
\verb|\pA| & \pA \\
\verb|\pNuc| & \pNuc \\
\verb|\pdA| & \pdA \\
\verb|\dAu| & \dAu \\
\verb|\pPb| & \pPb \\
\verb|\Npart| & \Npart \\
\verb|\avgNpart| & \avgNpart \\
\verb|\Ncoll| & \Ncoll \\
\verb|\avgNcoll| & \avgNcoll \\
\verb|\TA| & \TA \\
\verb|\avgTA| & \avgTA \\
\verb|\TPb| & \TPb \\
\verb|\avgTPb| & \avgTPb \\
\verb|\TAA| & \TAA \\
\verb|\avgTAA| & \avgTAA \\
\verb|\TAB| & \TAB \\
\verb|\avgTAB| & \avgTAB \\
\verb|\TpPb| & \TpPb \\
\verb|\avgTpPb| & \avgTpPb \\
\verb|\Gl| & \Gl \\
\verb|\GG| & \GG \\
\verb|\sqn| & \sqn \\
\verb|\lns| & \lns \\
\verb|\sumETPb| & \sumETPb \\
\verb|\sumETp| & \sumETp \\
\verb|\sumETA| & \sumETA \\
\verb|\RAA| & \RAA \\
\verb|\RCP| & \RCP \\
\verb|\RpA| & \RpA \\
\verb|\RpPb| & \RpPb \\
\verb|\dif| & \dif \\
\verb|\dNchdeta| & \dNchdeta \\
\verb|\dNevtdET| & \dNevtdET \\
\verb|\ystar| & \ystar \\
\verb|\ycms| & \ycms \\
\verb|\ygappb| & \ygappb \\
\verb|\ygapp| & \ygapp \\
\verb|\fgap| & \fgap \\
\end{xtabular}


%The following symbols were removed or modified with respect to the original submission
%\input{atlasheavyion-mod}


\newpage
%-------------------------------------------------------------------------------
\section{\File{atlasjetetmiss.sty}}

Turn on including these definitions with the option \Option{jetetmiss=true} and off with the option \Option{jetetmiss=false}.

%-------------------------------------------------------------------------------
% Useful definitions from the Jet/Etmiss group.
% Include with jetetmiss option in atlasphysics.sty.
% Not included by default.
% Also needs atlasmisc.sty (option misc).
% Compiled by David Miller.
% Adapted for atlaslatex by Ian Brock.
%
% Note that this file can be overwritten when atlaslatex is updated.
%
% Copyright (C) 2002-2024 CERN for the benefit of the ATLAS collaboration.
%-------------------------------------------------------------------------------

%%% Topoclusters
\newcommand*{\topo}{topo-cluster}
\newcommand*{\Topo}{Topo-cluster}
\newcommand*{\topos}{topo-clusters}
\newcommand*{\Topos}{Topo-clusters}


%%% Standard terms
\newcommand*{\insitu}{in~situ\xspace}
\newcommand*{\Insitu}{In~situ\xspace}


%%% Event generator
\newcommand*{\LS}{\ensuremath{\text{LS}}\xspace}
\newcommand*{\NLOjet}{\textsc{NLOJet}\texttt{++}\xspace}
\newcommand*{\Fastjet}{\textsc{FastJet}\xspace}
\newcommand*{\TwoToTwo}{\ensuremath{2\rightarrow2}\xspace}


%%% Jets
\newcommand*{\largeR}{large-\ensuremath{R}\xspace}
\newcommand*{\LargeR}{Large-\ensuremath{R}\xspace}
%\newcommand*{\kt}{\ensuremath{k_{t}}\xspace}
%\newcommand*{\antikt}{anti-\kt}
%\newcommand*{\Antikt}{Anti-\kt}
\newcommand*{\akt}{\antikt}
\newcommand*{\Akt}{\Antikt}
\newcommand*{\AKT}{\akt}
\newcommand*{\AKTFat}{\antikt, \ensuremath{R=1.0}\xspace}
\newcommand*{\AKTPrune}{\antikt, \ensuremath{R=1.0} (pruned)\xspace}
\newcommand*{\AKTFilt}{\antikt, \ensuremath{R=1.0} (filtered)\xspace}
\newcommand*{\KTSix}{\kt, \ensuremath{R=0.6}\xspace}
\newcommand*{\ca}{Cambridge--Aachen\xspace}
\newcommand*{\CamKt}{\ensuremath{\text{C/A}}\xspace}
\newcommand*{\CASix}{\CamKt, \ensuremath{R=0.6}\xspace}
\newcommand*{\CAFat}{\CamKt, \ensuremath{R=1.2}\xspace}
\newcommand*{\CAPrune}{\CamKt, \ensuremath{R=1.2} (pruned)\xspace}
\newcommand*{\CAFilt}{\CamKt, \ensuremath{R=1.2} (filtered)\xspace}

%%% HEPTopTagger
\newcommand*{\htt}{HEPTopTagger\xspace}
\newcommand*{\mcut}{\ensuremath{m_{\text{cut}}}\xspace}
\newcommand*{\Nfilt}{\ensuremath{N_{\text{filt}}}\xspace}

%%% Groomed jets
\newcommand*{\Rfilt}{\ensuremath{R_{\text{filt}}}\xspace}
\newcommand*{\ymin}{\ensuremath{y_{\min}}\xspace}
\newcommand*{\fcut}{\ensuremath{f_{\text{cut}}}\xspace}
\newcommand*{\Rsub}{\ensuremath{R_{\text{sub}}}\xspace}
%\newcommand*{\drsub}{\ensuremath{\Rsub}\xspace}
\newcommand*{\mufrac}{\ensuremath{\mu_{\text{frac}}}\xspace}
\newcommand*{\Rcut}{\ensuremath{R_{\text{cut}}}\xspace}
\newcommand*{\zcut}{\ensuremath{z_{\text{cut}}}\xspace}

% Jet calibration (GSC)
\newcommand*{\ftile}{\ensuremath{f_{\text{Tile}0}}\xspace}
\newcommand*{\fem}{\ensuremath{f_{\text{LAr}3}}\xspace}
\newcommand*{\fpres}{\ensuremath{f_{\text{PS}}}\xspace}
\newcommand*{\fhec}{\ensuremath{f_{\text{HEC}0}}\xspace}
\newcommand*{\ffcal}{\ensuremath{f_{\text{FCal}1}}\xspace}

\newcommand*{\central}{\ensuremath{0.3 \leq |\eta| < 0.8}\xspace}
\newcommand*{\ecap}{\ensuremath{2.1 \leq |\eta| < 2.8}\xspace}
\newcommand*{\forward}{\ensuremath{3.6 \leq |\eta| < 4.5}\xspace}


%%% Pileup
\newcommand*{\avg}[1]{\ensuremath{\langle #1 \rangle}\xspace} % for e.g. <\mu>
\newcommand*{\Npv}{\ensuremath{N_{\text{PV}}}\xspace}
\newcommand*{\Nref}{\ensuremath{\Npv^{\text{ref}}}\xspace}
\newcommand*{\Navg}{\ensuremath{\avg{\Npv}}\xspace}
\newcommand*{\avgmu}{\ensuremath{\avg{\mu}}\xspace}


%%% JES
\newcommand*{\JES}{\text{JES}\xspace}
\newcommand*{\JMS}{\text{JMS}\xspace}
\newcommand*{\EMJES}{\text{EM+JES}\xspace}
\newcommand*{\GCWJES}{\text{GCW+JES}\xspace}
\newcommand*{\LCWJES}{\text{LCW+JES}\xspace}
\newcommand*{\EM}{\text{EM}\xspace}
\newcommand*{\GCW}{\text{GCW}\xspace}
\newcommand*{\LCW}{\text{LCW}\xspace}
\newcommand*{\GSL}{\text{GSL}\xspace}
\newcommand*{\GS}{\text{GS}\xspace}
\newcommand*{\MTF}{\text{MTF}\xspace}
\newcommand*{\MPF}{\text{MPF}\xspace}


%%% Jet multiplicity
\newcommand*{\Njet}{\ensuremath{N_\text{jet}}\xspace}
\newcommand*{\njet}{\Njet}


%%% Kinematics
\newcommand*{\ETjet}{\ensuremath{\ET^{\text{jet}}}\xspace}
\newcommand*{\etjet}{\ETjet}
\newcommand*{\pTavg}{\ensuremath{\pT^\text{avg}}\xspace}
\newcommand*{\ptavg}{\pTavg}
\newcommand*{\pTjet}{\ensuremath{\pT^\text{jet}}\xspace}
\newcommand*{\ptjet}{\pTjet}
\newcommand*{\pTcorr}{\ensuremath{\pT^\text{corr}}\xspace}
\newcommand*{\ptcorr}{\pTcorr}
\newcommand*{\pTjeti}{\ensuremath{p_{\text{T}, i}^\text{jet}}\xspace}
\newcommand*{\ptjeti}{\pTjeti}
\newcommand*{\pTrecoil}{\ensuremath{\pT^\text{recoil}}\xspace}
\newcommand*{\ptrecoil}{\pTrecoil}
\newcommand*{\pTleading}{\ensuremath{\pT^{\text{leading}}}\xspace}
\newcommand*{\ptleading}{\pTleading}
\newcommand*{\pTjetEM}{\ensuremath{p_{\text{T, \EM}}^{\text{jet}}}\xspace}
\newcommand*{\ptjetEM}{\pTjetEM}
\newcommand*{\pThat}{\ensuremath{\hat{p}_\text{T}}\xspace}
\newcommand*{\pthat}{\pThat}
\newcommand*{\pTprobe}{\ensuremath{\pT^{\text{probe}}}\xspace}
\newcommand*{\ptprobe}{\pTprobe}
\newcommand*{\pTref}{\ensuremath{\pT^{\text{ref}}}\xspace}
\newcommand*{\ptref}{\pTref}
\newcommand*{\pToff}{\ensuremath{\mathcal{O}}\xspace}
\newcommand*{\ptoff}{\pToff}
\newcommand*{\pToffjet}{\ensuremath{\pToff^{\text{jet}}}\xspace}
\newcommand*{\ptoffjet}{\pToffjet}
\newcommand*{\pTZ}{\ensuremath{\pT^{Z}}\xspace}
\newcommand*{\ptZ}{\pTZ}

% Use siunitx definitions if available. Otherwise assume \GeV is defined.
\@ifpackageloaded{siunitx}{%
  \newcommand*{\pTRange}[2]{\ensuremath{{#1} \leq \pTjet < \qty{#2}{\GeV}}\xspace}
  \newcommand*{\ptRange}[2]{\ensuremath{{#1} \leq \ptjet < \qty{#2}{\GeV}}\xspace}
}{%
  \newcommand*{\pTRange}[2]{\ensuremath{{#1} \leq \pTjet < {#2} \GeV}\xspace}
  \newcommand*{\ptRange}[2]{\ensuremath{{#1} \leq \ptjet < {#2} \GeV}\xspace}
}
\newcommand*{\pTtrue}{\ensuremath{\pT^{\text{true}}}\xspace}
\newcommand*{\pttrue}{\pTtrue}
\newcommand*{\pTtruth}{\ensuremath{\pT^{\text{true}}}\xspace}
\newcommand*{\pttruth}{\pTtruth}
\newcommand*{\pTreco}{\ensuremath{\pT^{\text{reco}}}\xspace}
\newcommand*{\ptreco}{\pTreco}
\newcommand*{\pTtrk}{\ensuremath{\pT^{\text{track}}}\xspace} 
\newcommand*{\pttrk}{\pTtrk}
\newcommand*{\ptrk}{\ensuremath{p^{\text{track}}}\xspace} 
\newcommand*{\pTtrkjet}{\ensuremath{\pT^{\text{track jet}}}\xspace}
\newcommand*{\pttrkjet}{\pTtrkjet}
\newcommand*{\ntrk}{\ensuremath{n_{\text{track}}}\xspace}
\newcommand*{\EoverP}{\ensuremath{E/p}\xspace}
\newcommand*{\Etrue}{\ensuremath{E^{\text{true}}}\xspace}
\newcommand*{\Etruth}{\ensuremath{E^{\text{true}}}\xspace}
\newcommand*{\Ecalo}{\ensuremath{E^{\text{jet}}}\xspace}
\newcommand*{\EcaloEM}{\ensuremath{E^{\text{jet}}_{\EM}}\xspace}
\newcommand*{\asym}{\ensuremath{\mathcal{A}}\xspace}
\newcommand*{\Response}{\ensuremath{\mathcal{R}}\xspace} 
\newcommand*{\Rcalo}{\ensuremath{\Response^{\text{jet}}}\xspace} 
\newcommand*{\Rcalom}{\ensuremath{\Response^{\text{jet}}_{m}}\xspace} 
\newcommand*{\RcaloEM}{\ensuremath{\Rcalo_{\EM}}\xspace}
\newcommand*{\RMPF}{\ensuremath{\Response_{\MPF}}\xspace}
\newcommand*{\EcaloCALIB}{\ensuremath{E^{\text{jet}}}\xspace}
\newcommand*{\RcaloCALIB}{\ensuremath{\mathcal{R}^{\text{jet}}}\xspace} 
\newcommand*{\EcaloEMJES}{\ensuremath{E^{\text{jet}}_{\EMJES}}\xspace}
\newcommand*{\RcaloEMJES}{\ensuremath{\mathcal{R}^{\text{jet}}_{\EMJES}}\xspace}
\newcommand*{\EcaloGCWJES}{\ensuremath{E^{\text{jet}}_{\GCWJES}}\xspace}
\newcommand*{\RcaloGCWJES}{\ensuremath{\mathcal{R}^{\text{jet}}_{\GCWJES}}\xspace}
\newcommand*{\EcaloLCWJES}{\ensuremath{E^{\text{jet}}_{\LCWJES}}\xspace}
\newcommand*{\RcaloLCWJES}{\ensuremath{\mathcal{R}^{\text{jet}}_{\LCWJES}}\xspace}
\newcommand*{\Rtrack}{\ensuremath{\mathcal{R}^{\text{track jet}}}\xspace}
\newcommand*{\rtrk}{\ensuremath{r_{\text{trk}}}\xspace}
\newcommand*{\Rtrk}{\ensuremath{R_{\text{trk}}}\xspace}
\newcommand*{\rtrackjet}{\ensuremath{r^{\text{calo / track jet}}}\xspace}
\newcommand*{\rtrackjetiso}{\ensuremath{\rtrackjet_{\text{iso}}}\xspace}
\newcommand*{\rtrackjetnoniso}{\ensuremath{\rtrackjet_{\text{non-iso}}}\xspace}
\newcommand*{\rtrackjetisoratio}{\ensuremath{\rtrackjet_{\text{non-iso/iso}}}\xspace}
\newcommand*{\gammajet}{\ensuremath{\gamma\text{+jet}}\xspace}
\newcommand*{\deltaphijetgamma}{\ensuremath{\Delta \phi_{\text{jet--}\gamma}}\xspace}


%%% Spatial quantities
\newcommand*{\rapjet}{\ensuremath{y}\xspace}
\newcommand*{\etajet}{\ensuremath{\eta}\xspace}
\newcommand*{\phijet}{\ensuremath{\phi}\xspace}
\newcommand*{\etadet}{\ensuremath{\eta_{\text{det}}}\xspace}
\newcommand*{\etatrk}{\ensuremath{\eta^{\text{track}}}\xspace}
\newcommand*{\etaRange}[2]{\ensuremath{{#1} \leq |\eta| < {#2}}\xspace}
\newcommand*{\AetaRange}[1]{\ensuremath{|\eta| < {#1}}\xspace}

\newcommand*{\Rmin}{\ensuremath{R_{\min}}\xspace}
\newcommand*{\DeltaR}{\ensuremath{\Delta R}\xspace}
\newcommand*{\DetaDphi}{\ensuremath{\sqrt{(\Delta\eta)^{2}+(\Delta\phi)^{2}}}\xspace}
\newcommand*{\Deta}{\ensuremath{|\Delta\eta|}\xspace}
\newcommand*{\Drap}{\ensuremath{|\Delta y|}\xspace}
\newcommand*{\DetaOneTwo}{\ensuremath{|\Delta\eta(\text{jet}1, \text{jet}2)|}\xspace}
\newcommand*{\DyDphi}{\ensuremath{\sqrt{(\Delta y)^{2} + (\Delta\phi)^{2}}}\xspace}
\newcommand*{\DeltaRdef}{\ensuremath{\DeltaR = \DetaDphi}\xspace}
\newcommand*{\DeltaRydef}{\ensuremath{\DeltaR = \DyDphi}\xspace}
\newcommand*{\DeltaRtrk}{\ensuremath{\Delta R(\text{trk}_1,\text{trk}_2)}\xspace}


%%% Pile-up correction and mitigation related
\newcommand*{\JVF}{\ensuremath{\text{JVF}}\xspace}
\newcommand*{\cJVF}{\ensuremath{\text{corrJVF}}\xspace}
\newcommand*{\RpT}{\ensuremath{R_\text{\pT}}\xspace}
\newcommand*{\JVT}{\ensuremath{\text{JVT}}\xspace}
\newcommand*{\ghostpt}{\ensuremath{g_{t}}\xspace}
\newcommand*{\ghostptavg}{\ensuremath{\avg{\ghostpt}}\xspace}
\newcommand*{\ghostfm}{\ensuremath{g_{\mu}}\xspace}
\newcommand*{\ghostfmi}{\ensuremath{g_{\mu,i}}\xspace}
\newcommand*{\ghostdensity}{\ensuremath{\nu_{g}}\xspace}
\newcommand*{\ghostrho}{\ensuremath{\ghostdensity\ghostptavg}\xspace}
\newcommand*{\Aghost}{\ensuremath{A_{g}}\xspace}
\newcommand*{\Amu}{\ensuremath{A_{\mu}}\xspace}
\newcommand*{\Amui}{\ensuremath{A_{\mu, i}}\xspace}
\newcommand*{\jetarea}{\ensuremath{A^{\text{jet}}}\xspace}
\newcommand*{\jetareafm}{\ensuremath{A^{\text{jet}}_{\mu}}\xspace}
\newcommand*{\jetareai}{\ensuremath{\jetarea_{i}}\xspace}
\newcommand*{\Rkt}{\ensuremath{R_{\kt}}\xspace}
\newcommand*{\pTmuslope}{\ensuremath{\partial\langle\Delta \pt\rangle/\partial\avgmu}\xspace}
\newcommand*{\ptmuslope}{\pTmuslope}
\newcommand*{\pTnpvslope}{\ensuremath{\partial\langle\Delta \pt\rangle/\partial\Npv}\xspace}
\newcommand*{\ptnpvslope}{\pTnpvslope}
\newcommand*{\pTmuunc}{\ensuremath{\Delta\left(\ptmuslope\right)}\xspace}
\newcommand*{\ptmuunc}{\pTmuunc}
\newcommand*{\pTnpvunc}{\ensuremath{\Delta\left(\ptnpvslope\right)}\xspace}
\newcommand*{\ptnpvunc}{\pTnpvunc}

\newcommand*{\sumPt}{\ensuremath{\sum \vec{p}_{\text{T}}}\xspace}
\newcommand*{\sumpt}{\sumPt}
%\newcommand*{\sumPtTr}{\ensuremath{|\sum \vec{p}_{\text{T}}^{\text{ track}}|}\xspace}
%\newcommand*{\sumpt}{\sumPt}
\newcommand*{\sumpTtrk}{\ensuremath{\sum\pt^\text{track}}\xspace}
\newcommand*{\sumpttrk}{\sumpTtrk}
\newcommand*{\nPUtrk}{\ensuremath{n_\text{trk}^\text{PU}}\xspace}

%%% Jet structure observables
\newcommand*{\mjet}{\ensuremath{m^{\text{jet}}}\xspace}
\newcommand*{\mlead}{\ensuremath{\mjet_{1}}\xspace}
\newcommand*{\mleadavg}{\ensuremath{\langle\mlead\rangle}\xspace}
\newcommand*{\Mjet}{\mjet}
\newcommand*{\massjet}{\mjet}
\newcommand*{\masscorr}{\ensuremath{m^{\text{corr}}}}
\newcommand*{\mthresh}{\ensuremath{M_{\text{threshold}}}\xspace}
\newcommand*{\mjetavg}{\ensuremath{\langle\mjet\rangle}\xspace}
\newcommand*{\masstrkjet}{\ensuremath{m^{\text{track jet}}}\xspace}
\newcommand*{\width}{\ensuremath{w}\xspace}
\newcommand*{\wcalo}{\ensuremath{\width^{\text{calo}}}\xspace}
\newcommand*{\wtrk}{\ensuremath{\width^{\text{track}}}\xspace}
\newcommand*{\shapeV}{\ensuremath{\mathcal{V}}\xspace}
\newcommand*{\pTsubjet}{\ensuremath{\pT^\text{subjet}}\xspace}
\newcommand*{\ptsubjet}{\pTsubjet}
\newcommand*{\sjone}{\ensuremath{j_{1}}\xspace}
\newcommand*{\sjtwo}{\ensuremath{j_{2}}\xspace}
\newcommand*{\msubjone}{\ensuremath{m^{\sjone}}\xspace}
\newcommand*{\msubjtwo}{\ensuremath{m^{\sjtwo}}\xspace}
\newcommand*{\pTsubji}{\ensuremath{\pT^{i}}\xspace}
\newcommand*{\ptsubji}{\pTsubji}
\newcommand*{\pTsubjone}{\ensuremath{\pT^{\sjone}}\xspace}
\newcommand*{\ptsubjone}{\pTsubjone}
\newcommand*{\pTsubjtwo}{\ensuremath{\pT^{\sjtwo}}\xspace}
\newcommand*{\ptsubjtwo}{\pTsubjtwo}
\newcommand*{\Rsubjets}{\ensuremath{R_{\sjone,\sjtwo}}\xspace}
\newcommand*{\DRsubjets}{\ensuremath{\Delta\Rsubjets}\xspace}
\newcommand*{\yij}{\ensuremath{y_{ij}}\xspace}
\newcommand*{\dcut}{\ensuremath{d_{\text{cut}}}\xspace}
\newcommand*{\dmin}{\ensuremath{d_{\min}}\xspace}
\newcommand*{\dij}{\ensuremath{d_{ij}}\xspace}
\newcommand*{\Dij}{\ensuremath{\sqrt{\dij}}\xspace}
\newcommand*{\Donetwo}{\ensuremath{\sqrt{d_{12}}}\xspace}
\newcommand*{\Dtwothr}{\ensuremath{\sqrt{d_{23}}}\xspace}
\newcommand*{\yonetwo}{\ensuremath{y_{1}}\xspace}
\newcommand*{\ytwothr}{\ensuremath{y_{2}}\xspace}
\newcommand*{\yonetwoDef}{\ensuremath{\yonetwo=\Donetwo/\mjet}\xspace}
\newcommand*{\ytwothrDef}{\ensuremath{\ytwothr=\Dtwothr/\mjet}\xspace}
\newcommand*{\xj}{\ensuremath{x_{J}}\xspace}
\newcommand*{\jetFunc}{\ensuremath{J^{(eik),c} (\mjet,\pT,R)}\xspace}
\newcommand*{\tauone}{\ensuremath{\tau_1}\xspace}
\newcommand*{\tautwo}{\ensuremath{\tau_2}\xspace}
\newcommand*{\tauthr}{\ensuremath{\tau_3}\xspace}
\newcommand*{\tauN}{\ensuremath{\tau_N}\xspace}
\newcommand*{\tautwoone}{\ensuremath{\tau_{21}}\xspace}
\newcommand*{\tauthrtwo}{\ensuremath{\tau_{32}}\xspace}
\newcommand*{\dip}{\ensuremath{\mathcal{D}}\xspace}
\newcommand*{\diponetwo}{\ensuremath{\dip_{12}}\xspace}
\newcommand*{\diptwothr}{\ensuremath{\dip_{23}}\xspace}
\newcommand*{\diponethr}{\ensuremath{\dip_{13}}\xspace}
\newcommand*{\mtaSup}{\ensuremath{m^{\text{TA}}}\xspace}
\newcommand*{\mcalo}{\ensuremath{m^{\text{calo}}}\xspace}
\newcommand*{\mcomb}{\ensuremath{m^{\text{comb}}}\xspace}
\newcommand*{\ECFOne}{\ensuremath{ECF_1}\xspace}
\newcommand*{\ECFTwo}{\ensuremath{ECF_2}\xspace}
\newcommand*{\ECFThr}{\ensuremath{ECF_3}\xspace}
\newcommand*{\ECFThrNorm}{\ensuremath{e_3}\xspace}
\newcommand*{\DTwo}{\ensuremath{D_{2}}\xspace}
\newcommand*{\CTwo}{\ensuremath{C_{2}}\xspace}
\newcommand*{\FoxWolfRatio}{\ensuremath{R^{\text{FW}}_{2}}\xspace}
\newcommand*{\PlanarFlow}{\ensuremath{\mathcal{P}}\xspace}
\newcommand*{\Angularity}{\ensuremath{a_{3}}\xspace}
\newcommand*{\Aplanarity}{\ensuremath{A}\xspace}
\newcommand*{\KtDR}{\ensuremath{KtDR}\xspace}
\newcommand*{\Qw}{\ensuremath{Q_w}\xspace}
\newcommand*{\NConst}{\ensuremath{N^{\text{const}}}\xspace}

%-------------------------------------------------------------------------------
% The following definitions were in a first version of this style file
% that I received. They have either been moved into another file or
% are not really generic enough, or do not use a units package.
% They are kept for reference.
%-------------------------------------------------------------------------------

%%% b-tagging
%\newcommand*{\btag}{\ensuremath{b\text{-tagging}}\xspace}
%\newcommand*{\btagged}{\ensuremath{b\text{-tagged}}\xspace}
%\newcommand*{\bquark}{\ensuremath{b\text{-quark}}\xspace}
%\newcommand*{\bquarks}{\ensuremath{b\text{-quarks}}\xspace}
%\newcommand*{\bjet}{\ensuremath{b\text{-jet}}\xspace}
%\newcommand*{\bjets}{\ensuremath{b\text{-jets}}\xspace}


%%% Processes
%\newcommand*{\Zbb}{\ensuremath{Z \rightarrow \bbbar}\xspace}
%\newcommand*{\tWb}{\ensuremath{t \rightarrow Wb}\xspace}
%\newcommand*{\Wqqbar}{\ensuremath{W \rightarrow \qqbar}\xspace}
%%\newcommand*{\enu}{\ensuremath{e\nu}\xspace}  
%\newcommand*{\Zmumu}{\ensuremath{Z \rightarrow \mumu}\xspace}
%\newcommand*{\Wlnu}{\ensuremath{W \rightarrow \lnu}\xspace}
%\newcommand*{\Wenu}{\ensuremath{W \rightarrow \enu}\xspace}
%%\newcommand*{\munu}{\ensuremath{\mu\nu}\xspace}
%\newcommand*{\Wmunu}{\ensuremath{W \rightarrow \munu}\xspace}
%\newcommand*{\Wjets}{\ensuremath{\Wboson\text{+jets}}\xspace}
%\newcommand*{\Zjets}{\ensuremath{\Zboson\text{+jets}}\xspace}

%\newcommand*{\W}{\Wboson\xspace}
%\newcommand*{\Z}{\Zboson\xspace}


%%% Energy
%\newcommand*{\fourfgev}{\ensuremath{\qty{450}{\GeV}}\xspace}
%\newcommand*{\ninehgev}{\ensuremath{\qty{900}{\GeV}}\xspace}
%\newcommand*{\seventev}{\ensuremath{\qty{7}{\TeV}}\xspace}
%\newcommand*{\eighttev}{\ensuremath{\qty{8}{\TeV}}\xspace}
%\newcommand*{\sqsthirt}{\ensuremath{\sqrt{s} = \qty{13}{\TeV}}\xspace}
%\newcommand*{\sqsfourt}{\ensuremath{\sqrt{s} = \qty{14}{\TeV}}\xspace}
%\newcommand*{\sqsTev}{\ensuremath{\qty{1.96}{\TeV}}\xspace}
%\newcommand*{\threehtev}{\ensuremath{\qty{3.5}{\TeV}}\xspace}
%\newcommand*{\sqsnineh}{\ensuremath{\sqrt{s} = \ninehgev}\xspace}
%\newcommand*{\sqsseven}{\ensuremath{\sqrt{s} = \seventev}\xspace}
%\newcommand*{\sqseight}{\ensuremath{\sqrt{s} = \eighttev}\xspace}
%\newcommand*{\bunchSp}{\ensuremath{\tau_{\text{bunch}}}\xspace}


%%% Runs
%\newcommand*{\RunOne}{Run 1\xspace}
%\newcommand*{\RunTwo}{Run 2\xspace}
%\newcommand*{\RunThr}{Run 3\xspace}
%\newcommand*{\pp}{\ensuremath{pp}\xspace} -> atlasparticle.sty


%%% Luminosity
%\newcommand*{\invpb}{~\ensuremath{\text{pb}^{-1}}\xspace}
%\newcommand*{\invnb}{~\ensuremath{\text{nb}^{-1}}\xspace}
%\newcommand*{\invfb}{~\ensuremath{\text{fb}^{-1}}\xspace}
%\newcommand*{\invmub}{~\ensuremath{\mu\text{b}^{-1}}\xspace}
%\newcommand*{\cmSqSec}{cm\ensuremath{^{-2}}s\ensuremath{^{-1}}\xspace}
%\newcommand*{\lumi}{\ensuremath{\mathcal L}\xspace}
%\newcommand*{\calL}{\lumi}
%\newcommand*{\dldz}{\ensuremath{d\lumi/dz}\xspace}
%\newcommand*{\lofz}{\ensuremath{{\mathcal L}(z)}\xspace}
%\newcommand*{\sigyofz}{\ensuremath{\sigma_y(z)}\xspace}
%\newcommand*{\Lsp}{\ensuremath{{\mathcal L}_{sp}}\xspace}
%\newcommand*{\StoB}{\ensuremath{S/B}\xspace}


%%% Detectors -> atlasmisc
%\newcommand*{\radlength} {\ensuremath{X_0}}
%\newcommand*{\Lone}      {\texttt{L1}}
%\newcommand*{\HLT}       {\texttt{HLT}}
%\newcommand*{\ID}        {\texttt{ID}}
%\newcommand*{\Pixel}     {\texttt{Pixel}}
%\newcommand*{\SCT}       {\texttt{SCT}}
%\newcommand*{\TRT}       {\texttt{TRT}}
%
%\newcommand*{\LHC}       {LHC} - Dropped
%\newcommand*{\EMB}       {\texttt{EMB}}
%\newcommand*{\EME}       {\texttt{EME}}
%\newcommand*{\EMEC}      {\texttt{EMEC}}
%\newcommand*{\FCAL}      {\texttt{FCAL}}
%\newcommand*{\Cryo}      {\texttt{Cryo}}
%\newcommand*{\Gap}       {\texttt{Gap}}
%\newcommand*{\Scint}     {\texttt{Scint}}
%\newcommand*{\HEC}       {\texttt{HEC}}
%\newcommand*{\LAr}       {\texttt{LAr}}
%\newcommand*{\FCal}      {\texttt{FCal}}
%\newcommand*{\Tile}      {\texttt{Tile}}
%\newcommand*{\TileExt}   {\texttt{TileExt}}
%\newcommand*{\TileBar}   {\texttt{TileBar}}
%\newcommand*{\MBTS}      {\texttt{MBTS}}
%\newcommand*{\Presampler}{\texttt{Presampler}}


%%% Event generator stuff
%%\newcommand*{\Herwig}{\texttt{HERWIG}\xspace}
%\newcommand*{\herwigpp}{\texttt{HERWIG}++\xspace}
%%\newcommand*{\Herwigpp}{\herwigpp}
%\newcommand*{\geant}{\texttt{GEANT}4\xspace}
%\newcommand*{\jimmy}{\texttt{Jimmy}\xspace}
%\newcommand*{\Alpgen}{\texttt{Alpgen}\xspace}
%\newcommand*{\Sherpa}{\texttt{SHERPA}\xspace}
%\newcommand*{\Comphep}{\texttt{CompHep}\xspace}
%\newcommand*{\Madgraph}{\texttt{MadGraph}\xspace}
%\newcommand*{\Pythia}{\texttt{PYTHIA}\xspace}
%\newcommand*{\Powheg}{\texttt{POWHEG}\xspace}
%\newcommand*{\PowhegBox}{\texttt{POWHEG-BOX}\xspace}
%\newcommand*{\PowPythia}{\texttt{POWHEG}+\Pythia}
%\newcommand*{\Perugia}{\texttt{Perugia}\xspace}
%\newcommand*{\Prospino}{\texttt{Prospino}\xspace}
%\newcommand*{\Mcatnlo}{\texttt{MC@NLO}\xspace}
%\newcommand*{\Acermc}{\texttt{ACERMC}\xspace}
%\newcommand*{\PythiaP}{\Pythia (\Perugia 2010)\xspace}

%\newcommand*{\LO}        {\ensuremath{\text{LO}}\xspace}
%\newcommand*{\NLO}       {\ensuremath{\text{NLO}}\xspace}
%\newcommand*{\NLL}       {\ensuremath{\text{NLL}}\xspace}
%\newcommand*{\NNLO}      {\ensuremath{\text{N}}\NLO\xspace}
%\newcommand*{\muF}       {\ensuremath{\mu_\textsc{f}}\xspace}
%\newcommand*{\muR}       {\ensuremath{\mu_\textsc{r}}\xspace}


\noindent The macro \Macro{etaRange} produces what you would expect:
\verb|\etaRange{-2.5}{+2.5}| produces \etaRange{-2.5}{+2.5} while
\verb|\AetaRange{1.0}| produces \AetaRange{1.0}.
The macro \Macro{avg} can be used for average values:
\verb|\avg{\mu}| produces \avg{\mu}.


% \newpage
%-------------------------------------------------------------------------------
% \section{\File{atlasmath.sty}}

% Turn on including these definitions with the option \Option{math=true} and off with the option \Option{math=false}.

% \begin{xtabular}{ll}
\verb|\boxsq| & \boxsq \\
\verb|\grad| & \grad \\
\end{xtabular}


% \noindent The macro \Macro{spinor} is also defined.
% \verb|\spinor{u}| produces \spinor{u}.


% \newpage
%-------------------------------------------------------------------------------
% \section{\File{atlasother.sty}}

% Turn on including these definitions with the option \Option{other} and off with the option \Option{other=false}.

% \input{atlasother}
