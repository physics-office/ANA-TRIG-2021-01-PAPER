%-------------------------------------------------------------------------------
% This file provides a skeleton ATLAS paper.
%-------------------------------------------------------------------------------
% \pdfoutput=1
% The \pdfoutput command is needed by arXiv/JHEP/JINST to ensure use of pdflatex.
% It should be included in the first 5 lines of the file.
\pdfinclusioncopyfonts=1
% This command may be needed in order to get \ell in PDF plots to appear. Found in
% https://tex.stackexchange.com/questions/322010/pdflatex-glyph-undefined-symbols-disappear-from-included-pdf
%-------------------------------------------------------------------------------
% Specify where ATLAS LaTeX style files can be found.
\RequirePackage{latex/atlaslatexpath}
% Comment out the above line if the files are in a central location, e.g. $HOME/texmf.
%-------------------------------------------------------------------------------
\documentclass[PAPER, atlasdraft=true, UKenglish]{atlasdoc}
% The language of the document must be set: usually UKenglish or USenglish.
% british and american also work!
% Commonly used options:
%  atlasdraft=true|false This document is an ATLAS draft.
%  coverpage             Create ATLAS draft cover page for collaboration circulation.
%                        See atlas-draft-cover.tex for a list of variables that should be defined.
%  cernpreprint          Create front page for a CERN preprint.
%                        See atlas-preprint-cover.tex for a list of variables that should be defined.
%  NOTE                  The document is an ATLAS note (draft).
%  PAPER                 The document is an ATLAS paper (draft).
%  CONF                  The document is a CONF note (draft).
%  PUB                   The document is a PUB note (draft).
%  BOOK                  The document is of book form, like an LOI or TDR (draft).
%  paper=a4|letter       Set paper size to A4 (default) or letter.

%-------------------------------------------------------------------------------
% Extra packages:
\usepackage{atlaspackage}
% Commonly used options:
%  subfigure|subfig|subcaption  to use one of these packages for figures in figures.
%  minimal               Minimal set of packages.
%  default               Standard set of packages.
%  full                  Full set of packages.
%-------------------------------------------------------------------------------

% Style file with biblatex options for ATLAS documents.
\usepackage{atlasbiblatex}

% Useful macros
\usepackage{atlasphysics}
% See doc/atlas_physics.pdf for a list of the defined symbols.
% Default options are:
%   true:  journal, misc, particle, unit, xref
%   false: BSM, hepparticle, hepprocess, hion, jetetmiss, math, process,
%          other, snippets
% See the package for details on the options.

% Macro to add to-do notes (for several authors). Uses the todonotes package.
% \ATLnote{JS}{Jane}{green!20}{green!50!black!60}
% add macros \JSnote and \JSinote for notes in the margin and inline.
% The first colour is for the body and the second for the border of the note.
% Set output=false in order not to print out the notes.
% Set shift=false to avoid adjustment of margins.
% This package should not be used in the final version of the paper.
\ifthenelse{\boolean{AtlasDraft}}{\usepackage[output=true, shift=true]{atlastodo}}{}

% Files with references for use with biblatex.
% Note that biber gives an error if it finds empty bib files.
% \addbibresource{atlas-document.bib}
\addbibresource{bib/ATLAS.bib}
\addbibresource{bib/CMS.bib}
\addbibresource{bib/ConfNotes.bib}
\addbibresource{bib/PubNotes.bib}

% Paths for figures - do not forget the / at the end of the directory name.
\graphicspath{{logos/}{figures/}}

% Add your own definitions here (file atlas-document-defs.sty).
\usepackage{atlas-document-defs}

%-------------------------------------------------------------------------------
% Generic document information.
%-------------------------------------------------------------------------------

% Title, abstract and document.
%-------------------------------------------------------------------------------
% This file contains the title, author and abstract.
% It also contains all relevant document numbers used by the different cover pages.
%-------------------------------------------------------------------------------

%% Title
\AtlasTitle{A simple ATLAS paper draft}
%
% Author - this does not work with revtex
% \author{The ATLAS Collaboration}
%
% Date: if not given, uses current date
% \date{\today}

% Draft version:
% Should be 1.0 for the first circulation, and 2.0 for the second circulation.
% If given, adds draft version on front page, a 'DRAFT' box on top of each other page, 
% and line numbers.
% Comment or remove in final version.
\AtlasVersion{1.0}

% ATLAS reference code, to help ATLAS members to locate the paper
\AtlasRefCode{GROUP-2024-XX}

% Submission journal
\AtlasJournal{Phys.\ Lett.\ B.}

% Abstract - % directly after { is important for correct indentation
\AtlasAbstract{%
	This is a bare bones ATLAS document used to demonstrate how to use journal templates.

	This document is typeset using the \DocumentStyle.
}

%-------------------------------------------------------------------------------
% The following information is needed for the cover page. The commands are only defined
% if you use the coverpage option in atlasdoc or use the atlascover package
%-------------------------------------------------------------------------------

% Explicit list of supporting notes  (leave as null \AtlasCoverSupportingNote{} if you want to skip this option)
\AtlasCoverSupportingNote{A supporting note}{http://cds.cern.ch/record/XXXXXX}
\AtlasCoverSupportingNote{Another supporting note}{http://cds.cern.ch/record/YYYYYY}
%
% OR (the 2nd option is deprecated, especially for CONF and PUB notes)
%
% Supporting material TWiki page  (leave as null \AtlasCoverTwikiURL{} if you want to skip this option)
% \AtlasCoverTwikiURL{https://twiki.cern.ch/twiki/bin/view/Atlas/WebHome}

% Comment deadline
\AtlasCoverCommentsDeadline{DD Month 2024}

% Analysis team members - contact editors should no longer be specified
% as there is a generic email list name for the editors
\AtlasCoverAnalysisTeam{Peter Analyser, Susan Editor1, Jenny Editor2, Alphonse Physicien}

% Editorial Board Members - indicate the Chair by a (chair) after his/her name
\AtlasCoverEdBoardMember{EdBoard~Chair~(chair), EB~Member~1, EB~Member~2, EB~Member~3}

% Analysis team egroup
\AtlasCoverEgroupAnalysisTeam{atlas-GROUP-2024-XX-analysis-team@cern.ch}

% EdBoard and conveners egroup
\AtlasCoverEgroupEdBoard{atlas-GROUP-2024-XX-edboard-conveners@cern.ch}

% CERN preprintnumber
\PreprintIdNumber{CERN-EP-2024-XX}

% Author and title for the PDF file.
\hypersetup{pdftitle={ATLAS document},pdfauthor={The ATLAS Collaboration}}

%-------------------------------------------------------------------------------
% Content
%-------------------------------------------------------------------------------
\begin{document}

\maketitle

\tableofcontents

% List of to-do notes.
% \listoftodos

%-------------------------------------------------------------------------------
\section{Introduction}
\label{sec:intro}
%-------------------------------------------------------------------------------

Place your introduction here.

%-------------------------------------------------------------------------------
\section{ATLAS detector}
\label{sec:detector}
%-------------------------------------------------------------------------------

The ATLAS detector~\cite{PERF-2007-01} ...
% % Footnote with ATLAS coordinate system
\newcommand{\AtlasCoordFootnote}{%
ATLAS uses a right-handed coordinate system with its origin at the nominal interaction point (IP)
in the centre of the detector and the \(z\)-axis along the beam pipe.
The \(x\)-axis points from the IP to the centre of the LHC ring,
and the \(y\)-axis points upwards.
Polar coordinates \((r,\phi)\) are used in the transverse plane, 
\(\phi\) being the azimuthal angle around the \(z\)-axis.
The pseudorapidity is defined in terms of the polar angle \(\theta\) as \(\eta = -\ln \tan(\theta/2)\) and is equal to the rapidity
$ y = \frac{1}{2} \ln \left( \frac{E + p_z c}{E - p_z c} \right) $ in the relativistic limit.
Angular distance is measured in units of \(\Delta R \equiv \sqrt{(\Delta y)^{2} + (\Delta\phi)^{2}}\).}


%-------------------------------------------------------------------------------
\subsection{Run 3 ATLAS detector example for a Letter}
\label{sec:atlas3a}
%-------------------------------------------------------------------------------

The ATLAS experiment~\cite{PERF-2007-01,GENR-2019-02} at the LHC is a multipurpose particle detector
with a forward--backward symmetric cylindrical geometry and a near \(4\pi\) coverage in 
solid angle.\footnote{\AtlasCoordFootnote}
It consists of an inner tracking detector surrounded by a thin superconducting solenoid
providing a \qty{2}{\tesla} axial magnetic field, electromagnetic and hadronic calorimeters, and a muon spectrometer.
The inner tracking detector covers the pseudorapidity range \(|\eta| < 2.5\).
It consists of silicon pixel, silicon microstrip, and transition radiation tracking detectors.
Lead/liquid-argon (LAr) sampling calorimeters provide electromagnetic (EM) energy measurements
with high granularity within the region \(|\eta|< 3.2\).
A steel/scintillator-tile hadronic calorimeter covers the central pseudorapidity range (\(|\eta| < 1.7\)).
The endcap and forward regions are instrumented with LAr calorimeters
for EM and hadronic energy measurements up to \(|\eta| = 4.9\).
The muon spectrometer surrounds the calorimeters and is based on
three large superconducting air-core toroidal magnets with eight coils each.
The field integral of the toroids ranges between \num{2.0} and \qty{6.0}{\tesla\metre}
across most of the detector. 
The muon spectrometer includes a system of precision tracking chambers up to \(|\eta| = 2.7\) and fast detectors for triggering up to \(|\eta| = 2.4\).
The luminosity is measured mainly by the LUCID--2 detector which is located close to the beampipe.
A two-level trigger system is used to select events~\cite{TRIG-2016-01}. 
The first-level trigger is implemented in hardware and uses a subset of the detector information
to accept events at a rate below \qty{100}{\kHz}.
This is followed by a software-based trigger that
reduces the accepted event rate to \qty{3}{\kHz} on average,
depending on the data-taking conditions.
A software suite~\cite{SOFT-2022-02} is used in data simulation, in the reconstruction
and analysis of real and simulated data, in detector operations, and in the trigger and data acquisition
systems of the experiment.

%-------------------------------------------------------------------------------
\subsection{Run 3 ATLAS detector example for a paper making use of the whole detector}
\label{sec:atlas3b}
%-------------------------------------------------------------------------------

The ATLAS detector~\cite{PERF-2007-01,GENR-2019-02} at the LHC covers nearly the entire solid angle around the collision point.\footnote{\AtlasCoordFootnote}
It consists of an inner tracking detector surrounded by a thin superconducting solenoid, electromagnetic and hadronic calorimeters,
and a muon spectrometer incorporating three large superconducting air-core toroidal magnets.

The inner-detector system (ID) is immersed in a \qty{2}{\tesla} axial magnetic field 
and provides charged-particle tracking in the range \(|\eta| < 2.5\).
The high-granularity silicon pixel detector covers the interaction region and typically provides four measurements per track, 
the first hit generally being in the insertable B-layer (IBL). % installed before Run~2~\cite{ATLAS-TDR-19,PIX-2018-001}.
It is followed by the SemiConductor Tracker (SCT), which usually provides eight measurements per track.
These silicon detectors are complemented by the transition radiation tracker (TRT),
which enables radially extended track reconstruction up to \(|\eta| = 2.0\). 
The TRT also provides electron identification information 
based on the fraction of hits (typically 30 in total) above a higher energy-deposit threshold corresponding to transition radiation.

The calorimeter system covers the pseudorapidity range \(|\eta| < 4.9\).
Within the region \(|\eta|< 3.2\), electromagnetic calorimetry is provided by barrel and 
endcap high-granularity lead/liquid-argon (LAr) calorimeters,
with an additional thin LAr presampler covering \(|\eta| < 1.8\)
to correct for energy loss in material upstream of the calorimeters.
Hadronic calorimetry is provided by the steel/scintillator-tile calorimeter,
segmented into three barrel structures within \(|\eta| < 1.7\), and two copper/LAr hadronic endcap calorimeters.
The solid angle coverage is completed with forward copper/LAr and tungsten/LAr calorimeter modules
optimised for electromagnetic and hadronic energy measurements, respectively.

The muon spectrometer (MS) comprises separate trigger and
high-precision tracking chambers measuring the deflection of muons in a magnetic field generated by the superconducting air-core toroidal magnets.
The field integral of the toroids ranges between \num{2.0} and \qty{6.0}{\tesla\metre}
across most of the detector. 
Three layers of precision chambers, each consisting of layers of monitored drift tubes, cover the region \(|\eta| < 2.7\),
except in the innermost layer of the end-cap region, where layers of small-strip thin-gap chambers and Micromegas chambers both provide precision tracking in the region \(1.3 < |\eta| < 2.7\).
The muon trigger system covers the range \(|\eta| < 2.4\) with resistive-plate chambers in the barrel, thin-gap chambers in the endcap regions, and the aforementioned small-strip thin-gap chambers and Micromegas chambers in the innermost layer of the endcap.

The luminosity is measured mainly by the LUCID--2 detector that records Cherenkov light produced in the quartz windows of photomultipliers located close to the beampipe.

Events are selected by the first-level trigger system implemented in custom hardware,
followed by selections made by algorithms implemented in software in the high-level trigger~\cite{TRIG-2016-01}. 
The first-level trigger accepts events from the \qty{40}{\MHz} bunch crossings at a rate below \qty{100}{\kHz},
which the high-level trigger further reduces in order to record complete events to disk at an average rate of about \qty{3}{\kHz}.

The Run-3 detector configuration benefits from several upgrades compared with that of Run~2 to maintain high detector performance at the higher pileup levels of Run~3. The improvements include a new innermost layer of the muon spectrometer in the end-cap region, which provides higher redundancy and a large reduction in fake muon triggers. The trigger system also benefits from new LAr digital electronics with significantly increased granularity. Other updates and further details are provided in Ref.~\cite{GENR-2019-02}.

A software suite~\cite{SOFT-2022-02} is used in data simulation, in the reconstruction
and analysis of real and simulated data, in detector operations, and in the trigger and data acquisition
systems of the experiment.

%-------------------------------------------------------------------------------
\subsection{Run 2 ATLAS detector example for a Letter}
\label{sec:atlas2a}
%-------------------------------------------------------------------------------

The ATLAS experiment~\cite{PERF-2007-01} at the LHC is a multipurpose particle detector
with a forward--backward symmetric cylindrical geometry and a near \(4\pi\) coverage in 
solid angle.\footnote{\AtlasCoordFootnote}
It consists of an inner tracking detector surrounded by a thin superconducting solenoid
providing a \qty{2}{\tesla} axial magnetic field, electromagnetic and hadronic calorimeters, and a muon spectrometer.
The inner tracking detector covers the pseudorapidity range \(|\eta| < 2.5\).
It consists of silicon pixel, silicon microstrip, and transition radiation tracking detectors.
Lead/liquid-argon (LAr) sampling calorimeters provide electromagnetic (EM) energy measurements
with high granularity within the region \(|\eta|< 3.2\).
A steel/scintillator-tile hadronic calorimeter covers the central pseudorapidity range (\(|\eta| < 1.7\)).
The endcap and forward regions are instrumented with LAr calorimeters
for EM and hadronic energy measurements up to \(|\eta| = 4.9\).
The muon spectrometer surrounds the calorimeters and is based on
three large superconducting air-core toroidal magnets with eight coils each.
The field integral of the toroids ranges between \num{2.0} and \qty{6.0}{\tesla\metre}
across most of the detector. 
The muon spectrometer includes a system of precision tracking chambers up to \(|\eta| = 2.7\) and fast detectors for triggering up to \(|\eta| = 2.4\).
The luminosity is measured mainly by the LUCID--2~\cite{LUCID2} detector, which is located close to the beampipe.
A two-level trigger system is used to select events~\cite{TRIG-2016-01}. 
The first-level trigger is implemented in hardware and uses a subset of the detector information
to accept events at a rate below \qty{100}{\kHz}.
This is followed by a software-based trigger that
reduces the accepted event rate to \qty{1}{\kHz} on average
depending on the data-taking conditions.
A software suite~\cite{SOFT-2022-02} is used in data simulation, in the reconstruction
and analysis of real and simulated data, in detector operations, and in the trigger and data acquisition
systems of the experiment.


%-------------------------------------------------------------------------------
\subsection{Run 2 ATLAS detector example for a paper making use of the whole detector}
\label{sec:atlas2b}
%-------------------------------------------------------------------------------

The ATLAS detector~\cite{PERF-2007-01} at the LHC covers nearly the entire solid angle around the collision point.\footnote{\AtlasCoordFootnote}
It consists of an inner tracking detector surrounded by a thin superconducting solenoid, electromagnetic and hadronic calorimeters,
and a muon spectrometer incorporating three large superconducting air-core toroidal magnets.

The inner-detector system (ID) is immersed in a \qty{2}{\tesla} axial magnetic field 
and provides charged-particle tracking in the range \(|\eta| < 2.5\).
The high-granularity silicon pixel detector covers the vertex region and typically provides four measurements per track, 
the first hit generally being in the insertable B-layer (IBL) installed before Run~2~\cite{ATLAS-TDR-19,PIX-2018-001}.
It is followed by the SemiConductor Tracker (SCT), which usually provides eight measurements per track.
These silicon detectors are complemented by the transition radiation tracker (TRT),
which enables radially extended track reconstruction up to \(|\eta| = 2.0\). 
The TRT also provides electron identification information 
based on the fraction of hits (typically 30 in total) above a higher energy-deposit threshold corresponding to transition radiation.

The calorimeter system covers the pseudorapidity range \(|\eta| < 4.9\).
Within the region \(|\eta|< 3.2\), electromagnetic calorimetry is provided by barrel and 
endcap high-granularity lead/liquid-argon (LAr) calorimeters,
with an additional thin LAr presampler covering \(|\eta| < 1.8\)
to correct for energy loss in material upstream of the calorimeters.
Hadronic calorimetry is provided by the steel/scintillator-tile calorimeter,
segmented into three barrel structures within \(|\eta| < 1.7\), and two copper/LAr hadronic endcap calorimeters.
The solid angle coverage is completed with forward copper/LAr and tungsten/LAr calorimeter modules
optimised for electromagnetic and hadronic energy measurements respectively.

The muon spectrometer (MS) comprises separate trigger and
high-precision tracking chambers measuring the deflection of muons in a magnetic field generated by the superconducting air-core toroidal magnets.
The field integral of the toroids ranges between \num{2.0} and \qty{6.0}{\tesla\metre}
across most of the detector. 
Three layers of precision chambers, each consisting of layers of monitored drift tubes, cover the region \(|\eta| < 2.7\),
complemented by cathode-strip chambers in the forward region, where the background is highest.
The muon trigger system covers the range \(|\eta| < 2.4\) with resistive-plate chambers in the barrel, and thin-gap chambers in the endcap regions.

The luminosity is measured mainly by the LUCID--2~\cite{LUCID2} detector that records Cherenkov light produced in the quartz windows of photomultipliers located close to the beampipe.

Events are selected by the first-level trigger system implemented in custom hardware,
followed by selections made by algorithms implemented in software in the high-level trigger~\cite{TRIG-2016-01}. 
The first-level trigger accepts events from the \qty{40}{\MHz} bunch crossings at a rate below \qty{100}{\kHz},
which the high-level trigger further reduces in order to record complete events to disk at about \qty{1}{\kHz}.

A software suite~\cite{SOFT-2022-02} is used in data simulation, in the reconstruction
and analysis of real and simulated data, in detector operations, and in the trigger and data acquisition
systems of the experiment.



%-------------------------------------------------------------------------------
\section{Analysis}
\label{sec:analysis}
%-------------------------------------------------------------------------------

You can find some text snippets that can be used in papers in \texttt{latex/atlassnippets.sty}.
To use them, provide the \texttt{snippets} option to \texttt{atlasphysics}.

%-------------------------------------------------------------------------------
\section{Results}
\label{sec:result}
%-------------------------------------------------------------------------------

Place your results here.

% All figures and tables should appear before the summary and conclusion.
% The package placeins provides the macro \FloatBarrier to achieve this.
% \FloatBarrier

%-------------------------------------------------------------------------------
\section{Conclusion}
\label{sec:conclusion}
%-------------------------------------------------------------------------------

Place your conclusion here.

%-------------------------------------------------------------------------------
% When your ATLAS paper or PUB/CONF note is ready to be published,
% the acknowledgements should be automatically included/updated by PO-GitLab.
% The acknowledgements can be inspected on the web page:
% \url{https://atlas-glance.cern.ch/atlas/membership/funding_agencies/history}.
\section*{Acknowledgements}
%-------------------------------------------------------------------------------

% Acknowledgements - comment in this line when you are ready to include them.
% % Acknowledgements for papers with collision data
% Version 2024-07-11
 
% Standard acknowledgements start here
%----------------------------------------------
 
We thank CERN for the very successful operation of the LHC and its injectors, as well as the support staff at
CERN and at our institutions worldwide without whom ATLAS could not be operated efficiently.
 
The crucial computing support from all WLCG partners is acknowledged gratefully, in particular from CERN, the ATLAS Tier-1 facilities at TRIUMF/SFU (Canada), NDGF (Denmark, Norway, Sweden), CC-IN2P3 (France), KIT/GridKA (Germany), INFN-CNAF (Italy), NL-T1 (Netherlands), PIC (Spain), RAL (UK) and BNL (USA), the Tier-2 facilities worldwide and large non-WLCG resource providers. Major contributors of computing resources are listed in Ref.~\cite{ATL-SOFT-PUB-2023-001}.
 
We gratefully acknowledge the support of ANPCyT, Argentina; YerPhI, Armenia; ARC, Australia; BMWFW and FWF, Austria; ANAS, Azerbaijan; CNPq and FAPESP, Brazil; NSERC, NRC and CFI, Canada; CERN; ANID, Chile; CAS, MOST and NSFC, China; Minciencias, Colombia; MEYS CR, Czech Republic; DNRF and DNSRC, Denmark; IN2P3-CNRS and CEA-DRF/IRFU, France; SRNSFG, Georgia; BMBF, HGF and MPG, Germany; GSRI, Greece; RGC and Hong Kong SAR, China; ISF and Benoziyo Center, Israel; INFN, Italy; MEXT and JSPS, Japan; CNRST, Morocco; NWO, Netherlands; RCN, Norway; MNiSW, Poland; FCT, Portugal; MNE/IFA, Romania; MESTD, Serbia; MSSR, Slovakia; ARRS and MIZ\v{S}, Slovenia; DSI/NRF, South Africa; MICINN, Spain; SRC and Wallenberg Foundation, Sweden; SERI, SNSF and Cantons of Bern and Geneva, Switzerland; MOST, Taipei; TENMAK, T\"urkiye; STFC, United Kingdom; DOE and NSF, United States of America.
 
Individual groups and members have received support from BCKDF, CANARIE, CRC and DRAC, Canada; CERN-CZ, FORTE and PRIMUS, Czech Republic; COST, ERC, ERDF, Horizon 2020, ICSC-NextGenerationEU and Marie Sk{\l}odowska-Curie Actions, European Union; Investissements d'Avenir Labex, Investissements d'Avenir Idex and ANR, France; DFG and AvH Foundation, Germany; Herakleitos, Thales and Aristeia programmes co-financed by EU-ESF and the Greek NSRF, Greece; BSF-NSF and MINERVA, Israel; NCN and NAWA, Poland; La Caixa Banking Foundation, CERCA Programme Generalitat de Catalunya and PROMETEO and GenT Programmes Generalitat Valenciana, Spain; G\"{o}ran Gustafssons Stiftelse, Sweden; The Royal Society and Leverhulme Trust, United Kingdom.
 
In addition, individual members wish to acknowledge support from Armenia: Yerevan Physics Institute (FAPERJ); CERN: European Organization for Nuclear Research (CERN PJAS); Chile: Agencia Nacional de Investigaci\'on y Desarrollo (FONDECYT 1230812, FONDECYT 1230987, FONDECYT 1240864); China: Chinese Ministry of Science and Technology (MOST-2023YFA1605700), National Natural Science Foundation of China (NSFC - 12175119, NSFC 12275265, NSFC-12075060); Czech Republic: Czech Science Foundation (GACR - 24-11373S), Ministry of Education Youth and Sports (FORTE CZ.02.01.01/00/22\_008/0004632), PRIMUS Research Programme (PRIMUS/21/SCI/017); EU: H2020 European Research Council (ERC - 101002463); European Union: European Research Council (ERC - 948254, ERC 101089007), Horizon 2020 Framework Programme (MUCCA - CHIST-ERA-19-XAI-00), European Union, Future Artificial Intelligence Research (FAIR-NextGenerationEU PE00000013), Italian Center for High Performance Computing, Big Data and Quantum Computing (ICSC, NextGenerationEU); France: Agence Nationale de la Recherche (ANR-20-CE31-0013, ANR-21-CE31-0013, ANR-21-CE31-0022), Investissements d'Avenir Labex (ANR-11-LABX-0012); Germany: Baden-Württemberg Stiftung (BW Stiftung-Postdoc Eliteprogramme), Deutsche Forschungsgemeinschaft (DFG - 469666862, DFG - CR 312/5-2); Italy: Istituto Nazionale di Fisica Nucleare (ICSC, NextGenerationEU); Japan: Japan Society for the Promotion of Science (JSPS KAKENHI JP22H01227, JSPS KAKENHI JP22H04944, JSPS KAKENHI JP22KK0227, JSPS KAKENHI JP23KK0245); Netherlands: Netherlands Organisation for Scientific Research (NWO Veni 2020 - VI.Veni.202.179); Norway: Research Council of Norway (RCN-314472); Poland: Ministry of Science and Higher Education (IDUB AGH, POB8, D4 no 9722), Polish National Agency for Academic Exchange (PPN/PPO/2020/1/00002/U/00001), Polish National Science Centre (NCN 2021/42/E/ST2/00350, NCN OPUS nr 2022/47/B/ST2/03059, NCN UMO-2019/34/E/ST2/00393, UMO-2020/37/B/ST2/01043, UMO-2021/40/C/ST2/00187, UMO-2022/47/O/ST2/00148, UMO-2023/49/B/ST2/04085, UMO-2023/51/B/ST2/00920); Slovenia: Slovenian Research Agency (ARIS grant J1-3010); Spain: Generalitat Valenciana (Artemisa, FEDER, IDIFEDER/2018/048), Ministry of Science and Innovation (MCIN \& NextGenEU PCI2022-135018-2, MICIN \& FEDER PID2021-125273NB, RYC2019-028510-I, RYC2020-030254-I, RYC2021-031273-I, RYC2022-038164-I); Sweden: Swedish Research Council (Swedish Research Council 2023-04654, VR 2018-00482, VR 2022-03845, VR 2022-04683, VR 2023-03403, VR grant 2021-03651), Knut and Alice Wallenberg Foundation (KAW 2018.0458, KAW 2019.0447, KAW 2022.0358); Switzerland: Swiss National Science Foundation (SNSF - PCEFP2\_194658); United Kingdom: Leverhulme Trust (Leverhulme Trust RPG-2020-004), Royal Society (NIF-R1-231091); United States of America: U.S. Department of Energy (ECA DE-AC02-76SF00515), Neubauer Family Foundation.
 
%----------------------------------------------
% Created with Glance


%-------------------------------------------------------------------------------
\clearpage
\appendix
\part*{Appendix}
\addcontentsline{toc}{part}{Appendix}
%-------------------------------------------------------------------------------

In a paper, an appendix is used for technical details that would otherwise disturb the flow of the paper.
Such an appendix should be printed before the Bibliography.

%-------------------------------------------------------------------------------
% If you use biblatex and either biber or bibtex to process the bibliography
% just say \printbibliography here.
\printbibliography
% If you want to use the traditional BibTeX you need to use the syntax below.
% \bibliographystyle{obsolete/bst/atlasBibStyleWoTitle}
% \bibliography{atlas-document,bib/ATLAS,bib/CMS,bib/ConfNotes,bib/PubNotes}
%-------------------------------------------------------------------------------

%-------------------------------------------------------------------------------
% Author list - comment in these two lines when you are ready to include them.
% \clearpage
% % ATLAS Collaboration author list
% Reference date of TRIG-2021-01 is 2024-07-11
% Data extracted on 11-Jul-2024 for paper reference TRIG-2021-01
% at 7:19pm
 
\begin{flushleft}
\hypersetup{urlcolor=black}
{\Large The ATLAS Collaboration}

\bigskip

\AtlasOrcid[0000-0002-6665-4934]{G.~Aad}$^\textrm{\scriptsize 105}$,
\AtlasOrcid[0000-0001-7616-1554]{E.~Aakvaag}$^\textrm{\scriptsize 17}$,
\AtlasOrcid[0000-0002-5888-2734]{B.~Abbott}$^\textrm{\scriptsize 124}$,
\AtlasOrcid[0000-0002-0287-5869]{S.~Abdelhameed}$^\textrm{\scriptsize 120a}$,
\AtlasOrcid[0000-0002-1002-1652]{K.~Abeling}$^\textrm{\scriptsize 57}$,
\AtlasOrcid[0000-0001-5763-2760]{N.J.~Abicht}$^\textrm{\scriptsize 51}$,
\AtlasOrcid[0000-0002-8496-9294]{S.H.~Abidi}$^\textrm{\scriptsize 30}$,
\AtlasOrcid[0009-0003-6578-220X]{M.~Aboelela}$^\textrm{\scriptsize 46}$,
\AtlasOrcid[0000-0002-9987-2292]{A.~Aboulhorma}$^\textrm{\scriptsize 36e}$,
\AtlasOrcid[0000-0001-5329-6640]{H.~Abramowicz}$^\textrm{\scriptsize 155}$,
\AtlasOrcid[0000-0002-1599-2896]{H.~Abreu}$^\textrm{\scriptsize 154}$,
\AtlasOrcid[0000-0003-0403-3697]{Y.~Abulaiti}$^\textrm{\scriptsize 121}$,
\AtlasOrcid[0000-0002-8588-9157]{B.S.~Acharya}$^\textrm{\scriptsize 71a,71b,k}$,
\AtlasOrcid[0000-0003-4699-7275]{A.~Ackermann}$^\textrm{\scriptsize 65a}$,
\AtlasOrcid[0000-0002-2634-4958]{C.~Adam~Bourdarios}$^\textrm{\scriptsize 4}$,
\AtlasOrcid[0000-0002-5859-2075]{L.~Adamczyk}$^\textrm{\scriptsize 88a}$,
\AtlasOrcid[0000-0002-2919-6663]{S.V.~Addepalli}$^\textrm{\scriptsize 27}$,
\AtlasOrcid[0000-0002-8387-3661]{M.J.~Addison}$^\textrm{\scriptsize 104}$,
\AtlasOrcid[0000-0002-1041-3496]{J.~Adelman}$^\textrm{\scriptsize 119}$,
\AtlasOrcid[0000-0001-6644-0517]{A.~Adiguzel}$^\textrm{\scriptsize 22c}$,
\AtlasOrcid[0000-0003-0627-5059]{T.~Adye}$^\textrm{\scriptsize 138}$,
\AtlasOrcid[0000-0002-9058-7217]{A.A.~Affolder}$^\textrm{\scriptsize 140}$,
\AtlasOrcid[0000-0001-8102-356X]{Y.~Afik}$^\textrm{\scriptsize 41}$,
\AtlasOrcid[0000-0002-4355-5589]{M.N.~Agaras}$^\textrm{\scriptsize 13}$,
\AtlasOrcid[0000-0002-1922-2039]{A.~Aggarwal}$^\textrm{\scriptsize 103}$,
\AtlasOrcid[0000-0003-3695-1847]{C.~Agheorghiesei}$^\textrm{\scriptsize 28c}$,
\AtlasOrcid[0000-0003-3644-540X]{F.~Ahmadov}$^\textrm{\scriptsize 40,x}$,
\AtlasOrcid[0000-0003-4368-9285]{S.~Ahuja}$^\textrm{\scriptsize 98}$,
\AtlasOrcid[0000-0003-3856-2415]{X.~Ai}$^\textrm{\scriptsize 64e}$,
\AtlasOrcid[0000-0002-0573-8114]{G.~Aielli}$^\textrm{\scriptsize 78a,78b}$,
\AtlasOrcid[0000-0001-6578-6890]{A.~Aikot}$^\textrm{\scriptsize 168}$,
\AtlasOrcid[0000-0002-1322-4666]{M.~Ait~Tamlihat}$^\textrm{\scriptsize 36e}$,
\AtlasOrcid[0000-0002-8020-1181]{B.~Aitbenchikh}$^\textrm{\scriptsize 36a}$,
\AtlasOrcid[0000-0002-7342-3130]{M.~Akbiyik}$^\textrm{\scriptsize 103}$,
\AtlasOrcid[0000-0003-4141-5408]{T.P.A.~{\AA}kesson}$^\textrm{\scriptsize 101}$,
\AtlasOrcid[0000-0002-2846-2958]{A.V.~Akimov}$^\textrm{\scriptsize 38}$,
\AtlasOrcid[0000-0001-7623-6421]{D.~Akiyama}$^\textrm{\scriptsize 173}$,
\AtlasOrcid[0000-0003-3424-2123]{N.N.~Akolkar}$^\textrm{\scriptsize 25}$,
\AtlasOrcid[0000-0002-8250-6501]{S.~Aktas}$^\textrm{\scriptsize 22a}$,
\AtlasOrcid[0000-0002-0547-8199]{K.~Al~Khoury}$^\textrm{\scriptsize 43}$,
\AtlasOrcid[0000-0003-2388-987X]{G.L.~Alberghi}$^\textrm{\scriptsize 24b}$,
\AtlasOrcid[0000-0003-0253-2505]{J.~Albert}$^\textrm{\scriptsize 170}$,
\AtlasOrcid[0000-0001-6430-1038]{P.~Albicocco}$^\textrm{\scriptsize 55}$,
\AtlasOrcid[0000-0003-0830-0107]{G.L.~Albouy}$^\textrm{\scriptsize 62}$,
\AtlasOrcid[0000-0002-8224-7036]{S.~Alderweireldt}$^\textrm{\scriptsize 54}$,
\AtlasOrcid[0000-0002-1977-0799]{Z.L.~Alegria}$^\textrm{\scriptsize 125}$,
\AtlasOrcid[0000-0002-1936-9217]{M.~Aleksa}$^\textrm{\scriptsize 37}$,
\AtlasOrcid[0000-0001-7381-6762]{I.N.~Aleksandrov}$^\textrm{\scriptsize 40}$,
\AtlasOrcid[0000-0003-0922-7669]{C.~Alexa}$^\textrm{\scriptsize 28b}$,
\AtlasOrcid[0000-0002-8977-279X]{T.~Alexopoulos}$^\textrm{\scriptsize 10}$,
\AtlasOrcid[0000-0002-0966-0211]{F.~Alfonsi}$^\textrm{\scriptsize 24b}$,
\AtlasOrcid[0000-0003-1793-1787]{M.~Algren}$^\textrm{\scriptsize 58}$,
\AtlasOrcid[0000-0001-7569-7111]{M.~Alhroob}$^\textrm{\scriptsize 172}$,
\AtlasOrcid[0000-0001-8653-5556]{B.~Ali}$^\textrm{\scriptsize 136}$,
\AtlasOrcid[0000-0002-4507-7349]{H.M.J.~Ali}$^\textrm{\scriptsize 94,r}$,
\AtlasOrcid[0000-0001-5216-3133]{S.~Ali}$^\textrm{\scriptsize 32}$,
\AtlasOrcid[0000-0002-9377-8852]{S.W.~Alibocus}$^\textrm{\scriptsize 95}$,
\AtlasOrcid[0000-0002-9012-3746]{M.~Aliev}$^\textrm{\scriptsize 34c}$,
\AtlasOrcid[0000-0002-7128-9046]{G.~Alimonti}$^\textrm{\scriptsize 73a}$,
\AtlasOrcid[0000-0001-9355-4245]{W.~Alkakhi}$^\textrm{\scriptsize 57}$,
\AtlasOrcid[0000-0003-4745-538X]{C.~Allaire}$^\textrm{\scriptsize 68}$,
\AtlasOrcid[0000-0002-5738-2471]{B.M.M.~Allbrooke}$^\textrm{\scriptsize 150}$,
\AtlasOrcid[0000-0001-9990-7486]{J.F.~Allen}$^\textrm{\scriptsize 54}$,
\AtlasOrcid[0000-0002-1509-3217]{C.A.~Allendes~Flores}$^\textrm{\scriptsize 141f}$,
\AtlasOrcid[0000-0001-7303-2570]{P.P.~Allport}$^\textrm{\scriptsize 21}$,
\AtlasOrcid[0000-0002-3883-6693]{A.~Aloisio}$^\textrm{\scriptsize 74a,74b}$,
\AtlasOrcid[0000-0001-9431-8156]{F.~Alonso}$^\textrm{\scriptsize 93}$,
\AtlasOrcid[0000-0002-7641-5814]{C.~Alpigiani}$^\textrm{\scriptsize 142}$,
\AtlasOrcid[0000-0002-3785-0709]{Z.M.K.~Alsolami}$^\textrm{\scriptsize 94}$,
\AtlasOrcid[0000-0002-8181-6532]{M.~Alvarez~Estevez}$^\textrm{\scriptsize 102}$,
\AtlasOrcid[0000-0003-1525-4620]{A.~Alvarez~Fernandez}$^\textrm{\scriptsize 103}$,
\AtlasOrcid[0000-0002-0042-292X]{M.~Alves~Cardoso}$^\textrm{\scriptsize 58}$,
\AtlasOrcid[0000-0003-0026-982X]{M.G.~Alviggi}$^\textrm{\scriptsize 74a,74b}$,
\AtlasOrcid[0000-0003-3043-3715]{M.~Aly}$^\textrm{\scriptsize 104}$,
\AtlasOrcid[0000-0002-1798-7230]{Y.~Amaral~Coutinho}$^\textrm{\scriptsize 85b}$,
\AtlasOrcid[0000-0003-2184-3480]{A.~Ambler}$^\textrm{\scriptsize 107}$,
\AtlasOrcid{C.~Amelung}$^\textrm{\scriptsize 37}$,
\AtlasOrcid[0000-0003-1155-7982]{M.~Amerl}$^\textrm{\scriptsize 104}$,
\AtlasOrcid[0000-0002-2126-4246]{C.G.~Ames}$^\textrm{\scriptsize 112}$,
\AtlasOrcid[0000-0002-6814-0355]{D.~Amidei}$^\textrm{\scriptsize 109}$,
\AtlasOrcid[0000-0002-4692-0369]{B.~Amini}$^\textrm{\scriptsize 56}$,
\AtlasOrcid[0000-0002-8029-7347]{K.J.~Amirie}$^\textrm{\scriptsize 159}$,
\AtlasOrcid[0000-0001-7566-6067]{S.P.~Amor~Dos~Santos}$^\textrm{\scriptsize 134a}$,
\AtlasOrcid[0000-0003-1757-5620]{K.R.~Amos}$^\textrm{\scriptsize 168}$,
\AtlasOrcid[0000-0003-0205-6887]{D.~Amperiadou}$^\textrm{\scriptsize 156}$,
\AtlasOrcid{S.~An}$^\textrm{\scriptsize 86}$,
\AtlasOrcid[0000-0003-3649-7621]{V.~Ananiev}$^\textrm{\scriptsize 129}$,
\AtlasOrcid[0000-0003-1587-5830]{C.~Anastopoulos}$^\textrm{\scriptsize 143}$,
\AtlasOrcid[0000-0002-4413-871X]{T.~Andeen}$^\textrm{\scriptsize 11}$,
\AtlasOrcid[0000-0002-1846-0262]{J.K.~Anders}$^\textrm{\scriptsize 37}$,
\AtlasOrcid[0009-0009-9682-4656]{A.C.~Anderson}$^\textrm{\scriptsize 61}$,
\AtlasOrcid[0000-0002-9766-2670]{S.Y.~Andrean}$^\textrm{\scriptsize 49a,49b}$,
\AtlasOrcid[0000-0001-5161-5759]{A.~Andreazza}$^\textrm{\scriptsize 73a,73b}$,
\AtlasOrcid[0000-0002-8274-6118]{S.~Angelidakis}$^\textrm{\scriptsize 9}$,
\AtlasOrcid[0000-0001-7834-8750]{A.~Angerami}$^\textrm{\scriptsize 43}$,
\AtlasOrcid[0000-0002-7201-5936]{A.V.~Anisenkov}$^\textrm{\scriptsize 38}$,
\AtlasOrcid[0000-0002-4649-4398]{A.~Annovi}$^\textrm{\scriptsize 76a}$,
\AtlasOrcid[0000-0001-9683-0890]{C.~Antel}$^\textrm{\scriptsize 58}$,
\AtlasOrcid[0000-0002-6678-7665]{E.~Antipov}$^\textrm{\scriptsize 149}$,
\AtlasOrcid[0000-0002-2293-5726]{M.~Antonelli}$^\textrm{\scriptsize 55}$,
\AtlasOrcid[0000-0003-2734-130X]{F.~Anulli}$^\textrm{\scriptsize 77a}$,
\AtlasOrcid[0000-0001-7498-0097]{M.~Aoki}$^\textrm{\scriptsize 86}$,
\AtlasOrcid[0000-0002-6618-5170]{T.~Aoki}$^\textrm{\scriptsize 157}$,
\AtlasOrcid[0000-0003-4675-7810]{M.A.~Aparo}$^\textrm{\scriptsize 150}$,
\AtlasOrcid[0000-0003-3942-1702]{L.~Aperio~Bella}$^\textrm{\scriptsize 50}$,
\AtlasOrcid[0000-0003-1205-6784]{C.~Appelt}$^\textrm{\scriptsize 155}$,
\AtlasOrcid[0000-0002-9418-6656]{A.~Apyan}$^\textrm{\scriptsize 27}$,
\AtlasOrcid[0000-0002-8849-0360]{S.J.~Arbiol~Val}$^\textrm{\scriptsize 89}$,
\AtlasOrcid[0000-0001-8648-2896]{C.~Arcangeletti}$^\textrm{\scriptsize 55}$,
\AtlasOrcid[0000-0002-7255-0832]{A.T.H.~Arce}$^\textrm{\scriptsize 53}$,
\AtlasOrcid[0000-0003-0229-3858]{J-F.~Arguin}$^\textrm{\scriptsize 111}$,
\AtlasOrcid[0000-0001-7748-1429]{S.~Argyropoulos}$^\textrm{\scriptsize 156}$,
\AtlasOrcid[0000-0002-1577-5090]{J.-H.~Arling}$^\textrm{\scriptsize 50}$,
\AtlasOrcid[0000-0002-6096-0893]{O.~Arnaez}$^\textrm{\scriptsize 4}$,
\AtlasOrcid[0000-0003-3578-2228]{H.~Arnold}$^\textrm{\scriptsize 149}$,
\AtlasOrcid[0000-0002-3477-4499]{G.~Artoni}$^\textrm{\scriptsize 77a,77b}$,
\AtlasOrcid[0000-0003-1420-4955]{H.~Asada}$^\textrm{\scriptsize 114}$,
\AtlasOrcid[0000-0002-3670-6908]{K.~Asai}$^\textrm{\scriptsize 122}$,
\AtlasOrcid[0000-0001-5279-2298]{S.~Asai}$^\textrm{\scriptsize 157}$,
\AtlasOrcid[0000-0001-8381-2255]{N.A.~Asbah}$^\textrm{\scriptsize 37}$,
\AtlasOrcid[0000-0002-4340-4932]{R.A.~Ashby~Pickering}$^\textrm{\scriptsize 172}$,
\AtlasOrcid[0000-0002-4826-2662]{K.~Assamagan}$^\textrm{\scriptsize 30}$,
\AtlasOrcid[0000-0001-5095-605X]{R.~Astalos}$^\textrm{\scriptsize 29a}$,
\AtlasOrcid[0000-0001-9424-6607]{K.S.V.~Astrand}$^\textrm{\scriptsize 101}$,
\AtlasOrcid[0000-0002-3624-4475]{S.~Atashi}$^\textrm{\scriptsize 163}$,
\AtlasOrcid[0000-0002-1972-1006]{R.J.~Atkin}$^\textrm{\scriptsize 34a}$,
\AtlasOrcid{M.~Atkinson}$^\textrm{\scriptsize 167}$,
\AtlasOrcid{H.~Atmani}$^\textrm{\scriptsize 36f}$,
\AtlasOrcid[0000-0002-7639-9703]{P.A.~Atmasiddha}$^\textrm{\scriptsize 132}$,
\AtlasOrcid[0000-0001-8324-0576]{K.~Augsten}$^\textrm{\scriptsize 136}$,
\AtlasOrcid[0000-0002-3623-1228]{A.D.~Auriol}$^\textrm{\scriptsize 21}$,
\AtlasOrcid[0000-0001-6918-9065]{V.A.~Austrup}$^\textrm{\scriptsize 104}$,
\AtlasOrcid[0000-0003-2664-3437]{G.~Avolio}$^\textrm{\scriptsize 37}$,
\AtlasOrcid[0000-0003-3664-8186]{K.~Axiotis}$^\textrm{\scriptsize 58}$,
\AtlasOrcid[0000-0003-4241-022X]{G.~Azuelos}$^\textrm{\scriptsize 111,ab}$,
\AtlasOrcid[0000-0001-7657-6004]{D.~Babal}$^\textrm{\scriptsize 29b}$,
\AtlasOrcid[0000-0002-2256-4515]{H.~Bachacou}$^\textrm{\scriptsize 139}$,
\AtlasOrcid[0000-0002-9047-6517]{K.~Bachas}$^\textrm{\scriptsize 156,o}$,
\AtlasOrcid[0000-0001-8599-024X]{A.~Bachiu}$^\textrm{\scriptsize 35}$,
\AtlasOrcid[0000-0001-5199-9588]{A.~Badea}$^\textrm{\scriptsize 41}$,
\AtlasOrcid[0000-0002-2469-513X]{T.M.~Baer}$^\textrm{\scriptsize 109}$,
\AtlasOrcid[0000-0003-4578-2651]{P.~Bagnaia}$^\textrm{\scriptsize 77a,77b}$,
\AtlasOrcid[0000-0003-4173-0926]{M.~Bahmani}$^\textrm{\scriptsize 19}$,
\AtlasOrcid[0000-0001-8061-9978]{D.~Bahner}$^\textrm{\scriptsize 56}$,
\AtlasOrcid[0000-0001-8508-1169]{K.~Bai}$^\textrm{\scriptsize 127}$,
\AtlasOrcid[0000-0003-0770-2702]{J.T.~Baines}$^\textrm{\scriptsize 138}$,
\AtlasOrcid[0000-0002-9326-1415]{L.~Baines}$^\textrm{\scriptsize 97}$,
\AtlasOrcid[0000-0003-1346-5774]{O.K.~Baker}$^\textrm{\scriptsize 177}$,
\AtlasOrcid[0000-0002-1110-4433]{E.~Bakos}$^\textrm{\scriptsize 16}$,
\AtlasOrcid[0000-0002-6580-008X]{D.~Bakshi~Gupta}$^\textrm{\scriptsize 8}$,
\AtlasOrcid[0009-0006-1619-1261]{L.E.~Balabram~Filho}$^\textrm{\scriptsize 85b}$,
\AtlasOrcid[0000-0003-2580-2520]{V.~Balakrishnan}$^\textrm{\scriptsize 124}$,
\AtlasOrcid[0000-0001-5840-1788]{R.~Balasubramanian}$^\textrm{\scriptsize 4}$,
\AtlasOrcid[0000-0002-9854-975X]{E.M.~Baldin}$^\textrm{\scriptsize 38}$,
\AtlasOrcid[0000-0002-0942-1966]{P.~Balek}$^\textrm{\scriptsize 88a}$,
\AtlasOrcid[0000-0001-9700-2587]{E.~Ballabene}$^\textrm{\scriptsize 24b,24a}$,
\AtlasOrcid[0000-0003-0844-4207]{F.~Balli}$^\textrm{\scriptsize 139}$,
\AtlasOrcid[0000-0001-7041-7096]{L.M.~Baltes}$^\textrm{\scriptsize 65a}$,
\AtlasOrcid[0000-0002-7048-4915]{W.K.~Balunas}$^\textrm{\scriptsize 33}$,
\AtlasOrcid[0000-0003-2866-9446]{J.~Balz}$^\textrm{\scriptsize 103}$,
\AtlasOrcid[0000-0002-4382-1541]{I.~Bamwidhi}$^\textrm{\scriptsize 120b}$,
\AtlasOrcid[0000-0001-5325-6040]{E.~Banas}$^\textrm{\scriptsize 89}$,
\AtlasOrcid[0000-0003-2014-9489]{M.~Bandieramonte}$^\textrm{\scriptsize 133}$,
\AtlasOrcid[0000-0002-5256-839X]{A.~Bandyopadhyay}$^\textrm{\scriptsize 25}$,
\AtlasOrcid[0000-0002-8754-1074]{S.~Bansal}$^\textrm{\scriptsize 25}$,
\AtlasOrcid[0000-0002-3436-2726]{L.~Barak}$^\textrm{\scriptsize 155}$,
\AtlasOrcid[0000-0001-5740-1866]{M.~Barakat}$^\textrm{\scriptsize 50}$,
\AtlasOrcid[0000-0002-3111-0910]{E.L.~Barberio}$^\textrm{\scriptsize 108}$,
\AtlasOrcid[0000-0002-3938-4553]{D.~Barberis}$^\textrm{\scriptsize 59b,59a}$,
\AtlasOrcid[0000-0002-7824-3358]{M.~Barbero}$^\textrm{\scriptsize 105}$,
\AtlasOrcid[0000-0002-5572-2372]{M.Z.~Barel}$^\textrm{\scriptsize 118}$,
\AtlasOrcid[0000-0001-7326-0565]{T.~Barillari}$^\textrm{\scriptsize 113}$,
\AtlasOrcid[0000-0003-0253-106X]{M-S.~Barisits}$^\textrm{\scriptsize 37}$,
\AtlasOrcid[0000-0002-7709-037X]{T.~Barklow}$^\textrm{\scriptsize 147}$,
\AtlasOrcid[0000-0002-5170-0053]{P.~Baron}$^\textrm{\scriptsize 126}$,
\AtlasOrcid[0000-0001-9864-7985]{D.A.~Baron~Moreno}$^\textrm{\scriptsize 104}$,
\AtlasOrcid[0000-0001-7090-7474]{A.~Baroncelli}$^\textrm{\scriptsize 64a}$,
\AtlasOrcid[0000-0002-3533-3740]{A.J.~Barr}$^\textrm{\scriptsize 130}$,
\AtlasOrcid[0000-0002-9752-9204]{J.D.~Barr}$^\textrm{\scriptsize 99}$,
\AtlasOrcid[0000-0002-3021-0258]{F.~Barreiro}$^\textrm{\scriptsize 102}$,
\AtlasOrcid[0000-0003-2387-0386]{J.~Barreiro~Guimar\~{a}es~da~Costa}$^\textrm{\scriptsize 14}$,
\AtlasOrcid[0000-0003-0914-8178]{M.G.~Barros~Teixeira}$^\textrm{\scriptsize 134a}$,
\AtlasOrcid[0000-0003-2872-7116]{S.~Barsov}$^\textrm{\scriptsize 38}$,
\AtlasOrcid[0000-0002-3407-0918]{F.~Bartels}$^\textrm{\scriptsize 65a}$,
\AtlasOrcid[0000-0001-5317-9794]{R.~Bartoldus}$^\textrm{\scriptsize 147}$,
\AtlasOrcid[0000-0001-9696-9497]{A.E.~Barton}$^\textrm{\scriptsize 94}$,
\AtlasOrcid[0000-0003-1419-3213]{P.~Bartos}$^\textrm{\scriptsize 29a}$,
\AtlasOrcid[0000-0001-8021-8525]{A.~Basan}$^\textrm{\scriptsize 103}$,
\AtlasOrcid[0000-0002-1533-0876]{M.~Baselga}$^\textrm{\scriptsize 51}$,
\AtlasOrcid[0000-0002-0129-1423]{A.~Bassalat}$^\textrm{\scriptsize 68,b}$,
\AtlasOrcid[0000-0001-9278-3863]{M.J.~Basso}$^\textrm{\scriptsize 160a}$,
\AtlasOrcid[0009-0004-5048-9104]{S.~Bataju}$^\textrm{\scriptsize 46}$,
\AtlasOrcid[0009-0004-7639-1869]{R.~Bate}$^\textrm{\scriptsize 169}$,
\AtlasOrcid[0000-0002-6923-5372]{R.L.~Bates}$^\textrm{\scriptsize 61}$,
\AtlasOrcid{S.~Batlamous}$^\textrm{\scriptsize 102}$,
\AtlasOrcid[0000-0001-6544-9376]{B.~Batool}$^\textrm{\scriptsize 145}$,
\AtlasOrcid[0000-0001-9608-543X]{M.~Battaglia}$^\textrm{\scriptsize 140}$,
\AtlasOrcid[0000-0001-6389-5364]{D.~Battulga}$^\textrm{\scriptsize 19}$,
\AtlasOrcid[0000-0002-9148-4658]{M.~Bauce}$^\textrm{\scriptsize 77a,77b}$,
\AtlasOrcid[0000-0002-4819-0419]{M.~Bauer}$^\textrm{\scriptsize 81}$,
\AtlasOrcid[0000-0002-4568-5360]{P.~Bauer}$^\textrm{\scriptsize 25}$,
\AtlasOrcid[0000-0002-8985-6934]{L.T.~Bazzano~Hurrell}$^\textrm{\scriptsize 31}$,
\AtlasOrcid[0000-0003-3623-3335]{J.B.~Beacham}$^\textrm{\scriptsize 53}$,
\AtlasOrcid[0000-0002-2022-2140]{T.~Beau}$^\textrm{\scriptsize 131}$,
\AtlasOrcid[0000-0002-0660-1558]{J.Y.~Beaucamp}$^\textrm{\scriptsize 93}$,
\AtlasOrcid[0000-0003-4889-8748]{P.H.~Beauchemin}$^\textrm{\scriptsize 162}$,
\AtlasOrcid[0000-0003-3479-2221]{P.~Bechtle}$^\textrm{\scriptsize 25}$,
\AtlasOrcid[0000-0001-7212-1096]{H.P.~Beck}$^\textrm{\scriptsize 20,n}$,
\AtlasOrcid[0000-0002-6691-6498]{K.~Becker}$^\textrm{\scriptsize 172}$,
\AtlasOrcid[0000-0002-8451-9672]{A.J.~Beddall}$^\textrm{\scriptsize 84}$,
\AtlasOrcid[0000-0003-4864-8909]{V.A.~Bednyakov}$^\textrm{\scriptsize 40}$,
\AtlasOrcid[0000-0001-6294-6561]{C.P.~Bee}$^\textrm{\scriptsize 149}$,
\AtlasOrcid[0009-0000-5402-0697]{L.J.~Beemster}$^\textrm{\scriptsize 16}$,
\AtlasOrcid[0000-0001-9805-2893]{T.A.~Beermann}$^\textrm{\scriptsize 37}$,
\AtlasOrcid[0000-0003-4868-6059]{M.~Begalli}$^\textrm{\scriptsize 85d}$,
\AtlasOrcid[0000-0002-1634-4399]{M.~Begel}$^\textrm{\scriptsize 30}$,
\AtlasOrcid[0000-0002-7739-295X]{A.~Behera}$^\textrm{\scriptsize 149}$,
\AtlasOrcid[0000-0002-5501-4640]{J.K.~Behr}$^\textrm{\scriptsize 50}$,
\AtlasOrcid[0000-0001-9024-4989]{J.F.~Beirer}$^\textrm{\scriptsize 37}$,
\AtlasOrcid[0000-0002-7659-8948]{F.~Beisiegel}$^\textrm{\scriptsize 25}$,
\AtlasOrcid[0000-0001-9974-1527]{M.~Belfkir}$^\textrm{\scriptsize 120b}$,
\AtlasOrcid[0000-0002-4009-0990]{G.~Bella}$^\textrm{\scriptsize 155}$,
\AtlasOrcid[0000-0001-7098-9393]{L.~Bellagamba}$^\textrm{\scriptsize 24b}$,
\AtlasOrcid[0000-0001-6775-0111]{A.~Bellerive}$^\textrm{\scriptsize 35}$,
\AtlasOrcid[0000-0003-2049-9622]{P.~Bellos}$^\textrm{\scriptsize 21}$,
\AtlasOrcid[0000-0003-0945-4087]{K.~Beloborodov}$^\textrm{\scriptsize 38}$,
\AtlasOrcid[0000-0001-5196-8327]{D.~Benchekroun}$^\textrm{\scriptsize 36a}$,
\AtlasOrcid[0000-0002-5360-5973]{F.~Bendebba}$^\textrm{\scriptsize 36a}$,
\AtlasOrcid[0000-0002-0392-1783]{Y.~Benhammou}$^\textrm{\scriptsize 155}$,
\AtlasOrcid[0000-0003-4466-1196]{K.C.~Benkendorfer}$^\textrm{\scriptsize 63}$,
\AtlasOrcid[0000-0002-3080-1824]{L.~Beresford}$^\textrm{\scriptsize 50}$,
\AtlasOrcid[0000-0002-7026-8171]{M.~Beretta}$^\textrm{\scriptsize 55}$,
\AtlasOrcid[0000-0002-1253-8583]{E.~Bergeaas~Kuutmann}$^\textrm{\scriptsize 166}$,
\AtlasOrcid[0000-0002-7963-9725]{N.~Berger}$^\textrm{\scriptsize 4}$,
\AtlasOrcid[0000-0002-8076-5614]{B.~Bergmann}$^\textrm{\scriptsize 136}$,
\AtlasOrcid[0000-0002-9975-1781]{J.~Beringer}$^\textrm{\scriptsize 18a}$,
\AtlasOrcid[0000-0002-2837-2442]{G.~Bernardi}$^\textrm{\scriptsize 5}$,
\AtlasOrcid[0000-0003-3433-1687]{C.~Bernius}$^\textrm{\scriptsize 147}$,
\AtlasOrcid[0000-0001-8153-2719]{F.U.~Bernlochner}$^\textrm{\scriptsize 25}$,
\AtlasOrcid[0000-0003-0499-8755]{F.~Bernon}$^\textrm{\scriptsize 37}$,
\AtlasOrcid[0000-0002-1976-5703]{A.~Berrocal~Guardia}$^\textrm{\scriptsize 13}$,
\AtlasOrcid[0000-0002-9569-8231]{T.~Berry}$^\textrm{\scriptsize 98}$,
\AtlasOrcid[0000-0003-0780-0345]{P.~Berta}$^\textrm{\scriptsize 137}$,
\AtlasOrcid[0000-0002-3824-409X]{A.~Berthold}$^\textrm{\scriptsize 52}$,
\AtlasOrcid[0000-0003-0073-3821]{S.~Bethke}$^\textrm{\scriptsize 113}$,
\AtlasOrcid[0000-0003-0839-9311]{A.~Betti}$^\textrm{\scriptsize 77a,77b}$,
\AtlasOrcid[0000-0002-4105-9629]{A.J.~Bevan}$^\textrm{\scriptsize 97}$,
\AtlasOrcid[0000-0003-2677-5675]{N.K.~Bhalla}$^\textrm{\scriptsize 56}$,
\AtlasOrcid[0000-0002-9045-3278]{S.~Bhatta}$^\textrm{\scriptsize 149}$,
\AtlasOrcid[0000-0003-3837-4166]{D.S.~Bhattacharya}$^\textrm{\scriptsize 171}$,
\AtlasOrcid[0000-0001-9977-0416]{P.~Bhattarai}$^\textrm{\scriptsize 147}$,
\AtlasOrcid[0000-0003-1621-6036]{Z.M.~Bhatti}$^\textrm{\scriptsize 121}$,
\AtlasOrcid[0000-0001-8686-4026]{K.D.~Bhide}$^\textrm{\scriptsize 56}$,
\AtlasOrcid[0000-0003-3024-587X]{V.S.~Bhopatkar}$^\textrm{\scriptsize 125}$,
\AtlasOrcid[0000-0001-7345-7798]{R.M.~Bianchi}$^\textrm{\scriptsize 133}$,
\AtlasOrcid[0000-0003-4473-7242]{G.~Bianco}$^\textrm{\scriptsize 24b,24a}$,
\AtlasOrcid[0000-0002-8663-6856]{O.~Biebel}$^\textrm{\scriptsize 112}$,
\AtlasOrcid[0000-0002-2079-5344]{R.~Bielski}$^\textrm{\scriptsize 127}$,
\AtlasOrcid[0000-0001-5442-1351]{M.~Biglietti}$^\textrm{\scriptsize 79a}$,
\AtlasOrcid{C.S.~Billingsley}$^\textrm{\scriptsize 46}$,
\AtlasOrcid[0009-0002-0240-0270]{Y.~Bimgdi}$^\textrm{\scriptsize 36f}$,
\AtlasOrcid[0000-0001-6172-545X]{M.~Bindi}$^\textrm{\scriptsize 57}$,
\AtlasOrcid[0000-0002-2455-8039]{A.~Bingul}$^\textrm{\scriptsize 22b}$,
\AtlasOrcid[0000-0001-6674-7869]{C.~Bini}$^\textrm{\scriptsize 77a,77b}$,
\AtlasOrcid[0000-0003-2025-5935]{G.A.~Bird}$^\textrm{\scriptsize 33}$,
\AtlasOrcid[0000-0002-3835-0968]{M.~Birman}$^\textrm{\scriptsize 174}$,
\AtlasOrcid[0000-0003-2781-623X]{M.~Biros}$^\textrm{\scriptsize 137}$,
\AtlasOrcid[0000-0003-3386-9397]{S.~Biryukov}$^\textrm{\scriptsize 150}$,
\AtlasOrcid[0000-0002-7820-3065]{T.~Bisanz}$^\textrm{\scriptsize 51}$,
\AtlasOrcid[0000-0001-6410-9046]{E.~Bisceglie}$^\textrm{\scriptsize 45b,45a}$,
\AtlasOrcid[0000-0001-8361-2309]{J.P.~Biswal}$^\textrm{\scriptsize 138}$,
\AtlasOrcid[0000-0002-7543-3471]{D.~Biswas}$^\textrm{\scriptsize 145}$,
\AtlasOrcid[0000-0002-6696-5169]{I.~Bloch}$^\textrm{\scriptsize 50}$,
\AtlasOrcid[0000-0002-7716-5626]{A.~Blue}$^\textrm{\scriptsize 61}$,
\AtlasOrcid[0000-0002-6134-0303]{U.~Blumenschein}$^\textrm{\scriptsize 97}$,
\AtlasOrcid[0000-0001-5412-1236]{J.~Blumenthal}$^\textrm{\scriptsize 103}$,
\AtlasOrcid[0000-0002-2003-0261]{V.S.~Bobrovnikov}$^\textrm{\scriptsize 38}$,
\AtlasOrcid[0000-0001-9734-574X]{M.~Boehler}$^\textrm{\scriptsize 56}$,
\AtlasOrcid[0000-0002-8462-443X]{B.~Boehm}$^\textrm{\scriptsize 171}$,
\AtlasOrcid[0000-0003-2138-9062]{D.~Bogavac}$^\textrm{\scriptsize 37}$,
\AtlasOrcid[0000-0002-8635-9342]{A.G.~Bogdanchikov}$^\textrm{\scriptsize 38}$,
\AtlasOrcid[0000-0002-9924-7489]{L.S.~Boggia}$^\textrm{\scriptsize 131}$,
\AtlasOrcid[0000-0003-3807-7831]{C.~Bohm}$^\textrm{\scriptsize 49a}$,
\AtlasOrcid[0000-0002-7736-0173]{V.~Boisvert}$^\textrm{\scriptsize 98}$,
\AtlasOrcid[0000-0002-2668-889X]{P.~Bokan}$^\textrm{\scriptsize 37}$,
\AtlasOrcid[0000-0002-2432-411X]{T.~Bold}$^\textrm{\scriptsize 88a}$,
\AtlasOrcid[0000-0002-9807-861X]{M.~Bomben}$^\textrm{\scriptsize 5}$,
\AtlasOrcid[0000-0002-9660-580X]{M.~Bona}$^\textrm{\scriptsize 97}$,
\AtlasOrcid[0000-0003-0078-9817]{M.~Boonekamp}$^\textrm{\scriptsize 139}$,
\AtlasOrcid[0000-0002-6890-1601]{A.G.~Borb\'ely}$^\textrm{\scriptsize 61}$,
\AtlasOrcid[0000-0002-9249-2158]{I.S.~Bordulev}$^\textrm{\scriptsize 38}$,
\AtlasOrcid[0000-0002-4226-9521]{G.~Borissov}$^\textrm{\scriptsize 94}$,
\AtlasOrcid[0000-0002-1287-4712]{D.~Bortoletto}$^\textrm{\scriptsize 130}$,
\AtlasOrcid[0000-0001-9207-6413]{D.~Boscherini}$^\textrm{\scriptsize 24b}$,
\AtlasOrcid[0000-0002-7290-643X]{M.~Bosman}$^\textrm{\scriptsize 13}$,
\AtlasOrcid[0000-0002-7723-5030]{K.~Bouaouda}$^\textrm{\scriptsize 36a}$,
\AtlasOrcid[0000-0002-5129-5705]{N.~Bouchhar}$^\textrm{\scriptsize 168}$,
\AtlasOrcid[0000-0002-3613-3142]{L.~Boudet}$^\textrm{\scriptsize 4}$,
\AtlasOrcid[0000-0002-9314-5860]{J.~Boudreau}$^\textrm{\scriptsize 133}$,
\AtlasOrcid[0000-0002-5103-1558]{E.V.~Bouhova-Thacker}$^\textrm{\scriptsize 94}$,
\AtlasOrcid[0000-0002-7809-3118]{D.~Boumediene}$^\textrm{\scriptsize 42}$,
\AtlasOrcid[0000-0001-9683-7101]{R.~Bouquet}$^\textrm{\scriptsize 59b,59a}$,
\AtlasOrcid[0000-0002-6647-6699]{A.~Boveia}$^\textrm{\scriptsize 123}$,
\AtlasOrcid[0000-0001-7360-0726]{J.~Boyd}$^\textrm{\scriptsize 37}$,
\AtlasOrcid[0000-0002-2704-835X]{D.~Boye}$^\textrm{\scriptsize 30}$,
\AtlasOrcid[0000-0002-3355-4662]{I.R.~Boyko}$^\textrm{\scriptsize 40}$,
\AtlasOrcid[0000-0002-1243-9980]{L.~Bozianu}$^\textrm{\scriptsize 58}$,
\AtlasOrcid[0000-0001-5762-3477]{J.~Bracinik}$^\textrm{\scriptsize 21}$,
\AtlasOrcid[0000-0003-0992-3509]{N.~Brahimi}$^\textrm{\scriptsize 4}$,
\AtlasOrcid[0000-0001-7992-0309]{G.~Brandt}$^\textrm{\scriptsize 176}$,
\AtlasOrcid[0000-0001-5219-1417]{O.~Brandt}$^\textrm{\scriptsize 33}$,
\AtlasOrcid[0000-0003-4339-4727]{F.~Braren}$^\textrm{\scriptsize 50}$,
\AtlasOrcid[0000-0001-9726-4376]{B.~Brau}$^\textrm{\scriptsize 106}$,
\AtlasOrcid[0000-0003-1292-9725]{J.E.~Brau}$^\textrm{\scriptsize 127}$,
\AtlasOrcid[0000-0001-5791-4872]{R.~Brener}$^\textrm{\scriptsize 174}$,
\AtlasOrcid[0000-0001-5350-7081]{L.~Brenner}$^\textrm{\scriptsize 118}$,
\AtlasOrcid[0000-0002-8204-4124]{R.~Brenner}$^\textrm{\scriptsize 166}$,
\AtlasOrcid[0000-0003-4194-2734]{S.~Bressler}$^\textrm{\scriptsize 174}$,
\AtlasOrcid[0009-0000-8406-368X]{G.~Brianti}$^\textrm{\scriptsize 80a,80b}$,
\AtlasOrcid[0000-0001-9998-4342]{D.~Britton}$^\textrm{\scriptsize 61}$,
\AtlasOrcid[0000-0002-9246-7366]{D.~Britzger}$^\textrm{\scriptsize 113}$,
\AtlasOrcid[0000-0003-0903-8948]{I.~Brock}$^\textrm{\scriptsize 25}$,
\AtlasOrcid[0000-0002-3354-1810]{G.~Brooijmans}$^\textrm{\scriptsize 43}$,
\AtlasOrcid[0000-0002-8090-6181]{E.M.~Brooks}$^\textrm{\scriptsize 160b}$,
\AtlasOrcid[0000-0002-6800-9808]{E.~Brost}$^\textrm{\scriptsize 30}$,
\AtlasOrcid[0000-0002-5485-7419]{L.M.~Brown}$^\textrm{\scriptsize 170}$,
\AtlasOrcid[0009-0006-4398-5526]{L.E.~Bruce}$^\textrm{\scriptsize 63}$,
\AtlasOrcid[0000-0002-6199-8041]{T.L.~Bruckler}$^\textrm{\scriptsize 130}$,
\AtlasOrcid[0000-0002-0206-1160]{P.A.~Bruckman~de~Renstrom}$^\textrm{\scriptsize 89}$,
\AtlasOrcid[0000-0002-1479-2112]{B.~Br\"{u}ers}$^\textrm{\scriptsize 50}$,
\AtlasOrcid[0000-0003-4806-0718]{A.~Bruni}$^\textrm{\scriptsize 24b}$,
\AtlasOrcid[0000-0001-5667-7748]{G.~Bruni}$^\textrm{\scriptsize 24b}$,
\AtlasOrcid[0000-0002-4319-4023]{M.~Bruschi}$^\textrm{\scriptsize 24b}$,
\AtlasOrcid[0000-0002-6168-689X]{N.~Bruscino}$^\textrm{\scriptsize 77a,77b}$,
\AtlasOrcid[0000-0002-8977-121X]{T.~Buanes}$^\textrm{\scriptsize 17}$,
\AtlasOrcid[0000-0001-7318-5251]{Q.~Buat}$^\textrm{\scriptsize 142}$,
\AtlasOrcid[0000-0001-8272-1108]{D.~Buchin}$^\textrm{\scriptsize 113}$,
\AtlasOrcid[0000-0001-8355-9237]{A.G.~Buckley}$^\textrm{\scriptsize 61}$,
\AtlasOrcid[0000-0002-5687-2073]{O.~Bulekov}$^\textrm{\scriptsize 38}$,
\AtlasOrcid[0000-0001-7148-6536]{B.A.~Bullard}$^\textrm{\scriptsize 147}$,
\AtlasOrcid[0000-0003-4831-4132]{S.~Burdin}$^\textrm{\scriptsize 95}$,
\AtlasOrcid[0000-0002-6900-825X]{C.D.~Burgard}$^\textrm{\scriptsize 51}$,
\AtlasOrcid[0000-0003-0685-4122]{A.M.~Burger}$^\textrm{\scriptsize 37}$,
\AtlasOrcid[0000-0001-5686-0948]{B.~Burghgrave}$^\textrm{\scriptsize 8}$,
\AtlasOrcid[0000-0001-8283-935X]{O.~Burlayenko}$^\textrm{\scriptsize 56}$,
\AtlasOrcid[0000-0002-7898-2230]{J.~Burleson}$^\textrm{\scriptsize 167}$,
\AtlasOrcid[0000-0001-6726-6362]{J.T.P.~Burr}$^\textrm{\scriptsize 33}$,
\AtlasOrcid[0000-0002-4690-0528]{J.C.~Burzynski}$^\textrm{\scriptsize 146}$,
\AtlasOrcid[0000-0003-4482-2666]{E.L.~Busch}$^\textrm{\scriptsize 43}$,
\AtlasOrcid[0000-0001-9196-0629]{V.~B\"uscher}$^\textrm{\scriptsize 103}$,
\AtlasOrcid[0000-0003-0988-7878]{P.J.~Bussey}$^\textrm{\scriptsize 61}$,
\AtlasOrcid[0000-0003-2834-836X]{J.M.~Butler}$^\textrm{\scriptsize 26}$,
\AtlasOrcid[0000-0003-0188-6491]{C.M.~Buttar}$^\textrm{\scriptsize 61}$,
\AtlasOrcid[0000-0002-5905-5394]{J.M.~Butterworth}$^\textrm{\scriptsize 99}$,
\AtlasOrcid[0000-0002-5116-1897]{W.~Buttinger}$^\textrm{\scriptsize 138}$,
\AtlasOrcid[0009-0007-8811-9135]{C.J.~Buxo~Vazquez}$^\textrm{\scriptsize 110}$,
\AtlasOrcid[0000-0002-5458-5564]{A.R.~Buzykaev}$^\textrm{\scriptsize 38}$,
\AtlasOrcid[0000-0001-7640-7913]{S.~Cabrera~Urb\'an}$^\textrm{\scriptsize 168}$,
\AtlasOrcid[0000-0001-8789-610X]{L.~Cadamuro}$^\textrm{\scriptsize 68}$,
\AtlasOrcid[0000-0001-7808-8442]{D.~Caforio}$^\textrm{\scriptsize 60}$,
\AtlasOrcid[0000-0001-7575-3603]{H.~Cai}$^\textrm{\scriptsize 133}$,
\AtlasOrcid[0000-0003-4946-153X]{Y.~Cai}$^\textrm{\scriptsize 14,115c}$,
\AtlasOrcid[0000-0003-2246-7456]{Y.~Cai}$^\textrm{\scriptsize 115a}$,
\AtlasOrcid[0000-0002-0758-7575]{V.M.M.~Cairo}$^\textrm{\scriptsize 37}$,
\AtlasOrcid[0000-0002-9016-138X]{O.~Cakir}$^\textrm{\scriptsize 3a}$,
\AtlasOrcid[0000-0002-1494-9538]{N.~Calace}$^\textrm{\scriptsize 37}$,
\AtlasOrcid[0000-0002-1692-1678]{P.~Calafiura}$^\textrm{\scriptsize 18a}$,
\AtlasOrcid[0000-0002-9495-9145]{G.~Calderini}$^\textrm{\scriptsize 131}$,
\AtlasOrcid[0000-0003-1600-464X]{P.~Calfayan}$^\textrm{\scriptsize 35}$,
\AtlasOrcid[0000-0001-5969-3786]{G.~Callea}$^\textrm{\scriptsize 61}$,
\AtlasOrcid{L.P.~Caloba}$^\textrm{\scriptsize 85b}$,
\AtlasOrcid[0000-0002-9953-5333]{D.~Calvet}$^\textrm{\scriptsize 42}$,
\AtlasOrcid[0000-0002-2531-3463]{S.~Calvet}$^\textrm{\scriptsize 42}$,
\AtlasOrcid[0000-0003-0125-2165]{M.~Calvetti}$^\textrm{\scriptsize 76a,76b}$,
\AtlasOrcid[0000-0002-9192-8028]{R.~Camacho~Toro}$^\textrm{\scriptsize 131}$,
\AtlasOrcid[0000-0003-0479-7689]{S.~Camarda}$^\textrm{\scriptsize 37}$,
\AtlasOrcid[0000-0002-2855-7738]{D.~Camarero~Munoz}$^\textrm{\scriptsize 27}$,
\AtlasOrcid[0000-0002-5732-5645]{P.~Camarri}$^\textrm{\scriptsize 78a,78b}$,
\AtlasOrcid[0000-0002-9417-8613]{M.T.~Camerlingo}$^\textrm{\scriptsize 74a,74b}$,
\AtlasOrcid[0000-0001-6097-2256]{D.~Cameron}$^\textrm{\scriptsize 37}$,
\AtlasOrcid[0000-0001-5929-1357]{C.~Camincher}$^\textrm{\scriptsize 170}$,
\AtlasOrcid[0000-0001-6746-3374]{M.~Campanelli}$^\textrm{\scriptsize 99}$,
\AtlasOrcid[0000-0002-6386-9788]{A.~Camplani}$^\textrm{\scriptsize 44}$,
\AtlasOrcid[0000-0003-2303-9306]{V.~Canale}$^\textrm{\scriptsize 74a,74b}$,
\AtlasOrcid[0000-0003-4602-473X]{A.C.~Canbay}$^\textrm{\scriptsize 3a}$,
\AtlasOrcid[0000-0002-7180-4562]{E.~Canonero}$^\textrm{\scriptsize 98}$,
\AtlasOrcid[0000-0001-8449-1019]{J.~Cantero}$^\textrm{\scriptsize 168}$,
\AtlasOrcid[0000-0001-8747-2809]{Y.~Cao}$^\textrm{\scriptsize 167}$,
\AtlasOrcid[0000-0002-3562-9592]{F.~Capocasa}$^\textrm{\scriptsize 27}$,
\AtlasOrcid[0000-0002-2443-6525]{M.~Capua}$^\textrm{\scriptsize 45b,45a}$,
\AtlasOrcid[0000-0002-4117-3800]{A.~Carbone}$^\textrm{\scriptsize 73a,73b}$,
\AtlasOrcid[0000-0003-4541-4189]{R.~Cardarelli}$^\textrm{\scriptsize 78a}$,
\AtlasOrcid[0000-0002-6511-7096]{J.C.J.~Cardenas}$^\textrm{\scriptsize 8}$,
\AtlasOrcid[0000-0002-4376-4911]{G.~Carducci}$^\textrm{\scriptsize 45b,45a}$,
\AtlasOrcid[0000-0003-4058-5376]{T.~Carli}$^\textrm{\scriptsize 37}$,
\AtlasOrcid[0000-0002-3924-0445]{G.~Carlino}$^\textrm{\scriptsize 74a}$,
\AtlasOrcid[0000-0003-1718-307X]{J.I.~Carlotto}$^\textrm{\scriptsize 13}$,
\AtlasOrcid[0000-0002-7550-7821]{B.T.~Carlson}$^\textrm{\scriptsize 133,p}$,
\AtlasOrcid[0000-0002-4139-9543]{E.M.~Carlson}$^\textrm{\scriptsize 170,160a}$,
\AtlasOrcid[0000-0002-1705-1061]{J.~Carmignani}$^\textrm{\scriptsize 95}$,
\AtlasOrcid[0000-0003-4535-2926]{L.~Carminati}$^\textrm{\scriptsize 73a,73b}$,
\AtlasOrcid[0000-0002-8405-0886]{A.~Carnelli}$^\textrm{\scriptsize 139}$,
\AtlasOrcid[0000-0003-3570-7332]{M.~Carnesale}$^\textrm{\scriptsize 77a,77b}$,
\AtlasOrcid[0000-0003-2941-2829]{S.~Caron}$^\textrm{\scriptsize 117}$,
\AtlasOrcid[0000-0002-7863-1166]{E.~Carquin}$^\textrm{\scriptsize 141f}$,
\AtlasOrcid[0000-0001-7431-4211]{I.B.~Carr}$^\textrm{\scriptsize 108}$,
\AtlasOrcid[0000-0001-8650-942X]{S.~Carr\'a}$^\textrm{\scriptsize 73a}$,
\AtlasOrcid[0000-0002-8846-2714]{G.~Carratta}$^\textrm{\scriptsize 24b,24a}$,
\AtlasOrcid[0000-0003-1692-2029]{A.M.~Carroll}$^\textrm{\scriptsize 127}$,
\AtlasOrcid[0000-0002-0394-5646]{M.P.~Casado}$^\textrm{\scriptsize 13,h}$,
\AtlasOrcid[0000-0001-9116-0461]{M.~Caspar}$^\textrm{\scriptsize 50}$,
\AtlasOrcid[0000-0002-1172-1052]{F.L.~Castillo}$^\textrm{\scriptsize 4}$,
\AtlasOrcid[0000-0003-1396-2826]{L.~Castillo~Garcia}$^\textrm{\scriptsize 13}$,
\AtlasOrcid[0000-0002-8245-1790]{V.~Castillo~Gimenez}$^\textrm{\scriptsize 168}$,
\AtlasOrcid[0000-0001-8491-4376]{N.F.~Castro}$^\textrm{\scriptsize 134a,134e}$,
\AtlasOrcid[0000-0001-8774-8887]{A.~Catinaccio}$^\textrm{\scriptsize 37}$,
\AtlasOrcid[0000-0001-8915-0184]{J.R.~Catmore}$^\textrm{\scriptsize 129}$,
\AtlasOrcid[0000-0003-2897-0466]{T.~Cavaliere}$^\textrm{\scriptsize 4}$,
\AtlasOrcid[0000-0002-4297-8539]{V.~Cavaliere}$^\textrm{\scriptsize 30}$,
\AtlasOrcid{L.J.~Caviedes~Betancourt}$^\textrm{\scriptsize 23b}$,
\AtlasOrcid[0000-0002-5107-7134]{Y.C.~Cekmecelioglu}$^\textrm{\scriptsize 50}$,
\AtlasOrcid[0000-0003-3793-0159]{E.~Celebi}$^\textrm{\scriptsize 84}$,
\AtlasOrcid[0000-0001-7593-0243]{S.~Cella}$^\textrm{\scriptsize 37}$,
\AtlasOrcid[0000-0002-4809-4056]{V.~Cepaitis}$^\textrm{\scriptsize 58}$,
\AtlasOrcid[0000-0003-0683-2177]{K.~Cerny}$^\textrm{\scriptsize 126}$,
\AtlasOrcid[0000-0002-4300-703X]{A.S.~Cerqueira}$^\textrm{\scriptsize 85a}$,
\AtlasOrcid[0000-0002-1904-6661]{A.~Cerri}$^\textrm{\scriptsize 150}$,
\AtlasOrcid[0000-0002-8077-7850]{L.~Cerrito}$^\textrm{\scriptsize 78a,78b}$,
\AtlasOrcid[0000-0001-9669-9642]{F.~Cerutti}$^\textrm{\scriptsize 18a}$,
\AtlasOrcid[0000-0002-5200-0016]{B.~Cervato}$^\textrm{\scriptsize 145}$,
\AtlasOrcid[0000-0002-0518-1459]{A.~Cervelli}$^\textrm{\scriptsize 24b}$,
\AtlasOrcid[0000-0001-9073-0725]{G.~Cesarini}$^\textrm{\scriptsize 55}$,
\AtlasOrcid[0000-0001-5050-8441]{S.A.~Cetin}$^\textrm{\scriptsize 84}$,
\AtlasOrcid[0000-0002-9865-4146]{D.~Chakraborty}$^\textrm{\scriptsize 119}$,
\AtlasOrcid[0000-0001-7069-0295]{J.~Chan}$^\textrm{\scriptsize 18a}$,
\AtlasOrcid[0000-0002-5369-8540]{W.Y.~Chan}$^\textrm{\scriptsize 157}$,
\AtlasOrcid[0000-0002-2926-8962]{J.D.~Chapman}$^\textrm{\scriptsize 33}$,
\AtlasOrcid[0000-0001-6968-9828]{E.~Chapon}$^\textrm{\scriptsize 139}$,
\AtlasOrcid[0000-0002-5376-2397]{B.~Chargeishvili}$^\textrm{\scriptsize 153b}$,
\AtlasOrcid[0000-0003-0211-2041]{D.G.~Charlton}$^\textrm{\scriptsize 21}$,
\AtlasOrcid[0000-0003-4241-7405]{M.~Chatterjee}$^\textrm{\scriptsize 20}$,
\AtlasOrcid[0000-0001-5725-9134]{C.~Chauhan}$^\textrm{\scriptsize 137}$,
\AtlasOrcid[0000-0001-6623-1205]{Y.~Che}$^\textrm{\scriptsize 115a}$,
\AtlasOrcid[0000-0001-7314-7247]{S.~Chekanov}$^\textrm{\scriptsize 6}$,
\AtlasOrcid[0000-0002-4034-2326]{S.V.~Chekulaev}$^\textrm{\scriptsize 160a}$,
\AtlasOrcid[0000-0002-3468-9761]{G.A.~Chelkov}$^\textrm{\scriptsize 40,a}$,
\AtlasOrcid[0000-0001-9973-7966]{A.~Chen}$^\textrm{\scriptsize 109}$,
\AtlasOrcid[0000-0002-3034-8943]{B.~Chen}$^\textrm{\scriptsize 155}$,
\AtlasOrcid[0000-0002-7985-9023]{B.~Chen}$^\textrm{\scriptsize 170}$,
\AtlasOrcid[0000-0002-5895-6799]{H.~Chen}$^\textrm{\scriptsize 115a}$,
\AtlasOrcid[0000-0002-9936-0115]{H.~Chen}$^\textrm{\scriptsize 30}$,
\AtlasOrcid[0000-0002-2554-2725]{J.~Chen}$^\textrm{\scriptsize 64c}$,
\AtlasOrcid[0000-0003-1586-5253]{J.~Chen}$^\textrm{\scriptsize 146}$,
\AtlasOrcid[0000-0001-7021-3720]{M.~Chen}$^\textrm{\scriptsize 130}$,
\AtlasOrcid[0000-0001-7987-9764]{S.~Chen}$^\textrm{\scriptsize 90}$,
\AtlasOrcid[0000-0003-0447-5348]{S.J.~Chen}$^\textrm{\scriptsize 115a}$,
\AtlasOrcid[0000-0003-4977-2717]{X.~Chen}$^\textrm{\scriptsize 64c}$,
\AtlasOrcid[0000-0003-4027-3305]{X.~Chen}$^\textrm{\scriptsize 15,aa}$,
\AtlasOrcid[0000-0001-6793-3604]{Y.~Chen}$^\textrm{\scriptsize 64a}$,
\AtlasOrcid[0000-0002-4086-1847]{C.L.~Cheng}$^\textrm{\scriptsize 175}$,
\AtlasOrcid[0000-0002-8912-4389]{H.C.~Cheng}$^\textrm{\scriptsize 66a}$,
\AtlasOrcid[0000-0002-2797-6383]{S.~Cheong}$^\textrm{\scriptsize 147}$,
\AtlasOrcid[0000-0002-0967-2351]{A.~Cheplakov}$^\textrm{\scriptsize 40}$,
\AtlasOrcid[0000-0002-8772-0961]{E.~Cheremushkina}$^\textrm{\scriptsize 50}$,
\AtlasOrcid[0000-0002-3150-8478]{E.~Cherepanova}$^\textrm{\scriptsize 118}$,
\AtlasOrcid[0000-0002-5842-2818]{R.~Cherkaoui~El~Moursli}$^\textrm{\scriptsize 36e}$,
\AtlasOrcid[0000-0002-2562-9724]{E.~Cheu}$^\textrm{\scriptsize 7}$,
\AtlasOrcid[0000-0003-2176-4053]{K.~Cheung}$^\textrm{\scriptsize 67}$,
\AtlasOrcid[0000-0003-3762-7264]{L.~Chevalier}$^\textrm{\scriptsize 139}$,
\AtlasOrcid[0000-0002-4210-2924]{V.~Chiarella}$^\textrm{\scriptsize 55}$,
\AtlasOrcid[0000-0001-9851-4816]{G.~Chiarelli}$^\textrm{\scriptsize 76a}$,
\AtlasOrcid[0000-0003-1256-1043]{N.~Chiedde}$^\textrm{\scriptsize 105}$,
\AtlasOrcid[0000-0002-2458-9513]{G.~Chiodini}$^\textrm{\scriptsize 72a}$,
\AtlasOrcid[0000-0001-9214-8528]{A.S.~Chisholm}$^\textrm{\scriptsize 21}$,
\AtlasOrcid[0000-0003-2262-4773]{A.~Chitan}$^\textrm{\scriptsize 28b}$,
\AtlasOrcid[0000-0003-1523-7783]{M.~Chitishvili}$^\textrm{\scriptsize 168}$,
\AtlasOrcid[0000-0001-5841-3316]{M.V.~Chizhov}$^\textrm{\scriptsize 40}$,
\AtlasOrcid[0000-0003-0748-694X]{K.~Choi}$^\textrm{\scriptsize 11}$,
\AtlasOrcid[0000-0002-2204-5731]{Y.~Chou}$^\textrm{\scriptsize 142}$,
\AtlasOrcid[0000-0002-4549-2219]{E.Y.S.~Chow}$^\textrm{\scriptsize 117}$,
\AtlasOrcid[0000-0002-7442-6181]{K.L.~Chu}$^\textrm{\scriptsize 174}$,
\AtlasOrcid[0000-0002-1971-0403]{M.C.~Chu}$^\textrm{\scriptsize 66a}$,
\AtlasOrcid[0000-0003-2848-0184]{X.~Chu}$^\textrm{\scriptsize 14,115c}$,
\AtlasOrcid[0000-0003-2005-5992]{Z.~Chubinidze}$^\textrm{\scriptsize 55}$,
\AtlasOrcid[0000-0002-6425-2579]{J.~Chudoba}$^\textrm{\scriptsize 135}$,
\AtlasOrcid[0000-0002-6190-8376]{J.J.~Chwastowski}$^\textrm{\scriptsize 89}$,
\AtlasOrcid[0000-0002-3533-3847]{D.~Cieri}$^\textrm{\scriptsize 113}$,
\AtlasOrcid[0000-0003-2751-3474]{K.M.~Ciesla}$^\textrm{\scriptsize 88a}$,
\AtlasOrcid[0000-0002-2037-7185]{V.~Cindro}$^\textrm{\scriptsize 96}$,
\AtlasOrcid[0000-0002-3081-4879]{A.~Ciocio}$^\textrm{\scriptsize 18a}$,
\AtlasOrcid[0000-0001-6556-856X]{F.~Cirotto}$^\textrm{\scriptsize 74a,74b}$,
\AtlasOrcid[0000-0003-1831-6452]{Z.H.~Citron}$^\textrm{\scriptsize 174}$,
\AtlasOrcid[0000-0002-0842-0654]{M.~Citterio}$^\textrm{\scriptsize 73a}$,
\AtlasOrcid{D.A.~Ciubotaru}$^\textrm{\scriptsize 28b}$,
\AtlasOrcid[0000-0001-8341-5911]{A.~Clark}$^\textrm{\scriptsize 58}$,
\AtlasOrcid[0000-0002-3777-0880]{P.J.~Clark}$^\textrm{\scriptsize 54}$,
\AtlasOrcid[0000-0001-9236-7325]{N.~Clarke~Hall}$^\textrm{\scriptsize 99}$,
\AtlasOrcid[0000-0002-6031-8788]{C.~Clarry}$^\textrm{\scriptsize 159}$,
\AtlasOrcid[0000-0003-3210-1722]{J.M.~Clavijo~Columbie}$^\textrm{\scriptsize 50}$,
\AtlasOrcid[0000-0001-9952-934X]{S.E.~Clawson}$^\textrm{\scriptsize 50}$,
\AtlasOrcid[0000-0003-3122-3605]{C.~Clement}$^\textrm{\scriptsize 49a,49b}$,
\AtlasOrcid[0000-0001-8195-7004]{Y.~Coadou}$^\textrm{\scriptsize 105}$,
\AtlasOrcid[0000-0003-3309-0762]{M.~Cobal}$^\textrm{\scriptsize 71a,71c}$,
\AtlasOrcid[0000-0003-2368-4559]{A.~Coccaro}$^\textrm{\scriptsize 59b}$,
\AtlasOrcid[0000-0001-8985-5379]{R.F.~Coelho~Barrue}$^\textrm{\scriptsize 134a}$,
\AtlasOrcid[0000-0001-5200-9195]{R.~Coelho~Lopes~De~Sa}$^\textrm{\scriptsize 106}$,
\AtlasOrcid[0000-0002-5145-3646]{S.~Coelli}$^\textrm{\scriptsize 73a}$,
\AtlasOrcid[0009-0009-2414-9989]{L.S.~Colangeli}$^\textrm{\scriptsize 159}$,
\AtlasOrcid[0000-0002-5092-2148]{B.~Cole}$^\textrm{\scriptsize 43}$,
\AtlasOrcid[0000-0002-9412-7090]{J.~Collot}$^\textrm{\scriptsize 62}$,
\AtlasOrcid[0000-0002-9187-7478]{P.~Conde~Mui\~no}$^\textrm{\scriptsize 134a,134g}$,
\AtlasOrcid[0000-0002-4799-7560]{M.P.~Connell}$^\textrm{\scriptsize 34c}$,
\AtlasOrcid[0000-0001-6000-7245]{S.H.~Connell}$^\textrm{\scriptsize 34c}$,
\AtlasOrcid[0000-0002-0215-2767]{E.I.~Conroy}$^\textrm{\scriptsize 130}$,
\AtlasOrcid[0000-0002-5575-1413]{F.~Conventi}$^\textrm{\scriptsize 74a,ac}$,
\AtlasOrcid[0000-0001-9297-1063]{H.G.~Cooke}$^\textrm{\scriptsize 21}$,
\AtlasOrcid[0000-0002-7107-5902]{A.M.~Cooper-Sarkar}$^\textrm{\scriptsize 130}$,
\AtlasOrcid[0000-0002-1788-3204]{F.A.~Corchia}$^\textrm{\scriptsize 24b,24a}$,
\AtlasOrcid[0000-0001-7687-8299]{A.~Cordeiro~Oudot~Choi}$^\textrm{\scriptsize 131}$,
\AtlasOrcid[0000-0003-2136-4842]{L.D.~Corpe}$^\textrm{\scriptsize 42}$,
\AtlasOrcid[0000-0001-8729-466X]{M.~Corradi}$^\textrm{\scriptsize 77a,77b}$,
\AtlasOrcid[0000-0002-4970-7600]{F.~Corriveau}$^\textrm{\scriptsize 107,w}$,
\AtlasOrcid[0000-0002-3279-3370]{A.~Cortes-Gonzalez}$^\textrm{\scriptsize 19}$,
\AtlasOrcid[0000-0002-2064-2954]{M.J.~Costa}$^\textrm{\scriptsize 168}$,
\AtlasOrcid[0000-0002-8056-8469]{F.~Costanza}$^\textrm{\scriptsize 4}$,
\AtlasOrcid[0000-0003-4920-6264]{D.~Costanzo}$^\textrm{\scriptsize 143}$,
\AtlasOrcid[0000-0003-2444-8267]{B.M.~Cote}$^\textrm{\scriptsize 123}$,
\AtlasOrcid[0009-0004-3577-576X]{J.~Couthures}$^\textrm{\scriptsize 4}$,
\AtlasOrcid[0000-0001-8363-9827]{G.~Cowan}$^\textrm{\scriptsize 98}$,
\AtlasOrcid[0000-0002-5769-7094]{K.~Cranmer}$^\textrm{\scriptsize 175}$,
\AtlasOrcid[0009-0009-6459-2723]{L.~Cremer}$^\textrm{\scriptsize 51}$,
\AtlasOrcid[0000-0003-1687-3079]{D.~Cremonini}$^\textrm{\scriptsize 24b,24a}$,
\AtlasOrcid[0000-0001-5980-5805]{S.~Cr\'ep\'e-Renaudin}$^\textrm{\scriptsize 62}$,
\AtlasOrcid[0000-0001-6457-2575]{F.~Crescioli}$^\textrm{\scriptsize 131}$,
\AtlasOrcid[0000-0003-3893-9171]{M.~Cristinziani}$^\textrm{\scriptsize 145}$,
\AtlasOrcid[0000-0002-0127-1342]{M.~Cristoforetti}$^\textrm{\scriptsize 80a,80b}$,
\AtlasOrcid[0000-0002-8731-4525]{V.~Croft}$^\textrm{\scriptsize 118}$,
\AtlasOrcid[0000-0002-6579-3334]{J.E.~Crosby}$^\textrm{\scriptsize 125}$,
\AtlasOrcid[0000-0001-5990-4811]{G.~Crosetti}$^\textrm{\scriptsize 45b,45a}$,
\AtlasOrcid[0000-0003-1494-7898]{A.~Cueto}$^\textrm{\scriptsize 102}$,
\AtlasOrcid[0009-0009-3212-0967]{H.~Cui}$^\textrm{\scriptsize 99}$,
\AtlasOrcid[0000-0002-4317-2449]{Z.~Cui}$^\textrm{\scriptsize 7}$,
\AtlasOrcid[0000-0001-5517-8795]{W.R.~Cunningham}$^\textrm{\scriptsize 61}$,
\AtlasOrcid[0000-0002-8682-9316]{F.~Curcio}$^\textrm{\scriptsize 168}$,
\AtlasOrcid[0000-0001-9637-0484]{J.R.~Curran}$^\textrm{\scriptsize 54}$,
\AtlasOrcid[0000-0003-0723-1437]{P.~Czodrowski}$^\textrm{\scriptsize 37}$,
\AtlasOrcid[0000-0001-7991-593X]{M.J.~Da~Cunha~Sargedas~De~Sousa}$^\textrm{\scriptsize 59b,59a}$,
\AtlasOrcid[0000-0003-1746-1914]{J.V.~Da~Fonseca~Pinto}$^\textrm{\scriptsize 85b}$,
\AtlasOrcid[0000-0001-6154-7323]{C.~Da~Via}$^\textrm{\scriptsize 104}$,
\AtlasOrcid[0000-0001-9061-9568]{W.~Dabrowski}$^\textrm{\scriptsize 88a}$,
\AtlasOrcid[0000-0002-7050-2669]{T.~Dado}$^\textrm{\scriptsize 37}$,
\AtlasOrcid[0000-0002-5222-7894]{S.~Dahbi}$^\textrm{\scriptsize 152}$,
\AtlasOrcid[0000-0002-9607-5124]{T.~Dai}$^\textrm{\scriptsize 109}$,
\AtlasOrcid[0000-0001-7176-7979]{D.~Dal~Santo}$^\textrm{\scriptsize 20}$,
\AtlasOrcid[0000-0002-1391-2477]{C.~Dallapiccola}$^\textrm{\scriptsize 106}$,
\AtlasOrcid[0000-0001-6278-9674]{M.~Dam}$^\textrm{\scriptsize 44}$,
\AtlasOrcid[0000-0002-9742-3709]{G.~D'amen}$^\textrm{\scriptsize 30}$,
\AtlasOrcid[0000-0002-2081-0129]{V.~D'Amico}$^\textrm{\scriptsize 112}$,
\AtlasOrcid[0000-0002-7290-1372]{J.~Damp}$^\textrm{\scriptsize 103}$,
\AtlasOrcid[0000-0002-9271-7126]{J.R.~Dandoy}$^\textrm{\scriptsize 35}$,
\AtlasOrcid[0000-0001-8325-7650]{D.~Dannheim}$^\textrm{\scriptsize 37}$,
\AtlasOrcid[0000-0002-7807-7484]{M.~Danninger}$^\textrm{\scriptsize 146}$,
\AtlasOrcid[0000-0003-1645-8393]{V.~Dao}$^\textrm{\scriptsize 149}$,
\AtlasOrcid[0000-0003-2165-0638]{G.~Darbo}$^\textrm{\scriptsize 59b}$,
\AtlasOrcid[0000-0003-2693-3389]{S.J.~Das}$^\textrm{\scriptsize 30,ad}$,
\AtlasOrcid[0000-0003-3316-8574]{F.~Dattola}$^\textrm{\scriptsize 50}$,
\AtlasOrcid[0000-0003-3393-6318]{S.~D'Auria}$^\textrm{\scriptsize 73a,73b}$,
\AtlasOrcid[0000-0002-1104-3650]{A.~D'avanzo}$^\textrm{\scriptsize 74a,74b}$,
\AtlasOrcid[0000-0002-1794-1443]{C.~David}$^\textrm{\scriptsize 34a}$,
\AtlasOrcid[0000-0002-3770-8307]{T.~Davidek}$^\textrm{\scriptsize 137}$,
\AtlasOrcid[0000-0002-5177-8950]{I.~Dawson}$^\textrm{\scriptsize 97}$,
\AtlasOrcid[0000-0002-9710-2980]{H.A.~Day-hall}$^\textrm{\scriptsize 136}$,
\AtlasOrcid[0000-0002-5647-4489]{K.~De}$^\textrm{\scriptsize 8}$,
\AtlasOrcid[0000-0002-7268-8401]{R.~De~Asmundis}$^\textrm{\scriptsize 74a}$,
\AtlasOrcid[0000-0002-5586-8224]{N.~De~Biase}$^\textrm{\scriptsize 50}$,
\AtlasOrcid[0000-0003-2178-5620]{S.~De~Castro}$^\textrm{\scriptsize 24b,24a}$,
\AtlasOrcid[0000-0001-6850-4078]{N.~De~Groot}$^\textrm{\scriptsize 117}$,
\AtlasOrcid[0000-0002-5330-2614]{P.~de~Jong}$^\textrm{\scriptsize 118}$,
\AtlasOrcid[0000-0002-4516-5269]{H.~De~la~Torre}$^\textrm{\scriptsize 119}$,
\AtlasOrcid[0000-0001-6651-845X]{A.~De~Maria}$^\textrm{\scriptsize 115a}$,
\AtlasOrcid[0000-0001-8099-7821]{A.~De~Salvo}$^\textrm{\scriptsize 77a}$,
\AtlasOrcid[0000-0003-4704-525X]{U.~De~Sanctis}$^\textrm{\scriptsize 78a,78b}$,
\AtlasOrcid[0000-0003-0120-2096]{F.~De~Santis}$^\textrm{\scriptsize 72a,72b}$,
\AtlasOrcid[0000-0002-9158-6646]{A.~De~Santo}$^\textrm{\scriptsize 150}$,
\AtlasOrcid[0000-0001-9163-2211]{J.B.~De~Vivie~De~Regie}$^\textrm{\scriptsize 62}$,
\AtlasOrcid[0000-0001-9324-719X]{J.~Debevc}$^\textrm{\scriptsize 96}$,
\AtlasOrcid{D.V.~Dedovich}$^\textrm{\scriptsize 40}$,
\AtlasOrcid[0000-0002-6966-4935]{J.~Degens}$^\textrm{\scriptsize 95}$,
\AtlasOrcid[0000-0003-0360-6051]{A.M.~Deiana}$^\textrm{\scriptsize 46}$,
\AtlasOrcid[0000-0001-7799-577X]{F.~Del~Corso}$^\textrm{\scriptsize 24b,24a}$,
\AtlasOrcid[0000-0001-7090-4134]{J.~Del~Peso}$^\textrm{\scriptsize 102}$,
\AtlasOrcid[0000-0002-9169-1884]{L.~Delagrange}$^\textrm{\scriptsize 131}$,
\AtlasOrcid[0000-0003-0777-6031]{F.~Deliot}$^\textrm{\scriptsize 139}$,
\AtlasOrcid[0000-0001-7021-3333]{C.M.~Delitzsch}$^\textrm{\scriptsize 51}$,
\AtlasOrcid[0000-0003-4446-3368]{M.~Della~Pietra}$^\textrm{\scriptsize 74a,74b}$,
\AtlasOrcid[0000-0001-8530-7447]{D.~Della~Volpe}$^\textrm{\scriptsize 58}$,
\AtlasOrcid[0000-0003-2453-7745]{A.~Dell'Acqua}$^\textrm{\scriptsize 37}$,
\AtlasOrcid[0000-0002-9601-4225]{L.~Dell'Asta}$^\textrm{\scriptsize 73a,73b}$,
\AtlasOrcid[0000-0003-2992-3805]{M.~Delmastro}$^\textrm{\scriptsize 4}$,
\AtlasOrcid[0000-0002-9556-2924]{P.A.~Delsart}$^\textrm{\scriptsize 62}$,
\AtlasOrcid[0000-0002-7282-1786]{S.~Demers}$^\textrm{\scriptsize 177}$,
\AtlasOrcid[0000-0002-7730-3072]{M.~Demichev}$^\textrm{\scriptsize 40}$,
\AtlasOrcid[0000-0002-4028-7881]{S.P.~Denisov}$^\textrm{\scriptsize 38}$,
\AtlasOrcid[0000-0002-4910-5378]{L.~D'Eramo}$^\textrm{\scriptsize 42}$,
\AtlasOrcid[0000-0001-5660-3095]{D.~Derendarz}$^\textrm{\scriptsize 89}$,
\AtlasOrcid[0000-0002-3505-3503]{F.~Derue}$^\textrm{\scriptsize 131}$,
\AtlasOrcid[0000-0003-3929-8046]{P.~Dervan}$^\textrm{\scriptsize 95}$,
\AtlasOrcid[0000-0001-5836-6118]{K.~Desch}$^\textrm{\scriptsize 25}$,
\AtlasOrcid[0000-0002-6477-764X]{C.~Deutsch}$^\textrm{\scriptsize 25}$,
\AtlasOrcid[0000-0002-9870-2021]{F.A.~Di~Bello}$^\textrm{\scriptsize 59b,59a}$,
\AtlasOrcid[0000-0001-8289-5183]{A.~Di~Ciaccio}$^\textrm{\scriptsize 78a,78b}$,
\AtlasOrcid[0000-0003-0751-8083]{L.~Di~Ciaccio}$^\textrm{\scriptsize 4}$,
\AtlasOrcid[0000-0001-8078-2759]{A.~Di~Domenico}$^\textrm{\scriptsize 77a,77b}$,
\AtlasOrcid[0000-0003-2213-9284]{C.~Di~Donato}$^\textrm{\scriptsize 74a,74b}$,
\AtlasOrcid[0000-0002-9508-4256]{A.~Di~Girolamo}$^\textrm{\scriptsize 37}$,
\AtlasOrcid[0000-0002-7838-576X]{G.~Di~Gregorio}$^\textrm{\scriptsize 37}$,
\AtlasOrcid[0000-0002-9074-2133]{A.~Di~Luca}$^\textrm{\scriptsize 80a,80b}$,
\AtlasOrcid[0000-0002-4067-1592]{B.~Di~Micco}$^\textrm{\scriptsize 79a,79b}$,
\AtlasOrcid[0000-0003-1111-3783]{R.~Di~Nardo}$^\textrm{\scriptsize 79a,79b}$,
\AtlasOrcid[0000-0001-8001-4602]{K.F.~Di~Petrillo}$^\textrm{\scriptsize 41}$,
\AtlasOrcid[0009-0009-9679-1268]{M.~Diamantopoulou}$^\textrm{\scriptsize 35}$,
\AtlasOrcid[0000-0001-6882-5402]{F.A.~Dias}$^\textrm{\scriptsize 118}$,
\AtlasOrcid[0000-0001-8855-3520]{T.~Dias~Do~Vale}$^\textrm{\scriptsize 146}$,
\AtlasOrcid[0000-0003-1258-8684]{M.A.~Diaz}$^\textrm{\scriptsize 141a,141b}$,
\AtlasOrcid{A.R.~Didenko}$^\textrm{\scriptsize 40}$,
\AtlasOrcid[0000-0001-9942-6543]{M.~Didenko}$^\textrm{\scriptsize 168}$,
\AtlasOrcid[0000-0002-7611-355X]{E.B.~Diehl}$^\textrm{\scriptsize 109}$,
\AtlasOrcid[0000-0003-3694-6167]{S.~D\'iez~Cornell}$^\textrm{\scriptsize 50}$,
\AtlasOrcid[0000-0002-0482-1127]{C.~Diez~Pardos}$^\textrm{\scriptsize 145}$,
\AtlasOrcid[0000-0002-9605-3558]{C.~Dimitriadi}$^\textrm{\scriptsize 166}$,
\AtlasOrcid[0000-0003-0086-0599]{A.~Dimitrievska}$^\textrm{\scriptsize 21}$,
\AtlasOrcid[0000-0001-5767-2121]{J.~Dingfelder}$^\textrm{\scriptsize 25}$,
\AtlasOrcid[0000-0002-5384-8246]{T.~Dingley}$^\textrm{\scriptsize 130}$,
\AtlasOrcid[0000-0002-2683-7349]{I-M.~Dinu}$^\textrm{\scriptsize 28b}$,
\AtlasOrcid[0000-0002-5172-7520]{S.J.~Dittmeier}$^\textrm{\scriptsize 65b}$,
\AtlasOrcid[0000-0002-1760-8237]{F.~Dittus}$^\textrm{\scriptsize 37}$,
\AtlasOrcid[0000-0002-5981-1719]{M.~Divisek}$^\textrm{\scriptsize 137}$,
\AtlasOrcid[0000-0003-3532-1173]{B.~Dixit}$^\textrm{\scriptsize 95}$,
\AtlasOrcid[0000-0003-1881-3360]{F.~Djama}$^\textrm{\scriptsize 105}$,
\AtlasOrcid[0000-0002-9414-8350]{T.~Djobava}$^\textrm{\scriptsize 153b}$,
\AtlasOrcid[0000-0002-1509-0390]{C.~Doglioni}$^\textrm{\scriptsize 104,101}$,
\AtlasOrcid[0000-0001-5271-5153]{A.~Dohnalova}$^\textrm{\scriptsize 29a}$,
\AtlasOrcid[0000-0001-5821-7067]{J.~Dolejsi}$^\textrm{\scriptsize 137}$,
\AtlasOrcid[0000-0002-5662-3675]{Z.~Dolezal}$^\textrm{\scriptsize 137}$,
\AtlasOrcid[0009-0001-4200-1592]{K.~Domijan}$^\textrm{\scriptsize 88a}$,
\AtlasOrcid[0000-0002-9753-6498]{K.M.~Dona}$^\textrm{\scriptsize 41}$,
\AtlasOrcid[0000-0001-8329-4240]{M.~Donadelli}$^\textrm{\scriptsize 85d}$,
\AtlasOrcid[0000-0002-6075-0191]{B.~Dong}$^\textrm{\scriptsize 110}$,
\AtlasOrcid[0000-0002-8998-0839]{J.~Donini}$^\textrm{\scriptsize 42}$,
\AtlasOrcid[0000-0002-0343-6331]{A.~D'Onofrio}$^\textrm{\scriptsize 74a,74b}$,
\AtlasOrcid[0000-0003-2408-5099]{M.~D'Onofrio}$^\textrm{\scriptsize 95}$,
\AtlasOrcid[0000-0002-0683-9910]{J.~Dopke}$^\textrm{\scriptsize 138}$,
\AtlasOrcid[0000-0002-5381-2649]{A.~Doria}$^\textrm{\scriptsize 74a}$,
\AtlasOrcid[0000-0001-9909-0090]{N.~Dos~Santos~Fernandes}$^\textrm{\scriptsize 134a}$,
\AtlasOrcid[0000-0001-9884-3070]{P.~Dougan}$^\textrm{\scriptsize 104}$,
\AtlasOrcid[0000-0001-6113-0878]{M.T.~Dova}$^\textrm{\scriptsize 93}$,
\AtlasOrcid[0000-0001-6322-6195]{A.T.~Doyle}$^\textrm{\scriptsize 61}$,
\AtlasOrcid[0000-0003-1530-0519]{M.A.~Draguet}$^\textrm{\scriptsize 130}$,
\AtlasOrcid[0009-0008-3244-6804]{M.P.~Drescher}$^\textrm{\scriptsize 57}$,
\AtlasOrcid[0000-0001-8955-9510]{E.~Dreyer}$^\textrm{\scriptsize 174}$,
\AtlasOrcid[0000-0002-2885-9779]{I.~Drivas-koulouris}$^\textrm{\scriptsize 10}$,
\AtlasOrcid[0009-0004-5587-1804]{M.~Drnevich}$^\textrm{\scriptsize 121}$,
\AtlasOrcid[0000-0003-0699-3931]{M.~Drozdova}$^\textrm{\scriptsize 58}$,
\AtlasOrcid[0000-0002-6758-0113]{D.~Du}$^\textrm{\scriptsize 64a}$,
\AtlasOrcid[0000-0001-8703-7938]{T.A.~du~Pree}$^\textrm{\scriptsize 118}$,
\AtlasOrcid[0000-0003-2182-2727]{F.~Dubinin}$^\textrm{\scriptsize 38}$,
\AtlasOrcid[0000-0002-3847-0775]{M.~Dubovsky}$^\textrm{\scriptsize 29a}$,
\AtlasOrcid[0000-0002-7276-6342]{E.~Duchovni}$^\textrm{\scriptsize 174}$,
\AtlasOrcid[0000-0002-7756-7801]{G.~Duckeck}$^\textrm{\scriptsize 112}$,
\AtlasOrcid[0000-0001-5914-0524]{O.A.~Ducu}$^\textrm{\scriptsize 28b}$,
\AtlasOrcid[0000-0002-5916-3467]{D.~Duda}$^\textrm{\scriptsize 54}$,
\AtlasOrcid[0000-0002-8713-8162]{A.~Dudarev}$^\textrm{\scriptsize 37}$,
\AtlasOrcid[0000-0002-9092-9344]{E.R.~Duden}$^\textrm{\scriptsize 27}$,
\AtlasOrcid[0000-0003-2499-1649]{M.~D'uffizi}$^\textrm{\scriptsize 104}$,
\AtlasOrcid[0000-0002-4871-2176]{L.~Duflot}$^\textrm{\scriptsize 68}$,
\AtlasOrcid[0000-0002-5833-7058]{M.~D\"uhrssen}$^\textrm{\scriptsize 37}$,
\AtlasOrcid[0000-0003-4089-3416]{I.~Duminica}$^\textrm{\scriptsize 28g}$,
\AtlasOrcid[0000-0003-3310-4642]{A.E.~Dumitriu}$^\textrm{\scriptsize 28b}$,
\AtlasOrcid[0000-0002-7667-260X]{M.~Dunford}$^\textrm{\scriptsize 65a}$,
\AtlasOrcid[0000-0001-9935-6397]{S.~Dungs}$^\textrm{\scriptsize 51}$,
\AtlasOrcid[0000-0003-2626-2247]{K.~Dunne}$^\textrm{\scriptsize 49a,49b}$,
\AtlasOrcid[0000-0002-5789-9825]{A.~Duperrin}$^\textrm{\scriptsize 105}$,
\AtlasOrcid[0000-0003-3469-6045]{H.~Duran~Yildiz}$^\textrm{\scriptsize 3a}$,
\AtlasOrcid[0000-0002-6066-4744]{M.~D\"uren}$^\textrm{\scriptsize 60}$,
\AtlasOrcid[0000-0003-4157-592X]{A.~Durglishvili}$^\textrm{\scriptsize 153b}$,
\AtlasOrcid[0000-0001-5430-4702]{B.L.~Dwyer}$^\textrm{\scriptsize 119}$,
\AtlasOrcid[0000-0003-1464-0335]{G.I.~Dyckes}$^\textrm{\scriptsize 18a}$,
\AtlasOrcid[0000-0001-9632-6352]{M.~Dyndal}$^\textrm{\scriptsize 88a}$,
\AtlasOrcid[0000-0002-0805-9184]{B.S.~Dziedzic}$^\textrm{\scriptsize 37}$,
\AtlasOrcid[0000-0002-2878-261X]{Z.O.~Earnshaw}$^\textrm{\scriptsize 150}$,
\AtlasOrcid[0000-0003-3300-9717]{G.H.~Eberwein}$^\textrm{\scriptsize 130}$,
\AtlasOrcid[0000-0003-0336-3723]{B.~Eckerova}$^\textrm{\scriptsize 29a}$,
\AtlasOrcid[0000-0001-5238-4921]{S.~Eggebrecht}$^\textrm{\scriptsize 57}$,
\AtlasOrcid[0000-0001-5370-8377]{E.~Egidio~Purcino~De~Souza}$^\textrm{\scriptsize 85e}$,
\AtlasOrcid[0000-0002-2701-968X]{L.F.~Ehrke}$^\textrm{\scriptsize 58}$,
\AtlasOrcid[0000-0003-3529-5171]{G.~Eigen}$^\textrm{\scriptsize 17}$,
\AtlasOrcid[0000-0002-4391-9100]{K.~Einsweiler}$^\textrm{\scriptsize 18a}$,
\AtlasOrcid[0000-0002-7341-9115]{T.~Ekelof}$^\textrm{\scriptsize 166}$,
\AtlasOrcid[0000-0002-7032-2799]{P.A.~Ekman}$^\textrm{\scriptsize 101}$,
\AtlasOrcid[0000-0002-7999-3767]{S.~El~Farkh}$^\textrm{\scriptsize 36b}$,
\AtlasOrcid[0000-0001-9172-2946]{Y.~El~Ghazali}$^\textrm{\scriptsize 64a}$,
\AtlasOrcid[0000-0002-8955-9681]{H.~El~Jarrari}$^\textrm{\scriptsize 37}$,
\AtlasOrcid[0000-0002-9669-5374]{A.~El~Moussaouy}$^\textrm{\scriptsize 36a}$,
\AtlasOrcid[0000-0001-5997-3569]{V.~Ellajosyula}$^\textrm{\scriptsize 166}$,
\AtlasOrcid[0000-0001-5265-3175]{M.~Ellert}$^\textrm{\scriptsize 166}$,
\AtlasOrcid[0000-0003-3596-5331]{F.~Ellinghaus}$^\textrm{\scriptsize 176}$,
\AtlasOrcid[0000-0002-1920-4930]{N.~Ellis}$^\textrm{\scriptsize 37}$,
\AtlasOrcid[0000-0001-8899-051X]{J.~Elmsheuser}$^\textrm{\scriptsize 30}$,
\AtlasOrcid[0000-0002-3012-9986]{M.~Elsawy}$^\textrm{\scriptsize 120a}$,
\AtlasOrcid[0000-0002-1213-0545]{M.~Elsing}$^\textrm{\scriptsize 37}$,
\AtlasOrcid[0000-0002-1363-9175]{D.~Emeliyanov}$^\textrm{\scriptsize 138}$,
\AtlasOrcid[0000-0002-9916-3349]{Y.~Enari}$^\textrm{\scriptsize 86}$,
\AtlasOrcid[0000-0003-2296-1112]{I.~Ene}$^\textrm{\scriptsize 18a}$,
\AtlasOrcid[0000-0002-4095-4808]{S.~Epari}$^\textrm{\scriptsize 13}$,
\AtlasOrcid[0000-0003-4543-6599]{P.A.~Erland}$^\textrm{\scriptsize 89}$,
\AtlasOrcid[0000-0003-2793-5335]{D.~Ernani~Martins~Neto}$^\textrm{\scriptsize 89}$,
\AtlasOrcid[0000-0003-4656-3936]{M.~Errenst}$^\textrm{\scriptsize 176}$,
\AtlasOrcid[0000-0003-4270-2775]{M.~Escalier}$^\textrm{\scriptsize 68}$,
\AtlasOrcid[0000-0003-4442-4537]{C.~Escobar}$^\textrm{\scriptsize 168}$,
\AtlasOrcid[0000-0001-6871-7794]{E.~Etzion}$^\textrm{\scriptsize 155}$,
\AtlasOrcid[0000-0003-0434-6925]{G.~Evans}$^\textrm{\scriptsize 134a}$,
\AtlasOrcid[0000-0003-2183-3127]{H.~Evans}$^\textrm{\scriptsize 70}$,
\AtlasOrcid[0000-0002-4333-5084]{L.S.~Evans}$^\textrm{\scriptsize 98}$,
\AtlasOrcid[0000-0002-7520-293X]{A.~Ezhilov}$^\textrm{\scriptsize 38}$,
\AtlasOrcid[0000-0002-7912-2830]{S.~Ezzarqtouni}$^\textrm{\scriptsize 36a}$,
\AtlasOrcid[0000-0001-8474-0978]{F.~Fabbri}$^\textrm{\scriptsize 24b,24a}$,
\AtlasOrcid[0000-0002-4002-8353]{L.~Fabbri}$^\textrm{\scriptsize 24b,24a}$,
\AtlasOrcid[0000-0002-4056-4578]{G.~Facini}$^\textrm{\scriptsize 99}$,
\AtlasOrcid[0000-0003-0154-4328]{V.~Fadeyev}$^\textrm{\scriptsize 140}$,
\AtlasOrcid[0000-0001-7882-2125]{R.M.~Fakhrutdinov}$^\textrm{\scriptsize 38}$,
\AtlasOrcid[0009-0006-2877-7710]{D.~Fakoudis}$^\textrm{\scriptsize 103}$,
\AtlasOrcid[0000-0002-7118-341X]{S.~Falciano}$^\textrm{\scriptsize 77a}$,
\AtlasOrcid[0000-0002-2298-3605]{L.F.~Falda~Ulhoa~Coelho}$^\textrm{\scriptsize 37}$,
\AtlasOrcid[0000-0003-2315-2499]{F.~Fallavollita}$^\textrm{\scriptsize 113}$,
\AtlasOrcid[0000-0002-1919-4250]{G.~Falsetti}$^\textrm{\scriptsize 45b,45a}$,
\AtlasOrcid[0000-0003-4278-7182]{J.~Faltova}$^\textrm{\scriptsize 137}$,
\AtlasOrcid[0000-0003-2611-1975]{C.~Fan}$^\textrm{\scriptsize 167}$,
\AtlasOrcid[0009-0009-7615-6275]{K.Y.~Fan}$^\textrm{\scriptsize 66b}$,
\AtlasOrcid[0000-0001-7868-3858]{Y.~Fan}$^\textrm{\scriptsize 14}$,
\AtlasOrcid[0000-0001-8630-6585]{Y.~Fang}$^\textrm{\scriptsize 14,115c}$,
\AtlasOrcid[0000-0002-8773-145X]{M.~Fanti}$^\textrm{\scriptsize 73a,73b}$,
\AtlasOrcid[0000-0001-9442-7598]{M.~Faraj}$^\textrm{\scriptsize 71a,71b}$,
\AtlasOrcid[0000-0003-2245-150X]{Z.~Farazpay}$^\textrm{\scriptsize 100}$,
\AtlasOrcid[0000-0003-0000-2439]{A.~Farbin}$^\textrm{\scriptsize 8}$,
\AtlasOrcid[0000-0002-3983-0728]{A.~Farilla}$^\textrm{\scriptsize 79a}$,
\AtlasOrcid[0000-0003-1363-9324]{T.~Farooque}$^\textrm{\scriptsize 110}$,
\AtlasOrcid[0000-0001-5350-9271]{S.M.~Farrington}$^\textrm{\scriptsize 54}$,
\AtlasOrcid[0000-0002-6423-7213]{F.~Fassi}$^\textrm{\scriptsize 36e}$,
\AtlasOrcid[0000-0003-1289-2141]{D.~Fassouliotis}$^\textrm{\scriptsize 9}$,
\AtlasOrcid[0000-0003-3731-820X]{M.~Faucci~Giannelli}$^\textrm{\scriptsize 78a,78b}$,
\AtlasOrcid[0000-0003-2596-8264]{W.J.~Fawcett}$^\textrm{\scriptsize 33}$,
\AtlasOrcid[0000-0002-2190-9091]{L.~Fayard}$^\textrm{\scriptsize 68}$,
\AtlasOrcid[0000-0001-5137-473X]{P.~Federic}$^\textrm{\scriptsize 137}$,
\AtlasOrcid[0000-0003-4176-2768]{P.~Federicova}$^\textrm{\scriptsize 135}$,
\AtlasOrcid[0000-0002-1733-7158]{O.L.~Fedin}$^\textrm{\scriptsize 38,a}$,
\AtlasOrcid[0000-0003-4124-7862]{M.~Feickert}$^\textrm{\scriptsize 175}$,
\AtlasOrcid[0000-0002-1403-0951]{L.~Feligioni}$^\textrm{\scriptsize 105}$,
\AtlasOrcid[0000-0002-0731-9562]{D.E.~Fellers}$^\textrm{\scriptsize 127}$,
\AtlasOrcid[0000-0001-9138-3200]{C.~Feng}$^\textrm{\scriptsize 64b}$,
\AtlasOrcid[0000-0001-5155-3420]{Z.~Feng}$^\textrm{\scriptsize 118}$,
\AtlasOrcid[0000-0003-1002-6880]{M.J.~Fenton}$^\textrm{\scriptsize 163}$,
\AtlasOrcid[0000-0001-5489-1759]{L.~Ferencz}$^\textrm{\scriptsize 50}$,
\AtlasOrcid[0000-0003-2352-7334]{R.A.M.~Ferguson}$^\textrm{\scriptsize 94}$,
\AtlasOrcid[0000-0003-0172-9373]{S.I.~Fernandez~Luengo}$^\textrm{\scriptsize 141f}$,
\AtlasOrcid[0000-0002-7818-6971]{P.~Fernandez~Martinez}$^\textrm{\scriptsize 13}$,
\AtlasOrcid[0000-0003-2372-1444]{M.J.V.~Fernoux}$^\textrm{\scriptsize 105}$,
\AtlasOrcid[0000-0002-1007-7816]{J.~Ferrando}$^\textrm{\scriptsize 94}$,
\AtlasOrcid[0000-0003-2887-5311]{A.~Ferrari}$^\textrm{\scriptsize 166}$,
\AtlasOrcid[0000-0002-1387-153X]{P.~Ferrari}$^\textrm{\scriptsize 118,117}$,
\AtlasOrcid[0000-0001-5566-1373]{R.~Ferrari}$^\textrm{\scriptsize 75a}$,
\AtlasOrcid[0000-0002-5687-9240]{D.~Ferrere}$^\textrm{\scriptsize 58}$,
\AtlasOrcid[0000-0002-5562-7893]{C.~Ferretti}$^\textrm{\scriptsize 109}$,
\AtlasOrcid[0000-0002-0678-1667]{D.~Fiacco}$^\textrm{\scriptsize 77a,77b}$,
\AtlasOrcid[0000-0002-4610-5612]{F.~Fiedler}$^\textrm{\scriptsize 103}$,
\AtlasOrcid[0000-0002-1217-4097]{P.~Fiedler}$^\textrm{\scriptsize 136}$,
\AtlasOrcid[0000-0001-5671-1555]{A.~Filip\v{c}i\v{c}}$^\textrm{\scriptsize 96}$,
\AtlasOrcid[0000-0001-6967-7325]{E.K.~Filmer}$^\textrm{\scriptsize 160a}$,
\AtlasOrcid[0000-0003-3338-2247]{F.~Filthaut}$^\textrm{\scriptsize 117}$,
\AtlasOrcid[0000-0001-9035-0335]{M.C.N.~Fiolhais}$^\textrm{\scriptsize 134a,134c,c}$,
\AtlasOrcid[0000-0002-5070-2735]{L.~Fiorini}$^\textrm{\scriptsize 168}$,
\AtlasOrcid[0000-0003-3043-3045]{W.C.~Fisher}$^\textrm{\scriptsize 110}$,
\AtlasOrcid[0000-0002-1152-7372]{T.~Fitschen}$^\textrm{\scriptsize 104}$,
\AtlasOrcid{P.M.~Fitzhugh}$^\textrm{\scriptsize 139}$,
\AtlasOrcid[0000-0003-1461-8648]{I.~Fleck}$^\textrm{\scriptsize 145}$,
\AtlasOrcid[0000-0001-6968-340X]{P.~Fleischmann}$^\textrm{\scriptsize 109}$,
\AtlasOrcid[0000-0002-8356-6987]{T.~Flick}$^\textrm{\scriptsize 176}$,
\AtlasOrcid[0000-0002-4462-2851]{M.~Flores}$^\textrm{\scriptsize 34d,y}$,
\AtlasOrcid[0000-0003-1551-5974]{L.R.~Flores~Castillo}$^\textrm{\scriptsize 66a}$,
\AtlasOrcid[0000-0002-4006-3597]{L.~Flores~Sanz~De~Acedo}$^\textrm{\scriptsize 37}$,
\AtlasOrcid[0000-0003-2317-9560]{F.M.~Follega}$^\textrm{\scriptsize 80a,80b}$,
\AtlasOrcid[0000-0001-9457-394X]{N.~Fomin}$^\textrm{\scriptsize 33}$,
\AtlasOrcid[0000-0003-4577-0685]{J.H.~Foo}$^\textrm{\scriptsize 159}$,
\AtlasOrcid[0000-0001-8308-2643]{A.~Formica}$^\textrm{\scriptsize 139}$,
\AtlasOrcid[0000-0002-0532-7921]{A.C.~Forti}$^\textrm{\scriptsize 104}$,
\AtlasOrcid[0000-0002-6418-9522]{E.~Fortin}$^\textrm{\scriptsize 37}$,
\AtlasOrcid[0000-0001-9454-9069]{A.W.~Fortman}$^\textrm{\scriptsize 18a}$,
\AtlasOrcid[0000-0002-0976-7246]{M.G.~Foti}$^\textrm{\scriptsize 18a}$,
\AtlasOrcid[0000-0002-9986-6597]{L.~Fountas}$^\textrm{\scriptsize 9,i}$,
\AtlasOrcid[0000-0003-4836-0358]{D.~Fournier}$^\textrm{\scriptsize 68}$,
\AtlasOrcid[0000-0003-3089-6090]{H.~Fox}$^\textrm{\scriptsize 94}$,
\AtlasOrcid[0000-0003-1164-6870]{P.~Francavilla}$^\textrm{\scriptsize 76a,76b}$,
\AtlasOrcid[0000-0001-5315-9275]{S.~Francescato}$^\textrm{\scriptsize 63}$,
\AtlasOrcid[0000-0003-0695-0798]{S.~Franchellucci}$^\textrm{\scriptsize 58}$,
\AtlasOrcid[0000-0002-4554-252X]{M.~Franchini}$^\textrm{\scriptsize 24b,24a}$,
\AtlasOrcid[0000-0002-8159-8010]{S.~Franchino}$^\textrm{\scriptsize 65a}$,
\AtlasOrcid{D.~Francis}$^\textrm{\scriptsize 37}$,
\AtlasOrcid[0000-0002-1687-4314]{L.~Franco}$^\textrm{\scriptsize 117}$,
\AtlasOrcid[0000-0002-3761-209X]{V.~Franco~Lima}$^\textrm{\scriptsize 37}$,
\AtlasOrcid[0000-0002-0647-6072]{L.~Franconi}$^\textrm{\scriptsize 50}$,
\AtlasOrcid[0000-0002-6595-883X]{M.~Franklin}$^\textrm{\scriptsize 63}$,
\AtlasOrcid[0000-0002-7829-6564]{G.~Frattari}$^\textrm{\scriptsize 27}$,
\AtlasOrcid[0000-0003-1565-1773]{Y.Y.~Frid}$^\textrm{\scriptsize 155}$,
\AtlasOrcid[0009-0001-8430-1454]{J.~Friend}$^\textrm{\scriptsize 61}$,
\AtlasOrcid[0000-0002-9350-1060]{N.~Fritzsche}$^\textrm{\scriptsize 37}$,
\AtlasOrcid[0000-0002-8259-2622]{A.~Froch}$^\textrm{\scriptsize 56}$,
\AtlasOrcid[0000-0003-3986-3922]{D.~Froidevaux}$^\textrm{\scriptsize 37}$,
\AtlasOrcid[0000-0003-3562-9944]{J.A.~Frost}$^\textrm{\scriptsize 130}$,
\AtlasOrcid[0000-0002-7370-7395]{Y.~Fu}$^\textrm{\scriptsize 64a}$,
\AtlasOrcid[0000-0002-7835-5157]{S.~Fuenzalida~Garrido}$^\textrm{\scriptsize 141f}$,
\AtlasOrcid[0000-0002-6701-8198]{M.~Fujimoto}$^\textrm{\scriptsize 105}$,
\AtlasOrcid[0000-0003-2131-2970]{K.Y.~Fung}$^\textrm{\scriptsize 66a}$,
\AtlasOrcid[0000-0001-8707-785X]{E.~Furtado~De~Simas~Filho}$^\textrm{\scriptsize 85e}$,
\AtlasOrcid[0000-0003-4888-2260]{M.~Furukawa}$^\textrm{\scriptsize 157}$,
\AtlasOrcid[0000-0002-1290-2031]{J.~Fuster}$^\textrm{\scriptsize 168}$,
\AtlasOrcid[0000-0003-4011-5550]{A.~Gaa}$^\textrm{\scriptsize 57}$,
\AtlasOrcid[0000-0001-5346-7841]{A.~Gabrielli}$^\textrm{\scriptsize 24b,24a}$,
\AtlasOrcid[0000-0003-0768-9325]{A.~Gabrielli}$^\textrm{\scriptsize 159}$,
\AtlasOrcid[0000-0003-4475-6734]{P.~Gadow}$^\textrm{\scriptsize 37}$,
\AtlasOrcid[0000-0002-3550-4124]{G.~Gagliardi}$^\textrm{\scriptsize 59b,59a}$,
\AtlasOrcid[0000-0003-3000-8479]{L.G.~Gagnon}$^\textrm{\scriptsize 18a}$,
\AtlasOrcid[0009-0001-6883-9166]{S.~Gaid}$^\textrm{\scriptsize 165}$,
\AtlasOrcid[0000-0001-5047-5889]{S.~Galantzan}$^\textrm{\scriptsize 155}$,
\AtlasOrcid[0000-0001-9284-6270]{J.~Gallagher}$^\textrm{\scriptsize 1}$,
\AtlasOrcid[0000-0002-1259-1034]{E.J.~Gallas}$^\textrm{\scriptsize 130}$,
\AtlasOrcid[0000-0001-7401-5043]{B.J.~Gallop}$^\textrm{\scriptsize 138}$,
\AtlasOrcid[0000-0002-1550-1487]{K.K.~Gan}$^\textrm{\scriptsize 123}$,
\AtlasOrcid[0000-0003-1285-9261]{S.~Ganguly}$^\textrm{\scriptsize 157}$,
\AtlasOrcid[0000-0001-6326-4773]{Y.~Gao}$^\textrm{\scriptsize 54}$,
\AtlasOrcid[0000-0002-6670-1104]{F.M.~Garay~Walls}$^\textrm{\scriptsize 141a,141b}$,
\AtlasOrcid{B.~Garcia}$^\textrm{\scriptsize 30}$,
\AtlasOrcid[0000-0003-1625-7452]{C.~Garc\'ia}$^\textrm{\scriptsize 168}$,
\AtlasOrcid[0000-0002-9566-7793]{A.~Garcia~Alonso}$^\textrm{\scriptsize 118}$,
\AtlasOrcid[0000-0001-9095-4710]{A.G.~Garcia~Caffaro}$^\textrm{\scriptsize 177}$,
\AtlasOrcid[0000-0002-0279-0523]{J.E.~Garc\'ia~Navarro}$^\textrm{\scriptsize 168}$,
\AtlasOrcid[0000-0002-5800-4210]{M.~Garcia-Sciveres}$^\textrm{\scriptsize 18a}$,
\AtlasOrcid[0000-0002-8980-3314]{G.L.~Gardner}$^\textrm{\scriptsize 132}$,
\AtlasOrcid[0000-0003-1433-9366]{R.W.~Gardner}$^\textrm{\scriptsize 41}$,
\AtlasOrcid[0000-0003-0534-9634]{N.~Garelli}$^\textrm{\scriptsize 162}$,
\AtlasOrcid[0000-0001-8383-9343]{D.~Garg}$^\textrm{\scriptsize 82}$,
\AtlasOrcid[0000-0002-2691-7963]{R.B.~Garg}$^\textrm{\scriptsize 147}$,
\AtlasOrcid{J.M.~Gargan}$^\textrm{\scriptsize 54}$,
\AtlasOrcid{C.A.~Garner}$^\textrm{\scriptsize 159}$,
\AtlasOrcid[0000-0001-8849-4970]{C.M.~Garvey}$^\textrm{\scriptsize 34a}$,
\AtlasOrcid{V.K.~Gassmann}$^\textrm{\scriptsize 162}$,
\AtlasOrcid[0000-0002-6833-0933]{G.~Gaudio}$^\textrm{\scriptsize 75a}$,
\AtlasOrcid{V.~Gautam}$^\textrm{\scriptsize 13}$,
\AtlasOrcid[0000-0003-4841-5822]{P.~Gauzzi}$^\textrm{\scriptsize 77a,77b}$,
\AtlasOrcid[0000-0002-8760-9518]{J.~Gavranovic}$^\textrm{\scriptsize 96}$,
\AtlasOrcid[0000-0001-7219-2636]{I.L.~Gavrilenko}$^\textrm{\scriptsize 38}$,
\AtlasOrcid[0000-0003-3837-6567]{A.~Gavrilyuk}$^\textrm{\scriptsize 38}$,
\AtlasOrcid[0000-0002-9354-9507]{C.~Gay}$^\textrm{\scriptsize 169}$,
\AtlasOrcid[0000-0002-2941-9257]{G.~Gaycken}$^\textrm{\scriptsize 127}$,
\AtlasOrcid[0000-0002-9272-4254]{E.N.~Gazis}$^\textrm{\scriptsize 10}$,
\AtlasOrcid[0000-0003-2781-2933]{A.A.~Geanta}$^\textrm{\scriptsize 28b}$,
\AtlasOrcid[0000-0002-3271-7861]{C.M.~Gee}$^\textrm{\scriptsize 140}$,
\AtlasOrcid{A.~Gekow}$^\textrm{\scriptsize 123}$,
\AtlasOrcid[0000-0002-1702-5699]{C.~Gemme}$^\textrm{\scriptsize 59b}$,
\AtlasOrcid[0000-0002-4098-2024]{M.H.~Genest}$^\textrm{\scriptsize 62}$,
\AtlasOrcid[0009-0003-8477-0095]{A.D.~Gentry}$^\textrm{\scriptsize 116}$,
\AtlasOrcid[0000-0003-3565-3290]{S.~George}$^\textrm{\scriptsize 98}$,
\AtlasOrcid[0000-0003-3674-7475]{W.F.~George}$^\textrm{\scriptsize 21}$,
\AtlasOrcid[0000-0001-7188-979X]{T.~Geralis}$^\textrm{\scriptsize 48}$,
\AtlasOrcid[0000-0002-3056-7417]{P.~Gessinger-Befurt}$^\textrm{\scriptsize 37}$,
\AtlasOrcid[0000-0002-7491-0838]{M.E.~Geyik}$^\textrm{\scriptsize 176}$,
\AtlasOrcid[0000-0002-4123-508X]{M.~Ghani}$^\textrm{\scriptsize 172}$,
\AtlasOrcid[0000-0002-7985-9445]{K.~Ghorbanian}$^\textrm{\scriptsize 97}$,
\AtlasOrcid[0000-0003-0661-9288]{A.~Ghosal}$^\textrm{\scriptsize 145}$,
\AtlasOrcid[0000-0003-0819-1553]{A.~Ghosh}$^\textrm{\scriptsize 163}$,
\AtlasOrcid[0000-0002-5716-356X]{A.~Ghosh}$^\textrm{\scriptsize 7}$,
\AtlasOrcid[0000-0003-2987-7642]{B.~Giacobbe}$^\textrm{\scriptsize 24b}$,
\AtlasOrcid[0000-0001-9192-3537]{S.~Giagu}$^\textrm{\scriptsize 77a,77b}$,
\AtlasOrcid[0000-0001-7135-6731]{T.~Giani}$^\textrm{\scriptsize 118}$,
\AtlasOrcid[0000-0002-5683-814X]{A.~Giannini}$^\textrm{\scriptsize 64a}$,
\AtlasOrcid[0000-0002-1236-9249]{S.M.~Gibson}$^\textrm{\scriptsize 98}$,
\AtlasOrcid[0000-0003-4155-7844]{M.~Gignac}$^\textrm{\scriptsize 140}$,
\AtlasOrcid[0000-0001-9021-8836]{D.T.~Gil}$^\textrm{\scriptsize 88b}$,
\AtlasOrcid[0000-0002-8813-4446]{A.K.~Gilbert}$^\textrm{\scriptsize 88a}$,
\AtlasOrcid[0000-0003-0731-710X]{B.J.~Gilbert}$^\textrm{\scriptsize 43}$,
\AtlasOrcid[0000-0003-0341-0171]{D.~Gillberg}$^\textrm{\scriptsize 35}$,
\AtlasOrcid[0000-0001-8451-4604]{G.~Gilles}$^\textrm{\scriptsize 118}$,
\AtlasOrcid[0000-0002-7834-8117]{L.~Ginabat}$^\textrm{\scriptsize 131}$,
\AtlasOrcid[0000-0002-2552-1449]{D.M.~Gingrich}$^\textrm{\scriptsize 2,ab}$,
\AtlasOrcid[0000-0002-0792-6039]{M.P.~Giordani}$^\textrm{\scriptsize 71a,71c}$,
\AtlasOrcid[0000-0002-8485-9351]{P.F.~Giraud}$^\textrm{\scriptsize 139}$,
\AtlasOrcid[0000-0001-5765-1750]{G.~Giugliarelli}$^\textrm{\scriptsize 71a,71c}$,
\AtlasOrcid[0000-0002-6976-0951]{D.~Giugni}$^\textrm{\scriptsize 73a}$,
\AtlasOrcid[0000-0002-8506-274X]{F.~Giuli}$^\textrm{\scriptsize 78a,78b}$,
\AtlasOrcid[0000-0002-8402-723X]{I.~Gkialas}$^\textrm{\scriptsize 9,i}$,
\AtlasOrcid[0000-0001-9422-8636]{L.K.~Gladilin}$^\textrm{\scriptsize 38}$,
\AtlasOrcid[0000-0003-2025-3817]{C.~Glasman}$^\textrm{\scriptsize 102}$,
\AtlasOrcid[0000-0001-7701-5030]{G.R.~Gledhill}$^\textrm{\scriptsize 127}$,
\AtlasOrcid[0000-0003-4977-5256]{G.~Glem\v{z}a}$^\textrm{\scriptsize 50}$,
\AtlasOrcid{M.~Glisic}$^\textrm{\scriptsize 127}$,
\AtlasOrcid[0000-0002-0772-7312]{I.~Gnesi}$^\textrm{\scriptsize 45b}$,
\AtlasOrcid[0000-0003-1253-1223]{Y.~Go}$^\textrm{\scriptsize 30}$,
\AtlasOrcid[0000-0002-2785-9654]{M.~Goblirsch-Kolb}$^\textrm{\scriptsize 37}$,
\AtlasOrcid[0000-0001-8074-2538]{B.~Gocke}$^\textrm{\scriptsize 51}$,
\AtlasOrcid{D.~Godin}$^\textrm{\scriptsize 111}$,
\AtlasOrcid[0000-0002-6045-8617]{B.~Gokturk}$^\textrm{\scriptsize 22a}$,
\AtlasOrcid[0000-0002-1677-3097]{S.~Goldfarb}$^\textrm{\scriptsize 108}$,
\AtlasOrcid[0000-0001-8535-6687]{T.~Golling}$^\textrm{\scriptsize 58}$,
\AtlasOrcid[0000-0002-0689-5402]{M.G.D.~Gololo}$^\textrm{\scriptsize 34g}$,
\AtlasOrcid[0000-0002-5521-9793]{D.~Golubkov}$^\textrm{\scriptsize 38}$,
\AtlasOrcid[0000-0002-8285-3570]{J.P.~Gombas}$^\textrm{\scriptsize 110}$,
\AtlasOrcid[0000-0002-5940-9893]{A.~Gomes}$^\textrm{\scriptsize 134a,134b}$,
\AtlasOrcid[0000-0002-3552-1266]{G.~Gomes~Da~Silva}$^\textrm{\scriptsize 145}$,
\AtlasOrcid[0000-0003-4315-2621]{A.J.~Gomez~Delegido}$^\textrm{\scriptsize 168}$,
\AtlasOrcid[0000-0002-3826-3442]{R.~Gon\c{c}alo}$^\textrm{\scriptsize 134a}$,
\AtlasOrcid[0000-0002-4919-0808]{L.~Gonella}$^\textrm{\scriptsize 21}$,
\AtlasOrcid[0000-0001-8183-1612]{A.~Gongadze}$^\textrm{\scriptsize 153c}$,
\AtlasOrcid[0000-0003-0885-1654]{F.~Gonnella}$^\textrm{\scriptsize 21}$,
\AtlasOrcid[0000-0003-2037-6315]{J.L.~Gonski}$^\textrm{\scriptsize 147}$,
\AtlasOrcid[0000-0002-0700-1757]{R.Y.~Gonz\'alez~Andana}$^\textrm{\scriptsize 54}$,
\AtlasOrcid[0000-0001-5304-5390]{S.~Gonz\'alez~de~la~Hoz}$^\textrm{\scriptsize 168}$,
\AtlasOrcid[0000-0003-2302-8754]{R.~Gonzalez~Lopez}$^\textrm{\scriptsize 95}$,
\AtlasOrcid[0000-0003-0079-8924]{C.~Gonzalez~Renteria}$^\textrm{\scriptsize 18a}$,
\AtlasOrcid[0000-0002-7906-8088]{M.V.~Gonzalez~Rodrigues}$^\textrm{\scriptsize 50}$,
\AtlasOrcid[0000-0002-6126-7230]{R.~Gonzalez~Suarez}$^\textrm{\scriptsize 166}$,
\AtlasOrcid[0000-0003-4458-9403]{S.~Gonzalez-Sevilla}$^\textrm{\scriptsize 58}$,
\AtlasOrcid[0000-0002-2536-4498]{L.~Goossens}$^\textrm{\scriptsize 37}$,
\AtlasOrcid[0000-0003-4177-9666]{B.~Gorini}$^\textrm{\scriptsize 37}$,
\AtlasOrcid[0000-0002-7688-2797]{E.~Gorini}$^\textrm{\scriptsize 72a,72b}$,
\AtlasOrcid[0000-0002-3903-3438]{A.~Gori\v{s}ek}$^\textrm{\scriptsize 96}$,
\AtlasOrcid[0000-0002-8867-2551]{T.C.~Gosart}$^\textrm{\scriptsize 132}$,
\AtlasOrcid[0000-0002-5704-0885]{A.T.~Goshaw}$^\textrm{\scriptsize 53}$,
\AtlasOrcid[0000-0002-4311-3756]{M.I.~Gostkin}$^\textrm{\scriptsize 40}$,
\AtlasOrcid[0000-0001-9566-4640]{S.~Goswami}$^\textrm{\scriptsize 125}$,
\AtlasOrcid[0000-0003-0348-0364]{C.A.~Gottardo}$^\textrm{\scriptsize 37}$,
\AtlasOrcid[0000-0002-7518-7055]{S.A.~Gotz}$^\textrm{\scriptsize 112}$,
\AtlasOrcid[0000-0002-9551-0251]{M.~Gouighri}$^\textrm{\scriptsize 36b}$,
\AtlasOrcid[0000-0002-1294-9091]{V.~Goumarre}$^\textrm{\scriptsize 50}$,
\AtlasOrcid[0000-0001-6211-7122]{A.G.~Goussiou}$^\textrm{\scriptsize 142}$,
\AtlasOrcid[0000-0002-5068-5429]{N.~Govender}$^\textrm{\scriptsize 34c}$,
\AtlasOrcid[0009-0007-1845-0762]{R.P.~Grabarczyk}$^\textrm{\scriptsize 130}$,
\AtlasOrcid[0000-0001-9159-1210]{I.~Grabowska-Bold}$^\textrm{\scriptsize 88a}$,
\AtlasOrcid[0000-0002-5832-8653]{K.~Graham}$^\textrm{\scriptsize 35}$,
\AtlasOrcid[0000-0001-5792-5352]{E.~Gramstad}$^\textrm{\scriptsize 129}$,
\AtlasOrcid[0000-0001-8490-8304]{S.~Grancagnolo}$^\textrm{\scriptsize 72a,72b}$,
\AtlasOrcid{C.M.~Grant}$^\textrm{\scriptsize 1,139}$,
\AtlasOrcid[0000-0002-0154-577X]{P.M.~Gravila}$^\textrm{\scriptsize 28f}$,
\AtlasOrcid[0000-0003-2422-5960]{F.G.~Gravili}$^\textrm{\scriptsize 72a,72b}$,
\AtlasOrcid[0000-0002-5293-4716]{H.M.~Gray}$^\textrm{\scriptsize 18a}$,
\AtlasOrcid[0000-0001-8687-7273]{M.~Greco}$^\textrm{\scriptsize 72a,72b}$,
\AtlasOrcid[0000-0003-4402-7160]{M.J.~Green}$^\textrm{\scriptsize 1}$,
\AtlasOrcid[0000-0001-7050-5301]{C.~Grefe}$^\textrm{\scriptsize 25}$,
\AtlasOrcid[0009-0005-9063-4131]{A.S.~Grefsrud}$^\textrm{\scriptsize 17}$,
\AtlasOrcid[0000-0002-5976-7818]{I.M.~Gregor}$^\textrm{\scriptsize 50}$,
\AtlasOrcid[0000-0001-6607-0595]{K.T.~Greif}$^\textrm{\scriptsize 163}$,
\AtlasOrcid[0000-0002-9926-5417]{P.~Grenier}$^\textrm{\scriptsize 147}$,
\AtlasOrcid{S.G.~Grewe}$^\textrm{\scriptsize 113}$,
\AtlasOrcid[0000-0003-2950-1872]{A.A.~Grillo}$^\textrm{\scriptsize 140}$,
\AtlasOrcid[0000-0001-6587-7397]{K.~Grimm}$^\textrm{\scriptsize 32}$,
\AtlasOrcid[0000-0002-6460-8694]{S.~Grinstein}$^\textrm{\scriptsize 13,s}$,
\AtlasOrcid[0000-0003-4793-7995]{J.-F.~Grivaz}$^\textrm{\scriptsize 68}$,
\AtlasOrcid[0000-0003-1244-9350]{E.~Gross}$^\textrm{\scriptsize 174}$,
\AtlasOrcid[0000-0003-3085-7067]{J.~Grosse-Knetter}$^\textrm{\scriptsize 57}$,
\AtlasOrcid[0000-0003-1897-1617]{L.~Guan}$^\textrm{\scriptsize 109}$,
\AtlasOrcid[0000-0001-8487-3594]{J.G.R.~Guerrero~Rojas}$^\textrm{\scriptsize 168}$,
\AtlasOrcid[0000-0002-3403-1177]{G.~Guerrieri}$^\textrm{\scriptsize 37}$,
\AtlasOrcid[0000-0002-3349-1163]{R.~Gugel}$^\textrm{\scriptsize 103}$,
\AtlasOrcid[0000-0002-9802-0901]{J.A.M.~Guhit}$^\textrm{\scriptsize 109}$,
\AtlasOrcid[0000-0001-9021-9038]{A.~Guida}$^\textrm{\scriptsize 19}$,
\AtlasOrcid[0000-0003-4814-6693]{E.~Guilloton}$^\textrm{\scriptsize 172}$,
\AtlasOrcid[0000-0001-7595-3859]{S.~Guindon}$^\textrm{\scriptsize 37}$,
\AtlasOrcid[0000-0002-3864-9257]{F.~Guo}$^\textrm{\scriptsize 14,115c}$,
\AtlasOrcid[0000-0001-8125-9433]{J.~Guo}$^\textrm{\scriptsize 64c}$,
\AtlasOrcid[0000-0002-6785-9202]{L.~Guo}$^\textrm{\scriptsize 50}$,
\AtlasOrcid[0009-0006-9125-5210]{L.~Guo}$^\textrm{\scriptsize 14}$,
\AtlasOrcid[0000-0002-6027-5132]{Y.~Guo}$^\textrm{\scriptsize 109}$,
\AtlasOrcid[0009-0003-7307-9741]{A.~Gupta}$^\textrm{\scriptsize 51}$,
\AtlasOrcid[0000-0002-8508-8405]{R.~Gupta}$^\textrm{\scriptsize 133}$,
\AtlasOrcid[0000-0002-9152-1455]{S.~Gurbuz}$^\textrm{\scriptsize 25}$,
\AtlasOrcid[0000-0002-8836-0099]{S.S.~Gurdasani}$^\textrm{\scriptsize 56}$,
\AtlasOrcid[0000-0002-5938-4921]{G.~Gustavino}$^\textrm{\scriptsize 77a,77b}$,
\AtlasOrcid[0000-0003-2326-3877]{P.~Gutierrez}$^\textrm{\scriptsize 124}$,
\AtlasOrcid[0000-0003-0374-1595]{L.F.~Gutierrez~Zagazeta}$^\textrm{\scriptsize 132}$,
\AtlasOrcid[0000-0002-0947-7062]{M.~Gutsche}$^\textrm{\scriptsize 52}$,
\AtlasOrcid[0000-0003-0857-794X]{C.~Gutschow}$^\textrm{\scriptsize 99}$,
\AtlasOrcid[0000-0002-3518-0617]{C.~Gwenlan}$^\textrm{\scriptsize 130}$,
\AtlasOrcid[0000-0002-9401-5304]{C.B.~Gwilliam}$^\textrm{\scriptsize 95}$,
\AtlasOrcid[0000-0002-3676-493X]{E.S.~Haaland}$^\textrm{\scriptsize 129}$,
\AtlasOrcid[0000-0002-4832-0455]{A.~Haas}$^\textrm{\scriptsize 121}$,
\AtlasOrcid[0000-0002-7412-9355]{M.~Habedank}$^\textrm{\scriptsize 61}$,
\AtlasOrcid[0000-0002-0155-1360]{C.~Haber}$^\textrm{\scriptsize 18a}$,
\AtlasOrcid[0000-0001-5447-3346]{H.K.~Hadavand}$^\textrm{\scriptsize 8}$,
\AtlasOrcid[0000-0003-2508-0628]{A.~Hadef}$^\textrm{\scriptsize 52}$,
\AtlasOrcid[0000-0002-2079-4739]{A.I.~Hagan}$^\textrm{\scriptsize 94}$,
\AtlasOrcid[0000-0002-1677-4735]{J.J.~Hahn}$^\textrm{\scriptsize 145}$,
\AtlasOrcid[0000-0002-5417-2081]{E.H.~Haines}$^\textrm{\scriptsize 99}$,
\AtlasOrcid[0000-0003-3826-6333]{M.~Haleem}$^\textrm{\scriptsize 171}$,
\AtlasOrcid[0000-0002-6938-7405]{J.~Haley}$^\textrm{\scriptsize 125}$,
\AtlasOrcid[0000-0001-6267-8560]{G.D.~Hallewell}$^\textrm{\scriptsize 105}$,
\AtlasOrcid[0000-0002-0759-7247]{L.~Halser}$^\textrm{\scriptsize 20}$,
\AtlasOrcid[0000-0002-9438-8020]{K.~Hamano}$^\textrm{\scriptsize 170}$,
\AtlasOrcid[0000-0003-1550-2030]{M.~Hamer}$^\textrm{\scriptsize 25}$,
\AtlasOrcid[0000-0001-7988-4504]{E.J.~Hampshire}$^\textrm{\scriptsize 98}$,
\AtlasOrcid[0000-0002-1008-0943]{J.~Han}$^\textrm{\scriptsize 64b}$,
\AtlasOrcid[0000-0003-3321-8412]{L.~Han}$^\textrm{\scriptsize 115a}$,
\AtlasOrcid[0000-0002-6353-9711]{L.~Han}$^\textrm{\scriptsize 64a}$,
\AtlasOrcid[0000-0001-8383-7348]{S.~Han}$^\textrm{\scriptsize 18a}$,
\AtlasOrcid[0000-0002-7084-8424]{Y.F.~Han}$^\textrm{\scriptsize 159}$,
\AtlasOrcid[0000-0003-0676-0441]{K.~Hanagaki}$^\textrm{\scriptsize 86}$,
\AtlasOrcid[0000-0001-8392-0934]{M.~Hance}$^\textrm{\scriptsize 140}$,
\AtlasOrcid[0000-0002-3826-7232]{D.A.~Hangal}$^\textrm{\scriptsize 43}$,
\AtlasOrcid[0000-0002-0984-7887]{H.~Hanif}$^\textrm{\scriptsize 146}$,
\AtlasOrcid[0000-0002-4731-6120]{M.D.~Hank}$^\textrm{\scriptsize 132}$,
\AtlasOrcid[0000-0002-3684-8340]{J.B.~Hansen}$^\textrm{\scriptsize 44}$,
\AtlasOrcid[0000-0002-6764-4789]{P.H.~Hansen}$^\textrm{\scriptsize 44}$,
\AtlasOrcid[0000-0002-0792-0569]{D.~Harada}$^\textrm{\scriptsize 58}$,
\AtlasOrcid[0000-0001-8682-3734]{T.~Harenberg}$^\textrm{\scriptsize 176}$,
\AtlasOrcid[0000-0002-0309-4490]{S.~Harkusha}$^\textrm{\scriptsize 178}$,
\AtlasOrcid[0009-0001-8882-5976]{M.L.~Harris}$^\textrm{\scriptsize 106}$,
\AtlasOrcid[0000-0001-5816-2158]{Y.T.~Harris}$^\textrm{\scriptsize 25}$,
\AtlasOrcid[0000-0003-2576-080X]{J.~Harrison}$^\textrm{\scriptsize 13}$,
\AtlasOrcid[0000-0002-7461-8351]{N.M.~Harrison}$^\textrm{\scriptsize 123}$,
\AtlasOrcid{P.F.~Harrison}$^\textrm{\scriptsize 172}$,
\AtlasOrcid[0000-0001-9111-4916]{N.M.~Hartman}$^\textrm{\scriptsize 113}$,
\AtlasOrcid[0000-0003-0047-2908]{N.M.~Hartmann}$^\textrm{\scriptsize 112}$,
\AtlasOrcid[0009-0009-5896-9141]{R.Z.~Hasan}$^\textrm{\scriptsize 98,138}$,
\AtlasOrcid[0000-0003-2683-7389]{Y.~Hasegawa}$^\textrm{\scriptsize 144}$,
\AtlasOrcid[0000-0002-1804-5747]{F.~Haslbeck}$^\textrm{\scriptsize 130}$,
\AtlasOrcid[0000-0002-5027-4320]{S.~Hassan}$^\textrm{\scriptsize 17}$,
\AtlasOrcid[0000-0001-7682-8857]{R.~Hauser}$^\textrm{\scriptsize 110}$,
\AtlasOrcid[0000-0001-9167-0592]{C.M.~Hawkes}$^\textrm{\scriptsize 21}$,
\AtlasOrcid[0000-0001-9719-0290]{R.J.~Hawkings}$^\textrm{\scriptsize 37}$,
\AtlasOrcid[0000-0002-1222-4672]{Y.~Hayashi}$^\textrm{\scriptsize 157}$,
\AtlasOrcid[0000-0001-5220-2972]{D.~Hayden}$^\textrm{\scriptsize 110}$,
\AtlasOrcid[0000-0002-0298-0351]{C.~Hayes}$^\textrm{\scriptsize 109}$,
\AtlasOrcid[0000-0001-7752-9285]{R.L.~Hayes}$^\textrm{\scriptsize 118}$,
\AtlasOrcid[0000-0003-2371-9723]{C.P.~Hays}$^\textrm{\scriptsize 130}$,
\AtlasOrcid[0000-0003-1554-5401]{J.M.~Hays}$^\textrm{\scriptsize 97}$,
\AtlasOrcid[0000-0002-0972-3411]{H.S.~Hayward}$^\textrm{\scriptsize 95}$,
\AtlasOrcid[0000-0003-3733-4058]{F.~He}$^\textrm{\scriptsize 64a}$,
\AtlasOrcid[0000-0003-0514-2115]{M.~He}$^\textrm{\scriptsize 14,115c}$,
\AtlasOrcid[0000-0001-8068-5596]{Y.~He}$^\textrm{\scriptsize 50}$,
\AtlasOrcid[0009-0005-3061-4294]{Y.~He}$^\textrm{\scriptsize 99}$,
\AtlasOrcid[0000-0003-2204-4779]{N.B.~Heatley}$^\textrm{\scriptsize 97}$,
\AtlasOrcid[0000-0002-4596-3965]{V.~Hedberg}$^\textrm{\scriptsize 101}$,
\AtlasOrcid[0000-0002-7736-2806]{A.L.~Heggelund}$^\textrm{\scriptsize 129}$,
\AtlasOrcid[0000-0003-0466-4472]{N.D.~Hehir}$^\textrm{\scriptsize 97,*}$,
\AtlasOrcid[0000-0001-8821-1205]{C.~Heidegger}$^\textrm{\scriptsize 56}$,
\AtlasOrcid[0000-0003-3113-0484]{K.K.~Heidegger}$^\textrm{\scriptsize 56}$,
\AtlasOrcid[0000-0001-6792-2294]{J.~Heilman}$^\textrm{\scriptsize 35}$,
\AtlasOrcid[0000-0002-2639-6571]{S.~Heim}$^\textrm{\scriptsize 50}$,
\AtlasOrcid[0000-0002-7669-5318]{T.~Heim}$^\textrm{\scriptsize 18a}$,
\AtlasOrcid[0000-0001-6878-9405]{J.G.~Heinlein}$^\textrm{\scriptsize 132}$,
\AtlasOrcid[0000-0002-0253-0924]{J.J.~Heinrich}$^\textrm{\scriptsize 127}$,
\AtlasOrcid[0000-0002-4048-7584]{L.~Heinrich}$^\textrm{\scriptsize 113,z}$,
\AtlasOrcid[0000-0002-4600-3659]{J.~Hejbal}$^\textrm{\scriptsize 135}$,
\AtlasOrcid[0000-0002-8924-5885]{A.~Held}$^\textrm{\scriptsize 175}$,
\AtlasOrcid[0000-0002-4424-4643]{S.~Hellesund}$^\textrm{\scriptsize 17}$,
\AtlasOrcid[0000-0002-2657-7532]{C.M.~Helling}$^\textrm{\scriptsize 169}$,
\AtlasOrcid[0000-0002-5415-1600]{S.~Hellman}$^\textrm{\scriptsize 49a,49b}$,
\AtlasOrcid{R.C.W.~Henderson}$^\textrm{\scriptsize 94}$,
\AtlasOrcid[0000-0001-8231-2080]{L.~Henkelmann}$^\textrm{\scriptsize 33}$,
\AtlasOrcid{A.M.~Henriques~Correia}$^\textrm{\scriptsize 37}$,
\AtlasOrcid[0000-0001-8926-6734]{H.~Herde}$^\textrm{\scriptsize 101}$,
\AtlasOrcid[0000-0001-9844-6200]{Y.~Hern\'andez~Jim\'enez}$^\textrm{\scriptsize 149}$,
\AtlasOrcid[0000-0002-8794-0948]{L.M.~Herrmann}$^\textrm{\scriptsize 25}$,
\AtlasOrcid[0000-0002-1478-3152]{T.~Herrmann}$^\textrm{\scriptsize 52}$,
\AtlasOrcid[0000-0001-7661-5122]{G.~Herten}$^\textrm{\scriptsize 56}$,
\AtlasOrcid[0000-0002-2646-5805]{R.~Hertenberger}$^\textrm{\scriptsize 112}$,
\AtlasOrcid[0000-0002-0778-2717]{L.~Hervas}$^\textrm{\scriptsize 37}$,
\AtlasOrcid[0000-0002-2447-904X]{M.E.~Hesping}$^\textrm{\scriptsize 103}$,
\AtlasOrcid[0000-0002-6698-9937]{N.P.~Hessey}$^\textrm{\scriptsize 160a}$,
\AtlasOrcid[0000-0002-4834-4596]{J.~Hessler}$^\textrm{\scriptsize 113}$,
\AtlasOrcid[0000-0003-2025-6495]{M.~Hidaoui}$^\textrm{\scriptsize 36b}$,
\AtlasOrcid[0000-0003-4695-2798]{N.~Hidic}$^\textrm{\scriptsize 137}$,
\AtlasOrcid[0000-0002-1725-7414]{E.~Hill}$^\textrm{\scriptsize 159}$,
\AtlasOrcid[0000-0002-7599-6469]{S.J.~Hillier}$^\textrm{\scriptsize 21}$,
\AtlasOrcid[0000-0001-7844-8815]{J.R.~Hinds}$^\textrm{\scriptsize 110}$,
\AtlasOrcid[0000-0002-0556-189X]{F.~Hinterkeuser}$^\textrm{\scriptsize 25}$,
\AtlasOrcid[0000-0003-4988-9149]{M.~Hirose}$^\textrm{\scriptsize 128}$,
\AtlasOrcid[0000-0002-2389-1286]{S.~Hirose}$^\textrm{\scriptsize 161}$,
\AtlasOrcid[0000-0002-7998-8925]{D.~Hirschbuehl}$^\textrm{\scriptsize 176}$,
\AtlasOrcid[0000-0001-8978-7118]{T.G.~Hitchings}$^\textrm{\scriptsize 104}$,
\AtlasOrcid[0000-0002-8668-6933]{B.~Hiti}$^\textrm{\scriptsize 96}$,
\AtlasOrcid[0000-0001-5404-7857]{J.~Hobbs}$^\textrm{\scriptsize 149}$,
\AtlasOrcid[0000-0001-7602-5771]{R.~Hobincu}$^\textrm{\scriptsize 28e}$,
\AtlasOrcid[0000-0001-5241-0544]{N.~Hod}$^\textrm{\scriptsize 174}$,
\AtlasOrcid[0000-0002-1040-1241]{M.C.~Hodgkinson}$^\textrm{\scriptsize 143}$,
\AtlasOrcid[0000-0002-2244-189X]{B.H.~Hodkinson}$^\textrm{\scriptsize 130}$,
\AtlasOrcid[0000-0002-6596-9395]{A.~Hoecker}$^\textrm{\scriptsize 37}$,
\AtlasOrcid[0000-0003-0028-6486]{D.D.~Hofer}$^\textrm{\scriptsize 109}$,
\AtlasOrcid[0000-0003-2799-5020]{J.~Hofer}$^\textrm{\scriptsize 168}$,
\AtlasOrcid[0000-0001-5407-7247]{T.~Holm}$^\textrm{\scriptsize 25}$,
\AtlasOrcid[0000-0001-8018-4185]{M.~Holzbock}$^\textrm{\scriptsize 37}$,
\AtlasOrcid[0000-0003-0684-600X]{L.B.A.H.~Hommels}$^\textrm{\scriptsize 33}$,
\AtlasOrcid[0000-0002-2698-4787]{B.P.~Honan}$^\textrm{\scriptsize 104}$,
\AtlasOrcid[0000-0002-1685-8090]{J.J.~Hong}$^\textrm{\scriptsize 70}$,
\AtlasOrcid[0000-0002-7494-5504]{J.~Hong}$^\textrm{\scriptsize 64c}$,
\AtlasOrcid[0000-0001-7834-328X]{T.M.~Hong}$^\textrm{\scriptsize 133}$,
\AtlasOrcid[0000-0002-4090-6099]{B.H.~Hooberman}$^\textrm{\scriptsize 167}$,
\AtlasOrcid[0000-0001-7814-8740]{W.H.~Hopkins}$^\textrm{\scriptsize 6}$,
\AtlasOrcid[0000-0002-7773-3654]{M.C.~Hoppesch}$^\textrm{\scriptsize 167}$,
\AtlasOrcid[0000-0003-0457-3052]{Y.~Horii}$^\textrm{\scriptsize 114}$,
\AtlasOrcid[0000-0002-4359-6364]{M.E.~Horstmann}$^\textrm{\scriptsize 113}$,
\AtlasOrcid[0000-0001-9861-151X]{S.~Hou}$^\textrm{\scriptsize 152}$,
\AtlasOrcid{M.R.~Housenga}$^\textrm{\scriptsize 167}$,
\AtlasOrcid[0000-0003-0625-8996]{A.S.~Howard}$^\textrm{\scriptsize 96}$,
\AtlasOrcid[0000-0002-0560-8985]{J.~Howarth}$^\textrm{\scriptsize 61}$,
\AtlasOrcid[0000-0002-7562-0234]{J.~Hoya}$^\textrm{\scriptsize 6}$,
\AtlasOrcid[0000-0003-4223-7316]{M.~Hrabovsky}$^\textrm{\scriptsize 126}$,
\AtlasOrcid[0000-0002-5411-114X]{A.~Hrynevich}$^\textrm{\scriptsize 50}$,
\AtlasOrcid[0000-0001-5914-8614]{T.~Hryn'ova}$^\textrm{\scriptsize 4}$,
\AtlasOrcid[0000-0003-3895-8356]{P.J.~Hsu}$^\textrm{\scriptsize 67}$,
\AtlasOrcid[0000-0001-6214-8500]{S.-C.~Hsu}$^\textrm{\scriptsize 142}$,
\AtlasOrcid[0000-0001-9157-295X]{T.~Hsu}$^\textrm{\scriptsize 68}$,
\AtlasOrcid[0000-0003-2858-6931]{M.~Hu}$^\textrm{\scriptsize 18a}$,
\AtlasOrcid[0000-0002-9705-7518]{Q.~Hu}$^\textrm{\scriptsize 64a}$,
\AtlasOrcid[0000-0002-1177-6758]{S.~Huang}$^\textrm{\scriptsize 33}$,
\AtlasOrcid[0009-0004-1494-0543]{X.~Huang}$^\textrm{\scriptsize 14,115c}$,
\AtlasOrcid[0000-0003-1826-2749]{Y.~Huang}$^\textrm{\scriptsize 143}$,
\AtlasOrcid[0000-0002-1499-6051]{Y.~Huang}$^\textrm{\scriptsize 103}$,
\AtlasOrcid[0000-0002-5972-2855]{Y.~Huang}$^\textrm{\scriptsize 14}$,
\AtlasOrcid[0000-0002-9008-1937]{Z.~Huang}$^\textrm{\scriptsize 104}$,
\AtlasOrcid[0000-0003-3250-9066]{Z.~Hubacek}$^\textrm{\scriptsize 136}$,
\AtlasOrcid[0000-0002-1162-8763]{M.~Huebner}$^\textrm{\scriptsize 25}$,
\AtlasOrcid[0000-0002-7472-3151]{F.~Huegging}$^\textrm{\scriptsize 25}$,
\AtlasOrcid[0000-0002-5332-2738]{T.B.~Huffman}$^\textrm{\scriptsize 130}$,
\AtlasOrcid[0000-0002-3654-5614]{C.A.~Hugli}$^\textrm{\scriptsize 50}$,
\AtlasOrcid[0000-0002-1752-3583]{M.~Huhtinen}$^\textrm{\scriptsize 37}$,
\AtlasOrcid[0000-0002-3277-7418]{S.K.~Huiberts}$^\textrm{\scriptsize 17}$,
\AtlasOrcid[0000-0002-0095-1290]{R.~Hulsken}$^\textrm{\scriptsize 107}$,
\AtlasOrcid[0000-0003-2201-5572]{N.~Huseynov}$^\textrm{\scriptsize 12,f}$,
\AtlasOrcid[0000-0001-9097-3014]{J.~Huston}$^\textrm{\scriptsize 110}$,
\AtlasOrcid[0000-0002-6867-2538]{J.~Huth}$^\textrm{\scriptsize 63}$,
\AtlasOrcid[0000-0002-9093-7141]{R.~Hyneman}$^\textrm{\scriptsize 147}$,
\AtlasOrcid[0000-0001-9965-5442]{G.~Iacobucci}$^\textrm{\scriptsize 58}$,
\AtlasOrcid[0000-0002-0330-5921]{G.~Iakovidis}$^\textrm{\scriptsize 30}$,
\AtlasOrcid[0000-0001-6334-6648]{L.~Iconomidou-Fayard}$^\textrm{\scriptsize 68}$,
\AtlasOrcid[0000-0002-2851-5554]{J.P.~Iddon}$^\textrm{\scriptsize 37}$,
\AtlasOrcid[0000-0002-5035-1242]{P.~Iengo}$^\textrm{\scriptsize 74a,74b}$,
\AtlasOrcid[0000-0002-0940-244X]{R.~Iguchi}$^\textrm{\scriptsize 157}$,
\AtlasOrcid[0000-0002-8297-5930]{Y.~Iiyama}$^\textrm{\scriptsize 157}$,
\AtlasOrcid[0000-0001-5312-4865]{T.~Iizawa}$^\textrm{\scriptsize 130}$,
\AtlasOrcid[0000-0001-7287-6579]{Y.~Ikegami}$^\textrm{\scriptsize 86}$,
\AtlasOrcid[0000-0003-0105-7634]{N.~Ilic}$^\textrm{\scriptsize 159}$,
\AtlasOrcid[0000-0002-7854-3174]{H.~Imam}$^\textrm{\scriptsize 85c}$,
\AtlasOrcid[0000-0002-6807-3172]{G.~Inacio~Goncalves}$^\textrm{\scriptsize 85d}$,
\AtlasOrcid[0000-0002-3699-8517]{T.~Ingebretsen~Carlson}$^\textrm{\scriptsize 49a,49b}$,
\AtlasOrcid[0000-0002-9130-4792]{J.M.~Inglis}$^\textrm{\scriptsize 97}$,
\AtlasOrcid[0000-0002-1314-2580]{G.~Introzzi}$^\textrm{\scriptsize 75a,75b}$,
\AtlasOrcid[0000-0003-4446-8150]{M.~Iodice}$^\textrm{\scriptsize 79a}$,
\AtlasOrcid[0000-0001-5126-1620]{V.~Ippolito}$^\textrm{\scriptsize 77a,77b}$,
\AtlasOrcid[0000-0001-6067-104X]{R.K.~Irwin}$^\textrm{\scriptsize 95}$,
\AtlasOrcid[0000-0002-7185-1334]{M.~Ishino}$^\textrm{\scriptsize 157}$,
\AtlasOrcid[0000-0002-5624-5934]{W.~Islam}$^\textrm{\scriptsize 175}$,
\AtlasOrcid[0000-0001-8259-1067]{C.~Issever}$^\textrm{\scriptsize 19}$,
\AtlasOrcid[0000-0001-8504-6291]{S.~Istin}$^\textrm{\scriptsize 22a,af}$,
\AtlasOrcid[0000-0003-2018-5850]{H.~Ito}$^\textrm{\scriptsize 173}$,
\AtlasOrcid[0000-0001-5038-2762]{R.~Iuppa}$^\textrm{\scriptsize 80a,80b}$,
\AtlasOrcid[0000-0002-9152-383X]{A.~Ivina}$^\textrm{\scriptsize 174}$,
\AtlasOrcid[0000-0002-9846-5601]{J.M.~Izen}$^\textrm{\scriptsize 47}$,
\AtlasOrcid[0000-0002-8770-1592]{V.~Izzo}$^\textrm{\scriptsize 74a}$,
\AtlasOrcid[0000-0003-2489-9930]{P.~Jacka}$^\textrm{\scriptsize 135}$,
\AtlasOrcid[0000-0002-0847-402X]{P.~Jackson}$^\textrm{\scriptsize 1}$,
\AtlasOrcid[0000-0002-1669-759X]{C.S.~Jagfeld}$^\textrm{\scriptsize 112}$,
\AtlasOrcid[0000-0001-8067-0984]{G.~Jain}$^\textrm{\scriptsize 160a}$,
\AtlasOrcid[0000-0001-7277-9912]{P.~Jain}$^\textrm{\scriptsize 50}$,
\AtlasOrcid[0000-0001-8885-012X]{K.~Jakobs}$^\textrm{\scriptsize 56}$,
\AtlasOrcid[0000-0001-7038-0369]{T.~Jakoubek}$^\textrm{\scriptsize 174}$,
\AtlasOrcid[0000-0001-9554-0787]{J.~Jamieson}$^\textrm{\scriptsize 61}$,
\AtlasOrcid[0000-0002-3665-7747]{W.~Jang}$^\textrm{\scriptsize 157}$,
\AtlasOrcid[0000-0001-8798-808X]{M.~Javurkova}$^\textrm{\scriptsize 106}$,
\AtlasOrcid[0000-0003-2501-249X]{P.~Jawahar}$^\textrm{\scriptsize 104}$,
\AtlasOrcid[0000-0001-6507-4623]{L.~Jeanty}$^\textrm{\scriptsize 127}$,
\AtlasOrcid[0000-0002-0159-6593]{J.~Jejelava}$^\textrm{\scriptsize 153a}$,
\AtlasOrcid[0000-0002-4539-4192]{P.~Jenni}$^\textrm{\scriptsize 56,e}$,
\AtlasOrcid[0000-0002-2839-801X]{C.E.~Jessiman}$^\textrm{\scriptsize 35}$,
\AtlasOrcid{C.~Jia}$^\textrm{\scriptsize 64b}$,
\AtlasOrcid[0000-0002-7391-4423]{H.~Jia}$^\textrm{\scriptsize 169}$,
\AtlasOrcid[0000-0002-5725-3397]{J.~Jia}$^\textrm{\scriptsize 149}$,
\AtlasOrcid[0000-0002-5254-9930]{X.~Jia}$^\textrm{\scriptsize 14,115c}$,
\AtlasOrcid[0000-0002-2657-3099]{Z.~Jia}$^\textrm{\scriptsize 115a}$,
\AtlasOrcid[0009-0005-0253-5716]{C.~Jiang}$^\textrm{\scriptsize 54}$,
\AtlasOrcid[0000-0003-2906-1977]{S.~Jiggins}$^\textrm{\scriptsize 50}$,
\AtlasOrcid[0000-0002-8705-628X]{J.~Jimenez~Pena}$^\textrm{\scriptsize 13}$,
\AtlasOrcid[0000-0002-5076-7803]{S.~Jin}$^\textrm{\scriptsize 115a}$,
\AtlasOrcid[0000-0001-7449-9164]{A.~Jinaru}$^\textrm{\scriptsize 28b}$,
\AtlasOrcid[0000-0001-5073-0974]{O.~Jinnouchi}$^\textrm{\scriptsize 158}$,
\AtlasOrcid[0000-0001-5410-1315]{P.~Johansson}$^\textrm{\scriptsize 143}$,
\AtlasOrcid[0000-0001-9147-6052]{K.A.~Johns}$^\textrm{\scriptsize 7}$,
\AtlasOrcid[0000-0002-4837-3733]{J.W.~Johnson}$^\textrm{\scriptsize 140}$,
\AtlasOrcid[0009-0001-1943-1658]{F.A.~Jolly}$^\textrm{\scriptsize 50}$,
\AtlasOrcid[0000-0002-9204-4689]{D.M.~Jones}$^\textrm{\scriptsize 150}$,
\AtlasOrcid[0000-0001-6289-2292]{E.~Jones}$^\textrm{\scriptsize 50}$,
\AtlasOrcid{K.S.~Jones}$^\textrm{\scriptsize 8}$,
\AtlasOrcid[0000-0002-6293-6432]{P.~Jones}$^\textrm{\scriptsize 33}$,
\AtlasOrcid[0000-0002-6427-3513]{R.W.L.~Jones}$^\textrm{\scriptsize 94}$,
\AtlasOrcid[0000-0002-2580-1977]{T.J.~Jones}$^\textrm{\scriptsize 95}$,
\AtlasOrcid[0000-0003-4313-4255]{H.L.~Joos}$^\textrm{\scriptsize 57,37}$,
\AtlasOrcid[0000-0001-6249-7444]{R.~Joshi}$^\textrm{\scriptsize 123}$,
\AtlasOrcid[0000-0001-5650-4556]{J.~Jovicevic}$^\textrm{\scriptsize 16}$,
\AtlasOrcid[0000-0002-9745-1638]{X.~Ju}$^\textrm{\scriptsize 18a}$,
\AtlasOrcid[0000-0001-7205-1171]{J.J.~Junggeburth}$^\textrm{\scriptsize 37}$,
\AtlasOrcid[0000-0002-1119-8820]{T.~Junkermann}$^\textrm{\scriptsize 65a}$,
\AtlasOrcid[0000-0002-1558-3291]{A.~Juste~Rozas}$^\textrm{\scriptsize 13,s}$,
\AtlasOrcid[0000-0002-7269-9194]{M.K.~Juzek}$^\textrm{\scriptsize 89}$,
\AtlasOrcid[0000-0003-0568-5750]{S.~Kabana}$^\textrm{\scriptsize 141e}$,
\AtlasOrcid[0000-0002-8880-4120]{A.~Kaczmarska}$^\textrm{\scriptsize 89}$,
\AtlasOrcid[0000-0002-1003-7638]{M.~Kado}$^\textrm{\scriptsize 113}$,
\AtlasOrcid[0000-0002-4693-7857]{H.~Kagan}$^\textrm{\scriptsize 123}$,
\AtlasOrcid[0000-0002-3386-6869]{M.~Kagan}$^\textrm{\scriptsize 147}$,
\AtlasOrcid[0000-0001-7131-3029]{A.~Kahn}$^\textrm{\scriptsize 132}$,
\AtlasOrcid[0000-0002-9003-5711]{C.~Kahra}$^\textrm{\scriptsize 103}$,
\AtlasOrcid[0000-0002-6532-7501]{T.~Kaji}$^\textrm{\scriptsize 157}$,
\AtlasOrcid[0000-0002-8464-1790]{E.~Kajomovitz}$^\textrm{\scriptsize 154}$,
\AtlasOrcid[0000-0003-2155-1859]{N.~Kakati}$^\textrm{\scriptsize 174}$,
\AtlasOrcid[0000-0002-4563-3253]{I.~Kalaitzidou}$^\textrm{\scriptsize 56}$,
\AtlasOrcid[0000-0002-2875-853X]{C.W.~Kalderon}$^\textrm{\scriptsize 30}$,
\AtlasOrcid[0000-0001-5009-0399]{N.J.~Kang}$^\textrm{\scriptsize 140}$,
\AtlasOrcid[0000-0002-4238-9822]{D.~Kar}$^\textrm{\scriptsize 34g}$,
\AtlasOrcid[0000-0002-5010-8613]{K.~Karava}$^\textrm{\scriptsize 130}$,
\AtlasOrcid[0000-0001-8967-1705]{M.J.~Kareem}$^\textrm{\scriptsize 160b}$,
\AtlasOrcid[0000-0002-1037-1206]{E.~Karentzos}$^\textrm{\scriptsize 25}$,
\AtlasOrcid[0000-0002-4907-9499]{O.~Karkout}$^\textrm{\scriptsize 118}$,
\AtlasOrcid[0000-0002-2230-5353]{S.N.~Karpov}$^\textrm{\scriptsize 40}$,
\AtlasOrcid[0000-0003-0254-4629]{Z.M.~Karpova}$^\textrm{\scriptsize 40}$,
\AtlasOrcid[0000-0002-1957-3787]{V.~Kartvelishvili}$^\textrm{\scriptsize 94}$,
\AtlasOrcid[0000-0001-9087-4315]{A.N.~Karyukhin}$^\textrm{\scriptsize 38}$,
\AtlasOrcid[0000-0002-7139-8197]{E.~Kasimi}$^\textrm{\scriptsize 156}$,
\AtlasOrcid[0000-0003-3121-395X]{J.~Katzy}$^\textrm{\scriptsize 50}$,
\AtlasOrcid[0000-0002-7602-1284]{S.~Kaur}$^\textrm{\scriptsize 35}$,
\AtlasOrcid[0000-0002-7874-6107]{K.~Kawade}$^\textrm{\scriptsize 144}$,
\AtlasOrcid[0009-0008-7282-7396]{M.P.~Kawale}$^\textrm{\scriptsize 124}$,
\AtlasOrcid[0000-0002-3057-8378]{C.~Kawamoto}$^\textrm{\scriptsize 90}$,
\AtlasOrcid[0000-0002-5841-5511]{T.~Kawamoto}$^\textrm{\scriptsize 64a}$,
\AtlasOrcid[0000-0002-6304-3230]{E.F.~Kay}$^\textrm{\scriptsize 37}$,
\AtlasOrcid[0000-0002-9775-7303]{F.I.~Kaya}$^\textrm{\scriptsize 162}$,
\AtlasOrcid[0000-0002-7252-3201]{S.~Kazakos}$^\textrm{\scriptsize 110}$,
\AtlasOrcid[0000-0002-4906-5468]{V.F.~Kazanin}$^\textrm{\scriptsize 38}$,
\AtlasOrcid[0000-0001-5798-6665]{Y.~Ke}$^\textrm{\scriptsize 149}$,
\AtlasOrcid[0000-0003-0766-5307]{J.M.~Keaveney}$^\textrm{\scriptsize 34a}$,
\AtlasOrcid[0000-0002-0510-4189]{R.~Keeler}$^\textrm{\scriptsize 170}$,
\AtlasOrcid[0000-0002-1119-1004]{G.V.~Kehris}$^\textrm{\scriptsize 63}$,
\AtlasOrcid[0000-0001-7140-9813]{J.S.~Keller}$^\textrm{\scriptsize 35}$,
\AtlasOrcid[0000-0003-4168-3373]{J.J.~Kempster}$^\textrm{\scriptsize 150}$,
\AtlasOrcid[0000-0002-2555-497X]{O.~Kepka}$^\textrm{\scriptsize 135}$,
\AtlasOrcid[0000-0003-4171-1768]{B.P.~Kerridge}$^\textrm{\scriptsize 138}$,
\AtlasOrcid[0000-0002-0511-2592]{S.~Kersten}$^\textrm{\scriptsize 176}$,
\AtlasOrcid[0000-0002-4529-452X]{B.P.~Ker\v{s}evan}$^\textrm{\scriptsize 96}$,
\AtlasOrcid[0000-0001-6830-4244]{L.~Keszeghova}$^\textrm{\scriptsize 29a}$,
\AtlasOrcid[0000-0002-8597-3834]{S.~Ketabchi~Haghighat}$^\textrm{\scriptsize 159}$,
\AtlasOrcid[0009-0005-8074-6156]{R.A.~Khan}$^\textrm{\scriptsize 133}$,
\AtlasOrcid[0000-0001-9621-422X]{A.~Khanov}$^\textrm{\scriptsize 125}$,
\AtlasOrcid[0000-0002-1051-3833]{A.G.~Kharlamov}$^\textrm{\scriptsize 38}$,
\AtlasOrcid[0000-0002-0387-6804]{T.~Kharlamova}$^\textrm{\scriptsize 38}$,
\AtlasOrcid[0000-0001-8720-6615]{E.E.~Khoda}$^\textrm{\scriptsize 142}$,
\AtlasOrcid[0000-0002-8340-9455]{M.~Kholodenko}$^\textrm{\scriptsize 134a}$,
\AtlasOrcid[0000-0002-5954-3101]{T.J.~Khoo}$^\textrm{\scriptsize 19}$,
\AtlasOrcid[0000-0002-6353-8452]{G.~Khoriauli}$^\textrm{\scriptsize 171}$,
\AtlasOrcid[0000-0003-2350-1249]{J.~Khubua}$^\textrm{\scriptsize 153b,*}$,
\AtlasOrcid[0000-0001-8538-1647]{Y.A.R.~Khwaira}$^\textrm{\scriptsize 131}$,
\AtlasOrcid{B.~Kibirige}$^\textrm{\scriptsize 34g}$,
\AtlasOrcid[0000-0002-0331-6559]{D.~Kim}$^\textrm{\scriptsize 6}$,
\AtlasOrcid[0000-0002-9635-1491]{D.W.~Kim}$^\textrm{\scriptsize 49a,49b}$,
\AtlasOrcid[0000-0003-3286-1326]{Y.K.~Kim}$^\textrm{\scriptsize 41}$,
\AtlasOrcid[0000-0002-8883-9374]{N.~Kimura}$^\textrm{\scriptsize 99}$,
\AtlasOrcid[0009-0003-7785-7803]{M.K.~Kingston}$^\textrm{\scriptsize 57}$,
\AtlasOrcid[0000-0001-5611-9543]{A.~Kirchhoff}$^\textrm{\scriptsize 57}$,
\AtlasOrcid[0000-0003-1679-6907]{C.~Kirfel}$^\textrm{\scriptsize 25}$,
\AtlasOrcid[0000-0001-6242-8852]{F.~Kirfel}$^\textrm{\scriptsize 25}$,
\AtlasOrcid[0000-0001-8096-7577]{J.~Kirk}$^\textrm{\scriptsize 138}$,
\AtlasOrcid[0000-0001-7490-6890]{A.E.~Kiryunin}$^\textrm{\scriptsize 113}$,
\AtlasOrcid[0000-0002-7246-0570]{S.~Kita}$^\textrm{\scriptsize 161}$,
\AtlasOrcid[0000-0003-4431-8400]{C.~Kitsaki}$^\textrm{\scriptsize 10}$,
\AtlasOrcid[0000-0002-6854-2717]{O.~Kivernyk}$^\textrm{\scriptsize 25}$,
\AtlasOrcid[0000-0002-4326-9742]{M.~Klassen}$^\textrm{\scriptsize 162}$,
\AtlasOrcid[0000-0002-3780-1755]{C.~Klein}$^\textrm{\scriptsize 35}$,
\AtlasOrcid[0000-0002-0145-4747]{L.~Klein}$^\textrm{\scriptsize 171}$,
\AtlasOrcid[0000-0002-9999-2534]{M.H.~Klein}$^\textrm{\scriptsize 46}$,
\AtlasOrcid[0000-0002-2999-6150]{S.B.~Klein}$^\textrm{\scriptsize 58}$,
\AtlasOrcid[0000-0001-7391-5330]{U.~Klein}$^\textrm{\scriptsize 95}$,
\AtlasOrcid[0000-0003-2748-4829]{A.~Klimentov}$^\textrm{\scriptsize 30}$,
\AtlasOrcid[0000-0002-9580-0363]{T.~Klioutchnikova}$^\textrm{\scriptsize 37}$,
\AtlasOrcid[0000-0001-6419-5829]{P.~Kluit}$^\textrm{\scriptsize 118}$,
\AtlasOrcid[0000-0001-8484-2261]{S.~Kluth}$^\textrm{\scriptsize 113}$,
\AtlasOrcid[0000-0002-6206-1912]{E.~Kneringer}$^\textrm{\scriptsize 81}$,
\AtlasOrcid[0000-0003-2486-7672]{T.M.~Knight}$^\textrm{\scriptsize 159}$,
\AtlasOrcid[0000-0002-1559-9285]{A.~Knue}$^\textrm{\scriptsize 51}$,
\AtlasOrcid[0009-0002-0070-5900]{D.~Kobylianskii}$^\textrm{\scriptsize 174}$,
\AtlasOrcid[0000-0002-2676-2842]{S.F.~Koch}$^\textrm{\scriptsize 130}$,
\AtlasOrcid[0000-0003-4559-6058]{M.~Kocian}$^\textrm{\scriptsize 147}$,
\AtlasOrcid[0000-0002-8644-2349]{P.~Kody\v{s}}$^\textrm{\scriptsize 137}$,
\AtlasOrcid[0000-0002-9090-5502]{D.M.~Koeck}$^\textrm{\scriptsize 127}$,
\AtlasOrcid[0000-0002-0497-3550]{P.T.~Koenig}$^\textrm{\scriptsize 25}$,
\AtlasOrcid[0000-0001-9612-4988]{T.~Koffas}$^\textrm{\scriptsize 35}$,
\AtlasOrcid[0000-0003-2526-4910]{O.~Kolay}$^\textrm{\scriptsize 52}$,
\AtlasOrcid[0000-0002-8560-8917]{I.~Koletsou}$^\textrm{\scriptsize 4}$,
\AtlasOrcid[0000-0002-3047-3146]{T.~Komarek}$^\textrm{\scriptsize 89}$,
\AtlasOrcid[0000-0002-6901-9717]{K.~K\"oneke}$^\textrm{\scriptsize 56}$,
\AtlasOrcid[0000-0001-8063-8765]{A.X.Y.~Kong}$^\textrm{\scriptsize 1}$,
\AtlasOrcid[0000-0003-1553-2950]{T.~Kono}$^\textrm{\scriptsize 122}$,
\AtlasOrcid[0000-0002-4140-6360]{N.~Konstantinidis}$^\textrm{\scriptsize 99}$,
\AtlasOrcid[0000-0002-4860-5979]{P.~Kontaxakis}$^\textrm{\scriptsize 58}$,
\AtlasOrcid[0000-0002-1859-6557]{B.~Konya}$^\textrm{\scriptsize 101}$,
\AtlasOrcid[0000-0002-8775-1194]{R.~Kopeliansky}$^\textrm{\scriptsize 43}$,
\AtlasOrcid[0000-0002-2023-5945]{S.~Koperny}$^\textrm{\scriptsize 88a}$,
\AtlasOrcid[0000-0001-8085-4505]{K.~Korcyl}$^\textrm{\scriptsize 89}$,
\AtlasOrcid[0000-0003-0486-2081]{K.~Kordas}$^\textrm{\scriptsize 156,d}$,
\AtlasOrcid[0000-0002-3962-2099]{A.~Korn}$^\textrm{\scriptsize 99}$,
\AtlasOrcid[0000-0001-9291-5408]{S.~Korn}$^\textrm{\scriptsize 57}$,
\AtlasOrcid[0000-0002-9211-9775]{I.~Korolkov}$^\textrm{\scriptsize 13}$,
\AtlasOrcid[0000-0003-3640-8676]{N.~Korotkova}$^\textrm{\scriptsize 38}$,
\AtlasOrcid[0000-0001-7081-3275]{B.~Kortman}$^\textrm{\scriptsize 118}$,
\AtlasOrcid[0000-0003-0352-3096]{O.~Kortner}$^\textrm{\scriptsize 113}$,
\AtlasOrcid[0000-0001-8667-1814]{S.~Kortner}$^\textrm{\scriptsize 113}$,
\AtlasOrcid[0000-0003-1772-6898]{W.H.~Kostecka}$^\textrm{\scriptsize 119}$,
\AtlasOrcid[0000-0002-0490-9209]{V.V.~Kostyukhin}$^\textrm{\scriptsize 145}$,
\AtlasOrcid[0000-0002-8057-9467]{A.~Kotsokechagia}$^\textrm{\scriptsize 37}$,
\AtlasOrcid[0000-0003-3384-5053]{A.~Kotwal}$^\textrm{\scriptsize 53}$,
\AtlasOrcid[0000-0003-1012-4675]{A.~Koulouris}$^\textrm{\scriptsize 37}$,
\AtlasOrcid[0000-0002-6614-108X]{A.~Kourkoumeli-Charalampidi}$^\textrm{\scriptsize 75a,75b}$,
\AtlasOrcid[0000-0003-0083-274X]{C.~Kourkoumelis}$^\textrm{\scriptsize 9}$,
\AtlasOrcid[0000-0001-6568-2047]{E.~Kourlitis}$^\textrm{\scriptsize 113,z}$,
\AtlasOrcid[0000-0003-0294-3953]{O.~Kovanda}$^\textrm{\scriptsize 127}$,
\AtlasOrcid[0000-0002-7314-0990]{R.~Kowalewski}$^\textrm{\scriptsize 170}$,
\AtlasOrcid[0000-0001-6226-8385]{W.~Kozanecki}$^\textrm{\scriptsize 127}$,
\AtlasOrcid[0000-0003-4724-9017]{A.S.~Kozhin}$^\textrm{\scriptsize 38}$,
\AtlasOrcid[0000-0002-8625-5586]{V.A.~Kramarenko}$^\textrm{\scriptsize 38}$,
\AtlasOrcid[0000-0002-7580-384X]{G.~Kramberger}$^\textrm{\scriptsize 96}$,
\AtlasOrcid[0000-0002-0296-5899]{P.~Kramer}$^\textrm{\scriptsize 25}$,
\AtlasOrcid[0000-0002-7440-0520]{M.W.~Krasny}$^\textrm{\scriptsize 131}$,
\AtlasOrcid[0000-0002-6468-1381]{A.~Krasznahorkay}$^\textrm{\scriptsize 37}$,
\AtlasOrcid[0000-0001-8701-4592]{A.C.~Kraus}$^\textrm{\scriptsize 119}$,
\AtlasOrcid[0000-0003-3492-2831]{J.W.~Kraus}$^\textrm{\scriptsize 176}$,
\AtlasOrcid[0000-0003-4487-6365]{J.A.~Kremer}$^\textrm{\scriptsize 50}$,
\AtlasOrcid[0000-0003-0546-1634]{T.~Kresse}$^\textrm{\scriptsize 52}$,
\AtlasOrcid[0000-0002-7404-8483]{L.~Kretschmann}$^\textrm{\scriptsize 176}$,
\AtlasOrcid[0000-0002-8515-1355]{J.~Kretzschmar}$^\textrm{\scriptsize 95}$,
\AtlasOrcid[0000-0002-1739-6596]{K.~Kreul}$^\textrm{\scriptsize 19}$,
\AtlasOrcid[0000-0001-9958-949X]{P.~Krieger}$^\textrm{\scriptsize 159}$,
\AtlasOrcid[0000-0001-6408-2648]{K.~Krizka}$^\textrm{\scriptsize 21}$,
\AtlasOrcid[0000-0001-9873-0228]{K.~Kroeninger}$^\textrm{\scriptsize 51}$,
\AtlasOrcid[0000-0003-1808-0259]{H.~Kroha}$^\textrm{\scriptsize 113}$,
\AtlasOrcid[0000-0001-6215-3326]{J.~Kroll}$^\textrm{\scriptsize 135}$,
\AtlasOrcid[0000-0002-0964-6815]{J.~Kroll}$^\textrm{\scriptsize 132}$,
\AtlasOrcid[0000-0001-9395-3430]{K.S.~Krowpman}$^\textrm{\scriptsize 110}$,
\AtlasOrcid[0000-0003-2116-4592]{U.~Kruchonak}$^\textrm{\scriptsize 40}$,
\AtlasOrcid[0000-0001-8287-3961]{H.~Kr\"uger}$^\textrm{\scriptsize 25}$,
\AtlasOrcid{N.~Krumnack}$^\textrm{\scriptsize 83}$,
\AtlasOrcid[0000-0001-5791-0345]{M.C.~Kruse}$^\textrm{\scriptsize 53}$,
\AtlasOrcid[0000-0002-3664-2465]{O.~Kuchinskaia}$^\textrm{\scriptsize 38}$,
\AtlasOrcid[0000-0002-0116-5494]{S.~Kuday}$^\textrm{\scriptsize 3a}$,
\AtlasOrcid[0000-0001-5270-0920]{S.~Kuehn}$^\textrm{\scriptsize 37}$,
\AtlasOrcid[0000-0002-8309-019X]{R.~Kuesters}$^\textrm{\scriptsize 56}$,
\AtlasOrcid[0000-0002-1473-350X]{T.~Kuhl}$^\textrm{\scriptsize 50}$,
\AtlasOrcid[0000-0003-4387-8756]{V.~Kukhtin}$^\textrm{\scriptsize 40}$,
\AtlasOrcid[0000-0002-3036-5575]{Y.~Kulchitsky}$^\textrm{\scriptsize 40}$,
\AtlasOrcid[0000-0002-3065-326X]{S.~Kuleshov}$^\textrm{\scriptsize 141d,141b}$,
\AtlasOrcid[0000-0003-3681-1588]{M.~Kumar}$^\textrm{\scriptsize 34g}$,
\AtlasOrcid[0000-0001-9174-6200]{N.~Kumari}$^\textrm{\scriptsize 50}$,
\AtlasOrcid[0000-0002-6623-8586]{P.~Kumari}$^\textrm{\scriptsize 160b}$,
\AtlasOrcid[0000-0003-3692-1410]{A.~Kupco}$^\textrm{\scriptsize 135}$,
\AtlasOrcid{T.~Kupfer}$^\textrm{\scriptsize 51}$,
\AtlasOrcid[0000-0002-6042-8776]{A.~Kupich}$^\textrm{\scriptsize 38}$,
\AtlasOrcid[0000-0002-7540-0012]{O.~Kuprash}$^\textrm{\scriptsize 56}$,
\AtlasOrcid[0000-0003-3932-016X]{H.~Kurashige}$^\textrm{\scriptsize 87}$,
\AtlasOrcid[0000-0001-9392-3936]{L.L.~Kurchaninov}$^\textrm{\scriptsize 160a}$,
\AtlasOrcid[0000-0002-1837-6984]{O.~Kurdysh}$^\textrm{\scriptsize 68}$,
\AtlasOrcid[0000-0002-1281-8462]{Y.A.~Kurochkin}$^\textrm{\scriptsize 39}$,
\AtlasOrcid[0000-0001-7924-1517]{A.~Kurova}$^\textrm{\scriptsize 38}$,
\AtlasOrcid[0000-0001-8858-8440]{M.~Kuze}$^\textrm{\scriptsize 158}$,
\AtlasOrcid[0000-0001-7243-0227]{A.K.~Kvam}$^\textrm{\scriptsize 106}$,
\AtlasOrcid[0000-0001-5973-8729]{J.~Kvita}$^\textrm{\scriptsize 126}$,
\AtlasOrcid[0000-0001-8717-4449]{T.~Kwan}$^\textrm{\scriptsize 107}$,
\AtlasOrcid[0000-0002-8523-5954]{N.G.~Kyriacou}$^\textrm{\scriptsize 109}$,
\AtlasOrcid[0000-0001-6578-8618]{L.A.O.~Laatu}$^\textrm{\scriptsize 105}$,
\AtlasOrcid[0000-0002-2623-6252]{C.~Lacasta}$^\textrm{\scriptsize 168}$,
\AtlasOrcid[0000-0003-4588-8325]{F.~Lacava}$^\textrm{\scriptsize 77a,77b}$,
\AtlasOrcid[0000-0002-7183-8607]{H.~Lacker}$^\textrm{\scriptsize 19}$,
\AtlasOrcid[0000-0002-1590-194X]{D.~Lacour}$^\textrm{\scriptsize 131}$,
\AtlasOrcid[0000-0002-3707-9010]{N.N.~Lad}$^\textrm{\scriptsize 99}$,
\AtlasOrcid[0000-0001-6206-8148]{E.~Ladygin}$^\textrm{\scriptsize 40}$,
\AtlasOrcid[0009-0001-9169-2270]{A.~Lafarge}$^\textrm{\scriptsize 42}$,
\AtlasOrcid[0000-0002-4209-4194]{B.~Laforge}$^\textrm{\scriptsize 131}$,
\AtlasOrcid[0000-0001-7509-7765]{T.~Lagouri}$^\textrm{\scriptsize 177}$,
\AtlasOrcid[0000-0002-3879-696X]{F.Z.~Lahbabi}$^\textrm{\scriptsize 36a}$,
\AtlasOrcid[0000-0002-9898-9253]{S.~Lai}$^\textrm{\scriptsize 57}$,
\AtlasOrcid[0000-0002-5606-4164]{J.E.~Lambert}$^\textrm{\scriptsize 170}$,
\AtlasOrcid[0000-0003-2958-986X]{S.~Lammers}$^\textrm{\scriptsize 70}$,
\AtlasOrcid[0000-0002-2337-0958]{W.~Lampl}$^\textrm{\scriptsize 7}$,
\AtlasOrcid[0000-0001-9782-9920]{C.~Lampoudis}$^\textrm{\scriptsize 156,d}$,
\AtlasOrcid{G.~Lamprinoudis}$^\textrm{\scriptsize 103}$,
\AtlasOrcid[0000-0001-6212-5261]{A.N.~Lancaster}$^\textrm{\scriptsize 119}$,
\AtlasOrcid[0000-0002-0225-187X]{E.~Lan\c{c}on}$^\textrm{\scriptsize 30}$,
\AtlasOrcid[0000-0002-8222-2066]{U.~Landgraf}$^\textrm{\scriptsize 56}$,
\AtlasOrcid[0000-0001-6828-9769]{M.P.J.~Landon}$^\textrm{\scriptsize 97}$,
\AtlasOrcid[0000-0001-9954-7898]{V.S.~Lang}$^\textrm{\scriptsize 56}$,
\AtlasOrcid[0000-0001-8099-9042]{O.K.B.~Langrekken}$^\textrm{\scriptsize 129}$,
\AtlasOrcid[0000-0001-8057-4351]{A.J.~Lankford}$^\textrm{\scriptsize 163}$,
\AtlasOrcid[0000-0002-7197-9645]{F.~Lanni}$^\textrm{\scriptsize 37}$,
\AtlasOrcid[0000-0002-0729-6487]{K.~Lantzsch}$^\textrm{\scriptsize 25}$,
\AtlasOrcid[0000-0003-4980-6032]{A.~Lanza}$^\textrm{\scriptsize 75a}$,
\AtlasOrcid[0009-0004-5966-6699]{M.~Lanzac~Berrocal}$^\textrm{\scriptsize 168}$,
\AtlasOrcid[0000-0002-4815-5314]{J.F.~Laporte}$^\textrm{\scriptsize 139}$,
\AtlasOrcid[0000-0002-1388-869X]{T.~Lari}$^\textrm{\scriptsize 73a}$,
\AtlasOrcid[0000-0001-6068-4473]{F.~Lasagni~Manghi}$^\textrm{\scriptsize 24b}$,
\AtlasOrcid[0000-0002-9541-0592]{M.~Lassnig}$^\textrm{\scriptsize 37}$,
\AtlasOrcid[0000-0001-9591-5622]{V.~Latonova}$^\textrm{\scriptsize 135}$,
\AtlasOrcid[0000-0002-2575-0743]{A.~Laurier}$^\textrm{\scriptsize 154}$,
\AtlasOrcid[0000-0003-3211-067X]{S.D.~Lawlor}$^\textrm{\scriptsize 143}$,
\AtlasOrcid[0000-0002-9035-9679]{Z.~Lawrence}$^\textrm{\scriptsize 104}$,
\AtlasOrcid{R.~Lazaridou}$^\textrm{\scriptsize 172}$,
\AtlasOrcid[0000-0002-4094-1273]{M.~Lazzaroni}$^\textrm{\scriptsize 73a,73b}$,
\AtlasOrcid{B.~Le}$^\textrm{\scriptsize 104}$,
\AtlasOrcid[0000-0002-5421-1589]{H.D.M.~Le}$^\textrm{\scriptsize 110}$,
\AtlasOrcid[0000-0002-8909-2508]{E.M.~Le~Boulicaut}$^\textrm{\scriptsize 177}$,
\AtlasOrcid[0000-0002-2625-5648]{L.T.~Le~Pottier}$^\textrm{\scriptsize 18a}$,
\AtlasOrcid[0000-0003-1501-7262]{B.~Leban}$^\textrm{\scriptsize 24b,24a}$,
\AtlasOrcid[0000-0002-9566-1850]{A.~Lebedev}$^\textrm{\scriptsize 83}$,
\AtlasOrcid[0000-0001-5977-6418]{M.~LeBlanc}$^\textrm{\scriptsize 104}$,
\AtlasOrcid[0000-0001-9398-1909]{F.~Ledroit-Guillon}$^\textrm{\scriptsize 62}$,
\AtlasOrcid[0000-0002-3353-2658]{S.C.~Lee}$^\textrm{\scriptsize 152}$,
\AtlasOrcid[0000-0003-0836-416X]{S.~Lee}$^\textrm{\scriptsize 49a,49b}$,
\AtlasOrcid[0000-0001-7232-6315]{T.F.~Lee}$^\textrm{\scriptsize 95}$,
\AtlasOrcid[0000-0002-3365-6781]{L.L.~Leeuw}$^\textrm{\scriptsize 34c}$,
\AtlasOrcid[0000-0002-5560-0586]{M.~Lefebvre}$^\textrm{\scriptsize 170}$,
\AtlasOrcid[0000-0002-9299-9020]{C.~Leggett}$^\textrm{\scriptsize 18a}$,
\AtlasOrcid[0000-0001-9045-7853]{G.~Lehmann~Miotto}$^\textrm{\scriptsize 37}$,
\AtlasOrcid[0000-0003-1406-1413]{M.~Leigh}$^\textrm{\scriptsize 58}$,
\AtlasOrcid[0000-0002-2968-7841]{W.A.~Leight}$^\textrm{\scriptsize 106}$,
\AtlasOrcid[0000-0002-1747-2544]{W.~Leinonen}$^\textrm{\scriptsize 117}$,
\AtlasOrcid[0000-0002-8126-3958]{A.~Leisos}$^\textrm{\scriptsize 156,q}$,
\AtlasOrcid[0000-0003-0392-3663]{M.A.L.~Leite}$^\textrm{\scriptsize 85c}$,
\AtlasOrcid[0000-0002-0335-503X]{C.E.~Leitgeb}$^\textrm{\scriptsize 19}$,
\AtlasOrcid[0000-0002-2994-2187]{R.~Leitner}$^\textrm{\scriptsize 137}$,
\AtlasOrcid[0000-0002-1525-2695]{K.J.C.~Leney}$^\textrm{\scriptsize 46}$,
\AtlasOrcid[0000-0002-9560-1778]{T.~Lenz}$^\textrm{\scriptsize 25}$,
\AtlasOrcid[0000-0001-6222-9642]{S.~Leone}$^\textrm{\scriptsize 76a}$,
\AtlasOrcid[0000-0002-7241-2114]{C.~Leonidopoulos}$^\textrm{\scriptsize 54}$,
\AtlasOrcid[0000-0001-9415-7903]{A.~Leopold}$^\textrm{\scriptsize 148}$,
\AtlasOrcid[0000-0002-8875-1399]{R.~Les}$^\textrm{\scriptsize 110}$,
\AtlasOrcid[0000-0001-5770-4883]{C.G.~Lester}$^\textrm{\scriptsize 33}$,
\AtlasOrcid[0000-0002-5495-0656]{M.~Levchenko}$^\textrm{\scriptsize 38}$,
\AtlasOrcid[0000-0002-0244-4743]{J.~Lev\^eque}$^\textrm{\scriptsize 4}$,
\AtlasOrcid[0000-0003-4679-0485]{L.J.~Levinson}$^\textrm{\scriptsize 174}$,
\AtlasOrcid[0009-0000-5431-0029]{G.~Levrini}$^\textrm{\scriptsize 24b,24a}$,
\AtlasOrcid[0000-0002-8972-3066]{M.P.~Lewicki}$^\textrm{\scriptsize 89}$,
\AtlasOrcid[0000-0002-7581-846X]{C.~Lewis}$^\textrm{\scriptsize 142}$,
\AtlasOrcid[0000-0002-7814-8596]{D.J.~Lewis}$^\textrm{\scriptsize 4}$,
\AtlasOrcid[0009-0002-5604-8823]{L.~Lewitt}$^\textrm{\scriptsize 143}$,
\AtlasOrcid[0000-0003-4317-3342]{A.~Li}$^\textrm{\scriptsize 30}$,
\AtlasOrcid[0000-0002-1974-2229]{B.~Li}$^\textrm{\scriptsize 64b}$,
\AtlasOrcid{C.~Li}$^\textrm{\scriptsize 64a}$,
\AtlasOrcid[0000-0003-3495-7778]{C-Q.~Li}$^\textrm{\scriptsize 113}$,
\AtlasOrcid[0000-0002-1081-2032]{H.~Li}$^\textrm{\scriptsize 64a}$,
\AtlasOrcid[0000-0002-4732-5633]{H.~Li}$^\textrm{\scriptsize 64b}$,
\AtlasOrcid[0000-0002-2459-9068]{H.~Li}$^\textrm{\scriptsize 115a}$,
\AtlasOrcid[0009-0003-1487-5940]{H.~Li}$^\textrm{\scriptsize 15}$,
\AtlasOrcid[0000-0001-9346-6982]{H.~Li}$^\textrm{\scriptsize 64b}$,
\AtlasOrcid[0009-0000-5782-8050]{J.~Li}$^\textrm{\scriptsize 64c}$,
\AtlasOrcid[0000-0002-2545-0329]{K.~Li}$^\textrm{\scriptsize 14}$,
\AtlasOrcid[0000-0001-6411-6107]{L.~Li}$^\textrm{\scriptsize 64c}$,
\AtlasOrcid[0000-0003-4317-3203]{M.~Li}$^\textrm{\scriptsize 14,115c}$,
\AtlasOrcid[0000-0003-1673-2794]{S.~Li}$^\textrm{\scriptsize 14,115c}$,
\AtlasOrcid[0000-0001-7879-3272]{S.~Li}$^\textrm{\scriptsize 64d,64c}$,
\AtlasOrcid[0000-0001-7775-4300]{T.~Li}$^\textrm{\scriptsize 5}$,
\AtlasOrcid[0000-0001-6975-102X]{X.~Li}$^\textrm{\scriptsize 107}$,
\AtlasOrcid[0000-0001-7096-2158]{Z.~Li}$^\textrm{\scriptsize 157}$,
\AtlasOrcid[0000-0003-1561-3435]{Z.~Li}$^\textrm{\scriptsize 14,115c}$,
\AtlasOrcid[0000-0003-1630-0668]{Z.~Li}$^\textrm{\scriptsize 64a}$,
\AtlasOrcid[0009-0006-1840-2106]{S.~Liang}$^\textrm{\scriptsize 14,115c}$,
\AtlasOrcid[0000-0003-0629-2131]{Z.~Liang}$^\textrm{\scriptsize 14}$,
\AtlasOrcid[0000-0002-8444-8827]{M.~Liberatore}$^\textrm{\scriptsize 139}$,
\AtlasOrcid[0000-0002-6011-2851]{B.~Liberti}$^\textrm{\scriptsize 78a}$,
\AtlasOrcid[0000-0002-5779-5989]{K.~Lie}$^\textrm{\scriptsize 66c}$,
\AtlasOrcid[0000-0003-0642-9169]{J.~Lieber~Marin}$^\textrm{\scriptsize 85e}$,
\AtlasOrcid[0000-0001-8884-2664]{H.~Lien}$^\textrm{\scriptsize 70}$,
\AtlasOrcid[0000-0001-5688-3330]{H.~Lin}$^\textrm{\scriptsize 109}$,
\AtlasOrcid[0000-0002-2269-3632]{K.~Lin}$^\textrm{\scriptsize 110}$,
\AtlasOrcid[0000-0002-2342-1452]{R.E.~Lindley}$^\textrm{\scriptsize 7}$,
\AtlasOrcid[0000-0001-9490-7276]{J.H.~Lindon}$^\textrm{\scriptsize 2}$,
\AtlasOrcid[0000-0002-3359-0380]{J.~Ling}$^\textrm{\scriptsize 63}$,
\AtlasOrcid[0000-0001-5982-7326]{E.~Lipeles}$^\textrm{\scriptsize 132}$,
\AtlasOrcid[0000-0002-8759-8564]{A.~Lipniacka}$^\textrm{\scriptsize 17}$,
\AtlasOrcid[0000-0002-1552-3651]{A.~Lister}$^\textrm{\scriptsize 169}$,
\AtlasOrcid[0000-0002-9372-0730]{J.D.~Little}$^\textrm{\scriptsize 70}$,
\AtlasOrcid[0000-0003-2823-9307]{B.~Liu}$^\textrm{\scriptsize 14}$,
\AtlasOrcid[0000-0002-0721-8331]{B.X.~Liu}$^\textrm{\scriptsize 115b}$,
\AtlasOrcid[0000-0002-0065-5221]{D.~Liu}$^\textrm{\scriptsize 64d,64c}$,
\AtlasOrcid[0009-0005-1438-8258]{E.H.L.~Liu}$^\textrm{\scriptsize 21}$,
\AtlasOrcid[0000-0003-3259-8775]{J.B.~Liu}$^\textrm{\scriptsize 64a}$,
\AtlasOrcid[0000-0001-5359-4541]{J.K.K.~Liu}$^\textrm{\scriptsize 33}$,
\AtlasOrcid[0000-0002-2639-0698]{K.~Liu}$^\textrm{\scriptsize 64d}$,
\AtlasOrcid[0000-0001-5807-0501]{K.~Liu}$^\textrm{\scriptsize 64d,64c}$,
\AtlasOrcid[0000-0003-0056-7296]{M.~Liu}$^\textrm{\scriptsize 64a}$,
\AtlasOrcid[0000-0002-0236-5404]{M.Y.~Liu}$^\textrm{\scriptsize 64a}$,
\AtlasOrcid[0000-0002-9815-8898]{P.~Liu}$^\textrm{\scriptsize 14}$,
\AtlasOrcid[0000-0001-5248-4391]{Q.~Liu}$^\textrm{\scriptsize 64d,142,64c}$,
\AtlasOrcid[0000-0003-1366-5530]{X.~Liu}$^\textrm{\scriptsize 64a}$,
\AtlasOrcid[0000-0003-1890-2275]{X.~Liu}$^\textrm{\scriptsize 64b}$,
\AtlasOrcid[0000-0003-3615-2332]{Y.~Liu}$^\textrm{\scriptsize 115b,115c}$,
\AtlasOrcid[0000-0001-9190-4547]{Y.L.~Liu}$^\textrm{\scriptsize 64b}$,
\AtlasOrcid[0000-0003-4448-4679]{Y.W.~Liu}$^\textrm{\scriptsize 64a}$,
\AtlasOrcid[0000-0002-5073-2264]{S.L.~Lloyd}$^\textrm{\scriptsize 97}$,
\AtlasOrcid[0000-0001-9012-3431]{E.M.~Lobodzinska}$^\textrm{\scriptsize 50}$,
\AtlasOrcid[0000-0002-2005-671X]{P.~Loch}$^\textrm{\scriptsize 7}$,
\AtlasOrcid[0000-0002-6506-6962]{E.~Lodhi}$^\textrm{\scriptsize 159}$,
\AtlasOrcid[0000-0002-9751-7633]{T.~Lohse}$^\textrm{\scriptsize 19}$,
\AtlasOrcid[0000-0003-1833-9160]{K.~Lohwasser}$^\textrm{\scriptsize 143}$,
\AtlasOrcid[0000-0002-2773-0586]{E.~Loiacono}$^\textrm{\scriptsize 50}$,
\AtlasOrcid[0000-0001-7456-494X]{J.D.~Lomas}$^\textrm{\scriptsize 21}$,
\AtlasOrcid[0000-0002-2115-9382]{J.D.~Long}$^\textrm{\scriptsize 43}$,
\AtlasOrcid[0000-0002-0352-2854]{I.~Longarini}$^\textrm{\scriptsize 163}$,
\AtlasOrcid[0000-0003-3984-6452]{R.~Longo}$^\textrm{\scriptsize 167}$,
\AtlasOrcid[0000-0002-4300-7064]{I.~Lopez~Paz}$^\textrm{\scriptsize 69}$,
\AtlasOrcid[0000-0002-0511-4766]{A.~Lopez~Solis}$^\textrm{\scriptsize 50}$,
\AtlasOrcid[0009-0007-0484-4322]{N.A.~Lopez-canelas}$^\textrm{\scriptsize 7}$,
\AtlasOrcid[0000-0002-7857-7606]{N.~Lorenzo~Martinez}$^\textrm{\scriptsize 4}$,
\AtlasOrcid[0000-0001-9657-0910]{A.M.~Lory}$^\textrm{\scriptsize 112}$,
\AtlasOrcid[0000-0001-8374-5806]{M.~Losada}$^\textrm{\scriptsize 120a}$,
\AtlasOrcid[0000-0001-7962-5334]{G.~L\"oschcke~Centeno}$^\textrm{\scriptsize 150}$,
\AtlasOrcid[0000-0002-7745-1649]{O.~Loseva}$^\textrm{\scriptsize 38}$,
\AtlasOrcid[0000-0002-8309-5548]{X.~Lou}$^\textrm{\scriptsize 49a,49b}$,
\AtlasOrcid[0000-0003-0867-2189]{X.~Lou}$^\textrm{\scriptsize 14,115c}$,
\AtlasOrcid[0000-0003-4066-2087]{A.~Lounis}$^\textrm{\scriptsize 68}$,
\AtlasOrcid[0000-0002-7803-6674]{P.A.~Love}$^\textrm{\scriptsize 94}$,
\AtlasOrcid[0000-0001-8133-3533]{G.~Lu}$^\textrm{\scriptsize 14,115c}$,
\AtlasOrcid[0000-0001-7610-3952]{M.~Lu}$^\textrm{\scriptsize 68}$,
\AtlasOrcid[0000-0002-8814-1670]{S.~Lu}$^\textrm{\scriptsize 132}$,
\AtlasOrcid[0000-0002-2497-0509]{Y.J.~Lu}$^\textrm{\scriptsize 67}$,
\AtlasOrcid[0000-0002-9285-7452]{H.J.~Lubatti}$^\textrm{\scriptsize 142}$,
\AtlasOrcid[0000-0001-7464-304X]{C.~Luci}$^\textrm{\scriptsize 77a,77b}$,
\AtlasOrcid[0000-0002-1626-6255]{F.L.~Lucio~Alves}$^\textrm{\scriptsize 115a}$,
\AtlasOrcid[0000-0001-8721-6901]{F.~Luehring}$^\textrm{\scriptsize 70}$,
\AtlasOrcid[0000-0002-3265-8371]{O.~Lukianchuk}$^\textrm{\scriptsize 68}$,
\AtlasOrcid[0000-0001-9790-4724]{B.S.~Lunday}$^\textrm{\scriptsize 132}$,
\AtlasOrcid[0009-0004-1439-5151]{O.~Lundberg}$^\textrm{\scriptsize 148}$,
\AtlasOrcid[0000-0003-3867-0336]{B.~Lund-Jensen}$^\textrm{\scriptsize 148,*}$,
\AtlasOrcid[0000-0001-6527-0253]{N.A.~Luongo}$^\textrm{\scriptsize 6}$,
\AtlasOrcid[0000-0003-4515-0224]{M.S.~Lutz}$^\textrm{\scriptsize 37}$,
\AtlasOrcid[0000-0002-3025-3020]{A.B.~Lux}$^\textrm{\scriptsize 26}$,
\AtlasOrcid[0000-0002-9634-542X]{D.~Lynn}$^\textrm{\scriptsize 30}$,
\AtlasOrcid[0000-0003-2990-1673]{R.~Lysak}$^\textrm{\scriptsize 135}$,
\AtlasOrcid[0000-0002-8141-3995]{E.~Lytken}$^\textrm{\scriptsize 101}$,
\AtlasOrcid[0000-0003-0136-233X]{V.~Lyubushkin}$^\textrm{\scriptsize 40}$,
\AtlasOrcid[0000-0001-8329-7994]{T.~Lyubushkina}$^\textrm{\scriptsize 40}$,
\AtlasOrcid[0000-0001-8343-9809]{M.M.~Lyukova}$^\textrm{\scriptsize 149}$,
\AtlasOrcid[0000-0003-1734-0610]{M.Firdaus~M.~Soberi}$^\textrm{\scriptsize 54}$,
\AtlasOrcid[0000-0002-8916-6220]{H.~Ma}$^\textrm{\scriptsize 30}$,
\AtlasOrcid[0009-0004-7076-0889]{K.~Ma}$^\textrm{\scriptsize 64a}$,
\AtlasOrcid[0000-0001-9717-1508]{L.L.~Ma}$^\textrm{\scriptsize 64b}$,
\AtlasOrcid[0009-0009-0770-2885]{W.~Ma}$^\textrm{\scriptsize 64a}$,
\AtlasOrcid[0000-0002-3577-9347]{Y.~Ma}$^\textrm{\scriptsize 125}$,
\AtlasOrcid[0000-0002-3150-3124]{J.C.~MacDonald}$^\textrm{\scriptsize 103}$,
\AtlasOrcid[0000-0002-8423-4933]{P.C.~Machado~De~Abreu~Farias}$^\textrm{\scriptsize 85e}$,
\AtlasOrcid[0000-0002-6875-6408]{R.~Madar}$^\textrm{\scriptsize 42}$,
\AtlasOrcid[0000-0001-7689-8628]{T.~Madula}$^\textrm{\scriptsize 99}$,
\AtlasOrcid[0000-0002-9084-3305]{J.~Maeda}$^\textrm{\scriptsize 87}$,
\AtlasOrcid[0000-0003-0901-1817]{T.~Maeno}$^\textrm{\scriptsize 30}$,
\AtlasOrcid[0000-0001-6218-4309]{H.~Maguire}$^\textrm{\scriptsize 143}$,
\AtlasOrcid[0000-0003-1056-3870]{V.~Maiboroda}$^\textrm{\scriptsize 139}$,
\AtlasOrcid[0000-0001-9099-0009]{A.~Maio}$^\textrm{\scriptsize 134a,134b,134d}$,
\AtlasOrcid[0000-0003-4819-9226]{K.~Maj}$^\textrm{\scriptsize 88a}$,
\AtlasOrcid[0000-0001-8857-5770]{O.~Majersky}$^\textrm{\scriptsize 50}$,
\AtlasOrcid[0000-0002-6871-3395]{S.~Majewski}$^\textrm{\scriptsize 127}$,
\AtlasOrcid[0000-0001-5124-904X]{N.~Makovec}$^\textrm{\scriptsize 68}$,
\AtlasOrcid[0000-0001-9418-3941]{V.~Maksimovic}$^\textrm{\scriptsize 16}$,
\AtlasOrcid[0000-0002-8813-3830]{B.~Malaescu}$^\textrm{\scriptsize 131}$,
\AtlasOrcid[0000-0001-8183-0468]{Pa.~Malecki}$^\textrm{\scriptsize 89}$,
\AtlasOrcid[0000-0003-1028-8602]{V.P.~Maleev}$^\textrm{\scriptsize 38}$,
\AtlasOrcid[0000-0002-0948-5775]{F.~Malek}$^\textrm{\scriptsize 62,m}$,
\AtlasOrcid[0000-0002-1585-4426]{M.~Mali}$^\textrm{\scriptsize 96}$,
\AtlasOrcid[0000-0002-3996-4662]{D.~Malito}$^\textrm{\scriptsize 98}$,
\AtlasOrcid[0000-0001-7934-1649]{U.~Mallik}$^\textrm{\scriptsize 82}$,
\AtlasOrcid{S.~Maltezos}$^\textrm{\scriptsize 10}$,
\AtlasOrcid{S.~Malyukov}$^\textrm{\scriptsize 40}$,
\AtlasOrcid[0000-0002-3203-4243]{J.~Mamuzic}$^\textrm{\scriptsize 13}$,
\AtlasOrcid[0000-0001-6158-2751]{G.~Mancini}$^\textrm{\scriptsize 55}$,
\AtlasOrcid[0000-0003-1103-0179]{M.N.~Mancini}$^\textrm{\scriptsize 27}$,
\AtlasOrcid[0000-0002-9909-1111]{G.~Manco}$^\textrm{\scriptsize 75a,75b}$,
\AtlasOrcid[0000-0001-5038-5154]{J.P.~Mandalia}$^\textrm{\scriptsize 97}$,
\AtlasOrcid[0000-0003-2597-2650]{S.S.~Mandarry}$^\textrm{\scriptsize 150}$,
\AtlasOrcid[0000-0002-0131-7523]{I.~Mandi\'{c}}$^\textrm{\scriptsize 96}$,
\AtlasOrcid[0000-0003-1792-6793]{L.~Manhaes~de~Andrade~Filho}$^\textrm{\scriptsize 85a}$,
\AtlasOrcid[0000-0002-4362-0088]{I.M.~Maniatis}$^\textrm{\scriptsize 174}$,
\AtlasOrcid[0000-0003-3896-5222]{J.~Manjarres~Ramos}$^\textrm{\scriptsize 92}$,
\AtlasOrcid[0000-0002-5708-0510]{D.C.~Mankad}$^\textrm{\scriptsize 174}$,
\AtlasOrcid[0000-0002-8497-9038]{A.~Mann}$^\textrm{\scriptsize 112}$,
\AtlasOrcid[0000-0002-2488-0511]{S.~Manzoni}$^\textrm{\scriptsize 37}$,
\AtlasOrcid[0000-0002-6123-7699]{L.~Mao}$^\textrm{\scriptsize 64c}$,
\AtlasOrcid[0000-0003-4046-0039]{X.~Mapekula}$^\textrm{\scriptsize 34c}$,
\AtlasOrcid[0000-0002-7020-4098]{A.~Marantis}$^\textrm{\scriptsize 156,q}$,
\AtlasOrcid[0000-0003-2655-7643]{G.~Marchiori}$^\textrm{\scriptsize 5}$,
\AtlasOrcid[0000-0003-0860-7897]{M.~Marcisovsky}$^\textrm{\scriptsize 135}$,
\AtlasOrcid[0000-0002-9889-8271]{C.~Marcon}$^\textrm{\scriptsize 73a}$,
\AtlasOrcid[0000-0002-4588-3578]{M.~Marinescu}$^\textrm{\scriptsize 21}$,
\AtlasOrcid[0000-0002-8431-1943]{S.~Marium}$^\textrm{\scriptsize 50}$,
\AtlasOrcid[0000-0002-4468-0154]{M.~Marjanovic}$^\textrm{\scriptsize 124}$,
\AtlasOrcid[0000-0002-9702-7431]{A.~Markhoos}$^\textrm{\scriptsize 56}$,
\AtlasOrcid[0000-0001-6231-3019]{M.~Markovitch}$^\textrm{\scriptsize 68}$,
\AtlasOrcid[0000-0003-3662-4694]{E.J.~Marshall}$^\textrm{\scriptsize 94}$,
\AtlasOrcid[0000-0003-0786-2570]{Z.~Marshall}$^\textrm{\scriptsize 18a}$,
\AtlasOrcid[0000-0002-3897-6223]{S.~Marti-Garcia}$^\textrm{\scriptsize 168}$,
\AtlasOrcid[0000-0002-3083-8782]{J.~Martin}$^\textrm{\scriptsize 99}$,
\AtlasOrcid[0000-0002-1477-1645]{T.A.~Martin}$^\textrm{\scriptsize 138}$,
\AtlasOrcid[0000-0003-3053-8146]{V.J.~Martin}$^\textrm{\scriptsize 54}$,
\AtlasOrcid[0000-0003-3420-2105]{B.~Martin~dit~Latour}$^\textrm{\scriptsize 17}$,
\AtlasOrcid[0000-0002-4466-3864]{L.~Martinelli}$^\textrm{\scriptsize 77a,77b}$,
\AtlasOrcid[0000-0002-3135-945X]{M.~Martinez}$^\textrm{\scriptsize 13,s}$,
\AtlasOrcid[0000-0001-8925-9518]{P.~Martinez~Agullo}$^\textrm{\scriptsize 168}$,
\AtlasOrcid[0000-0001-7102-6388]{V.I.~Martinez~Outschoorn}$^\textrm{\scriptsize 106}$,
\AtlasOrcid[0000-0001-6914-1168]{P.~Martinez~Suarez}$^\textrm{\scriptsize 13}$,
\AtlasOrcid[0000-0001-9457-1928]{S.~Martin-Haugh}$^\textrm{\scriptsize 138}$,
\AtlasOrcid[0000-0002-9144-2642]{G.~Martinovicova}$^\textrm{\scriptsize 137}$,
\AtlasOrcid[0000-0002-4963-9441]{V.S.~Martoiu}$^\textrm{\scriptsize 28b}$,
\AtlasOrcid[0000-0001-9080-2944]{A.C.~Martyniuk}$^\textrm{\scriptsize 99}$,
\AtlasOrcid[0000-0003-4364-4351]{A.~Marzin}$^\textrm{\scriptsize 37}$,
\AtlasOrcid[0000-0001-8660-9893]{D.~Mascione}$^\textrm{\scriptsize 80a,80b}$,
\AtlasOrcid[0000-0002-0038-5372]{L.~Masetti}$^\textrm{\scriptsize 103}$,
\AtlasOrcid[0000-0002-6813-8423]{J.~Masik}$^\textrm{\scriptsize 104}$,
\AtlasOrcid[0000-0002-4234-3111]{A.L.~Maslennikov}$^\textrm{\scriptsize 38}$,
\AtlasOrcid[0000-0002-9335-9690]{P.~Massarotti}$^\textrm{\scriptsize 74a,74b}$,
\AtlasOrcid[0000-0002-9853-0194]{P.~Mastrandrea}$^\textrm{\scriptsize 76a,76b}$,
\AtlasOrcid[0000-0002-8933-9494]{A.~Mastroberardino}$^\textrm{\scriptsize 45b,45a}$,
\AtlasOrcid[0000-0001-9984-8009]{T.~Masubuchi}$^\textrm{\scriptsize 128}$,
\AtlasOrcid[0009-0005-5396-4756]{T.T.~Mathew}$^\textrm{\scriptsize 127}$,
\AtlasOrcid[0000-0002-6248-953X]{T.~Mathisen}$^\textrm{\scriptsize 166}$,
\AtlasOrcid[0000-0002-2174-5517]{J.~Matousek}$^\textrm{\scriptsize 137}$,
\AtlasOrcid[0009-0002-0808-3798]{D.M.~Mattern}$^\textrm{\scriptsize 51}$,
\AtlasOrcid[0000-0002-5162-3713]{J.~Maurer}$^\textrm{\scriptsize 28b}$,
\AtlasOrcid[0000-0001-5914-5018]{T.~Maurin}$^\textrm{\scriptsize 61}$,
\AtlasOrcid[0000-0001-7331-2732]{A.J.~Maury}$^\textrm{\scriptsize 68}$,
\AtlasOrcid[0000-0002-1449-0317]{B.~Ma\v{c}ek}$^\textrm{\scriptsize 96}$,
\AtlasOrcid[0000-0001-8783-3758]{D.A.~Maximov}$^\textrm{\scriptsize 38}$,
\AtlasOrcid[0000-0003-4227-7094]{A.E.~May}$^\textrm{\scriptsize 104}$,
\AtlasOrcid[0000-0003-0954-0970]{R.~Mazini}$^\textrm{\scriptsize 34g}$,
\AtlasOrcid[0000-0001-8420-3742]{I.~Maznas}$^\textrm{\scriptsize 119}$,
\AtlasOrcid[0000-0002-8273-9532]{M.~Mazza}$^\textrm{\scriptsize 110}$,
\AtlasOrcid[0000-0003-3865-730X]{S.M.~Mazza}$^\textrm{\scriptsize 140}$,
\AtlasOrcid[0000-0002-8406-0195]{E.~Mazzeo}$^\textrm{\scriptsize 73a,73b}$,
\AtlasOrcid[0000-0001-7551-3386]{J.P.~Mc~Gowan}$^\textrm{\scriptsize 170}$,
\AtlasOrcid[0000-0002-4551-4502]{S.P.~Mc~Kee}$^\textrm{\scriptsize 109}$,
\AtlasOrcid[0000-0002-7450-4805]{C.A.~Mc~Lean}$^\textrm{\scriptsize 6}$,
\AtlasOrcid[0000-0002-9656-5692]{C.C.~McCracken}$^\textrm{\scriptsize 169}$,
\AtlasOrcid[0000-0002-8092-5331]{E.F.~McDonald}$^\textrm{\scriptsize 108}$,
\AtlasOrcid[0000-0002-2489-2598]{A.E.~McDougall}$^\textrm{\scriptsize 118}$,
\AtlasOrcid[0000-0001-9273-2564]{J.A.~Mcfayden}$^\textrm{\scriptsize 150}$,
\AtlasOrcid[0000-0001-9139-6896]{R.P.~McGovern}$^\textrm{\scriptsize 132}$,
\AtlasOrcid[0000-0001-9618-3689]{R.P.~Mckenzie}$^\textrm{\scriptsize 34g}$,
\AtlasOrcid[0000-0002-0930-5340]{T.C.~Mclachlan}$^\textrm{\scriptsize 50}$,
\AtlasOrcid[0000-0003-2424-5697]{D.J.~Mclaughlin}$^\textrm{\scriptsize 99}$,
\AtlasOrcid[0000-0002-3599-9075]{S.J.~McMahon}$^\textrm{\scriptsize 138}$,
\AtlasOrcid[0000-0003-1477-1407]{C.M.~Mcpartland}$^\textrm{\scriptsize 95}$,
\AtlasOrcid[0000-0001-9211-7019]{R.A.~McPherson}$^\textrm{\scriptsize 170,w}$,
\AtlasOrcid[0000-0002-1281-2060]{S.~Mehlhase}$^\textrm{\scriptsize 112}$,
\AtlasOrcid[0000-0003-2619-9743]{A.~Mehta}$^\textrm{\scriptsize 95}$,
\AtlasOrcid[0000-0002-7018-682X]{D.~Melini}$^\textrm{\scriptsize 168}$,
\AtlasOrcid[0000-0003-4838-1546]{B.R.~Mellado~Garcia}$^\textrm{\scriptsize 34g}$,
\AtlasOrcid[0000-0002-3964-6736]{A.H.~Melo}$^\textrm{\scriptsize 57}$,
\AtlasOrcid[0000-0001-7075-2214]{F.~Meloni}$^\textrm{\scriptsize 50}$,
\AtlasOrcid[0000-0001-6305-8400]{A.M.~Mendes~Jacques~Da~Costa}$^\textrm{\scriptsize 104}$,
\AtlasOrcid[0000-0002-7234-8351]{H.Y.~Meng}$^\textrm{\scriptsize 159}$,
\AtlasOrcid[0000-0002-2901-6589]{L.~Meng}$^\textrm{\scriptsize 94}$,
\AtlasOrcid[0000-0002-8186-4032]{S.~Menke}$^\textrm{\scriptsize 113}$,
\AtlasOrcid[0000-0001-9769-0578]{M.~Mentink}$^\textrm{\scriptsize 37}$,
\AtlasOrcid[0000-0002-6934-3752]{E.~Meoni}$^\textrm{\scriptsize 45b,45a}$,
\AtlasOrcid[0009-0009-4494-6045]{G.~Mercado}$^\textrm{\scriptsize 119}$,
\AtlasOrcid[0000-0001-6512-0036]{S.~Merianos}$^\textrm{\scriptsize 156}$,
\AtlasOrcid[0000-0002-5445-5938]{C.~Merlassino}$^\textrm{\scriptsize 71a,71c}$,
\AtlasOrcid[0000-0002-1822-1114]{L.~Merola}$^\textrm{\scriptsize 74a,74b}$,
\AtlasOrcid[0000-0003-4779-3522]{C.~Meroni}$^\textrm{\scriptsize 73a,73b}$,
\AtlasOrcid[0000-0001-5454-3017]{J.~Metcalfe}$^\textrm{\scriptsize 6}$,
\AtlasOrcid[0000-0002-5508-530X]{A.S.~Mete}$^\textrm{\scriptsize 6}$,
\AtlasOrcid[0000-0002-0473-2116]{E.~Meuser}$^\textrm{\scriptsize 103}$,
\AtlasOrcid[0000-0003-3552-6566]{C.~Meyer}$^\textrm{\scriptsize 70}$,
\AtlasOrcid[0000-0002-7497-0945]{J-P.~Meyer}$^\textrm{\scriptsize 139}$,
\AtlasOrcid[0000-0002-8396-9946]{R.P.~Middleton}$^\textrm{\scriptsize 138}$,
\AtlasOrcid[0000-0003-0162-2891]{L.~Mijovi\'{c}}$^\textrm{\scriptsize 54}$,
\AtlasOrcid[0000-0003-0460-3178]{G.~Mikenberg}$^\textrm{\scriptsize 174}$,
\AtlasOrcid[0000-0003-1277-2596]{M.~Mikestikova}$^\textrm{\scriptsize 135}$,
\AtlasOrcid[0000-0002-4119-6156]{M.~Miku\v{z}}$^\textrm{\scriptsize 96}$,
\AtlasOrcid[0000-0002-0384-6955]{H.~Mildner}$^\textrm{\scriptsize 103}$,
\AtlasOrcid[0000-0002-9173-8363]{A.~Milic}$^\textrm{\scriptsize 37}$,
\AtlasOrcid[0000-0002-9485-9435]{D.W.~Miller}$^\textrm{\scriptsize 41}$,
\AtlasOrcid[0000-0002-7083-1585]{E.H.~Miller}$^\textrm{\scriptsize 147}$,
\AtlasOrcid[0000-0001-5539-3233]{L.S.~Miller}$^\textrm{\scriptsize 35}$,
\AtlasOrcid[0000-0003-3863-3607]{A.~Milov}$^\textrm{\scriptsize 174}$,
\AtlasOrcid{D.A.~Milstead}$^\textrm{\scriptsize 49a,49b}$,
\AtlasOrcid{T.~Min}$^\textrm{\scriptsize 115a}$,
\AtlasOrcid[0000-0001-8055-4692]{A.A.~Minaenko}$^\textrm{\scriptsize 38}$,
\AtlasOrcid[0000-0002-4688-3510]{I.A.~Minashvili}$^\textrm{\scriptsize 153b}$,
\AtlasOrcid[0000-0002-6307-1418]{A.I.~Mincer}$^\textrm{\scriptsize 121}$,
\AtlasOrcid[0000-0002-5511-2611]{B.~Mindur}$^\textrm{\scriptsize 88a}$,
\AtlasOrcid[0000-0002-2236-3879]{M.~Mineev}$^\textrm{\scriptsize 40}$,
\AtlasOrcid[0000-0002-2984-8174]{Y.~Mino}$^\textrm{\scriptsize 90}$,
\AtlasOrcid[0000-0002-4276-715X]{L.M.~Mir}$^\textrm{\scriptsize 13}$,
\AtlasOrcid[0000-0001-7863-583X]{M.~Miralles~Lopez}$^\textrm{\scriptsize 61}$,
\AtlasOrcid[0000-0001-6381-5723]{M.~Mironova}$^\textrm{\scriptsize 18a}$,
\AtlasOrcid[0000-0002-0494-9753]{M.C.~Missio}$^\textrm{\scriptsize 117}$,
\AtlasOrcid[0000-0003-3714-0915]{A.~Mitra}$^\textrm{\scriptsize 172}$,
\AtlasOrcid[0000-0002-1533-8886]{V.A.~Mitsou}$^\textrm{\scriptsize 168}$,
\AtlasOrcid[0000-0003-4863-3272]{Y.~Mitsumori}$^\textrm{\scriptsize 114}$,
\AtlasOrcid[0000-0002-0287-8293]{O.~Miu}$^\textrm{\scriptsize 159}$,
\AtlasOrcid[0000-0002-4893-6778]{P.S.~Miyagawa}$^\textrm{\scriptsize 97}$,
\AtlasOrcid[0000-0002-5786-3136]{T.~Mkrtchyan}$^\textrm{\scriptsize 65a}$,
\AtlasOrcid[0000-0003-3587-646X]{M.~Mlinarevic}$^\textrm{\scriptsize 99}$,
\AtlasOrcid[0000-0002-6399-1732]{T.~Mlinarevic}$^\textrm{\scriptsize 99}$,
\AtlasOrcid[0000-0003-2028-1930]{M.~Mlynarikova}$^\textrm{\scriptsize 37}$,
\AtlasOrcid[0000-0001-5911-6815]{S.~Mobius}$^\textrm{\scriptsize 20}$,
\AtlasOrcid[0000-0003-2688-234X]{P.~Mogg}$^\textrm{\scriptsize 112}$,
\AtlasOrcid[0000-0002-2082-8134]{M.H.~Mohamed~Farook}$^\textrm{\scriptsize 116}$,
\AtlasOrcid[0000-0002-5003-1919]{A.F.~Mohammed}$^\textrm{\scriptsize 14,115c}$,
\AtlasOrcid[0000-0003-3006-6337]{S.~Mohapatra}$^\textrm{\scriptsize 43}$,
\AtlasOrcid[0000-0001-9878-4373]{G.~Mokgatitswane}$^\textrm{\scriptsize 34g}$,
\AtlasOrcid[0000-0003-0196-3602]{L.~Moleri}$^\textrm{\scriptsize 174}$,
\AtlasOrcid[0000-0003-1025-3741]{B.~Mondal}$^\textrm{\scriptsize 145}$,
\AtlasOrcid[0000-0002-6965-7380]{S.~Mondal}$^\textrm{\scriptsize 136}$,
\AtlasOrcid[0000-0002-3169-7117]{K.~M\"onig}$^\textrm{\scriptsize 50}$,
\AtlasOrcid[0000-0002-2551-5751]{E.~Monnier}$^\textrm{\scriptsize 105}$,
\AtlasOrcid{L.~Monsonis~Romero}$^\textrm{\scriptsize 168}$,
\AtlasOrcid[0000-0001-9213-904X]{J.~Montejo~Berlingen}$^\textrm{\scriptsize 13}$,
\AtlasOrcid[0000-0002-5578-6333]{A.~Montella}$^\textrm{\scriptsize 49a,49b}$,
\AtlasOrcid[0000-0001-5010-886X]{M.~Montella}$^\textrm{\scriptsize 123}$,
\AtlasOrcid[0000-0002-9939-8543]{F.~Montereali}$^\textrm{\scriptsize 79a,79b}$,
\AtlasOrcid[0000-0002-6974-1443]{F.~Monticelli}$^\textrm{\scriptsize 93}$,
\AtlasOrcid[0000-0002-0479-2207]{S.~Monzani}$^\textrm{\scriptsize 71a,71c}$,
\AtlasOrcid[0000-0002-4870-4758]{A.~Morancho~Tarda}$^\textrm{\scriptsize 44}$,
\AtlasOrcid[0000-0003-0047-7215]{N.~Morange}$^\textrm{\scriptsize 68}$,
\AtlasOrcid[0000-0002-1986-5720]{A.L.~Moreira~De~Carvalho}$^\textrm{\scriptsize 50}$,
\AtlasOrcid[0000-0003-1113-3645]{M.~Moreno~Ll\'acer}$^\textrm{\scriptsize 168}$,
\AtlasOrcid[0000-0002-5719-7655]{C.~Moreno~Martinez}$^\textrm{\scriptsize 58}$,
\AtlasOrcid{J.M.~Moreno~Perez}$^\textrm{\scriptsize 23b}$,
\AtlasOrcid[0000-0001-7139-7912]{P.~Morettini}$^\textrm{\scriptsize 59b}$,
\AtlasOrcid[0000-0002-7834-4781]{S.~Morgenstern}$^\textrm{\scriptsize 37}$,
\AtlasOrcid[0000-0001-9324-057X]{M.~Morii}$^\textrm{\scriptsize 63}$,
\AtlasOrcid[0000-0003-2129-1372]{M.~Morinaga}$^\textrm{\scriptsize 157}$,
\AtlasOrcid[0000-0001-6357-0543]{M.~Moritsu}$^\textrm{\scriptsize 91}$,
\AtlasOrcid[0000-0001-8251-7262]{F.~Morodei}$^\textrm{\scriptsize 77a,77b}$,
\AtlasOrcid[0000-0001-6993-9698]{P.~Moschovakos}$^\textrm{\scriptsize 37}$,
\AtlasOrcid[0000-0001-6750-5060]{B.~Moser}$^\textrm{\scriptsize 130}$,
\AtlasOrcid[0000-0002-1720-0493]{M.~Mosidze}$^\textrm{\scriptsize 153b}$,
\AtlasOrcid[0000-0001-6508-3968]{T.~Moskalets}$^\textrm{\scriptsize 46}$,
\AtlasOrcid[0000-0002-7926-7650]{P.~Moskvitina}$^\textrm{\scriptsize 117}$,
\AtlasOrcid[0000-0002-6729-4803]{J.~Moss}$^\textrm{\scriptsize 32,j}$,
\AtlasOrcid[0000-0001-5269-6191]{P.~Moszkowicz}$^\textrm{\scriptsize 88a}$,
\AtlasOrcid[0000-0003-2233-9120]{A.~Moussa}$^\textrm{\scriptsize 36d}$,
\AtlasOrcid{Y.~Moyal}$^\textrm{\scriptsize 174}$,
\AtlasOrcid[0000-0003-4449-6178]{E.J.W.~Moyse}$^\textrm{\scriptsize 106}$,
\AtlasOrcid[0000-0003-2168-4854]{O.~Mtintsilana}$^\textrm{\scriptsize 34g}$,
\AtlasOrcid[0000-0002-1786-2075]{S.~Muanza}$^\textrm{\scriptsize 105}$,
\AtlasOrcid[0000-0001-5099-4718]{J.~Mueller}$^\textrm{\scriptsize 133}$,
\AtlasOrcid[0000-0001-6223-2497]{D.~Muenstermann}$^\textrm{\scriptsize 94}$,
\AtlasOrcid[0000-0002-5835-0690]{R.~M\"uller}$^\textrm{\scriptsize 37}$,
\AtlasOrcid[0000-0001-6771-0937]{G.A.~Mullier}$^\textrm{\scriptsize 166}$,
\AtlasOrcid{A.J.~Mullin}$^\textrm{\scriptsize 33}$,
\AtlasOrcid{J.J.~Mullin}$^\textrm{\scriptsize 132}$,
\AtlasOrcid[0000-0001-6187-9344]{A.E.~Mulski}$^\textrm{\scriptsize 63}$,
\AtlasOrcid[0000-0002-2567-7857]{D.P.~Mungo}$^\textrm{\scriptsize 159}$,
\AtlasOrcid[0000-0003-3215-6467]{D.~Munoz~Perez}$^\textrm{\scriptsize 168}$,
\AtlasOrcid[0000-0002-6374-458X]{F.J.~Munoz~Sanchez}$^\textrm{\scriptsize 104}$,
\AtlasOrcid[0000-0002-2388-1969]{M.~Murin}$^\textrm{\scriptsize 104}$,
\AtlasOrcid[0000-0003-1710-6306]{W.J.~Murray}$^\textrm{\scriptsize 172,138}$,
\AtlasOrcid[0000-0001-8442-2718]{M.~Mu\v{s}kinja}$^\textrm{\scriptsize 96}$,
\AtlasOrcid[0000-0002-3504-0366]{C.~Mwewa}$^\textrm{\scriptsize 30}$,
\AtlasOrcid[0000-0003-4189-4250]{A.G.~Myagkov}$^\textrm{\scriptsize 38,a}$,
\AtlasOrcid[0000-0003-1691-4643]{A.J.~Myers}$^\textrm{\scriptsize 8}$,
\AtlasOrcid[0000-0002-2562-0930]{G.~Myers}$^\textrm{\scriptsize 109}$,
\AtlasOrcid[0000-0003-0982-3380]{M.~Myska}$^\textrm{\scriptsize 136}$,
\AtlasOrcid[0000-0003-1024-0932]{B.P.~Nachman}$^\textrm{\scriptsize 18a}$,
\AtlasOrcid[0000-0002-4285-0578]{K.~Nagai}$^\textrm{\scriptsize 130}$,
\AtlasOrcid[0000-0003-2741-0627]{K.~Nagano}$^\textrm{\scriptsize 86}$,
\AtlasOrcid[0000-0003-0056-6613]{J.L.~Nagle}$^\textrm{\scriptsize 30,ad}$,
\AtlasOrcid[0000-0001-5420-9537]{E.~Nagy}$^\textrm{\scriptsize 105}$,
\AtlasOrcid[0000-0003-3561-0880]{A.M.~Nairz}$^\textrm{\scriptsize 37}$,
\AtlasOrcid[0000-0003-3133-7100]{Y.~Nakahama}$^\textrm{\scriptsize 86}$,
\AtlasOrcid[0000-0002-1560-0434]{K.~Nakamura}$^\textrm{\scriptsize 86}$,
\AtlasOrcid[0000-0002-5662-3907]{K.~Nakkalil}$^\textrm{\scriptsize 5}$,
\AtlasOrcid[0000-0003-0703-103X]{H.~Nanjo}$^\textrm{\scriptsize 128}$,
\AtlasOrcid[0000-0001-6042-6781]{E.A.~Narayanan}$^\textrm{\scriptsize 116}$,
\AtlasOrcid[0000-0001-6412-4801]{I.~Naryshkin}$^\textrm{\scriptsize 38}$,
\AtlasOrcid[0000-0002-4871-784X]{L.~Nasella}$^\textrm{\scriptsize 73a,73b}$,
\AtlasOrcid[0000-0002-5985-4567]{S.~Nasri}$^\textrm{\scriptsize 120b}$,
\AtlasOrcid[0000-0002-8098-4948]{C.~Nass}$^\textrm{\scriptsize 25}$,
\AtlasOrcid[0000-0002-5108-0042]{G.~Navarro}$^\textrm{\scriptsize 23a}$,
\AtlasOrcid[0000-0002-4172-7965]{J.~Navarro-Gonzalez}$^\textrm{\scriptsize 168}$,
\AtlasOrcid[0000-0001-6988-0606]{R.~Nayak}$^\textrm{\scriptsize 155}$,
\AtlasOrcid[0000-0003-1418-3437]{A.~Nayaz}$^\textrm{\scriptsize 19}$,
\AtlasOrcid[0000-0002-5910-4117]{P.Y.~Nechaeva}$^\textrm{\scriptsize 38}$,
\AtlasOrcid[0000-0002-0623-9034]{S.~Nechaeva}$^\textrm{\scriptsize 24b,24a}$,
\AtlasOrcid[0000-0002-2684-9024]{F.~Nechansky}$^\textrm{\scriptsize 135}$,
\AtlasOrcid[0000-0002-7672-7367]{L.~Nedic}$^\textrm{\scriptsize 130}$,
\AtlasOrcid[0000-0003-0056-8651]{T.J.~Neep}$^\textrm{\scriptsize 21}$,
\AtlasOrcid[0000-0002-7386-901X]{A.~Negri}$^\textrm{\scriptsize 75a,75b}$,
\AtlasOrcid[0000-0003-0101-6963]{M.~Negrini}$^\textrm{\scriptsize 24b}$,
\AtlasOrcid[0000-0002-5171-8579]{C.~Nellist}$^\textrm{\scriptsize 118}$,
\AtlasOrcid[0000-0002-5713-3803]{C.~Nelson}$^\textrm{\scriptsize 107}$,
\AtlasOrcid[0000-0003-4194-1790]{K.~Nelson}$^\textrm{\scriptsize 109}$,
\AtlasOrcid[0000-0001-8978-7150]{S.~Nemecek}$^\textrm{\scriptsize 135}$,
\AtlasOrcid[0000-0001-7316-0118]{M.~Nessi}$^\textrm{\scriptsize 37,g}$,
\AtlasOrcid[0000-0001-8434-9274]{M.S.~Neubauer}$^\textrm{\scriptsize 167}$,
\AtlasOrcid[0000-0002-3819-2453]{F.~Neuhaus}$^\textrm{\scriptsize 103}$,
\AtlasOrcid[0000-0002-8565-0015]{J.~Neundorf}$^\textrm{\scriptsize 50}$,
\AtlasOrcid[0000-0001-6917-2802]{J.~Newell}$^\textrm{\scriptsize 95}$,
\AtlasOrcid[0000-0002-6252-266X]{P.R.~Newman}$^\textrm{\scriptsize 21}$,
\AtlasOrcid[0000-0001-8190-4017]{C.W.~Ng}$^\textrm{\scriptsize 133}$,
\AtlasOrcid[0000-0001-9135-1321]{Y.W.Y.~Ng}$^\textrm{\scriptsize 50}$,
\AtlasOrcid[0000-0002-5807-8535]{B.~Ngair}$^\textrm{\scriptsize 120a}$,
\AtlasOrcid[0000-0002-4326-9283]{H.D.N.~Nguyen}$^\textrm{\scriptsize 111}$,
\AtlasOrcid[0000-0002-2157-9061]{R.B.~Nickerson}$^\textrm{\scriptsize 130}$,
\AtlasOrcid[0000-0003-3723-1745]{R.~Nicolaidou}$^\textrm{\scriptsize 139}$,
\AtlasOrcid[0000-0002-9175-4419]{J.~Nielsen}$^\textrm{\scriptsize 140}$,
\AtlasOrcid[0000-0003-4222-8284]{M.~Niemeyer}$^\textrm{\scriptsize 57}$,
\AtlasOrcid[0000-0003-0069-8907]{J.~Niermann}$^\textrm{\scriptsize 57}$,
\AtlasOrcid[0000-0003-1267-7740]{N.~Nikiforou}$^\textrm{\scriptsize 37}$,
\AtlasOrcid[0000-0001-6545-1820]{V.~Nikolaenko}$^\textrm{\scriptsize 38,a}$,
\AtlasOrcid[0000-0003-1681-1118]{I.~Nikolic-Audit}$^\textrm{\scriptsize 131}$,
\AtlasOrcid[0000-0002-3048-489X]{K.~Nikolopoulos}$^\textrm{\scriptsize 21}$,
\AtlasOrcid[0000-0002-6848-7463]{P.~Nilsson}$^\textrm{\scriptsize 30}$,
\AtlasOrcid[0000-0001-8158-8966]{I.~Ninca}$^\textrm{\scriptsize 50}$,
\AtlasOrcid[0000-0003-4014-7253]{G.~Ninio}$^\textrm{\scriptsize 155}$,
\AtlasOrcid[0000-0002-5080-2293]{A.~Nisati}$^\textrm{\scriptsize 77a}$,
\AtlasOrcid[0000-0002-9048-1332]{N.~Nishu}$^\textrm{\scriptsize 2}$,
\AtlasOrcid[0000-0003-2257-0074]{R.~Nisius}$^\textrm{\scriptsize 113}$,
\AtlasOrcid[0000-0003-0576-3122]{N.~Nitika}$^\textrm{\scriptsize 71a,71c}$,
\AtlasOrcid[0000-0002-0174-4816]{J-E.~Nitschke}$^\textrm{\scriptsize 52}$,
\AtlasOrcid[0000-0003-0800-7963]{E.K.~Nkadimeng}$^\textrm{\scriptsize 34g}$,
\AtlasOrcid[0000-0002-5809-325X]{T.~Nobe}$^\textrm{\scriptsize 157}$,
\AtlasOrcid[0000-0002-4542-6385]{T.~Nommensen}$^\textrm{\scriptsize 151}$,
\AtlasOrcid[0000-0001-7984-5783]{M.B.~Norfolk}$^\textrm{\scriptsize 143}$,
\AtlasOrcid[0000-0002-5736-1398]{B.J.~Norman}$^\textrm{\scriptsize 35}$,
\AtlasOrcid[0000-0003-0371-1521]{M.~Noury}$^\textrm{\scriptsize 36a}$,
\AtlasOrcid[0000-0002-3195-8903]{J.~Novak}$^\textrm{\scriptsize 96}$,
\AtlasOrcid[0000-0002-3053-0913]{T.~Novak}$^\textrm{\scriptsize 96}$,
\AtlasOrcid[0000-0001-5165-8425]{L.~Novotny}$^\textrm{\scriptsize 136}$,
\AtlasOrcid[0000-0002-1630-694X]{R.~Novotny}$^\textrm{\scriptsize 116}$,
\AtlasOrcid[0000-0002-8774-7099]{L.~Nozka}$^\textrm{\scriptsize 126}$,
\AtlasOrcid[0000-0001-9252-6509]{K.~Ntekas}$^\textrm{\scriptsize 163}$,
\AtlasOrcid[0000-0003-0828-6085]{N.M.J.~Nunes~De~Moura~Junior}$^\textrm{\scriptsize 85b}$,
\AtlasOrcid[0000-0003-2262-0780]{J.~Ocariz}$^\textrm{\scriptsize 131}$,
\AtlasOrcid[0000-0002-2024-5609]{A.~Ochi}$^\textrm{\scriptsize 87}$,
\AtlasOrcid[0000-0001-6156-1790]{I.~Ochoa}$^\textrm{\scriptsize 134a}$,
\AtlasOrcid[0000-0001-8763-0096]{S.~Oerdek}$^\textrm{\scriptsize 50,t}$,
\AtlasOrcid[0000-0002-6468-518X]{J.T.~Offermann}$^\textrm{\scriptsize 41}$,
\AtlasOrcid[0000-0002-6025-4833]{A.~Ogrodnik}$^\textrm{\scriptsize 137}$,
\AtlasOrcid[0000-0001-9025-0422]{A.~Oh}$^\textrm{\scriptsize 104}$,
\AtlasOrcid[0000-0002-8015-7512]{C.C.~Ohm}$^\textrm{\scriptsize 148}$,
\AtlasOrcid[0000-0002-2173-3233]{H.~Oide}$^\textrm{\scriptsize 86}$,
\AtlasOrcid[0000-0001-6930-7789]{R.~Oishi}$^\textrm{\scriptsize 157}$,
\AtlasOrcid[0000-0002-3834-7830]{M.L.~Ojeda}$^\textrm{\scriptsize 37}$,
\AtlasOrcid[0000-0002-7613-5572]{Y.~Okumura}$^\textrm{\scriptsize 157}$,
\AtlasOrcid[0000-0002-9320-8825]{L.F.~Oleiro~Seabra}$^\textrm{\scriptsize 134a}$,
\AtlasOrcid[0000-0002-4784-6340]{I.~Oleksiyuk}$^\textrm{\scriptsize 58}$,
\AtlasOrcid[0000-0003-4616-6973]{S.A.~Olivares~Pino}$^\textrm{\scriptsize 141d}$,
\AtlasOrcid[0000-0003-0700-0030]{G.~Oliveira~Correa}$^\textrm{\scriptsize 13}$,
\AtlasOrcid[0000-0002-8601-2074]{D.~Oliveira~Damazio}$^\textrm{\scriptsize 30}$,
\AtlasOrcid[0000-0002-0713-6627]{J.L.~Oliver}$^\textrm{\scriptsize 163}$,
\AtlasOrcid[0000-0001-8772-1705]{\"O.O.~\"Oncel}$^\textrm{\scriptsize 56}$,
\AtlasOrcid[0000-0002-8104-7227]{A.P.~O'Neill}$^\textrm{\scriptsize 20}$,
\AtlasOrcid[0000-0003-3471-2703]{A.~Onofre}$^\textrm{\scriptsize 134a,134e}$,
\AtlasOrcid[0000-0003-4201-7997]{P.U.E.~Onyisi}$^\textrm{\scriptsize 11}$,
\AtlasOrcid[0000-0001-6203-2209]{M.J.~Oreglia}$^\textrm{\scriptsize 41}$,
\AtlasOrcid[0000-0001-5103-5527]{D.~Orestano}$^\textrm{\scriptsize 79a,79b}$,
\AtlasOrcid[0000-0003-0616-245X]{N.~Orlando}$^\textrm{\scriptsize 13}$,
\AtlasOrcid[0000-0002-8690-9746]{R.S.~Orr}$^\textrm{\scriptsize 159}$,
\AtlasOrcid[0000-0002-9538-0514]{L.M.~Osojnak}$^\textrm{\scriptsize 132}$,
\AtlasOrcid[0000-0001-5091-9216]{R.~Ospanov}$^\textrm{\scriptsize 64a}$,
\AtlasOrcid[0000-0003-4803-5280]{G.~Otero~y~Garzon}$^\textrm{\scriptsize 31}$,
\AtlasOrcid[0000-0003-0760-5988]{H.~Otono}$^\textrm{\scriptsize 91}$,
\AtlasOrcid[0000-0003-1052-7925]{P.S.~Ott}$^\textrm{\scriptsize 65a}$,
\AtlasOrcid[0000-0001-8083-6411]{G.J.~Ottino}$^\textrm{\scriptsize 18a}$,
\AtlasOrcid[0000-0002-2954-1420]{M.~Ouchrif}$^\textrm{\scriptsize 36d}$,
\AtlasOrcid[0000-0002-9404-835X]{F.~Ould-Saada}$^\textrm{\scriptsize 129}$,
\AtlasOrcid[0000-0002-3890-9426]{T.~Ovsiannikova}$^\textrm{\scriptsize 142}$,
\AtlasOrcid[0000-0001-6820-0488]{M.~Owen}$^\textrm{\scriptsize 61}$,
\AtlasOrcid[0000-0002-2684-1399]{R.E.~Owen}$^\textrm{\scriptsize 138}$,
\AtlasOrcid[0000-0003-4643-6347]{V.E.~Ozcan}$^\textrm{\scriptsize 22a}$,
\AtlasOrcid[0000-0003-2481-8176]{F.~Ozturk}$^\textrm{\scriptsize 89}$,
\AtlasOrcid[0000-0003-1125-6784]{N.~Ozturk}$^\textrm{\scriptsize 8}$,
\AtlasOrcid[0000-0001-6533-6144]{S.~Ozturk}$^\textrm{\scriptsize 84}$,
\AtlasOrcid[0000-0002-2325-6792]{H.A.~Pacey}$^\textrm{\scriptsize 130}$,
\AtlasOrcid[0000-0001-8210-1734]{A.~Pacheco~Pages}$^\textrm{\scriptsize 13}$,
\AtlasOrcid[0000-0001-7951-0166]{C.~Padilla~Aranda}$^\textrm{\scriptsize 13}$,
\AtlasOrcid[0000-0003-0014-3901]{G.~Padovano}$^\textrm{\scriptsize 77a,77b}$,
\AtlasOrcid[0000-0003-0999-5019]{S.~Pagan~Griso}$^\textrm{\scriptsize 18a}$,
\AtlasOrcid[0000-0003-0278-9941]{G.~Palacino}$^\textrm{\scriptsize 70}$,
\AtlasOrcid[0000-0001-9794-2851]{A.~Palazzo}$^\textrm{\scriptsize 72a,72b}$,
\AtlasOrcid[0000-0001-8648-4891]{J.~Pampel}$^\textrm{\scriptsize 25}$,
\AtlasOrcid[0000-0002-0664-9199]{J.~Pan}$^\textrm{\scriptsize 177}$,
\AtlasOrcid[0000-0002-4700-1516]{T.~Pan}$^\textrm{\scriptsize 66a}$,
\AtlasOrcid[0000-0001-5732-9948]{D.K.~Panchal}$^\textrm{\scriptsize 11}$,
\AtlasOrcid[0000-0003-3838-1307]{C.E.~Pandini}$^\textrm{\scriptsize 118}$,
\AtlasOrcid[0000-0003-2605-8940]{J.G.~Panduro~Vazquez}$^\textrm{\scriptsize 138}$,
\AtlasOrcid[0000-0002-1199-945X]{H.D.~Pandya}$^\textrm{\scriptsize 1}$,
\AtlasOrcid[0000-0002-1946-1769]{H.~Pang}$^\textrm{\scriptsize 15}$,
\AtlasOrcid[0000-0003-2149-3791]{P.~Pani}$^\textrm{\scriptsize 50}$,
\AtlasOrcid[0000-0002-0352-4833]{G.~Panizzo}$^\textrm{\scriptsize 71a,71c}$,
\AtlasOrcid[0000-0003-2461-4907]{L.~Panwar}$^\textrm{\scriptsize 131}$,
\AtlasOrcid[0000-0002-9281-1972]{L.~Paolozzi}$^\textrm{\scriptsize 58}$,
\AtlasOrcid[0000-0003-1499-3990]{S.~Parajuli}$^\textrm{\scriptsize 167}$,
\AtlasOrcid[0000-0002-6492-3061]{A.~Paramonov}$^\textrm{\scriptsize 6}$,
\AtlasOrcid[0000-0002-2858-9182]{C.~Paraskevopoulos}$^\textrm{\scriptsize 55}$,
\AtlasOrcid[0000-0002-3179-8524]{D.~Paredes~Hernandez}$^\textrm{\scriptsize 66b}$,
\AtlasOrcid[0000-0003-3028-4895]{A.~Pareti}$^\textrm{\scriptsize 75a,75b}$,
\AtlasOrcid[0009-0003-6804-4288]{K.R.~Park}$^\textrm{\scriptsize 43}$,
\AtlasOrcid[0000-0002-1910-0541]{T.H.~Park}$^\textrm{\scriptsize 159}$,
\AtlasOrcid[0000-0001-9798-8411]{M.A.~Parker}$^\textrm{\scriptsize 33}$,
\AtlasOrcid[0000-0002-7160-4720]{F.~Parodi}$^\textrm{\scriptsize 59b,59a}$,
\AtlasOrcid[0000-0001-5164-9414]{V.A.~Parrish}$^\textrm{\scriptsize 54}$,
\AtlasOrcid[0000-0002-9470-6017]{J.A.~Parsons}$^\textrm{\scriptsize 43}$,
\AtlasOrcid[0000-0002-4858-6560]{U.~Parzefall}$^\textrm{\scriptsize 56}$,
\AtlasOrcid[0000-0002-7673-1067]{B.~Pascual~Dias}$^\textrm{\scriptsize 111}$,
\AtlasOrcid[0000-0003-4701-9481]{L.~Pascual~Dominguez}$^\textrm{\scriptsize 102}$,
\AtlasOrcid[0000-0001-8160-2545]{E.~Pasqualucci}$^\textrm{\scriptsize 77a}$,
\AtlasOrcid[0000-0001-9200-5738]{S.~Passaggio}$^\textrm{\scriptsize 59b}$,
\AtlasOrcid[0000-0001-5962-7826]{F.~Pastore}$^\textrm{\scriptsize 98}$,
\AtlasOrcid[0000-0002-7467-2470]{P.~Patel}$^\textrm{\scriptsize 89}$,
\AtlasOrcid[0000-0001-5191-2526]{U.M.~Patel}$^\textrm{\scriptsize 53}$,
\AtlasOrcid[0000-0002-0598-5035]{J.R.~Pater}$^\textrm{\scriptsize 104}$,
\AtlasOrcid[0000-0001-9082-035X]{T.~Pauly}$^\textrm{\scriptsize 37}$,
\AtlasOrcid{F.~Pauwels}$^\textrm{\scriptsize 137}$,
\AtlasOrcid[0000-0001-8533-3805]{C.I.~Pazos}$^\textrm{\scriptsize 162}$,
\AtlasOrcid[0000-0003-4281-0119]{M.~Pedersen}$^\textrm{\scriptsize 129}$,
\AtlasOrcid[0000-0002-7139-9587]{R.~Pedro}$^\textrm{\scriptsize 134a}$,
\AtlasOrcid[0000-0003-0907-7592]{S.V.~Peleganchuk}$^\textrm{\scriptsize 38}$,
\AtlasOrcid[0000-0002-5433-3981]{O.~Penc}$^\textrm{\scriptsize 37}$,
\AtlasOrcid[0009-0002-8629-4486]{E.A.~Pender}$^\textrm{\scriptsize 54}$,
\AtlasOrcid{S.~Peng}$^\textrm{\scriptsize 15}$,
\AtlasOrcid[0000-0002-6956-9970]{G.D.~Penn}$^\textrm{\scriptsize 177}$,
\AtlasOrcid[0000-0002-8082-424X]{K.E.~Penski}$^\textrm{\scriptsize 112}$,
\AtlasOrcid[0000-0002-0928-3129]{M.~Penzin}$^\textrm{\scriptsize 38}$,
\AtlasOrcid[0000-0003-1664-5658]{B.S.~Peralva}$^\textrm{\scriptsize 85d}$,
\AtlasOrcid[0000-0003-3424-7338]{A.P.~Pereira~Peixoto}$^\textrm{\scriptsize 142}$,
\AtlasOrcid[0000-0001-7913-3313]{L.~Pereira~Sanchez}$^\textrm{\scriptsize 147}$,
\AtlasOrcid[0000-0001-8732-6908]{D.V.~Perepelitsa}$^\textrm{\scriptsize 30,ad}$,
\AtlasOrcid[0000-0001-7292-2547]{G.~Perera}$^\textrm{\scriptsize 106}$,
\AtlasOrcid[0000-0003-0426-6538]{E.~Perez~Codina}$^\textrm{\scriptsize 160a}$,
\AtlasOrcid[0000-0003-3451-9938]{M.~Perganti}$^\textrm{\scriptsize 10}$,
\AtlasOrcid[0000-0001-6418-8784]{H.~Pernegger}$^\textrm{\scriptsize 37}$,
\AtlasOrcid[0000-0003-4955-5130]{S.~Perrella}$^\textrm{\scriptsize 77a,77b}$,
\AtlasOrcid[0000-0003-2078-6541]{O.~Perrin}$^\textrm{\scriptsize 42}$,
\AtlasOrcid[0000-0002-7654-1677]{K.~Peters}$^\textrm{\scriptsize 50}$,
\AtlasOrcid[0000-0003-1702-7544]{R.F.Y.~Peters}$^\textrm{\scriptsize 104}$,
\AtlasOrcid[0000-0002-7380-6123]{B.A.~Petersen}$^\textrm{\scriptsize 37}$,
\AtlasOrcid[0000-0003-0221-3037]{T.C.~Petersen}$^\textrm{\scriptsize 44}$,
\AtlasOrcid[0000-0002-3059-735X]{E.~Petit}$^\textrm{\scriptsize 105}$,
\AtlasOrcid[0000-0002-5575-6476]{V.~Petousis}$^\textrm{\scriptsize 136}$,
\AtlasOrcid[0000-0001-5957-6133]{C.~Petridou}$^\textrm{\scriptsize 156,d}$,
\AtlasOrcid[0000-0003-4903-9419]{T.~Petru}$^\textrm{\scriptsize 137}$,
\AtlasOrcid[0000-0003-0533-2277]{A.~Petrukhin}$^\textrm{\scriptsize 145}$,
\AtlasOrcid[0000-0001-9208-3218]{M.~Pettee}$^\textrm{\scriptsize 18a}$,
\AtlasOrcid[0000-0002-8126-9575]{A.~Petukhov}$^\textrm{\scriptsize 38}$,
\AtlasOrcid[0000-0002-0654-8398]{K.~Petukhova}$^\textrm{\scriptsize 37}$,
\AtlasOrcid[0000-0003-3344-791X]{R.~Pezoa}$^\textrm{\scriptsize 141f}$,
\AtlasOrcid[0000-0002-3802-8944]{L.~Pezzotti}$^\textrm{\scriptsize 37}$,
\AtlasOrcid[0000-0002-6653-1555]{G.~Pezzullo}$^\textrm{\scriptsize 177}$,
\AtlasOrcid[0000-0001-5524-7738]{A.J.~Pfleger}$^\textrm{\scriptsize 37}$,
\AtlasOrcid[0000-0003-2436-6317]{T.M.~Pham}$^\textrm{\scriptsize 175}$,
\AtlasOrcid[0000-0002-8859-1313]{T.~Pham}$^\textrm{\scriptsize 108}$,
\AtlasOrcid[0000-0003-3651-4081]{P.W.~Phillips}$^\textrm{\scriptsize 138}$,
\AtlasOrcid[0000-0002-4531-2900]{G.~Piacquadio}$^\textrm{\scriptsize 149}$,
\AtlasOrcid[0000-0001-9233-5892]{E.~Pianori}$^\textrm{\scriptsize 18a}$,
\AtlasOrcid[0000-0002-3664-8912]{F.~Piazza}$^\textrm{\scriptsize 127}$,
\AtlasOrcid[0000-0001-7850-8005]{R.~Piegaia}$^\textrm{\scriptsize 31}$,
\AtlasOrcid[0000-0003-1381-5949]{D.~Pietreanu}$^\textrm{\scriptsize 28b}$,
\AtlasOrcid[0000-0001-8007-0778]{A.D.~Pilkington}$^\textrm{\scriptsize 104}$,
\AtlasOrcid[0000-0002-5282-5050]{M.~Pinamonti}$^\textrm{\scriptsize 71a,71c}$,
\AtlasOrcid[0000-0002-2397-4196]{J.L.~Pinfold}$^\textrm{\scriptsize 2}$,
\AtlasOrcid[0000-0002-9639-7887]{B.C.~Pinheiro~Pereira}$^\textrm{\scriptsize 134a}$,
\AtlasOrcid[0000-0001-8524-1257]{J.~Pinol~Bel}$^\textrm{\scriptsize 13}$,
\AtlasOrcid[0000-0001-9616-1690]{A.E.~Pinto~Pinoargote}$^\textrm{\scriptsize 139,139}$,
\AtlasOrcid[0000-0001-9842-9830]{L.~Pintucci}$^\textrm{\scriptsize 71a,71c}$,
\AtlasOrcid[0000-0002-7669-4518]{K.M.~Piper}$^\textrm{\scriptsize 150}$,
\AtlasOrcid[0009-0002-3707-1446]{A.~Pirttikoski}$^\textrm{\scriptsize 58}$,
\AtlasOrcid[0000-0001-5193-1567]{D.A.~Pizzi}$^\textrm{\scriptsize 35}$,
\AtlasOrcid[0000-0002-1814-2758]{L.~Pizzimento}$^\textrm{\scriptsize 66b}$,
\AtlasOrcid[0000-0001-8891-1842]{A.~Pizzini}$^\textrm{\scriptsize 118}$,
\AtlasOrcid[0000-0002-9461-3494]{M.-A.~Pleier}$^\textrm{\scriptsize 30}$,
\AtlasOrcid[0000-0001-5435-497X]{V.~Pleskot}$^\textrm{\scriptsize 137}$,
\AtlasOrcid{E.~Plotnikova}$^\textrm{\scriptsize 40}$,
\AtlasOrcid[0000-0001-7424-4161]{G.~Poddar}$^\textrm{\scriptsize 97}$,
\AtlasOrcid[0000-0002-3304-0987]{R.~Poettgen}$^\textrm{\scriptsize 101}$,
\AtlasOrcid[0000-0003-3210-6646]{L.~Poggioli}$^\textrm{\scriptsize 131}$,
\AtlasOrcid[0000-0002-9929-9713]{S.~Polacek}$^\textrm{\scriptsize 137}$,
\AtlasOrcid[0000-0001-8636-0186]{G.~Polesello}$^\textrm{\scriptsize 75a}$,
\AtlasOrcid[0000-0002-4063-0408]{A.~Poley}$^\textrm{\scriptsize 146,160a}$,
\AtlasOrcid[0000-0002-4986-6628]{A.~Polini}$^\textrm{\scriptsize 24b}$,
\AtlasOrcid[0000-0002-3690-3960]{C.S.~Pollard}$^\textrm{\scriptsize 172}$,
\AtlasOrcid[0000-0001-6285-0658]{Z.B.~Pollock}$^\textrm{\scriptsize 123}$,
\AtlasOrcid[0000-0003-4528-6594]{E.~Pompa~Pacchi}$^\textrm{\scriptsize 77a,77b}$,
\AtlasOrcid[0000-0002-5966-0332]{N.I.~Pond}$^\textrm{\scriptsize 99}$,
\AtlasOrcid[0000-0003-4213-1511]{D.~Ponomarenko}$^\textrm{\scriptsize 70}$,
\AtlasOrcid[0000-0003-2284-3765]{L.~Pontecorvo}$^\textrm{\scriptsize 37}$,
\AtlasOrcid[0000-0001-9275-4536]{S.~Popa}$^\textrm{\scriptsize 28a}$,
\AtlasOrcid[0000-0001-9783-7736]{G.A.~Popeneciu}$^\textrm{\scriptsize 28d}$,
\AtlasOrcid[0000-0003-1250-0865]{A.~Poreba}$^\textrm{\scriptsize 37}$,
\AtlasOrcid[0000-0002-7042-4058]{D.M.~Portillo~Quintero}$^\textrm{\scriptsize 160a}$,
\AtlasOrcid[0000-0001-5424-9096]{S.~Pospisil}$^\textrm{\scriptsize 136}$,
\AtlasOrcid[0000-0002-0861-1776]{M.A.~Postill}$^\textrm{\scriptsize 143}$,
\AtlasOrcid[0000-0001-8797-012X]{P.~Postolache}$^\textrm{\scriptsize 28c}$,
\AtlasOrcid[0000-0001-7839-9785]{K.~Potamianos}$^\textrm{\scriptsize 172}$,
\AtlasOrcid[0000-0002-1325-7214]{P.A.~Potepa}$^\textrm{\scriptsize 88a}$,
\AtlasOrcid[0000-0002-0375-6909]{I.N.~Potrap}$^\textrm{\scriptsize 40}$,
\AtlasOrcid[0000-0002-9815-5208]{C.J.~Potter}$^\textrm{\scriptsize 33}$,
\AtlasOrcid[0000-0002-0800-9902]{H.~Potti}$^\textrm{\scriptsize 151}$,
\AtlasOrcid[0000-0001-8144-1964]{J.~Poveda}$^\textrm{\scriptsize 168}$,
\AtlasOrcid[0000-0002-3069-3077]{M.E.~Pozo~Astigarraga}$^\textrm{\scriptsize 37}$,
\AtlasOrcid[0000-0003-1418-2012]{A.~Prades~Ibanez}$^\textrm{\scriptsize 78a,78b}$,
\AtlasOrcid[0000-0001-7385-8874]{J.~Pretel}$^\textrm{\scriptsize 170}$,
\AtlasOrcid[0000-0003-2750-9977]{D.~Price}$^\textrm{\scriptsize 104}$,
\AtlasOrcid[0000-0002-6866-3818]{M.~Primavera}$^\textrm{\scriptsize 72a}$,
\AtlasOrcid[0000-0002-2699-9444]{L.~Primomo}$^\textrm{\scriptsize 71a,71c}$,
\AtlasOrcid[0000-0002-5085-2717]{M.A.~Principe~Martin}$^\textrm{\scriptsize 102}$,
\AtlasOrcid[0000-0002-2239-0586]{R.~Privara}$^\textrm{\scriptsize 126}$,
\AtlasOrcid[0000-0002-6534-9153]{T.~Procter}$^\textrm{\scriptsize 61}$,
\AtlasOrcid[0000-0003-0323-8252]{M.L.~Proffitt}$^\textrm{\scriptsize 142}$,
\AtlasOrcid[0000-0002-5237-0201]{N.~Proklova}$^\textrm{\scriptsize 132}$,
\AtlasOrcid[0000-0002-2177-6401]{K.~Prokofiev}$^\textrm{\scriptsize 66c}$,
\AtlasOrcid[0000-0002-3069-7297]{G.~Proto}$^\textrm{\scriptsize 113}$,
\AtlasOrcid[0000-0003-1032-9945]{J.~Proudfoot}$^\textrm{\scriptsize 6}$,
\AtlasOrcid[0000-0002-9235-2649]{M.~Przybycien}$^\textrm{\scriptsize 88a}$,
\AtlasOrcid[0000-0003-0984-0754]{W.W.~Przygoda}$^\textrm{\scriptsize 88b}$,
\AtlasOrcid[0000-0003-2901-6834]{A.~Psallidas}$^\textrm{\scriptsize 48}$,
\AtlasOrcid[0000-0001-9514-3597]{J.E.~Puddefoot}$^\textrm{\scriptsize 143}$,
\AtlasOrcid[0000-0002-7026-1412]{D.~Pudzha}$^\textrm{\scriptsize 56}$,
\AtlasOrcid[0000-0002-6659-8506]{D.~Pyatiizbyantseva}$^\textrm{\scriptsize 38}$,
\AtlasOrcid[0000-0003-4813-8167]{J.~Qian}$^\textrm{\scriptsize 109}$,
\AtlasOrcid[0000-0002-0117-7831]{D.~Qichen}$^\textrm{\scriptsize 104}$,
\AtlasOrcid[0000-0002-6960-502X]{Y.~Qin}$^\textrm{\scriptsize 13}$,
\AtlasOrcid[0000-0001-5047-3031]{T.~Qiu}$^\textrm{\scriptsize 54}$,
\AtlasOrcid[0000-0002-0098-384X]{A.~Quadt}$^\textrm{\scriptsize 57}$,
\AtlasOrcid[0000-0003-4643-515X]{M.~Queitsch-Maitland}$^\textrm{\scriptsize 104}$,
\AtlasOrcid[0000-0002-2957-3449]{G.~Quetant}$^\textrm{\scriptsize 58}$,
\AtlasOrcid[0000-0002-0879-6045]{R.P.~Quinn}$^\textrm{\scriptsize 169}$,
\AtlasOrcid[0000-0003-1526-5848]{G.~Rabanal~Bolanos}$^\textrm{\scriptsize 63}$,
\AtlasOrcid[0000-0002-7151-3343]{D.~Rafanoharana}$^\textrm{\scriptsize 56}$,
\AtlasOrcid[0000-0002-7728-3278]{F.~Raffaeli}$^\textrm{\scriptsize 78a,78b}$,
\AtlasOrcid[0000-0002-4064-0489]{F.~Ragusa}$^\textrm{\scriptsize 73a,73b}$,
\AtlasOrcid[0000-0001-7394-0464]{J.L.~Rainbolt}$^\textrm{\scriptsize 41}$,
\AtlasOrcid[0000-0002-5987-4648]{J.A.~Raine}$^\textrm{\scriptsize 58}$,
\AtlasOrcid[0000-0001-6543-1520]{S.~Rajagopalan}$^\textrm{\scriptsize 30}$,
\AtlasOrcid[0000-0003-4495-4335]{E.~Ramakoti}$^\textrm{\scriptsize 38}$,
\AtlasOrcid[0000-0002-9155-9453]{L.~Rambelli}$^\textrm{\scriptsize 59b,59a}$,
\AtlasOrcid[0000-0001-5821-1490]{I.A.~Ramirez-Berend}$^\textrm{\scriptsize 35}$,
\AtlasOrcid[0000-0003-3119-9924]{K.~Ran}$^\textrm{\scriptsize 50,115c}$,
\AtlasOrcid[0000-0001-8411-9620]{D.S.~Rankin}$^\textrm{\scriptsize 132}$,
\AtlasOrcid[0000-0001-8022-9697]{N.P.~Rapheeha}$^\textrm{\scriptsize 34g}$,
\AtlasOrcid[0000-0001-9234-4465]{H.~Rasheed}$^\textrm{\scriptsize 28b}$,
\AtlasOrcid[0000-0002-5773-6380]{V.~Raskina}$^\textrm{\scriptsize 131}$,
\AtlasOrcid[0000-0002-5756-4558]{D.F.~Rassloff}$^\textrm{\scriptsize 65a}$,
\AtlasOrcid[0000-0003-1245-6710]{A.~Rastogi}$^\textrm{\scriptsize 18a}$,
\AtlasOrcid[0000-0002-0050-8053]{S.~Rave}$^\textrm{\scriptsize 103}$,
\AtlasOrcid[0000-0002-3976-0985]{S.~Ravera}$^\textrm{\scriptsize 59b,59a}$,
\AtlasOrcid[0000-0002-1622-6640]{B.~Ravina}$^\textrm{\scriptsize 57}$,
\AtlasOrcid[0000-0001-9348-4363]{I.~Ravinovich}$^\textrm{\scriptsize 174}$,
\AtlasOrcid[0000-0001-8225-1142]{M.~Raymond}$^\textrm{\scriptsize 37}$,
\AtlasOrcid[0000-0002-5751-6636]{A.L.~Read}$^\textrm{\scriptsize 129}$,
\AtlasOrcid[0000-0002-3427-0688]{N.P.~Readioff}$^\textrm{\scriptsize 143}$,
\AtlasOrcid[0000-0003-4461-3880]{D.M.~Rebuzzi}$^\textrm{\scriptsize 75a,75b}$,
\AtlasOrcid[0000-0002-6437-9991]{G.~Redlinger}$^\textrm{\scriptsize 30}$,
\AtlasOrcid[0000-0002-4570-8673]{A.S.~Reed}$^\textrm{\scriptsize 113}$,
\AtlasOrcid[0000-0003-3504-4882]{K.~Reeves}$^\textrm{\scriptsize 27}$,
\AtlasOrcid[0000-0001-8507-4065]{J.A.~Reidelsturz}$^\textrm{\scriptsize 176}$,
\AtlasOrcid[0000-0001-5758-579X]{D.~Reikher}$^\textrm{\scriptsize 127}$,
\AtlasOrcid[0000-0002-5471-0118]{A.~Rej}$^\textrm{\scriptsize 51}$,
\AtlasOrcid[0000-0001-6139-2210]{C.~Rembser}$^\textrm{\scriptsize 37}$,
\AtlasOrcid[0000-0002-0429-6959]{M.~Renda}$^\textrm{\scriptsize 28b}$,
\AtlasOrcid[0000-0002-9475-3075]{F.~Renner}$^\textrm{\scriptsize 50}$,
\AtlasOrcid[0000-0002-8485-3734]{A.G.~Rennie}$^\textrm{\scriptsize 163}$,
\AtlasOrcid[0000-0003-2258-314X]{A.L.~Rescia}$^\textrm{\scriptsize 50}$,
\AtlasOrcid[0000-0003-2313-4020]{S.~Resconi}$^\textrm{\scriptsize 73a}$,
\AtlasOrcid[0000-0002-6777-1761]{M.~Ressegotti}$^\textrm{\scriptsize 59b,59a}$,
\AtlasOrcid[0000-0002-7092-3893]{S.~Rettie}$^\textrm{\scriptsize 37}$,
\AtlasOrcid[0000-0001-8335-0505]{J.G.~Reyes~Rivera}$^\textrm{\scriptsize 110}$,
\AtlasOrcid[0000-0002-1506-5750]{E.~Reynolds}$^\textrm{\scriptsize 18a}$,
\AtlasOrcid[0000-0001-7141-0304]{O.L.~Rezanova}$^\textrm{\scriptsize 38}$,
\AtlasOrcid[0000-0003-4017-9829]{P.~Reznicek}$^\textrm{\scriptsize 137}$,
\AtlasOrcid[0009-0001-6269-0954]{H.~Riani}$^\textrm{\scriptsize 36d}$,
\AtlasOrcid[0000-0003-3212-3681]{N.~Ribaric}$^\textrm{\scriptsize 53}$,
\AtlasOrcid[0000-0002-4222-9976]{E.~Ricci}$^\textrm{\scriptsize 80a,80b}$,
\AtlasOrcid[0000-0001-8981-1966]{R.~Richter}$^\textrm{\scriptsize 113}$,
\AtlasOrcid[0000-0001-6613-4448]{S.~Richter}$^\textrm{\scriptsize 49a,49b}$,
\AtlasOrcid[0000-0002-3823-9039]{E.~Richter-Was}$^\textrm{\scriptsize 88b}$,
\AtlasOrcid[0000-0002-2601-7420]{M.~Ridel}$^\textrm{\scriptsize 131}$,
\AtlasOrcid[0000-0002-9740-7549]{S.~Ridouani}$^\textrm{\scriptsize 36d}$,
\AtlasOrcid[0000-0003-0290-0566]{P.~Rieck}$^\textrm{\scriptsize 121}$,
\AtlasOrcid[0000-0002-4871-8543]{P.~Riedler}$^\textrm{\scriptsize 37}$,
\AtlasOrcid[0000-0001-7818-2324]{E.M.~Riefel}$^\textrm{\scriptsize 49a,49b}$,
\AtlasOrcid[0009-0008-3521-1920]{J.O.~Rieger}$^\textrm{\scriptsize 118}$,
\AtlasOrcid[0000-0002-3476-1575]{M.~Rijssenbeek}$^\textrm{\scriptsize 149}$,
\AtlasOrcid[0000-0003-1165-7940]{M.~Rimoldi}$^\textrm{\scriptsize 37}$,
\AtlasOrcid[0000-0001-9608-9940]{L.~Rinaldi}$^\textrm{\scriptsize 24b,24a}$,
\AtlasOrcid[0009-0000-3940-2355]{P.~Rincke}$^\textrm{\scriptsize 57,166}$,
\AtlasOrcid[0000-0002-1295-1538]{T.T.~Rinn}$^\textrm{\scriptsize 30}$,
\AtlasOrcid[0000-0003-4931-0459]{M.P.~Rinnagel}$^\textrm{\scriptsize 112}$,
\AtlasOrcid[0000-0002-4053-5144]{G.~Ripellino}$^\textrm{\scriptsize 166}$,
\AtlasOrcid[0000-0002-3742-4582]{I.~Riu}$^\textrm{\scriptsize 13}$,
\AtlasOrcid[0000-0002-8149-4561]{J.C.~Rivera~Vergara}$^\textrm{\scriptsize 170}$,
\AtlasOrcid[0000-0002-2041-6236]{F.~Rizatdinova}$^\textrm{\scriptsize 125}$,
\AtlasOrcid[0000-0001-9834-2671]{E.~Rizvi}$^\textrm{\scriptsize 97}$,
\AtlasOrcid[0000-0001-5235-8256]{B.R.~Roberts}$^\textrm{\scriptsize 18a}$,
\AtlasOrcid[0000-0003-1227-0852]{S.S.~Roberts}$^\textrm{\scriptsize 140}$,
\AtlasOrcid[0000-0003-4096-8393]{S.H.~Robertson}$^\textrm{\scriptsize 107,w}$,
\AtlasOrcid[0000-0001-6169-4868]{D.~Robinson}$^\textrm{\scriptsize 33}$,
\AtlasOrcid[0000-0001-7701-8864]{M.~Robles~Manzano}$^\textrm{\scriptsize 103}$,
\AtlasOrcid[0000-0002-1659-8284]{A.~Robson}$^\textrm{\scriptsize 61}$,
\AtlasOrcid[0000-0002-3125-8333]{A.~Rocchi}$^\textrm{\scriptsize 78a,78b}$,
\AtlasOrcid[0000-0002-3020-4114]{C.~Roda}$^\textrm{\scriptsize 76a,76b}$,
\AtlasOrcid[0000-0002-4571-2509]{S.~Rodriguez~Bosca}$^\textrm{\scriptsize 37}$,
\AtlasOrcid[0000-0003-2729-6086]{Y.~Rodriguez~Garcia}$^\textrm{\scriptsize 23a}$,
\AtlasOrcid[0000-0002-9609-3306]{A.M.~Rodr\'iguez~Vera}$^\textrm{\scriptsize 119}$,
\AtlasOrcid{S.~Roe}$^\textrm{\scriptsize 37}$,
\AtlasOrcid[0000-0002-8794-3209]{J.T.~Roemer}$^\textrm{\scriptsize 37}$,
\AtlasOrcid[0000-0001-5933-9357]{A.R.~Roepe-Gier}$^\textrm{\scriptsize 140}$,
\AtlasOrcid[0000-0001-7744-9584]{O.~R{\o}hne}$^\textrm{\scriptsize 129}$,
\AtlasOrcid[0000-0002-6888-9462]{R.A.~Rojas}$^\textrm{\scriptsize 106}$,
\AtlasOrcid[0000-0003-2084-369X]{C.P.A.~Roland}$^\textrm{\scriptsize 131}$,
\AtlasOrcid[0000-0001-6479-3079]{J.~Roloff}$^\textrm{\scriptsize 30}$,
\AtlasOrcid[0000-0001-9241-1189]{A.~Romaniouk}$^\textrm{\scriptsize 38}$,
\AtlasOrcid[0000-0003-3154-7386]{E.~Romano}$^\textrm{\scriptsize 75a,75b}$,
\AtlasOrcid[0000-0002-6609-7250]{M.~Romano}$^\textrm{\scriptsize 24b}$,
\AtlasOrcid[0000-0001-9434-1380]{A.C.~Romero~Hernandez}$^\textrm{\scriptsize 167}$,
\AtlasOrcid[0000-0003-2577-1875]{N.~Rompotis}$^\textrm{\scriptsize 95}$,
\AtlasOrcid[0000-0001-7151-9983]{L.~Roos}$^\textrm{\scriptsize 131}$,
\AtlasOrcid[0000-0003-0838-5980]{S.~Rosati}$^\textrm{\scriptsize 77a}$,
\AtlasOrcid[0000-0001-7492-831X]{B.J.~Rosser}$^\textrm{\scriptsize 41}$,
\AtlasOrcid[0000-0002-2146-677X]{E.~Rossi}$^\textrm{\scriptsize 130}$,
\AtlasOrcid[0000-0001-9476-9854]{E.~Rossi}$^\textrm{\scriptsize 74a,74b}$,
\AtlasOrcid[0000-0003-3104-7971]{L.P.~Rossi}$^\textrm{\scriptsize 63}$,
\AtlasOrcid[0000-0003-0424-5729]{L.~Rossini}$^\textrm{\scriptsize 56}$,
\AtlasOrcid[0000-0002-9095-7142]{R.~Rosten}$^\textrm{\scriptsize 123}$,
\AtlasOrcid[0000-0003-4088-6275]{M.~Rotaru}$^\textrm{\scriptsize 28b}$,
\AtlasOrcid[0000-0002-6762-2213]{B.~Rottler}$^\textrm{\scriptsize 56}$,
\AtlasOrcid[0000-0002-9853-7468]{C.~Rougier}$^\textrm{\scriptsize 92}$,
\AtlasOrcid[0000-0001-7613-8063]{D.~Rousseau}$^\textrm{\scriptsize 68}$,
\AtlasOrcid[0000-0003-1427-6668]{D.~Rousso}$^\textrm{\scriptsize 50}$,
\AtlasOrcid[0000-0002-0116-1012]{A.~Roy}$^\textrm{\scriptsize 167}$,
\AtlasOrcid[0000-0002-1966-8567]{S.~Roy-Garand}$^\textrm{\scriptsize 159}$,
\AtlasOrcid[0000-0003-0504-1453]{A.~Rozanov}$^\textrm{\scriptsize 105}$,
\AtlasOrcid[0000-0002-4887-9224]{Z.M.A.~Rozario}$^\textrm{\scriptsize 61}$,
\AtlasOrcid[0000-0001-6969-0634]{Y.~Rozen}$^\textrm{\scriptsize 154}$,
\AtlasOrcid[0000-0001-9085-2175]{A.~Rubio~Jimenez}$^\textrm{\scriptsize 168}$,
\AtlasOrcid[0000-0002-2116-048X]{V.H.~Ruelas~Rivera}$^\textrm{\scriptsize 19}$,
\AtlasOrcid[0000-0001-9941-1966]{T.A.~Ruggeri}$^\textrm{\scriptsize 1}$,
\AtlasOrcid[0000-0001-6436-8814]{A.~Ruggiero}$^\textrm{\scriptsize 130}$,
\AtlasOrcid[0000-0002-5742-2541]{A.~Ruiz-Martinez}$^\textrm{\scriptsize 168}$,
\AtlasOrcid[0000-0001-8945-8760]{A.~Rummler}$^\textrm{\scriptsize 37}$,
\AtlasOrcid[0000-0003-3051-9607]{Z.~Rurikova}$^\textrm{\scriptsize 56}$,
\AtlasOrcid[0000-0003-1927-5322]{N.A.~Rusakovich}$^\textrm{\scriptsize 40}$,
\AtlasOrcid[0000-0003-4181-0678]{H.L.~Russell}$^\textrm{\scriptsize 170}$,
\AtlasOrcid[0000-0002-5105-8021]{G.~Russo}$^\textrm{\scriptsize 77a,77b}$,
\AtlasOrcid[0000-0002-4682-0667]{J.P.~Rutherfoord}$^\textrm{\scriptsize 7}$,
\AtlasOrcid[0000-0001-8474-8531]{S.~Rutherford~Colmenares}$^\textrm{\scriptsize 33}$,
\AtlasOrcid[0000-0002-6033-004X]{M.~Rybar}$^\textrm{\scriptsize 137}$,
\AtlasOrcid[0000-0001-7088-1745]{E.B.~Rye}$^\textrm{\scriptsize 129}$,
\AtlasOrcid[0000-0002-0623-7426]{A.~Ryzhov}$^\textrm{\scriptsize 46}$,
\AtlasOrcid[0000-0003-2328-1952]{J.A.~Sabater~Iglesias}$^\textrm{\scriptsize 58}$,
\AtlasOrcid[0000-0003-0019-5410]{H.F-W.~Sadrozinski}$^\textrm{\scriptsize 140}$,
\AtlasOrcid[0000-0001-7796-0120]{F.~Safai~Tehrani}$^\textrm{\scriptsize 77a}$,
\AtlasOrcid[0000-0002-0338-9707]{B.~Safarzadeh~Samani}$^\textrm{\scriptsize 138}$,
\AtlasOrcid[0000-0001-9296-1498]{S.~Saha}$^\textrm{\scriptsize 1}$,
\AtlasOrcid[0000-0002-7400-7286]{M.~Sahinsoy}$^\textrm{\scriptsize 84}$,
\AtlasOrcid[0000-0002-9932-7622]{A.~Saibel}$^\textrm{\scriptsize 168}$,
\AtlasOrcid[0000-0002-3765-1320]{M.~Saimpert}$^\textrm{\scriptsize 139}$,
\AtlasOrcid[0000-0001-5564-0935]{M.~Saito}$^\textrm{\scriptsize 157}$,
\AtlasOrcid[0000-0003-2567-6392]{T.~Saito}$^\textrm{\scriptsize 157}$,
\AtlasOrcid[0000-0003-0824-7326]{A.~Sala}$^\textrm{\scriptsize 73a,73b}$,
\AtlasOrcid[0000-0002-8780-5885]{D.~Salamani}$^\textrm{\scriptsize 37}$,
\AtlasOrcid[0000-0002-3623-0161]{A.~Salnikov}$^\textrm{\scriptsize 147}$,
\AtlasOrcid[0000-0003-4181-2788]{J.~Salt}$^\textrm{\scriptsize 168}$,
\AtlasOrcid[0000-0001-5041-5659]{A.~Salvador~Salas}$^\textrm{\scriptsize 155}$,
\AtlasOrcid[0000-0002-8564-2373]{D.~Salvatore}$^\textrm{\scriptsize 45b,45a}$,
\AtlasOrcid[0000-0002-3709-1554]{F.~Salvatore}$^\textrm{\scriptsize 150}$,
\AtlasOrcid[0000-0001-6004-3510]{A.~Salzburger}$^\textrm{\scriptsize 37}$,
\AtlasOrcid[0000-0003-4484-1410]{D.~Sammel}$^\textrm{\scriptsize 56}$,
\AtlasOrcid[0009-0005-7228-1539]{E.~Sampson}$^\textrm{\scriptsize 94}$,
\AtlasOrcid[0000-0002-9571-2304]{D.~Sampsonidis}$^\textrm{\scriptsize 156,d}$,
\AtlasOrcid[0000-0003-0384-7672]{D.~Sampsonidou}$^\textrm{\scriptsize 127}$,
\AtlasOrcid[0000-0001-9913-310X]{J.~S\'anchez}$^\textrm{\scriptsize 168}$,
\AtlasOrcid[0000-0002-4143-6201]{V.~Sanchez~Sebastian}$^\textrm{\scriptsize 168}$,
\AtlasOrcid[0000-0001-5235-4095]{H.~Sandaker}$^\textrm{\scriptsize 129}$,
\AtlasOrcid[0000-0003-2576-259X]{C.O.~Sander}$^\textrm{\scriptsize 50}$,
\AtlasOrcid[0000-0002-6016-8011]{J.A.~Sandesara}$^\textrm{\scriptsize 106}$,
\AtlasOrcid[0000-0002-7601-8528]{M.~Sandhoff}$^\textrm{\scriptsize 176}$,
\AtlasOrcid[0000-0003-1038-723X]{C.~Sandoval}$^\textrm{\scriptsize 23b}$,
\AtlasOrcid[0000-0001-5923-6999]{L.~Sanfilippo}$^\textrm{\scriptsize 65a}$,
\AtlasOrcid[0000-0003-0955-4213]{D.P.C.~Sankey}$^\textrm{\scriptsize 138}$,
\AtlasOrcid[0000-0001-8655-0609]{T.~Sano}$^\textrm{\scriptsize 90}$,
\AtlasOrcid[0000-0002-9166-099X]{A.~Sansoni}$^\textrm{\scriptsize 55}$,
\AtlasOrcid[0000-0003-1766-2791]{L.~Santi}$^\textrm{\scriptsize 37,77b}$,
\AtlasOrcid[0000-0002-1642-7186]{C.~Santoni}$^\textrm{\scriptsize 42}$,
\AtlasOrcid[0000-0003-1710-9291]{H.~Santos}$^\textrm{\scriptsize 134a,134b}$,
\AtlasOrcid[0000-0003-4644-2579]{A.~Santra}$^\textrm{\scriptsize 174}$,
\AtlasOrcid[0000-0002-9478-0671]{E.~Sanzani}$^\textrm{\scriptsize 24b,24a}$,
\AtlasOrcid[0000-0001-9150-640X]{K.A.~Saoucha}$^\textrm{\scriptsize 165}$,
\AtlasOrcid[0000-0002-7006-0864]{J.G.~Saraiva}$^\textrm{\scriptsize 134a,134d}$,
\AtlasOrcid[0000-0002-6932-2804]{J.~Sardain}$^\textrm{\scriptsize 7}$,
\AtlasOrcid[0000-0002-2910-3906]{O.~Sasaki}$^\textrm{\scriptsize 86}$,
\AtlasOrcid[0000-0001-8988-4065]{K.~Sato}$^\textrm{\scriptsize 161}$,
\AtlasOrcid{C.~Sauer}$^\textrm{\scriptsize 37}$,
\AtlasOrcid[0000-0003-1921-2647]{E.~Sauvan}$^\textrm{\scriptsize 4}$,
\AtlasOrcid[0000-0001-5606-0107]{P.~Savard}$^\textrm{\scriptsize 159,ab}$,
\AtlasOrcid[0000-0002-2226-9874]{R.~Sawada}$^\textrm{\scriptsize 157}$,
\AtlasOrcid[0000-0002-2027-1428]{C.~Sawyer}$^\textrm{\scriptsize 138}$,
\AtlasOrcid[0000-0001-8295-0605]{L.~Sawyer}$^\textrm{\scriptsize 100}$,
\AtlasOrcid[0000-0002-8236-5251]{C.~Sbarra}$^\textrm{\scriptsize 24b}$,
\AtlasOrcid[0000-0002-1934-3041]{A.~Sbrizzi}$^\textrm{\scriptsize 24b,24a}$,
\AtlasOrcid[0000-0002-2746-525X]{T.~Scanlon}$^\textrm{\scriptsize 99}$,
\AtlasOrcid[0000-0002-0433-6439]{J.~Schaarschmidt}$^\textrm{\scriptsize 142}$,
\AtlasOrcid[0000-0003-4489-9145]{U.~Sch\"afer}$^\textrm{\scriptsize 103}$,
\AtlasOrcid[0000-0002-2586-7554]{A.C.~Schaffer}$^\textrm{\scriptsize 68,46}$,
\AtlasOrcid[0000-0001-7822-9663]{D.~Schaile}$^\textrm{\scriptsize 112}$,
\AtlasOrcid[0000-0003-1218-425X]{R.D.~Schamberger}$^\textrm{\scriptsize 149}$,
\AtlasOrcid[0000-0002-0294-1205]{C.~Scharf}$^\textrm{\scriptsize 19}$,
\AtlasOrcid[0000-0002-8403-8924]{M.M.~Schefer}$^\textrm{\scriptsize 20}$,
\AtlasOrcid[0000-0003-1870-1967]{V.A.~Schegelsky}$^\textrm{\scriptsize 38}$,
\AtlasOrcid[0000-0001-6012-7191]{D.~Scheirich}$^\textrm{\scriptsize 137}$,
\AtlasOrcid[0000-0002-0859-4312]{M.~Schernau}$^\textrm{\scriptsize 163}$,
\AtlasOrcid[0000-0002-9142-1948]{C.~Scheulen}$^\textrm{\scriptsize 57}$,
\AtlasOrcid[0000-0003-0957-4994]{C.~Schiavi}$^\textrm{\scriptsize 59b,59a}$,
\AtlasOrcid[0000-0003-0628-0579]{M.~Schioppa}$^\textrm{\scriptsize 45b,45a}$,
\AtlasOrcid[0000-0002-1284-4169]{B.~Schlag}$^\textrm{\scriptsize 147,l}$,
\AtlasOrcid[0000-0001-5239-3609]{S.~Schlenker}$^\textrm{\scriptsize 37}$,
\AtlasOrcid[0000-0002-2855-9549]{J.~Schmeing}$^\textrm{\scriptsize 176}$,
\AtlasOrcid[0000-0002-4467-2461]{M.A.~Schmidt}$^\textrm{\scriptsize 176}$,
\AtlasOrcid[0000-0003-1978-4928]{K.~Schmieden}$^\textrm{\scriptsize 103}$,
\AtlasOrcid[0000-0003-1471-690X]{C.~Schmitt}$^\textrm{\scriptsize 103}$,
\AtlasOrcid[0000-0002-1844-1723]{N.~Schmitt}$^\textrm{\scriptsize 103}$,
\AtlasOrcid[0000-0001-8387-1853]{S.~Schmitt}$^\textrm{\scriptsize 50}$,
\AtlasOrcid[0000-0002-8081-2353]{L.~Schoeffel}$^\textrm{\scriptsize 139}$,
\AtlasOrcid[0000-0002-4499-7215]{A.~Schoening}$^\textrm{\scriptsize 65b}$,
\AtlasOrcid[0000-0003-2882-9796]{P.G.~Scholer}$^\textrm{\scriptsize 35}$,
\AtlasOrcid[0000-0002-9340-2214]{E.~Schopf}$^\textrm{\scriptsize 130}$,
\AtlasOrcid[0000-0002-4235-7265]{M.~Schott}$^\textrm{\scriptsize 25}$,
\AtlasOrcid[0000-0003-0016-5246]{J.~Schovancova}$^\textrm{\scriptsize 37}$,
\AtlasOrcid[0000-0001-9031-6751]{S.~Schramm}$^\textrm{\scriptsize 58}$,
\AtlasOrcid[0000-0001-7967-6385]{T.~Schroer}$^\textrm{\scriptsize 58}$,
\AtlasOrcid[0000-0002-0860-7240]{H-C.~Schultz-Coulon}$^\textrm{\scriptsize 65a}$,
\AtlasOrcid[0000-0002-1733-8388]{M.~Schumacher}$^\textrm{\scriptsize 56}$,
\AtlasOrcid[0000-0002-5394-0317]{B.A.~Schumm}$^\textrm{\scriptsize 140}$,
\AtlasOrcid[0000-0002-3971-9595]{Ph.~Schune}$^\textrm{\scriptsize 139}$,
\AtlasOrcid[0000-0003-1230-2842]{A.J.~Schuy}$^\textrm{\scriptsize 142}$,
\AtlasOrcid[0000-0002-5014-1245]{H.R.~Schwartz}$^\textrm{\scriptsize 140}$,
\AtlasOrcid[0000-0002-6680-8366]{A.~Schwartzman}$^\textrm{\scriptsize 147}$,
\AtlasOrcid[0000-0001-5660-2690]{T.A.~Schwarz}$^\textrm{\scriptsize 109}$,
\AtlasOrcid[0000-0003-0989-5675]{Ph.~Schwemling}$^\textrm{\scriptsize 139}$,
\AtlasOrcid[0000-0001-6348-5410]{R.~Schwienhorst}$^\textrm{\scriptsize 110}$,
\AtlasOrcid[0000-0002-2000-6210]{F.G.~Sciacca}$^\textrm{\scriptsize 20}$,
\AtlasOrcid[0000-0001-7163-501X]{A.~Sciandra}$^\textrm{\scriptsize 30}$,
\AtlasOrcid[0000-0002-8482-1775]{G.~Sciolla}$^\textrm{\scriptsize 27}$,
\AtlasOrcid[0000-0001-9569-3089]{F.~Scuri}$^\textrm{\scriptsize 76a}$,
\AtlasOrcid[0000-0003-1073-035X]{C.D.~Sebastiani}$^\textrm{\scriptsize 95}$,
\AtlasOrcid[0000-0003-2052-2386]{K.~Sedlaczek}$^\textrm{\scriptsize 119}$,
\AtlasOrcid[0000-0002-1181-3061]{S.C.~Seidel}$^\textrm{\scriptsize 116}$,
\AtlasOrcid[0000-0003-4311-8597]{A.~Seiden}$^\textrm{\scriptsize 140}$,
\AtlasOrcid[0000-0002-4703-000X]{B.D.~Seidlitz}$^\textrm{\scriptsize 43}$,
\AtlasOrcid[0000-0003-4622-6091]{C.~Seitz}$^\textrm{\scriptsize 50}$,
\AtlasOrcid[0000-0001-5148-7363]{J.M.~Seixas}$^\textrm{\scriptsize 85b}$,
\AtlasOrcid[0000-0002-4116-5309]{G.~Sekhniaidze}$^\textrm{\scriptsize 74a}$,
\AtlasOrcid[0000-0002-8739-8554]{L.~Selem}$^\textrm{\scriptsize 62}$,
\AtlasOrcid[0000-0002-3946-377X]{N.~Semprini-Cesari}$^\textrm{\scriptsize 24b,24a}$,
\AtlasOrcid[0000-0003-2676-3498]{D.~Sengupta}$^\textrm{\scriptsize 58}$,
\AtlasOrcid[0000-0001-9783-8878]{V.~Senthilkumar}$^\textrm{\scriptsize 168}$,
\AtlasOrcid[0000-0003-3238-5382]{L.~Serin}$^\textrm{\scriptsize 68}$,
\AtlasOrcid[0000-0002-1402-7525]{M.~Sessa}$^\textrm{\scriptsize 78a,78b}$,
\AtlasOrcid[0000-0003-3316-846X]{H.~Severini}$^\textrm{\scriptsize 124}$,
\AtlasOrcid[0000-0002-4065-7352]{F.~Sforza}$^\textrm{\scriptsize 59b,59a}$,
\AtlasOrcid[0000-0002-3003-9905]{A.~Sfyrla}$^\textrm{\scriptsize 58}$,
\AtlasOrcid[0000-0002-0032-4473]{Q.~Sha}$^\textrm{\scriptsize 14}$,
\AtlasOrcid[0000-0003-4849-556X]{E.~Shabalina}$^\textrm{\scriptsize 57}$,
\AtlasOrcid[0000-0002-6157-2016]{A.H.~Shah}$^\textrm{\scriptsize 33}$,
\AtlasOrcid[0000-0002-2673-8527]{R.~Shaheen}$^\textrm{\scriptsize 148}$,
\AtlasOrcid[0000-0002-1325-3432]{J.D.~Shahinian}$^\textrm{\scriptsize 132}$,
\AtlasOrcid[0000-0002-5376-1546]{D.~Shaked~Renous}$^\textrm{\scriptsize 174}$,
\AtlasOrcid[0000-0001-9134-5925]{L.Y.~Shan}$^\textrm{\scriptsize 14}$,
\AtlasOrcid[0000-0001-8540-9654]{M.~Shapiro}$^\textrm{\scriptsize 18a}$,
\AtlasOrcid[0000-0002-5211-7177]{A.~Sharma}$^\textrm{\scriptsize 37}$,
\AtlasOrcid[0000-0003-2250-4181]{A.S.~Sharma}$^\textrm{\scriptsize 169}$,
\AtlasOrcid[0000-0002-3454-9558]{P.~Sharma}$^\textrm{\scriptsize 82}$,
\AtlasOrcid[0000-0001-7530-4162]{P.B.~Shatalov}$^\textrm{\scriptsize 38}$,
\AtlasOrcid[0000-0001-9182-0634]{K.~Shaw}$^\textrm{\scriptsize 150}$,
\AtlasOrcid[0000-0002-8958-7826]{S.M.~Shaw}$^\textrm{\scriptsize 104}$,
\AtlasOrcid[0000-0002-4085-1227]{Q.~Shen}$^\textrm{\scriptsize 64c}$,
\AtlasOrcid[0009-0003-3022-8858]{D.J.~Sheppard}$^\textrm{\scriptsize 146}$,
\AtlasOrcid[0000-0002-6621-4111]{P.~Sherwood}$^\textrm{\scriptsize 99}$,
\AtlasOrcid[0000-0001-9532-5075]{L.~Shi}$^\textrm{\scriptsize 99}$,
\AtlasOrcid[0000-0001-9910-9345]{X.~Shi}$^\textrm{\scriptsize 14}$,
\AtlasOrcid[0000-0001-8279-442X]{S.~Shimizu}$^\textrm{\scriptsize 86}$,
\AtlasOrcid[0000-0002-2228-2251]{C.O.~Shimmin}$^\textrm{\scriptsize 177}$,
\AtlasOrcid[0000-0003-4050-6420]{I.P.J.~Shipsey}$^\textrm{\scriptsize 130}$,
\AtlasOrcid[0000-0002-3191-0061]{S.~Shirabe}$^\textrm{\scriptsize 91}$,
\AtlasOrcid[0000-0002-4775-9669]{M.~Shiyakova}$^\textrm{\scriptsize 40,u}$,
\AtlasOrcid[0000-0002-3017-826X]{M.J.~Shochet}$^\textrm{\scriptsize 41}$,
\AtlasOrcid[0000-0002-9453-9415]{D.R.~Shope}$^\textrm{\scriptsize 129}$,
\AtlasOrcid[0009-0005-3409-7781]{B.~Shrestha}$^\textrm{\scriptsize 124}$,
\AtlasOrcid[0000-0001-7249-7456]{S.~Shrestha}$^\textrm{\scriptsize 123,ae}$,
\AtlasOrcid[0000-0002-0456-786X]{M.J.~Shroff}$^\textrm{\scriptsize 170}$,
\AtlasOrcid[0000-0002-5428-813X]{P.~Sicho}$^\textrm{\scriptsize 135}$,
\AtlasOrcid[0000-0002-3246-0330]{A.M.~Sickles}$^\textrm{\scriptsize 167}$,
\AtlasOrcid[0000-0002-3206-395X]{E.~Sideras~Haddad}$^\textrm{\scriptsize 34g,164}$,
\AtlasOrcid[0000-0002-4021-0374]{A.C.~Sidley}$^\textrm{\scriptsize 118}$,
\AtlasOrcid[0000-0002-3277-1999]{A.~Sidoti}$^\textrm{\scriptsize 24b}$,
\AtlasOrcid[0000-0002-2893-6412]{F.~Siegert}$^\textrm{\scriptsize 52}$,
\AtlasOrcid[0000-0002-5809-9424]{Dj.~Sijacki}$^\textrm{\scriptsize 16}$,
\AtlasOrcid[0000-0001-6035-8109]{F.~Sili}$^\textrm{\scriptsize 93}$,
\AtlasOrcid[0000-0002-5987-2984]{J.M.~Silva}$^\textrm{\scriptsize 54}$,
\AtlasOrcid[0000-0002-0666-7485]{I.~Silva~Ferreira}$^\textrm{\scriptsize 85b}$,
\AtlasOrcid[0000-0003-2285-478X]{M.V.~Silva~Oliveira}$^\textrm{\scriptsize 30}$,
\AtlasOrcid[0000-0001-7734-7617]{S.B.~Silverstein}$^\textrm{\scriptsize 49a}$,
\AtlasOrcid{S.~Simion}$^\textrm{\scriptsize 68}$,
\AtlasOrcid[0000-0003-2042-6394]{R.~Simoniello}$^\textrm{\scriptsize 37}$,
\AtlasOrcid[0000-0002-9899-7413]{E.L.~Simpson}$^\textrm{\scriptsize 104}$,
\AtlasOrcid[0000-0003-3354-6088]{H.~Simpson}$^\textrm{\scriptsize 150}$,
\AtlasOrcid[0000-0002-4689-3903]{L.R.~Simpson}$^\textrm{\scriptsize 109}$,
\AtlasOrcid[0000-0002-9650-3846]{S.~Simsek}$^\textrm{\scriptsize 84}$,
\AtlasOrcid[0000-0003-1235-5178]{S.~Sindhu}$^\textrm{\scriptsize 57}$,
\AtlasOrcid[0000-0002-5128-2373]{P.~Sinervo}$^\textrm{\scriptsize 159}$,
\AtlasOrcid[0000-0001-5641-5713]{S.~Singh}$^\textrm{\scriptsize 30}$,
\AtlasOrcid[0000-0002-3600-2804]{S.~Sinha}$^\textrm{\scriptsize 50}$,
\AtlasOrcid[0000-0002-2438-3785]{S.~Sinha}$^\textrm{\scriptsize 104}$,
\AtlasOrcid[0000-0002-0912-9121]{M.~Sioli}$^\textrm{\scriptsize 24b,24a}$,
\AtlasOrcid[0000-0003-4554-1831]{I.~Siral}$^\textrm{\scriptsize 37}$,
\AtlasOrcid[0000-0003-3745-0454]{E.~Sitnikova}$^\textrm{\scriptsize 50}$,
\AtlasOrcid[0000-0002-5285-8995]{J.~Sj\"{o}lin}$^\textrm{\scriptsize 49a,49b}$,
\AtlasOrcid[0000-0003-3614-026X]{A.~Skaf}$^\textrm{\scriptsize 57}$,
\AtlasOrcid[0000-0003-3973-9382]{E.~Skorda}$^\textrm{\scriptsize 21}$,
\AtlasOrcid[0000-0001-6342-9283]{P.~Skubic}$^\textrm{\scriptsize 124}$,
\AtlasOrcid[0000-0002-9386-9092]{M.~Slawinska}$^\textrm{\scriptsize 89}$,
\AtlasOrcid{V.~Smakhtin}$^\textrm{\scriptsize 174}$,
\AtlasOrcid[0000-0002-7192-4097]{B.H.~Smart}$^\textrm{\scriptsize 138}$,
\AtlasOrcid[0000-0002-6778-073X]{S.Yu.~Smirnov}$^\textrm{\scriptsize 38}$,
\AtlasOrcid[0000-0002-2891-0781]{Y.~Smirnov}$^\textrm{\scriptsize 38}$,
\AtlasOrcid[0000-0002-0447-2975]{L.N.~Smirnova}$^\textrm{\scriptsize 38,a}$,
\AtlasOrcid[0000-0003-2517-531X]{O.~Smirnova}$^\textrm{\scriptsize 101}$,
\AtlasOrcid[0000-0002-2488-407X]{A.C.~Smith}$^\textrm{\scriptsize 43}$,
\AtlasOrcid{D.R.~Smith}$^\textrm{\scriptsize 163}$,
\AtlasOrcid[0000-0001-6480-6829]{E.A.~Smith}$^\textrm{\scriptsize 41}$,
\AtlasOrcid[0000-0003-4231-6241]{J.L.~Smith}$^\textrm{\scriptsize 104}$,
\AtlasOrcid{R.~Smith}$^\textrm{\scriptsize 147}$,
\AtlasOrcid[0000-0002-3777-4734]{M.~Smizanska}$^\textrm{\scriptsize 94}$,
\AtlasOrcid[0000-0002-5996-7000]{K.~Smolek}$^\textrm{\scriptsize 136}$,
\AtlasOrcid[0000-0002-9067-8362]{A.A.~Snesarev}$^\textrm{\scriptsize 38}$,
\AtlasOrcid[0000-0003-4579-2120]{H.L.~Snoek}$^\textrm{\scriptsize 118}$,
\AtlasOrcid[0000-0001-8610-8423]{S.~Snyder}$^\textrm{\scriptsize 30}$,
\AtlasOrcid[0000-0001-7430-7599]{R.~Sobie}$^\textrm{\scriptsize 170,w}$,
\AtlasOrcid[0000-0002-0749-2146]{A.~Soffer}$^\textrm{\scriptsize 155}$,
\AtlasOrcid[0000-0002-0518-4086]{C.A.~Solans~Sanchez}$^\textrm{\scriptsize 37}$,
\AtlasOrcid[0000-0003-0694-3272]{E.Yu.~Soldatov}$^\textrm{\scriptsize 38}$,
\AtlasOrcid[0000-0002-7674-7878]{U.~Soldevila}$^\textrm{\scriptsize 168}$,
\AtlasOrcid[0000-0002-2737-8674]{A.A.~Solodkov}$^\textrm{\scriptsize 38}$,
\AtlasOrcid[0000-0002-7378-4454]{S.~Solomon}$^\textrm{\scriptsize 27}$,
\AtlasOrcid[0000-0001-9946-8188]{A.~Soloshenko}$^\textrm{\scriptsize 40}$,
\AtlasOrcid[0000-0003-2168-9137]{K.~Solovieva}$^\textrm{\scriptsize 56}$,
\AtlasOrcid[0000-0002-2598-5657]{O.V.~Solovyanov}$^\textrm{\scriptsize 42}$,
\AtlasOrcid[0000-0003-1703-7304]{P.~Sommer}$^\textrm{\scriptsize 52}$,
\AtlasOrcid[0000-0003-4435-4962]{A.~Sonay}$^\textrm{\scriptsize 13}$,
\AtlasOrcid[0000-0003-1338-2741]{W.Y.~Song}$^\textrm{\scriptsize 160b}$,
\AtlasOrcid[0000-0001-6981-0544]{A.~Sopczak}$^\textrm{\scriptsize 136}$,
\AtlasOrcid[0000-0001-9116-880X]{A.L.~Sopio}$^\textrm{\scriptsize 54}$,
\AtlasOrcid[0000-0002-6171-1119]{F.~Sopkova}$^\textrm{\scriptsize 29b}$,
\AtlasOrcid[0000-0003-1278-7691]{J.D.~Sorenson}$^\textrm{\scriptsize 116}$,
\AtlasOrcid[0009-0001-8347-0803]{I.R.~Sotarriva~Alvarez}$^\textrm{\scriptsize 158}$,
\AtlasOrcid{V.~Sothilingam}$^\textrm{\scriptsize 65a}$,
\AtlasOrcid[0000-0002-8613-0310]{O.J.~Soto~Sandoval}$^\textrm{\scriptsize 141c,141b}$,
\AtlasOrcid[0000-0002-1430-5994]{S.~Sottocornola}$^\textrm{\scriptsize 70}$,
\AtlasOrcid[0000-0003-0124-3410]{R.~Soualah}$^\textrm{\scriptsize 165}$,
\AtlasOrcid[0000-0002-8120-478X]{Z.~Soumaimi}$^\textrm{\scriptsize 36e}$,
\AtlasOrcid[0000-0002-0786-6304]{D.~South}$^\textrm{\scriptsize 50}$,
\AtlasOrcid[0000-0003-0209-0858]{N.~Soybelman}$^\textrm{\scriptsize 174}$,
\AtlasOrcid[0000-0001-7482-6348]{S.~Spagnolo}$^\textrm{\scriptsize 72a,72b}$,
\AtlasOrcid[0000-0001-5813-1693]{M.~Spalla}$^\textrm{\scriptsize 113}$,
\AtlasOrcid[0000-0003-4454-6999]{D.~Sperlich}$^\textrm{\scriptsize 56}$,
\AtlasOrcid[0000-0003-4183-2594]{G.~Spigo}$^\textrm{\scriptsize 37}$,
\AtlasOrcid[0000-0003-1491-6151]{B.~Spisso}$^\textrm{\scriptsize 74a,74b}$,
\AtlasOrcid[0000-0002-9226-2539]{D.P.~Spiteri}$^\textrm{\scriptsize 61}$,
\AtlasOrcid[0000-0001-5644-9526]{M.~Spousta}$^\textrm{\scriptsize 137}$,
\AtlasOrcid[0000-0002-6719-9726]{E.J.~Staats}$^\textrm{\scriptsize 35}$,
\AtlasOrcid[0000-0001-7282-949X]{R.~Stamen}$^\textrm{\scriptsize 65a}$,
\AtlasOrcid[0000-0002-7666-7544]{A.~Stampekis}$^\textrm{\scriptsize 21}$,
\AtlasOrcid[0000-0003-2546-0516]{E.~Stanecka}$^\textrm{\scriptsize 89}$,
\AtlasOrcid[0000-0002-7033-874X]{W.~Stanek-Maslouska}$^\textrm{\scriptsize 50}$,
\AtlasOrcid[0000-0003-4132-7205]{M.V.~Stange}$^\textrm{\scriptsize 52}$,
\AtlasOrcid[0000-0001-9007-7658]{B.~Stanislaus}$^\textrm{\scriptsize 18a}$,
\AtlasOrcid[0000-0002-7561-1960]{M.M.~Stanitzki}$^\textrm{\scriptsize 50}$,
\AtlasOrcid[0000-0001-5374-6402]{B.~Stapf}$^\textrm{\scriptsize 50}$,
\AtlasOrcid[0000-0002-8495-0630]{E.A.~Starchenko}$^\textrm{\scriptsize 38}$,
\AtlasOrcid[0000-0001-6616-3433]{G.H.~Stark}$^\textrm{\scriptsize 140}$,
\AtlasOrcid[0000-0002-1217-672X]{J.~Stark}$^\textrm{\scriptsize 92}$,
\AtlasOrcid[0000-0001-6009-6321]{P.~Staroba}$^\textrm{\scriptsize 135}$,
\AtlasOrcid[0000-0003-1990-0992]{P.~Starovoitov}$^\textrm{\scriptsize 65a}$,
\AtlasOrcid[0000-0002-2908-3909]{S.~St\"arz}$^\textrm{\scriptsize 107}$,
\AtlasOrcid[0000-0001-7708-9259]{R.~Staszewski}$^\textrm{\scriptsize 89}$,
\AtlasOrcid[0000-0002-8549-6855]{G.~Stavropoulos}$^\textrm{\scriptsize 48}$,
\AtlasOrcid[0000-0002-5349-8370]{P.~Steinberg}$^\textrm{\scriptsize 30}$,
\AtlasOrcid[0000-0003-4091-1784]{B.~Stelzer}$^\textrm{\scriptsize 146,160a}$,
\AtlasOrcid[0000-0003-0690-8573]{H.J.~Stelzer}$^\textrm{\scriptsize 133}$,
\AtlasOrcid[0000-0002-0791-9728]{O.~Stelzer-Chilton}$^\textrm{\scriptsize 160a}$,
\AtlasOrcid[0000-0002-4185-6484]{H.~Stenzel}$^\textrm{\scriptsize 60}$,
\AtlasOrcid[0000-0003-2399-8945]{T.J.~Stevenson}$^\textrm{\scriptsize 150}$,
\AtlasOrcid[0000-0003-0182-7088]{G.A.~Stewart}$^\textrm{\scriptsize 37}$,
\AtlasOrcid[0000-0002-8649-1917]{J.R.~Stewart}$^\textrm{\scriptsize 125}$,
\AtlasOrcid[0000-0001-9679-0323]{M.C.~Stockton}$^\textrm{\scriptsize 37}$,
\AtlasOrcid[0000-0002-7511-4614]{G.~Stoicea}$^\textrm{\scriptsize 28b}$,
\AtlasOrcid[0000-0003-0276-8059]{M.~Stolarski}$^\textrm{\scriptsize 134a}$,
\AtlasOrcid[0000-0001-7582-6227]{S.~Stonjek}$^\textrm{\scriptsize 113}$,
\AtlasOrcid[0000-0003-2460-6659]{A.~Straessner}$^\textrm{\scriptsize 52}$,
\AtlasOrcid[0000-0002-8913-0981]{J.~Strandberg}$^\textrm{\scriptsize 148}$,
\AtlasOrcid[0000-0001-7253-7497]{S.~Strandberg}$^\textrm{\scriptsize 49a,49b}$,
\AtlasOrcid[0000-0002-9542-1697]{M.~Stratmann}$^\textrm{\scriptsize 176}$,
\AtlasOrcid[0000-0002-0465-5472]{M.~Strauss}$^\textrm{\scriptsize 124}$,
\AtlasOrcid[0000-0002-6972-7473]{T.~Strebler}$^\textrm{\scriptsize 105}$,
\AtlasOrcid[0000-0003-0958-7656]{P.~Strizenec}$^\textrm{\scriptsize 29b}$,
\AtlasOrcid[0000-0002-0062-2438]{R.~Str\"ohmer}$^\textrm{\scriptsize 171}$,
\AtlasOrcid[0000-0002-8302-386X]{D.M.~Strom}$^\textrm{\scriptsize 127}$,
\AtlasOrcid[0000-0002-7863-3778]{R.~Stroynowski}$^\textrm{\scriptsize 46}$,
\AtlasOrcid[0000-0002-2382-6951]{A.~Strubig}$^\textrm{\scriptsize 49a,49b}$,
\AtlasOrcid[0000-0002-1639-4484]{S.A.~Stucci}$^\textrm{\scriptsize 30}$,
\AtlasOrcid[0000-0002-1728-9272]{B.~Stugu}$^\textrm{\scriptsize 17}$,
\AtlasOrcid[0000-0001-9610-0783]{J.~Stupak}$^\textrm{\scriptsize 124}$,
\AtlasOrcid[0000-0001-6976-9457]{N.A.~Styles}$^\textrm{\scriptsize 50}$,
\AtlasOrcid[0000-0001-6980-0215]{D.~Su}$^\textrm{\scriptsize 147}$,
\AtlasOrcid[0000-0002-7356-4961]{S.~Su}$^\textrm{\scriptsize 64a}$,
\AtlasOrcid[0000-0001-7755-5280]{W.~Su}$^\textrm{\scriptsize 64d}$,
\AtlasOrcid[0000-0001-9155-3898]{X.~Su}$^\textrm{\scriptsize 64a}$,
\AtlasOrcid[0009-0007-2966-1063]{D.~Suchy}$^\textrm{\scriptsize 29a}$,
\AtlasOrcid[0000-0003-4364-006X]{K.~Sugizaki}$^\textrm{\scriptsize 157}$,
\AtlasOrcid[0000-0003-3943-2495]{V.V.~Sulin}$^\textrm{\scriptsize 38}$,
\AtlasOrcid[0000-0002-4807-6448]{M.J.~Sullivan}$^\textrm{\scriptsize 95}$,
\AtlasOrcid[0000-0003-2925-279X]{D.M.S.~Sultan}$^\textrm{\scriptsize 130}$,
\AtlasOrcid[0000-0002-0059-0165]{L.~Sultanaliyeva}$^\textrm{\scriptsize 38}$,
\AtlasOrcid[0000-0003-2340-748X]{S.~Sultansoy}$^\textrm{\scriptsize 3b}$,
\AtlasOrcid[0000-0002-2685-6187]{T.~Sumida}$^\textrm{\scriptsize 90}$,
\AtlasOrcid[0000-0001-5295-6563]{S.~Sun}$^\textrm{\scriptsize 175}$,
\AtlasOrcid[0000-0002-6277-1877]{O.~Sunneborn~Gudnadottir}$^\textrm{\scriptsize 166}$,
\AtlasOrcid[0000-0001-5233-553X]{N.~Sur}$^\textrm{\scriptsize 105}$,
\AtlasOrcid[0000-0003-4893-8041]{M.R.~Sutton}$^\textrm{\scriptsize 150}$,
\AtlasOrcid[0000-0002-6375-5596]{H.~Suzuki}$^\textrm{\scriptsize 161}$,
\AtlasOrcid[0000-0002-7199-3383]{M.~Svatos}$^\textrm{\scriptsize 135}$,
\AtlasOrcid[0000-0001-7287-0468]{M.~Swiatlowski}$^\textrm{\scriptsize 160a}$,
\AtlasOrcid[0000-0002-4679-6767]{T.~Swirski}$^\textrm{\scriptsize 171}$,
\AtlasOrcid[0000-0003-3447-5621]{I.~Sykora}$^\textrm{\scriptsize 29a}$,
\AtlasOrcid[0000-0003-4422-6493]{M.~Sykora}$^\textrm{\scriptsize 137}$,
\AtlasOrcid[0000-0001-9585-7215]{T.~Sykora}$^\textrm{\scriptsize 137}$,
\AtlasOrcid[0000-0002-0918-9175]{D.~Ta}$^\textrm{\scriptsize 103}$,
\AtlasOrcid[0000-0003-3917-3761]{K.~Tackmann}$^\textrm{\scriptsize 50,t}$,
\AtlasOrcid[0000-0002-5800-4798]{A.~Taffard}$^\textrm{\scriptsize 163}$,
\AtlasOrcid[0000-0003-3425-794X]{R.~Tafirout}$^\textrm{\scriptsize 160a}$,
\AtlasOrcid[0000-0002-0703-4452]{J.S.~Tafoya~Vargas}$^\textrm{\scriptsize 68}$,
\AtlasOrcid[0000-0002-3143-8510]{Y.~Takubo}$^\textrm{\scriptsize 86}$,
\AtlasOrcid[0000-0001-9985-6033]{M.~Talby}$^\textrm{\scriptsize 105}$,
\AtlasOrcid[0000-0001-8560-3756]{A.A.~Talyshev}$^\textrm{\scriptsize 38}$,
\AtlasOrcid[0000-0002-1433-2140]{K.C.~Tam}$^\textrm{\scriptsize 66b}$,
\AtlasOrcid{N.M.~Tamir}$^\textrm{\scriptsize 155}$,
\AtlasOrcid[0000-0002-9166-7083]{A.~Tanaka}$^\textrm{\scriptsize 157}$,
\AtlasOrcid[0000-0001-9994-5802]{J.~Tanaka}$^\textrm{\scriptsize 157}$,
\AtlasOrcid[0000-0002-9929-1797]{R.~Tanaka}$^\textrm{\scriptsize 68}$,
\AtlasOrcid[0000-0002-6313-4175]{M.~Tanasini}$^\textrm{\scriptsize 149}$,
\AtlasOrcid[0000-0003-0362-8795]{Z.~Tao}$^\textrm{\scriptsize 169}$,
\AtlasOrcid[0000-0002-3659-7270]{S.~Tapia~Araya}$^\textrm{\scriptsize 141f}$,
\AtlasOrcid[0000-0003-1251-3332]{S.~Tapprogge}$^\textrm{\scriptsize 103}$,
\AtlasOrcid[0000-0002-9252-7605]{A.~Tarek~Abouelfadl~Mohamed}$^\textrm{\scriptsize 110}$,
\AtlasOrcid[0000-0002-9296-7272]{S.~Tarem}$^\textrm{\scriptsize 154}$,
\AtlasOrcid[0000-0002-0584-8700]{K.~Tariq}$^\textrm{\scriptsize 14}$,
\AtlasOrcid[0000-0002-5060-2208]{G.~Tarna}$^\textrm{\scriptsize 28b}$,
\AtlasOrcid[0000-0002-4244-502X]{G.F.~Tartarelli}$^\textrm{\scriptsize 73a}$,
\AtlasOrcid[0000-0002-3893-8016]{M.J.~Tartarin}$^\textrm{\scriptsize 92}$,
\AtlasOrcid[0000-0001-5785-7548]{P.~Tas}$^\textrm{\scriptsize 137}$,
\AtlasOrcid[0000-0002-1535-9732]{M.~Tasevsky}$^\textrm{\scriptsize 135}$,
\AtlasOrcid[0000-0002-3335-6500]{E.~Tassi}$^\textrm{\scriptsize 45b,45a}$,
\AtlasOrcid[0000-0003-1583-2611]{A.C.~Tate}$^\textrm{\scriptsize 167}$,
\AtlasOrcid[0000-0003-3348-0234]{G.~Tateno}$^\textrm{\scriptsize 157}$,
\AtlasOrcid[0000-0001-8760-7259]{Y.~Tayalati}$^\textrm{\scriptsize 36e,v}$,
\AtlasOrcid[0000-0002-1831-4871]{G.N.~Taylor}$^\textrm{\scriptsize 108}$,
\AtlasOrcid[0000-0002-6596-9125]{W.~Taylor}$^\textrm{\scriptsize 160b}$,
\AtlasOrcid[0000-0001-9977-3836]{P.~Teixeira-Dias}$^\textrm{\scriptsize 98}$,
\AtlasOrcid[0000-0003-4803-5213]{J.J.~Teoh}$^\textrm{\scriptsize 159}$,
\AtlasOrcid[0000-0001-6520-8070]{K.~Terashi}$^\textrm{\scriptsize 157}$,
\AtlasOrcid[0000-0003-0132-5723]{J.~Terron}$^\textrm{\scriptsize 102}$,
\AtlasOrcid[0000-0003-3388-3906]{S.~Terzo}$^\textrm{\scriptsize 13}$,
\AtlasOrcid[0000-0003-1274-8967]{M.~Testa}$^\textrm{\scriptsize 55}$,
\AtlasOrcid[0000-0002-8768-2272]{R.J.~Teuscher}$^\textrm{\scriptsize 159,w}$,
\AtlasOrcid[0000-0003-0134-4377]{A.~Thaler}$^\textrm{\scriptsize 81}$,
\AtlasOrcid[0000-0002-6558-7311]{O.~Theiner}$^\textrm{\scriptsize 58}$,
\AtlasOrcid[0000-0002-9746-4172]{T.~Theveneaux-Pelzer}$^\textrm{\scriptsize 105}$,
\AtlasOrcid[0000-0001-9454-2481]{O.~Thielmann}$^\textrm{\scriptsize 176}$,
\AtlasOrcid{D.W.~Thomas}$^\textrm{\scriptsize 98}$,
\AtlasOrcid[0000-0001-6965-6604]{J.P.~Thomas}$^\textrm{\scriptsize 21}$,
\AtlasOrcid[0000-0001-7050-8203]{E.A.~Thompson}$^\textrm{\scriptsize 18a}$,
\AtlasOrcid[0000-0002-6239-7715]{P.D.~Thompson}$^\textrm{\scriptsize 21}$,
\AtlasOrcid[0000-0001-6031-2768]{E.~Thomson}$^\textrm{\scriptsize 132}$,
\AtlasOrcid[0009-0006-4037-0972]{R.E.~Thornberry}$^\textrm{\scriptsize 46}$,
\AtlasOrcid[0009-0009-3407-6648]{C.~Tian}$^\textrm{\scriptsize 64a}$,
\AtlasOrcid[0000-0001-8739-9250]{Y.~Tian}$^\textrm{\scriptsize 58}$,
\AtlasOrcid[0000-0002-9634-0581]{V.~Tikhomirov}$^\textrm{\scriptsize 38,a}$,
\AtlasOrcid[0000-0002-8023-6448]{Yu.A.~Tikhonov}$^\textrm{\scriptsize 38}$,
\AtlasOrcid{S.~Timoshenko}$^\textrm{\scriptsize 38}$,
\AtlasOrcid[0000-0003-0439-9795]{D.~Timoshyn}$^\textrm{\scriptsize 137}$,
\AtlasOrcid[0000-0002-5886-6339]{E.X.L.~Ting}$^\textrm{\scriptsize 1}$,
\AtlasOrcid[0000-0002-3698-3585]{P.~Tipton}$^\textrm{\scriptsize 177}$,
\AtlasOrcid[0000-0002-7332-5098]{A.~Tishelman-Charny}$^\textrm{\scriptsize 30}$,
\AtlasOrcid[0000-0002-4934-1661]{S.H.~Tlou}$^\textrm{\scriptsize 34g}$,
\AtlasOrcid[0000-0003-2445-1132]{K.~Todome}$^\textrm{\scriptsize 158}$,
\AtlasOrcid[0000-0003-2433-231X]{S.~Todorova-Nova}$^\textrm{\scriptsize 137}$,
\AtlasOrcid{S.~Todt}$^\textrm{\scriptsize 52}$,
\AtlasOrcid[0000-0001-7170-410X]{L.~Toffolin}$^\textrm{\scriptsize 71a,71c}$,
\AtlasOrcid[0000-0002-1128-4200]{M.~Togawa}$^\textrm{\scriptsize 86}$,
\AtlasOrcid[0000-0003-4666-3208]{J.~Tojo}$^\textrm{\scriptsize 91}$,
\AtlasOrcid[0000-0001-8777-0590]{S.~Tok\'ar}$^\textrm{\scriptsize 29a}$,
\AtlasOrcid[0000-0002-8262-1577]{K.~Tokushuku}$^\textrm{\scriptsize 86}$,
\AtlasOrcid[0000-0002-8286-8780]{O.~Toldaiev}$^\textrm{\scriptsize 70}$,
\AtlasOrcid[0000-0002-4603-2070]{M.~Tomoto}$^\textrm{\scriptsize 86,114}$,
\AtlasOrcid[0000-0001-8127-9653]{L.~Tompkins}$^\textrm{\scriptsize 147,l}$,
\AtlasOrcid[0000-0002-9312-1842]{K.W.~Topolnicki}$^\textrm{\scriptsize 88b}$,
\AtlasOrcid[0000-0003-2911-8910]{E.~Torrence}$^\textrm{\scriptsize 127}$,
\AtlasOrcid[0000-0003-0822-1206]{H.~Torres}$^\textrm{\scriptsize 92}$,
\AtlasOrcid[0000-0002-5507-7924]{E.~Torr\'o~Pastor}$^\textrm{\scriptsize 168}$,
\AtlasOrcid[0000-0001-9898-480X]{M.~Toscani}$^\textrm{\scriptsize 31}$,
\AtlasOrcid[0000-0001-6485-2227]{C.~Tosciri}$^\textrm{\scriptsize 41}$,
\AtlasOrcid[0000-0002-1647-4329]{M.~Tost}$^\textrm{\scriptsize 11}$,
\AtlasOrcid[0000-0001-5543-6192]{D.R.~Tovey}$^\textrm{\scriptsize 143}$,
\AtlasOrcid[0000-0003-1094-6409]{I.S.~Trandafir}$^\textrm{\scriptsize 28b}$,
\AtlasOrcid[0000-0002-9820-1729]{T.~Trefzger}$^\textrm{\scriptsize 171}$,
\AtlasOrcid[0000-0002-8224-6105]{A.~Tricoli}$^\textrm{\scriptsize 30}$,
\AtlasOrcid[0000-0002-6127-5847]{I.M.~Trigger}$^\textrm{\scriptsize 160a}$,
\AtlasOrcid[0000-0001-5913-0828]{S.~Trincaz-Duvoid}$^\textrm{\scriptsize 131}$,
\AtlasOrcid[0000-0001-6204-4445]{D.A.~Trischuk}$^\textrm{\scriptsize 27}$,
\AtlasOrcid[0000-0001-9500-2487]{B.~Trocm\'e}$^\textrm{\scriptsize 62}$,
\AtlasOrcid{A.~Tropina}$^\textrm{\scriptsize 40}$,
\AtlasOrcid[0000-0001-8249-7150]{L.~Truong}$^\textrm{\scriptsize 34c}$,
\AtlasOrcid[0000-0002-5151-7101]{M.~Trzebinski}$^\textrm{\scriptsize 89}$,
\AtlasOrcid[0000-0001-6938-5867]{A.~Trzupek}$^\textrm{\scriptsize 89}$,
\AtlasOrcid[0000-0001-7878-6435]{F.~Tsai}$^\textrm{\scriptsize 149}$,
\AtlasOrcid[0000-0002-4728-9150]{M.~Tsai}$^\textrm{\scriptsize 109}$,
\AtlasOrcid[0000-0002-8761-4632]{A.~Tsiamis}$^\textrm{\scriptsize 156}$,
\AtlasOrcid{P.V.~Tsiareshka}$^\textrm{\scriptsize 40}$,
\AtlasOrcid[0000-0002-6393-2302]{S.~Tsigaridas}$^\textrm{\scriptsize 160a}$,
\AtlasOrcid[0000-0002-6632-0440]{A.~Tsirigotis}$^\textrm{\scriptsize 156,q}$,
\AtlasOrcid[0000-0002-2119-8875]{V.~Tsiskaridze}$^\textrm{\scriptsize 159}$,
\AtlasOrcid[0000-0002-6071-3104]{E.G.~Tskhadadze}$^\textrm{\scriptsize 153a}$,
\AtlasOrcid[0000-0002-9104-2884]{M.~Tsopoulou}$^\textrm{\scriptsize 156}$,
\AtlasOrcid[0000-0002-8784-5684]{Y.~Tsujikawa}$^\textrm{\scriptsize 90}$,
\AtlasOrcid[0000-0002-8965-6676]{I.I.~Tsukerman}$^\textrm{\scriptsize 38}$,
\AtlasOrcid[0000-0001-8157-6711]{V.~Tsulaia}$^\textrm{\scriptsize 18a}$,
\AtlasOrcid[0000-0002-2055-4364]{S.~Tsuno}$^\textrm{\scriptsize 86}$,
\AtlasOrcid[0000-0001-6263-9879]{K.~Tsuri}$^\textrm{\scriptsize 122}$,
\AtlasOrcid[0000-0001-8212-6894]{D.~Tsybychev}$^\textrm{\scriptsize 149}$,
\AtlasOrcid[0000-0002-5865-183X]{Y.~Tu}$^\textrm{\scriptsize 66b}$,
\AtlasOrcid[0000-0001-6307-1437]{A.~Tudorache}$^\textrm{\scriptsize 28b}$,
\AtlasOrcid[0000-0001-5384-3843]{V.~Tudorache}$^\textrm{\scriptsize 28b}$,
\AtlasOrcid[0000-0002-7672-7754]{A.N.~Tuna}$^\textrm{\scriptsize 63}$,
\AtlasOrcid[0000-0001-6506-3123]{S.~Turchikhin}$^\textrm{\scriptsize 59b,59a}$,
\AtlasOrcid[0000-0002-0726-5648]{I.~Turk~Cakir}$^\textrm{\scriptsize 3a}$,
\AtlasOrcid[0000-0001-8740-796X]{R.~Turra}$^\textrm{\scriptsize 73a}$,
\AtlasOrcid[0000-0001-9471-8627]{T.~Turtuvshin}$^\textrm{\scriptsize 40}$,
\AtlasOrcid[0000-0001-6131-5725]{P.M.~Tuts}$^\textrm{\scriptsize 43}$,
\AtlasOrcid[0000-0002-8363-1072]{S.~Tzamarias}$^\textrm{\scriptsize 156,d}$,
\AtlasOrcid[0000-0002-0410-0055]{E.~Tzovara}$^\textrm{\scriptsize 103}$,
\AtlasOrcid[0000-0002-9813-7931]{F.~Ukegawa}$^\textrm{\scriptsize 161}$,
\AtlasOrcid[0000-0002-0789-7581]{P.A.~Ulloa~Poblete}$^\textrm{\scriptsize 141c,141b}$,
\AtlasOrcid[0000-0001-7725-8227]{E.N.~Umaka}$^\textrm{\scriptsize 30}$,
\AtlasOrcid[0000-0001-8130-7423]{G.~Unal}$^\textrm{\scriptsize 37}$,
\AtlasOrcid[0000-0002-1384-286X]{A.~Undrus}$^\textrm{\scriptsize 30}$,
\AtlasOrcid[0000-0002-3274-6531]{G.~Unel}$^\textrm{\scriptsize 163}$,
\AtlasOrcid[0000-0002-7633-8441]{J.~Urban}$^\textrm{\scriptsize 29b}$,
\AtlasOrcid[0000-0001-8309-2227]{P.~Urrejola}$^\textrm{\scriptsize 141a}$,
\AtlasOrcid[0000-0001-5032-7907]{G.~Usai}$^\textrm{\scriptsize 8}$,
\AtlasOrcid[0000-0002-4241-8937]{R.~Ushioda}$^\textrm{\scriptsize 158}$,
\AtlasOrcid[0000-0003-1950-0307]{M.~Usman}$^\textrm{\scriptsize 111}$,
\AtlasOrcid[0009-0000-2512-020X]{F.~Ustuner}$^\textrm{\scriptsize 54}$,
\AtlasOrcid[0000-0002-7110-8065]{Z.~Uysal}$^\textrm{\scriptsize 84}$,
\AtlasOrcid[0000-0001-9584-0392]{V.~Vacek}$^\textrm{\scriptsize 136}$,
\AtlasOrcid[0000-0001-8703-6978]{B.~Vachon}$^\textrm{\scriptsize 107}$,
\AtlasOrcid[0000-0003-1492-5007]{T.~Vafeiadis}$^\textrm{\scriptsize 37}$,
\AtlasOrcid[0000-0002-0393-666X]{A.~Vaitkus}$^\textrm{\scriptsize 99}$,
\AtlasOrcid[0000-0001-9362-8451]{C.~Valderanis}$^\textrm{\scriptsize 112}$,
\AtlasOrcid[0000-0001-9931-2896]{E.~Valdes~Santurio}$^\textrm{\scriptsize 49a,49b}$,
\AtlasOrcid[0000-0002-0486-9569]{M.~Valente}$^\textrm{\scriptsize 160a}$,
\AtlasOrcid[0000-0003-2044-6539]{S.~Valentinetti}$^\textrm{\scriptsize 24b,24a}$,
\AtlasOrcid[0000-0002-9776-5880]{A.~Valero}$^\textrm{\scriptsize 168}$,
\AtlasOrcid[0000-0002-9784-5477]{E.~Valiente~Moreno}$^\textrm{\scriptsize 168}$,
\AtlasOrcid[0000-0002-5496-349X]{A.~Vallier}$^\textrm{\scriptsize 92}$,
\AtlasOrcid[0000-0002-3953-3117]{J.A.~Valls~Ferrer}$^\textrm{\scriptsize 168}$,
\AtlasOrcid[0000-0002-3895-8084]{D.R.~Van~Arneman}$^\textrm{\scriptsize 118}$,
\AtlasOrcid[0000-0002-2254-125X]{T.R.~Van~Daalen}$^\textrm{\scriptsize 142}$,
\AtlasOrcid[0000-0002-2854-3811]{A.~Van~Der~Graaf}$^\textrm{\scriptsize 51}$,
\AtlasOrcid[0000-0002-7227-4006]{P.~Van~Gemmeren}$^\textrm{\scriptsize 6}$,
\AtlasOrcid[0000-0003-3728-5102]{M.~Van~Rijnbach}$^\textrm{\scriptsize 37}$,
\AtlasOrcid[0000-0002-7969-0301]{S.~Van~Stroud}$^\textrm{\scriptsize 99}$,
\AtlasOrcid[0000-0001-7074-5655]{I.~Van~Vulpen}$^\textrm{\scriptsize 118}$,
\AtlasOrcid[0000-0002-9701-792X]{P.~Vana}$^\textrm{\scriptsize 137}$,
\AtlasOrcid[0000-0003-2684-276X]{M.~Vanadia}$^\textrm{\scriptsize 78a,78b}$,
\AtlasOrcid[0009-0007-3175-5325]{U.M.~Vande~Voorde}$^\textrm{\scriptsize 148}$,
\AtlasOrcid[0000-0001-6581-9410]{W.~Vandelli}$^\textrm{\scriptsize 37}$,
\AtlasOrcid[0000-0003-3453-6156]{E.R.~Vandewall}$^\textrm{\scriptsize 125}$,
\AtlasOrcid[0000-0001-6814-4674]{D.~Vannicola}$^\textrm{\scriptsize 155}$,
\AtlasOrcid[0000-0002-9866-6040]{L.~Vannoli}$^\textrm{\scriptsize 55}$,
\AtlasOrcid[0000-0002-2814-1337]{R.~Vari}$^\textrm{\scriptsize 77a}$,
\AtlasOrcid[0000-0001-7820-9144]{E.W.~Varnes}$^\textrm{\scriptsize 7}$,
\AtlasOrcid[0000-0001-6733-4310]{C.~Varni}$^\textrm{\scriptsize 18b}$,
\AtlasOrcid[0000-0002-0697-5808]{T.~Varol}$^\textrm{\scriptsize 152}$,
\AtlasOrcid[0000-0002-0734-4442]{D.~Varouchas}$^\textrm{\scriptsize 68}$,
\AtlasOrcid[0000-0003-4375-5190]{L.~Varriale}$^\textrm{\scriptsize 168}$,
\AtlasOrcid[0000-0003-1017-1295]{K.E.~Varvell}$^\textrm{\scriptsize 151}$,
\AtlasOrcid[0000-0001-8415-0759]{M.E.~Vasile}$^\textrm{\scriptsize 28b}$,
\AtlasOrcid{L.~Vaslin}$^\textrm{\scriptsize 86}$,
\AtlasOrcid[0000-0003-2460-1276]{A.~Vasyukov}$^\textrm{\scriptsize 40}$,
\AtlasOrcid[0009-0005-8446-5255]{L.M.~Vaughan}$^\textrm{\scriptsize 125}$,
\AtlasOrcid{R.~Vavricka}$^\textrm{\scriptsize 103}$,
\AtlasOrcid[0000-0002-9780-099X]{T.~Vazquez~Schroeder}$^\textrm{\scriptsize 37}$,
\AtlasOrcid[0000-0003-0855-0958]{J.~Veatch}$^\textrm{\scriptsize 32}$,
\AtlasOrcid[0000-0002-1351-6757]{V.~Vecchio}$^\textrm{\scriptsize 104}$,
\AtlasOrcid[0000-0001-5284-2451]{M.J.~Veen}$^\textrm{\scriptsize 106}$,
\AtlasOrcid[0000-0003-2432-3309]{I.~Veliscek}$^\textrm{\scriptsize 30}$,
\AtlasOrcid[0000-0003-1827-2955]{L.M.~Veloce}$^\textrm{\scriptsize 159}$,
\AtlasOrcid[0000-0002-5956-4244]{F.~Veloso}$^\textrm{\scriptsize 134a,134c}$,
\AtlasOrcid[0000-0002-2598-2659]{S.~Veneziano}$^\textrm{\scriptsize 77a}$,
\AtlasOrcid[0000-0002-3368-3413]{A.~Ventura}$^\textrm{\scriptsize 72a,72b}$,
\AtlasOrcid[0000-0001-5246-0779]{S.~Ventura~Gonzalez}$^\textrm{\scriptsize 139}$,
\AtlasOrcid[0000-0002-3713-8033]{A.~Verbytskyi}$^\textrm{\scriptsize 113}$,
\AtlasOrcid[0000-0001-8209-4757]{M.~Verducci}$^\textrm{\scriptsize 76a,76b}$,
\AtlasOrcid[0000-0002-3228-6715]{C.~Vergis}$^\textrm{\scriptsize 97}$,
\AtlasOrcid[0000-0001-8060-2228]{M.~Verissimo~De~Araujo}$^\textrm{\scriptsize 85b}$,
\AtlasOrcid[0000-0001-5468-2025]{W.~Verkerke}$^\textrm{\scriptsize 118}$,
\AtlasOrcid[0000-0003-4378-5736]{J.C.~Vermeulen}$^\textrm{\scriptsize 118}$,
\AtlasOrcid[0000-0002-0235-1053]{C.~Vernieri}$^\textrm{\scriptsize 147}$,
\AtlasOrcid[0000-0001-8669-9139]{M.~Vessella}$^\textrm{\scriptsize 106}$,
\AtlasOrcid[0000-0002-7223-2965]{M.C.~Vetterli}$^\textrm{\scriptsize 146,ab}$,
\AtlasOrcid[0000-0002-7011-9432]{A.~Vgenopoulos}$^\textrm{\scriptsize 103}$,
\AtlasOrcid[0000-0002-5102-9140]{N.~Viaux~Maira}$^\textrm{\scriptsize 141f}$,
\AtlasOrcid[0000-0002-1596-2611]{T.~Vickey}$^\textrm{\scriptsize 143}$,
\AtlasOrcid[0000-0002-6497-6809]{O.E.~Vickey~Boeriu}$^\textrm{\scriptsize 143}$,
\AtlasOrcid[0000-0002-0237-292X]{G.H.A.~Viehhauser}$^\textrm{\scriptsize 130}$,
\AtlasOrcid[0000-0002-6270-9176]{L.~Vigani}$^\textrm{\scriptsize 65b}$,
\AtlasOrcid[0000-0003-2281-3822]{M.~Vigl}$^\textrm{\scriptsize 113}$,
\AtlasOrcid[0000-0002-9181-8048]{M.~Villa}$^\textrm{\scriptsize 24b,24a}$,
\AtlasOrcid[0000-0002-0048-4602]{M.~Villaplana~Perez}$^\textrm{\scriptsize 168}$,
\AtlasOrcid{E.M.~Villhauer}$^\textrm{\scriptsize 54}$,
\AtlasOrcid[0000-0002-4839-6281]{E.~Vilucchi}$^\textrm{\scriptsize 55}$,
\AtlasOrcid[0000-0002-5338-8972]{M.G.~Vincter}$^\textrm{\scriptsize 35}$,
\AtlasOrcid{A.~Visibile}$^\textrm{\scriptsize 118}$,
\AtlasOrcid[0000-0001-9156-970X]{C.~Vittori}$^\textrm{\scriptsize 37}$,
\AtlasOrcid[0000-0003-0097-123X]{I.~Vivarelli}$^\textrm{\scriptsize 24b,24a}$,
\AtlasOrcid[0000-0003-2987-3772]{E.~Voevodina}$^\textrm{\scriptsize 113}$,
\AtlasOrcid[0000-0001-8891-8606]{F.~Vogel}$^\textrm{\scriptsize 112}$,
\AtlasOrcid[0009-0005-7503-3370]{J.C.~Voigt}$^\textrm{\scriptsize 52}$,
\AtlasOrcid[0000-0002-3429-4778]{P.~Vokac}$^\textrm{\scriptsize 136}$,
\AtlasOrcid[0000-0002-3114-3798]{Yu.~Volkotrub}$^\textrm{\scriptsize 88b}$,
\AtlasOrcid[0000-0001-8899-4027]{E.~Von~Toerne}$^\textrm{\scriptsize 25}$,
\AtlasOrcid[0000-0003-2607-7287]{B.~Vormwald}$^\textrm{\scriptsize 37}$,
\AtlasOrcid[0000-0001-8757-2180]{V.~Vorobel}$^\textrm{\scriptsize 137}$,
\AtlasOrcid[0000-0002-7110-8516]{K.~Vorobev}$^\textrm{\scriptsize 38}$,
\AtlasOrcid[0000-0001-8474-5357]{M.~Vos}$^\textrm{\scriptsize 168}$,
\AtlasOrcid[0000-0002-4157-0996]{K.~Voss}$^\textrm{\scriptsize 145}$,
\AtlasOrcid[0000-0002-7561-204X]{M.~Vozak}$^\textrm{\scriptsize 118}$,
\AtlasOrcid[0000-0003-2541-4827]{L.~Vozdecky}$^\textrm{\scriptsize 124}$,
\AtlasOrcid[0000-0001-5415-5225]{N.~Vranjes}$^\textrm{\scriptsize 16}$,
\AtlasOrcid[0000-0003-4477-9733]{M.~Vranjes~Milosavljevic}$^\textrm{\scriptsize 16}$,
\AtlasOrcid[0000-0001-8083-0001]{M.~Vreeswijk}$^\textrm{\scriptsize 118}$,
\AtlasOrcid[0000-0002-6251-1178]{N.K.~Vu}$^\textrm{\scriptsize 64d,64c}$,
\AtlasOrcid[0000-0003-3208-9209]{R.~Vuillermet}$^\textrm{\scriptsize 37}$,
\AtlasOrcid[0000-0003-3473-7038]{O.~Vujinovic}$^\textrm{\scriptsize 103}$,
\AtlasOrcid[0000-0003-0472-3516]{I.~Vukotic}$^\textrm{\scriptsize 41}$,
\AtlasOrcid[0009-0008-7683-7428]{I.K.~Vyas}$^\textrm{\scriptsize 35}$,
\AtlasOrcid[0000-0002-8600-9799]{S.~Wada}$^\textrm{\scriptsize 161}$,
\AtlasOrcid{C.~Wagner}$^\textrm{\scriptsize 147}$,
\AtlasOrcid[0000-0002-5588-0020]{J.M.~Wagner}$^\textrm{\scriptsize 18a}$,
\AtlasOrcid[0000-0002-9198-5911]{W.~Wagner}$^\textrm{\scriptsize 176}$,
\AtlasOrcid[0000-0002-6324-8551]{S.~Wahdan}$^\textrm{\scriptsize 176}$,
\AtlasOrcid[0000-0003-0616-7330]{H.~Wahlberg}$^\textrm{\scriptsize 93}$,
\AtlasOrcid{C.H.~Waits}$^\textrm{\scriptsize 124}$,
\AtlasOrcid[0000-0002-9039-8758]{J.~Walder}$^\textrm{\scriptsize 138}$,
\AtlasOrcid[0000-0001-8535-4809]{R.~Walker}$^\textrm{\scriptsize 112}$,
\AtlasOrcid[0000-0002-0385-3784]{W.~Walkowiak}$^\textrm{\scriptsize 145}$,
\AtlasOrcid[0000-0002-7867-7922]{A.~Wall}$^\textrm{\scriptsize 132}$,
\AtlasOrcid[0000-0002-4848-5540]{E.J.~Wallin}$^\textrm{\scriptsize 101}$,
\AtlasOrcid[0000-0001-5551-5456]{T.~Wamorkar}$^\textrm{\scriptsize 6}$,
\AtlasOrcid[0000-0003-2482-711X]{A.Z.~Wang}$^\textrm{\scriptsize 140}$,
\AtlasOrcid[0000-0001-9116-055X]{C.~Wang}$^\textrm{\scriptsize 103}$,
\AtlasOrcid[0000-0002-8487-8480]{C.~Wang}$^\textrm{\scriptsize 11}$,
\AtlasOrcid[0000-0003-3952-8139]{H.~Wang}$^\textrm{\scriptsize 18a}$,
\AtlasOrcid[0000-0002-5246-5497]{J.~Wang}$^\textrm{\scriptsize 66c}$,
\AtlasOrcid[0000-0001-7613-5997]{P.~Wang}$^\textrm{\scriptsize 99}$,
\AtlasOrcid[0000-0001-9839-608X]{R.~Wang}$^\textrm{\scriptsize 63}$,
\AtlasOrcid[0000-0001-8530-6487]{R.~Wang}$^\textrm{\scriptsize 6}$,
\AtlasOrcid[0000-0002-5821-4875]{S.M.~Wang}$^\textrm{\scriptsize 152}$,
\AtlasOrcid[0000-0001-7477-4955]{S.~Wang}$^\textrm{\scriptsize 14}$,
\AtlasOrcid[0000-0002-1152-2221]{T.~Wang}$^\textrm{\scriptsize 64a}$,
\AtlasOrcid[0000-0002-7184-9891]{W.T.~Wang}$^\textrm{\scriptsize 82}$,
\AtlasOrcid[0000-0001-9714-9319]{W.~Wang}$^\textrm{\scriptsize 14}$,
\AtlasOrcid[0000-0002-6229-1945]{X.~Wang}$^\textrm{\scriptsize 115a}$,
\AtlasOrcid[0000-0002-2411-7399]{X.~Wang}$^\textrm{\scriptsize 167}$,
\AtlasOrcid[0000-0001-5173-2234]{X.~Wang}$^\textrm{\scriptsize 64c}$,
\AtlasOrcid[0000-0003-2693-3442]{Y.~Wang}$^\textrm{\scriptsize 64d}$,
\AtlasOrcid[0000-0003-4693-5365]{Y.~Wang}$^\textrm{\scriptsize 115a}$,
\AtlasOrcid[0009-0003-3345-4359]{Y.~Wang}$^\textrm{\scriptsize 64a}$,
\AtlasOrcid[0000-0002-0928-2070]{Z.~Wang}$^\textrm{\scriptsize 109}$,
\AtlasOrcid[0000-0002-9862-3091]{Z.~Wang}$^\textrm{\scriptsize 64d,53,64c}$,
\AtlasOrcid[0000-0003-0756-0206]{Z.~Wang}$^\textrm{\scriptsize 109}$,
\AtlasOrcid[0000-0002-2298-7315]{A.~Warburton}$^\textrm{\scriptsize 107}$,
\AtlasOrcid[0000-0001-5530-9919]{R.J.~Ward}$^\textrm{\scriptsize 21}$,
\AtlasOrcid[0000-0002-8268-8325]{N.~Warrack}$^\textrm{\scriptsize 61}$,
\AtlasOrcid[0000-0002-6382-1573]{S.~Waterhouse}$^\textrm{\scriptsize 98}$,
\AtlasOrcid[0000-0001-7052-7973]{A.T.~Watson}$^\textrm{\scriptsize 21}$,
\AtlasOrcid[0000-0003-3704-5782]{H.~Watson}$^\textrm{\scriptsize 54}$,
\AtlasOrcid[0000-0002-9724-2684]{M.F.~Watson}$^\textrm{\scriptsize 21}$,
\AtlasOrcid[0000-0003-3352-126X]{E.~Watton}$^\textrm{\scriptsize 61,138}$,
\AtlasOrcid[0000-0002-0753-7308]{G.~Watts}$^\textrm{\scriptsize 142}$,
\AtlasOrcid[0000-0003-0872-8920]{B.M.~Waugh}$^\textrm{\scriptsize 99}$,
\AtlasOrcid[0000-0002-5294-6856]{J.M.~Webb}$^\textrm{\scriptsize 56}$,
\AtlasOrcid[0000-0002-8659-5767]{C.~Weber}$^\textrm{\scriptsize 30}$,
\AtlasOrcid[0000-0002-5074-0539]{H.A.~Weber}$^\textrm{\scriptsize 19}$,
\AtlasOrcid[0000-0002-2770-9031]{M.S.~Weber}$^\textrm{\scriptsize 20}$,
\AtlasOrcid[0000-0002-2841-1616]{S.M.~Weber}$^\textrm{\scriptsize 65a}$,
\AtlasOrcid[0000-0001-9524-8452]{C.~Wei}$^\textrm{\scriptsize 64a}$,
\AtlasOrcid[0000-0001-9725-2316]{Y.~Wei}$^\textrm{\scriptsize 56}$,
\AtlasOrcid[0000-0002-5158-307X]{A.R.~Weidberg}$^\textrm{\scriptsize 130}$,
\AtlasOrcid[0000-0003-4563-2346]{E.J.~Weik}$^\textrm{\scriptsize 121}$,
\AtlasOrcid[0000-0003-2165-871X]{J.~Weingarten}$^\textrm{\scriptsize 51}$,
\AtlasOrcid[0000-0002-6456-6834]{C.~Weiser}$^\textrm{\scriptsize 56}$,
\AtlasOrcid[0000-0002-5450-2511]{C.J.~Wells}$^\textrm{\scriptsize 50}$,
\AtlasOrcid[0000-0002-8678-893X]{T.~Wenaus}$^\textrm{\scriptsize 30}$,
\AtlasOrcid[0000-0003-1623-3899]{B.~Wendland}$^\textrm{\scriptsize 51}$,
\AtlasOrcid[0000-0002-4375-5265]{T.~Wengler}$^\textrm{\scriptsize 37}$,
\AtlasOrcid{N.S.~Wenke}$^\textrm{\scriptsize 113}$,
\AtlasOrcid[0000-0001-9971-0077]{N.~Wermes}$^\textrm{\scriptsize 25}$,
\AtlasOrcid[0000-0002-8192-8999]{M.~Wessels}$^\textrm{\scriptsize 65a}$,
\AtlasOrcid[0000-0002-9507-1869]{A.M.~Wharton}$^\textrm{\scriptsize 94}$,
\AtlasOrcid[0000-0003-0714-1466]{A.S.~White}$^\textrm{\scriptsize 63}$,
\AtlasOrcid[0000-0001-8315-9778]{A.~White}$^\textrm{\scriptsize 8}$,
\AtlasOrcid[0000-0001-5474-4580]{M.J.~White}$^\textrm{\scriptsize 1}$,
\AtlasOrcid[0000-0002-2005-3113]{D.~Whiteson}$^\textrm{\scriptsize 163}$,
\AtlasOrcid[0000-0002-2711-4820]{L.~Wickremasinghe}$^\textrm{\scriptsize 128}$,
\AtlasOrcid[0000-0003-3605-3633]{W.~Wiedenmann}$^\textrm{\scriptsize 175}$,
\AtlasOrcid[0000-0001-9232-4827]{M.~Wielers}$^\textrm{\scriptsize 138}$,
\AtlasOrcid[0000-0001-6219-8946]{C.~Wiglesworth}$^\textrm{\scriptsize 44}$,
\AtlasOrcid{D.J.~Wilbern}$^\textrm{\scriptsize 124}$,
\AtlasOrcid[0000-0002-8483-9502]{H.G.~Wilkens}$^\textrm{\scriptsize 37}$,
\AtlasOrcid[0000-0003-0924-7889]{J.J.H.~Wilkinson}$^\textrm{\scriptsize 33}$,
\AtlasOrcid[0000-0002-5646-1856]{D.M.~Williams}$^\textrm{\scriptsize 43}$,
\AtlasOrcid{H.H.~Williams}$^\textrm{\scriptsize 132}$,
\AtlasOrcid[0000-0001-6174-401X]{S.~Williams}$^\textrm{\scriptsize 33}$,
\AtlasOrcid[0000-0002-4120-1453]{S.~Willocq}$^\textrm{\scriptsize 106}$,
\AtlasOrcid[0000-0002-7811-7474]{B.J.~Wilson}$^\textrm{\scriptsize 104}$,
\AtlasOrcid[0000-0001-5038-1399]{P.J.~Windischhofer}$^\textrm{\scriptsize 41}$,
\AtlasOrcid[0000-0003-1532-6399]{F.I.~Winkel}$^\textrm{\scriptsize 31}$,
\AtlasOrcid[0000-0001-8290-3200]{F.~Winklmeier}$^\textrm{\scriptsize 127}$,
\AtlasOrcid[0000-0001-9606-7688]{B.T.~Winter}$^\textrm{\scriptsize 56}$,
\AtlasOrcid[0000-0002-6166-6979]{J.K.~Winter}$^\textrm{\scriptsize 104}$,
\AtlasOrcid{M.~Wittgen}$^\textrm{\scriptsize 147}$,
\AtlasOrcid[0000-0002-0688-3380]{M.~Wobisch}$^\textrm{\scriptsize 100}$,
\AtlasOrcid{T.~Wojtkowski}$^\textrm{\scriptsize 62}$,
\AtlasOrcid[0000-0001-5100-2522]{Z.~Wolffs}$^\textrm{\scriptsize 118}$,
\AtlasOrcid{J.~Wollrath}$^\textrm{\scriptsize 37}$,
\AtlasOrcid[0000-0001-9184-2921]{M.W.~Wolter}$^\textrm{\scriptsize 89}$,
\AtlasOrcid[0000-0002-9588-1773]{H.~Wolters}$^\textrm{\scriptsize 134a,134c}$,
\AtlasOrcid{M.C.~Wong}$^\textrm{\scriptsize 140}$,
\AtlasOrcid[0000-0003-3089-022X]{E.L.~Woodward}$^\textrm{\scriptsize 43}$,
\AtlasOrcid[0000-0002-3865-4996]{S.D.~Worm}$^\textrm{\scriptsize 50}$,
\AtlasOrcid[0000-0003-4273-6334]{B.K.~Wosiek}$^\textrm{\scriptsize 89}$,
\AtlasOrcid[0000-0003-1171-0887]{K.W.~Wo\'{z}niak}$^\textrm{\scriptsize 89}$,
\AtlasOrcid[0000-0001-8563-0412]{S.~Wozniewski}$^\textrm{\scriptsize 57}$,
\AtlasOrcid[0000-0002-3298-4900]{K.~Wraight}$^\textrm{\scriptsize 61}$,
\AtlasOrcid[0000-0003-3700-8818]{C.~Wu}$^\textrm{\scriptsize 21}$,
\AtlasOrcid[0000-0001-5283-4080]{M.~Wu}$^\textrm{\scriptsize 115b}$,
\AtlasOrcid[0000-0002-5252-2375]{M.~Wu}$^\textrm{\scriptsize 117}$,
\AtlasOrcid[0000-0001-5866-1504]{S.L.~Wu}$^\textrm{\scriptsize 175}$,
\AtlasOrcid[0000-0001-7655-389X]{X.~Wu}$^\textrm{\scriptsize 58}$,
\AtlasOrcid[0000-0002-1528-4865]{Y.~Wu}$^\textrm{\scriptsize 64a}$,
\AtlasOrcid[0000-0002-5392-902X]{Z.~Wu}$^\textrm{\scriptsize 4}$,
\AtlasOrcid[0000-0002-4055-218X]{J.~Wuerzinger}$^\textrm{\scriptsize 113,z}$,
\AtlasOrcid[0000-0001-9690-2997]{T.R.~Wyatt}$^\textrm{\scriptsize 104}$,
\AtlasOrcid[0000-0001-9895-4475]{B.M.~Wynne}$^\textrm{\scriptsize 54}$,
\AtlasOrcid[0000-0002-0988-1655]{S.~Xella}$^\textrm{\scriptsize 44}$,
\AtlasOrcid[0000-0003-3073-3662]{L.~Xia}$^\textrm{\scriptsize 115a}$,
\AtlasOrcid[0009-0007-3125-1880]{M.~Xia}$^\textrm{\scriptsize 15}$,
\AtlasOrcid[0000-0001-6707-5590]{M.~Xie}$^\textrm{\scriptsize 64a}$,
\AtlasOrcid[0000-0002-7153-4750]{S.~Xin}$^\textrm{\scriptsize 14,115c}$,
\AtlasOrcid[0009-0005-0548-6219]{A.~Xiong}$^\textrm{\scriptsize 127}$,
\AtlasOrcid[0000-0002-4853-7558]{J.~Xiong}$^\textrm{\scriptsize 18a}$,
\AtlasOrcid[0000-0001-6355-2767]{D.~Xu}$^\textrm{\scriptsize 14}$,
\AtlasOrcid[0000-0001-6110-2172]{H.~Xu}$^\textrm{\scriptsize 64a}$,
\AtlasOrcid[0000-0001-8997-3199]{L.~Xu}$^\textrm{\scriptsize 64a}$,
\AtlasOrcid[0000-0002-1928-1717]{R.~Xu}$^\textrm{\scriptsize 132}$,
\AtlasOrcid[0000-0002-0215-6151]{T.~Xu}$^\textrm{\scriptsize 109}$,
\AtlasOrcid[0000-0001-9563-4804]{Y.~Xu}$^\textrm{\scriptsize 142}$,
\AtlasOrcid[0000-0001-9571-3131]{Z.~Xu}$^\textrm{\scriptsize 54}$,
\AtlasOrcid{Z.~Xu}$^\textrm{\scriptsize 115a}$,
\AtlasOrcid[0000-0002-2680-0474]{B.~Yabsley}$^\textrm{\scriptsize 151}$,
\AtlasOrcid[0000-0001-6977-3456]{S.~Yacoob}$^\textrm{\scriptsize 34a}$,
\AtlasOrcid[0000-0002-3725-4800]{Y.~Yamaguchi}$^\textrm{\scriptsize 86}$,
\AtlasOrcid[0000-0003-1721-2176]{E.~Yamashita}$^\textrm{\scriptsize 157}$,
\AtlasOrcid[0000-0003-2123-5311]{H.~Yamauchi}$^\textrm{\scriptsize 161}$,
\AtlasOrcid[0000-0003-0411-3590]{T.~Yamazaki}$^\textrm{\scriptsize 18a}$,
\AtlasOrcid[0000-0003-3710-6995]{Y.~Yamazaki}$^\textrm{\scriptsize 87}$,
\AtlasOrcid[0000-0002-1512-5506]{S.~Yan}$^\textrm{\scriptsize 61}$,
\AtlasOrcid[0000-0002-2483-4937]{Z.~Yan}$^\textrm{\scriptsize 106}$,
\AtlasOrcid[0000-0001-7367-1380]{H.J.~Yang}$^\textrm{\scriptsize 64c,64d}$,
\AtlasOrcid[0000-0003-3554-7113]{H.T.~Yang}$^\textrm{\scriptsize 64a}$,
\AtlasOrcid[0000-0002-0204-984X]{S.~Yang}$^\textrm{\scriptsize 64a}$,
\AtlasOrcid[0000-0002-4996-1924]{T.~Yang}$^\textrm{\scriptsize 66c}$,
\AtlasOrcid[0000-0002-1452-9824]{X.~Yang}$^\textrm{\scriptsize 37}$,
\AtlasOrcid[0000-0002-9201-0972]{X.~Yang}$^\textrm{\scriptsize 14}$,
\AtlasOrcid[0000-0001-8524-1855]{Y.~Yang}$^\textrm{\scriptsize 46}$,
\AtlasOrcid{Y.~Yang}$^\textrm{\scriptsize 64a}$,
\AtlasOrcid[0000-0002-3335-1988]{W-M.~Yao}$^\textrm{\scriptsize 18a}$,
\AtlasOrcid[0000-0002-4886-9851]{H.~Ye}$^\textrm{\scriptsize 115a}$,
\AtlasOrcid[0000-0003-0552-5490]{H.~Ye}$^\textrm{\scriptsize 57}$,
\AtlasOrcid[0000-0001-9274-707X]{J.~Ye}$^\textrm{\scriptsize 14}$,
\AtlasOrcid[0000-0002-7864-4282]{S.~Ye}$^\textrm{\scriptsize 30}$,
\AtlasOrcid[0000-0002-3245-7676]{X.~Ye}$^\textrm{\scriptsize 64a}$,
\AtlasOrcid[0000-0002-8484-9655]{Y.~Yeh}$^\textrm{\scriptsize 99}$,
\AtlasOrcid[0000-0003-0586-7052]{I.~Yeletskikh}$^\textrm{\scriptsize 40}$,
\AtlasOrcid[0000-0002-3372-2590]{B.K.~Yeo}$^\textrm{\scriptsize 18b}$,
\AtlasOrcid[0000-0002-1827-9201]{M.R.~Yexley}$^\textrm{\scriptsize 99}$,
\AtlasOrcid[0000-0002-6689-0232]{T.P.~Yildirim}$^\textrm{\scriptsize 130}$,
\AtlasOrcid[0000-0003-2174-807X]{P.~Yin}$^\textrm{\scriptsize 43}$,
\AtlasOrcid[0000-0003-1988-8401]{K.~Yorita}$^\textrm{\scriptsize 173}$,
\AtlasOrcid[0000-0001-8253-9517]{S.~Younas}$^\textrm{\scriptsize 28b}$,
\AtlasOrcid[0000-0001-5858-6639]{C.J.S.~Young}$^\textrm{\scriptsize 37}$,
\AtlasOrcid[0000-0003-3268-3486]{C.~Young}$^\textrm{\scriptsize 147}$,
\AtlasOrcid[0009-0006-8942-5911]{C.~Yu}$^\textrm{\scriptsize 14,115c}$,
\AtlasOrcid[0000-0003-4762-8201]{Y.~Yu}$^\textrm{\scriptsize 64a}$,
\AtlasOrcid[0000-0001-9834-7309]{J.~Yuan}$^\textrm{\scriptsize 14,115c}$,
\AtlasOrcid[0000-0002-0991-5026]{M.~Yuan}$^\textrm{\scriptsize 109}$,
\AtlasOrcid[0000-0002-8452-0315]{R.~Yuan}$^\textrm{\scriptsize 64d,64c}$,
\AtlasOrcid[0000-0001-6470-4662]{L.~Yue}$^\textrm{\scriptsize 99}$,
\AtlasOrcid[0000-0002-4105-2988]{M.~Zaazoua}$^\textrm{\scriptsize 64a}$,
\AtlasOrcid[0000-0001-5626-0993]{B.~Zabinski}$^\textrm{\scriptsize 89}$,
\AtlasOrcid{E.~Zaid}$^\textrm{\scriptsize 54}$,
\AtlasOrcid[0000-0002-9330-8842]{Z.K.~Zak}$^\textrm{\scriptsize 89}$,
\AtlasOrcid[0000-0001-7909-4772]{T.~Zakareishvili}$^\textrm{\scriptsize 168}$,
\AtlasOrcid[0000-0002-4499-2545]{S.~Zambito}$^\textrm{\scriptsize 58}$,
\AtlasOrcid[0000-0002-5030-7516]{J.A.~Zamora~Saa}$^\textrm{\scriptsize 141d,141b}$,
\AtlasOrcid[0000-0003-2770-1387]{J.~Zang}$^\textrm{\scriptsize 157}$,
\AtlasOrcid[0000-0002-1222-7937]{D.~Zanzi}$^\textrm{\scriptsize 56}$,
\AtlasOrcid[0000-0002-4687-3662]{O.~Zaplatilek}$^\textrm{\scriptsize 136}$,
\AtlasOrcid[0000-0003-2280-8636]{C.~Zeitnitz}$^\textrm{\scriptsize 176}$,
\AtlasOrcid[0000-0002-2032-442X]{H.~Zeng}$^\textrm{\scriptsize 14}$,
\AtlasOrcid[0000-0002-2029-2659]{J.C.~Zeng}$^\textrm{\scriptsize 167}$,
\AtlasOrcid[0000-0002-4867-3138]{D.T.~Zenger~Jr}$^\textrm{\scriptsize 27}$,
\AtlasOrcid[0000-0002-5447-1989]{O.~Zenin}$^\textrm{\scriptsize 38}$,
\AtlasOrcid[0000-0001-8265-6916]{T.~\v{Z}eni\v{s}}$^\textrm{\scriptsize 29a}$,
\AtlasOrcid[0000-0002-9720-1794]{S.~Zenz}$^\textrm{\scriptsize 97}$,
\AtlasOrcid[0000-0001-9101-3226]{S.~Zerradi}$^\textrm{\scriptsize 36a}$,
\AtlasOrcid[0000-0002-4198-3029]{D.~Zerwas}$^\textrm{\scriptsize 68}$,
\AtlasOrcid[0000-0003-0524-1914]{M.~Zhai}$^\textrm{\scriptsize 14,115c}$,
\AtlasOrcid[0000-0001-7335-4983]{D.F.~Zhang}$^\textrm{\scriptsize 143}$,
\AtlasOrcid[0000-0002-4380-1655]{J.~Zhang}$^\textrm{\scriptsize 64b}$,
\AtlasOrcid[0000-0002-9907-838X]{J.~Zhang}$^\textrm{\scriptsize 6}$,
\AtlasOrcid[0000-0002-9778-9209]{K.~Zhang}$^\textrm{\scriptsize 14,115c}$,
\AtlasOrcid[0009-0000-4105-4564]{L.~Zhang}$^\textrm{\scriptsize 64a}$,
\AtlasOrcid[0000-0002-9336-9338]{L.~Zhang}$^\textrm{\scriptsize 115a}$,
\AtlasOrcid[0000-0002-9177-6108]{P.~Zhang}$^\textrm{\scriptsize 14,115c}$,
\AtlasOrcid[0000-0002-8265-474X]{R.~Zhang}$^\textrm{\scriptsize 175}$,
\AtlasOrcid[0000-0001-9039-9809]{S.~Zhang}$^\textrm{\scriptsize 109}$,
\AtlasOrcid[0000-0002-8480-2662]{S.~Zhang}$^\textrm{\scriptsize 92}$,
\AtlasOrcid[0000-0001-7729-085X]{T.~Zhang}$^\textrm{\scriptsize 157}$,
\AtlasOrcid[0000-0003-4731-0754]{X.~Zhang}$^\textrm{\scriptsize 64c}$,
\AtlasOrcid[0000-0001-6274-7714]{Y.~Zhang}$^\textrm{\scriptsize 142}$,
\AtlasOrcid[0000-0001-7287-9091]{Y.~Zhang}$^\textrm{\scriptsize 99}$,
\AtlasOrcid[0000-0003-2029-0300]{Y.~Zhang}$^\textrm{\scriptsize 115a}$,
\AtlasOrcid[0000-0002-1630-0986]{Z.~Zhang}$^\textrm{\scriptsize 18a}$,
\AtlasOrcid[0000-0002-7936-8419]{Z.~Zhang}$^\textrm{\scriptsize 64b}$,
\AtlasOrcid[0000-0002-7853-9079]{Z.~Zhang}$^\textrm{\scriptsize 68}$,
\AtlasOrcid[0000-0002-6638-847X]{H.~Zhao}$^\textrm{\scriptsize 142}$,
\AtlasOrcid[0000-0002-6427-0806]{T.~Zhao}$^\textrm{\scriptsize 64b}$,
\AtlasOrcid[0000-0003-0494-6728]{Y.~Zhao}$^\textrm{\scriptsize 140}$,
\AtlasOrcid[0000-0001-6758-3974]{Z.~Zhao}$^\textrm{\scriptsize 64a}$,
\AtlasOrcid[0000-0001-8178-8861]{Z.~Zhao}$^\textrm{\scriptsize 64a}$,
\AtlasOrcid[0000-0002-3360-4965]{A.~Zhemchugov}$^\textrm{\scriptsize 40}$,
\AtlasOrcid[0000-0002-9748-3074]{J.~Zheng}$^\textrm{\scriptsize 115a}$,
\AtlasOrcid[0009-0006-9951-2090]{K.~Zheng}$^\textrm{\scriptsize 167}$,
\AtlasOrcid[0000-0002-2079-996X]{X.~Zheng}$^\textrm{\scriptsize 64a}$,
\AtlasOrcid[0000-0002-8323-7753]{Z.~Zheng}$^\textrm{\scriptsize 147}$,
\AtlasOrcid[0000-0001-9377-650X]{D.~Zhong}$^\textrm{\scriptsize 167}$,
\AtlasOrcid[0000-0002-0034-6576]{B.~Zhou}$^\textrm{\scriptsize 109}$,
\AtlasOrcid[0000-0002-7986-9045]{H.~Zhou}$^\textrm{\scriptsize 7}$,
\AtlasOrcid[0000-0002-1775-2511]{N.~Zhou}$^\textrm{\scriptsize 64c}$,
\AtlasOrcid{Y.~Zhou}$^\textrm{\scriptsize 15}$,
\AtlasOrcid[0009-0009-4876-1611]{Y.~Zhou}$^\textrm{\scriptsize 115a}$,
\AtlasOrcid{Y.~Zhou}$^\textrm{\scriptsize 7}$,
\AtlasOrcid[0000-0001-8015-3901]{C.G.~Zhu}$^\textrm{\scriptsize 64b}$,
\AtlasOrcid[0000-0002-5278-2855]{J.~Zhu}$^\textrm{\scriptsize 109}$,
\AtlasOrcid{X.~Zhu}$^\textrm{\scriptsize 64d}$,
\AtlasOrcid[0000-0001-7964-0091]{Y.~Zhu}$^\textrm{\scriptsize 64c}$,
\AtlasOrcid[0000-0002-7306-1053]{Y.~Zhu}$^\textrm{\scriptsize 64a}$,
\AtlasOrcid[0000-0003-0996-3279]{X.~Zhuang}$^\textrm{\scriptsize 14}$,
\AtlasOrcid[0000-0003-2468-9634]{K.~Zhukov}$^\textrm{\scriptsize 70}$,
\AtlasOrcid[0000-0003-0277-4870]{N.I.~Zimine}$^\textrm{\scriptsize 40}$,
\AtlasOrcid[0000-0002-5117-4671]{J.~Zinsser}$^\textrm{\scriptsize 65b}$,
\AtlasOrcid[0000-0002-2891-8812]{M.~Ziolkowski}$^\textrm{\scriptsize 145}$,
\AtlasOrcid[0000-0003-4236-8930]{L.~\v{Z}ivkovi\'{c}}$^\textrm{\scriptsize 16}$,
\AtlasOrcid[0000-0002-0993-6185]{A.~Zoccoli}$^\textrm{\scriptsize 24b,24a}$,
\AtlasOrcid[0000-0003-2138-6187]{K.~Zoch}$^\textrm{\scriptsize 63}$,
\AtlasOrcid[0000-0003-2073-4901]{T.G.~Zorbas}$^\textrm{\scriptsize 143}$,
\AtlasOrcid[0000-0003-3177-903X]{O.~Zormpa}$^\textrm{\scriptsize 48}$,
\AtlasOrcid[0000-0002-0779-8815]{W.~Zou}$^\textrm{\scriptsize 43}$,
\AtlasOrcid[0000-0002-9397-2313]{L.~Zwalinski}$^\textrm{\scriptsize 37}$.
\bigskip
\\

$^{1}$Department of Physics, University of Adelaide, Adelaide; Australia.\\
$^{2}$Department of Physics, University of Alberta, Edmonton AB; Canada.\\
$^{3}$$^{(a)}$Department of Physics, Ankara University, Ankara;$^{(b)}$Division of Physics, TOBB University of Economics and Technology, Ankara; T\"urkiye.\\
$^{4}$LAPP, Université Savoie Mont Blanc, CNRS/IN2P3, Annecy; France.\\
$^{5}$APC, Universit\'e Paris Cit\'e, CNRS/IN2P3, Paris; France.\\
$^{6}$High Energy Physics Division, Argonne National Laboratory, Argonne IL; United States of America.\\
$^{7}$Department of Physics, University of Arizona, Tucson AZ; United States of America.\\
$^{8}$Department of Physics, University of Texas at Arlington, Arlington TX; United States of America.\\
$^{9}$Physics Department, National and Kapodistrian University of Athens, Athens; Greece.\\
$^{10}$Physics Department, National Technical University of Athens, Zografou; Greece.\\
$^{11}$Department of Physics, University of Texas at Austin, Austin TX; United States of America.\\
$^{12}$Institute of Physics, Azerbaijan Academy of Sciences, Baku; Azerbaijan.\\
$^{13}$Institut de F\'isica d'Altes Energies (IFAE), Barcelona Institute of Science and Technology, Barcelona; Spain.\\
$^{14}$Institute of High Energy Physics, Chinese Academy of Sciences, Beijing; China.\\
$^{15}$Physics Department, Tsinghua University, Beijing; China.\\
$^{16}$Institute of Physics, University of Belgrade, Belgrade; Serbia.\\
$^{17}$Department for Physics and Technology, University of Bergen, Bergen; Norway.\\
$^{18}$$^{(a)}$Physics Division, Lawrence Berkeley National Laboratory, Berkeley CA;$^{(b)}$University of California, Berkeley CA; United States of America.\\
$^{19}$Institut f\"{u}r Physik, Humboldt Universit\"{a}t zu Berlin, Berlin; Germany.\\
$^{20}$Albert Einstein Center for Fundamental Physics and Laboratory for High Energy Physics, University of Bern, Bern; Switzerland.\\
$^{21}$School of Physics and Astronomy, University of Birmingham, Birmingham; United Kingdom.\\
$^{22}$$^{(a)}$Department of Physics, Bogazici University, Istanbul;$^{(b)}$Department of Physics Engineering, Gaziantep University, Gaziantep;$^{(c)}$Department of Physics, Istanbul University, Istanbul; T\"urkiye.\\
$^{23}$$^{(a)}$Facultad de Ciencias y Centro de Investigaci\'ones, Universidad Antonio Nari\~no, Bogot\'a;$^{(b)}$Departamento de F\'isica, Universidad Nacional de Colombia, Bogot\'a; Colombia.\\
$^{24}$$^{(a)}$Dipartimento di Fisica e Astronomia A. Righi, Università di Bologna, Bologna;$^{(b)}$INFN Sezione di Bologna; Italy.\\
$^{25}$Physikalisches Institut, Universit\"{a}t Bonn, Bonn; Germany.\\
$^{26}$Department of Physics, Boston University, Boston MA; United States of America.\\
$^{27}$Department of Physics, Brandeis University, Waltham MA; United States of America.\\
$^{28}$$^{(a)}$Transilvania University of Brasov, Brasov;$^{(b)}$Horia Hulubei National Institute of Physics and Nuclear Engineering, Bucharest;$^{(c)}$Department of Physics, Alexandru Ioan Cuza University of Iasi, Iasi;$^{(d)}$National Institute for Research and Development of Isotopic and Molecular Technologies, Physics Department, Cluj-Napoca;$^{(e)}$National University of Science and Technology Politechnica, Bucharest;$^{(f)}$West University in Timisoara, Timisoara;$^{(g)}$Faculty of Physics, University of Bucharest, Bucharest; Romania.\\
$^{29}$$^{(a)}$Faculty of Mathematics, Physics and Informatics, Comenius University, Bratislava;$^{(b)}$Department of Subnuclear Physics, Institute of Experimental Physics of the Slovak Academy of Sciences, Kosice; Slovak Republic.\\
$^{30}$Physics Department, Brookhaven National Laboratory, Upton NY; United States of America.\\
$^{31}$Universidad de Buenos Aires, Facultad de Ciencias Exactas y Naturales, Departamento de F\'isica, y CONICET, Instituto de Física de Buenos Aires (IFIBA), Buenos Aires; Argentina.\\
$^{32}$California State University, CA; United States of America.\\
$^{33}$Cavendish Laboratory, University of Cambridge, Cambridge; United Kingdom.\\
$^{34}$$^{(a)}$Department of Physics, University of Cape Town, Cape Town;$^{(b)}$iThemba Labs, Western Cape;$^{(c)}$Department of Mechanical Engineering Science, University of Johannesburg, Johannesburg;$^{(d)}$National Institute of Physics, University of the Philippines Diliman (Philippines);$^{(e)}$University of South Africa, Department of Physics, Pretoria;$^{(f)}$University of Zululand, KwaDlangezwa;$^{(g)}$School of Physics, University of the Witwatersrand, Johannesburg; South Africa.\\
$^{35}$Department of Physics, Carleton University, Ottawa ON; Canada.\\
$^{36}$$^{(a)}$Facult\'e des Sciences Ain Chock, R\'eseau Universitaire de Physique des Hautes Energies - Universit\'e Hassan II, Casablanca;$^{(b)}$Facult\'{e} des Sciences, Universit\'{e} Ibn-Tofail, K\'{e}nitra;$^{(c)}$Facult\'e des Sciences Semlalia, Universit\'e Cadi Ayyad, LPHEA-Marrakech;$^{(d)}$LPMR, Facult\'e des Sciences, Universit\'e Mohamed Premier, Oujda;$^{(e)}$Facult\'e des sciences, Universit\'e Mohammed V, Rabat;$^{(f)}$Institute of Applied Physics, Mohammed VI Polytechnic University, Ben Guerir; Morocco.\\
$^{37}$CERN, Geneva; Switzerland.\\
$^{38}$Affiliated with an institute covered by a cooperation agreement with CERN.\\
$^{39}$Affiliated with an institute formerly covered by a cooperation agreement with CERN.\\
$^{40}$Affiliated with an international laboratory covered by a cooperation agreement with CERN.\\
$^{41}$Enrico Fermi Institute, University of Chicago, Chicago IL; United States of America.\\
$^{42}$LPC, Universit\'e Clermont Auvergne, CNRS/IN2P3, Clermont-Ferrand; France.\\
$^{43}$Nevis Laboratory, Columbia University, Irvington NY; United States of America.\\
$^{44}$Niels Bohr Institute, University of Copenhagen, Copenhagen; Denmark.\\
$^{45}$$^{(a)}$Dipartimento di Fisica, Universit\`a della Calabria, Rende;$^{(b)}$INFN Gruppo Collegato di Cosenza, Laboratori Nazionali di Frascati; Italy.\\
$^{46}$Physics Department, Southern Methodist University, Dallas TX; United States of America.\\
$^{47}$Physics Department, University of Texas at Dallas, Richardson TX; United States of America.\\
$^{48}$National Centre for Scientific Research "Demokritos", Agia Paraskevi; Greece.\\
$^{49}$$^{(a)}$Department of Physics, Stockholm University;$^{(b)}$Oskar Klein Centre, Stockholm; Sweden.\\
$^{50}$Deutsches Elektronen-Synchrotron DESY, Hamburg and Zeuthen; Germany.\\
$^{51}$Fakult\"{a}t Physik , Technische Universit{\"a}t Dortmund, Dortmund; Germany.\\
$^{52}$Institut f\"{u}r Kern-~und Teilchenphysik, Technische Universit\"{a}t Dresden, Dresden; Germany.\\
$^{53}$Department of Physics, Duke University, Durham NC; United States of America.\\
$^{54}$SUPA - School of Physics and Astronomy, University of Edinburgh, Edinburgh; United Kingdom.\\
$^{55}$INFN e Laboratori Nazionali di Frascati, Frascati; Italy.\\
$^{56}$Physikalisches Institut, Albert-Ludwigs-Universit\"{a}t Freiburg, Freiburg; Germany.\\
$^{57}$II. Physikalisches Institut, Georg-August-Universit\"{a}t G\"ottingen, G\"ottingen; Germany.\\
$^{58}$D\'epartement de Physique Nucl\'eaire et Corpusculaire, Universit\'e de Gen\`eve, Gen\`eve; Switzerland.\\
$^{59}$$^{(a)}$Dipartimento di Fisica, Universit\`a di Genova, Genova;$^{(b)}$INFN Sezione di Genova; Italy.\\
$^{60}$II. Physikalisches Institut, Justus-Liebig-Universit{\"a}t Giessen, Giessen; Germany.\\
$^{61}$SUPA - School of Physics and Astronomy, University of Glasgow, Glasgow; United Kingdom.\\
$^{62}$LPSC, Universit\'e Grenoble Alpes, CNRS/IN2P3, Grenoble INP, Grenoble; France.\\
$^{63}$Laboratory for Particle Physics and Cosmology, Harvard University, Cambridge MA; United States of America.\\
$^{64}$$^{(a)}$Department of Modern Physics and State Key Laboratory of Particle Detection and Electronics, University of Science and Technology of China, Hefei;$^{(b)}$Institute of Frontier and Interdisciplinary Science and Key Laboratory of Particle Physics and Particle Irradiation (MOE), Shandong University, Qingdao;$^{(c)}$School of Physics and Astronomy, Shanghai Jiao Tong University, Key Laboratory for Particle Astrophysics and Cosmology (MOE), SKLPPC, Shanghai;$^{(d)}$Tsung-Dao Lee Institute, Shanghai;$^{(e)}$School of Physics and Microelectronics, Zhengzhou University; China.\\
$^{65}$$^{(a)}$Kirchhoff-Institut f\"{u}r Physik, Ruprecht-Karls-Universit\"{a}t Heidelberg, Heidelberg;$^{(b)}$Physikalisches Institut, Ruprecht-Karls-Universit\"{a}t Heidelberg, Heidelberg; Germany.\\
$^{66}$$^{(a)}$Department of Physics, Chinese University of Hong Kong, Shatin, N.T., Hong Kong;$^{(b)}$Department of Physics, University of Hong Kong, Hong Kong;$^{(c)}$Department of Physics and Institute for Advanced Study, Hong Kong University of Science and Technology, Clear Water Bay, Kowloon, Hong Kong; China.\\
$^{67}$Department of Physics, National Tsing Hua University, Hsinchu; Taiwan.\\
$^{68}$IJCLab, Universit\'e Paris-Saclay, CNRS/IN2P3, 91405, Orsay; France.\\
$^{69}$Centro Nacional de Microelectrónica (IMB-CNM-CSIC), Barcelona; Spain.\\
$^{70}$Department of Physics, Indiana University, Bloomington IN; United States of America.\\
$^{71}$$^{(a)}$INFN Gruppo Collegato di Udine, Sezione di Trieste, Udine;$^{(b)}$ICTP, Trieste;$^{(c)}$Dipartimento Politecnico di Ingegneria e Architettura, Universit\`a di Udine, Udine; Italy.\\
$^{72}$$^{(a)}$INFN Sezione di Lecce;$^{(b)}$Dipartimento di Matematica e Fisica, Universit\`a del Salento, Lecce; Italy.\\
$^{73}$$^{(a)}$INFN Sezione di Milano;$^{(b)}$Dipartimento di Fisica, Universit\`a di Milano, Milano; Italy.\\
$^{74}$$^{(a)}$INFN Sezione di Napoli;$^{(b)}$Dipartimento di Fisica, Universit\`a di Napoli, Napoli; Italy.\\
$^{75}$$^{(a)}$INFN Sezione di Pavia;$^{(b)}$Dipartimento di Fisica, Universit\`a di Pavia, Pavia; Italy.\\
$^{76}$$^{(a)}$INFN Sezione di Pisa;$^{(b)}$Dipartimento di Fisica E. Fermi, Universit\`a di Pisa, Pisa; Italy.\\
$^{77}$$^{(a)}$INFN Sezione di Roma;$^{(b)}$Dipartimento di Fisica, Sapienza Universit\`a di Roma, Roma; Italy.\\
$^{78}$$^{(a)}$INFN Sezione di Roma Tor Vergata;$^{(b)}$Dipartimento di Fisica, Universit\`a di Roma Tor Vergata, Roma; Italy.\\
$^{79}$$^{(a)}$INFN Sezione di Roma Tre;$^{(b)}$Dipartimento di Matematica e Fisica, Universit\`a Roma Tre, Roma; Italy.\\
$^{80}$$^{(a)}$INFN-TIFPA;$^{(b)}$Universit\`a degli Studi di Trento, Trento; Italy.\\
$^{81}$Universit\"{a}t Innsbruck, Department of Astro and Particle Physics, Innsbruck; Austria.\\
$^{82}$University of Iowa, Iowa City IA; United States of America.\\
$^{83}$Department of Physics and Astronomy, Iowa State University, Ames IA; United States of America.\\
$^{84}$Istinye University, Sariyer, Istanbul; T\"urkiye.\\
$^{85}$$^{(a)}$Departamento de Engenharia El\'etrica, Universidade Federal de Juiz de Fora (UFJF), Juiz de Fora;$^{(b)}$Universidade Federal do Rio De Janeiro COPPE/EE/IF, Rio de Janeiro;$^{(c)}$Instituto de F\'isica, Universidade de S\~ao Paulo, S\~ao Paulo;$^{(d)}$Rio de Janeiro State University, Rio de Janeiro;$^{(e)}$Federal University of Bahia, Bahia; Brazil.\\
$^{86}$KEK, High Energy Accelerator Research Organization, Tsukuba; Japan.\\
$^{87}$Graduate School of Science, Kobe University, Kobe; Japan.\\
$^{88}$$^{(a)}$AGH University of Krakow, Faculty of Physics and Applied Computer Science, Krakow;$^{(b)}$Marian Smoluchowski Institute of Physics, Jagiellonian University, Krakow; Poland.\\
$^{89}$Institute of Nuclear Physics Polish Academy of Sciences, Krakow; Poland.\\
$^{90}$Faculty of Science, Kyoto University, Kyoto; Japan.\\
$^{91}$Research Center for Advanced Particle Physics and Department of Physics, Kyushu University, Fukuoka ; Japan.\\
$^{92}$L2IT, Universit\'e de Toulouse, CNRS/IN2P3, UPS, Toulouse; France.\\
$^{93}$Instituto de F\'{i}sica La Plata, Universidad Nacional de La Plata and CONICET, La Plata; Argentina.\\
$^{94}$Physics Department, Lancaster University, Lancaster; United Kingdom.\\
$^{95}$Oliver Lodge Laboratory, University of Liverpool, Liverpool; United Kingdom.\\
$^{96}$Department of Experimental Particle Physics, Jo\v{z}ef Stefan Institute and Department of Physics, University of Ljubljana, Ljubljana; Slovenia.\\
$^{97}$School of Physics and Astronomy, Queen Mary University of London, London; United Kingdom.\\
$^{98}$Department of Physics, Royal Holloway University of London, Egham; United Kingdom.\\
$^{99}$Department of Physics and Astronomy, University College London, London; United Kingdom.\\
$^{100}$Louisiana Tech University, Ruston LA; United States of America.\\
$^{101}$Fysiska institutionen, Lunds universitet, Lund; Sweden.\\
$^{102}$Departamento de F\'isica Teorica C-15 and CIAFF, Universidad Aut\'onoma de Madrid, Madrid; Spain.\\
$^{103}$Institut f\"{u}r Physik, Universit\"{a}t Mainz, Mainz; Germany.\\
$^{104}$School of Physics and Astronomy, University of Manchester, Manchester; United Kingdom.\\
$^{105}$CPPM, Aix-Marseille Universit\'e, CNRS/IN2P3, Marseille; France.\\
$^{106}$Department of Physics, University of Massachusetts, Amherst MA; United States of America.\\
$^{107}$Department of Physics, McGill University, Montreal QC; Canada.\\
$^{108}$School of Physics, University of Melbourne, Victoria; Australia.\\
$^{109}$Department of Physics, University of Michigan, Ann Arbor MI; United States of America.\\
$^{110}$Department of Physics and Astronomy, Michigan State University, East Lansing MI; United States of America.\\
$^{111}$Group of Particle Physics, University of Montreal, Montreal QC; Canada.\\
$^{112}$Fakult\"at f\"ur Physik, Ludwig-Maximilians-Universit\"at M\"unchen, M\"unchen; Germany.\\
$^{113}$Max-Planck-Institut f\"ur Physik (Werner-Heisenberg-Institut), M\"unchen; Germany.\\
$^{114}$Graduate School of Science and Kobayashi-Maskawa Institute, Nagoya University, Nagoya; Japan.\\
$^{115}$$^{(a)}$Department of Physics, Nanjing University, Nanjing;$^{(b)}$School of Science, Shenzhen Campus of Sun Yat-sen University;$^{(c)}$University of Chinese Academy of Science (UCAS), Beijing; China.\\
$^{116}$Department of Physics and Astronomy, University of New Mexico, Albuquerque NM; United States of America.\\
$^{117}$Institute for Mathematics, Astrophysics and Particle Physics, Radboud University/Nikhef, Nijmegen; Netherlands.\\
$^{118}$Nikhef National Institute for Subatomic Physics and University of Amsterdam, Amsterdam; Netherlands.\\
$^{119}$Department of Physics, Northern Illinois University, DeKalb IL; United States of America.\\
$^{120}$$^{(a)}$New York University Abu Dhabi, Abu Dhabi;$^{(b)}$United Arab Emirates University, Al Ain; United Arab Emirates.\\
$^{121}$Department of Physics, New York University, New York NY; United States of America.\\
$^{122}$Ochanomizu University, Otsuka, Bunkyo-ku, Tokyo; Japan.\\
$^{123}$Ohio State University, Columbus OH; United States of America.\\
$^{124}$Homer L. Dodge Department of Physics and Astronomy, University of Oklahoma, Norman OK; United States of America.\\
$^{125}$Department of Physics, Oklahoma State University, Stillwater OK; United States of America.\\
$^{126}$Palack\'y University, Joint Laboratory of Optics, Olomouc; Czech Republic.\\
$^{127}$Institute for Fundamental Science, University of Oregon, Eugene, OR; United States of America.\\
$^{128}$Graduate School of Science, Osaka University, Osaka; Japan.\\
$^{129}$Department of Physics, University of Oslo, Oslo; Norway.\\
$^{130}$Department of Physics, Oxford University, Oxford; United Kingdom.\\
$^{131}$LPNHE, Sorbonne Universit\'e, Universit\'e Paris Cit\'e, CNRS/IN2P3, Paris; France.\\
$^{132}$Department of Physics, University of Pennsylvania, Philadelphia PA; United States of America.\\
$^{133}$Department of Physics and Astronomy, University of Pittsburgh, Pittsburgh PA; United States of America.\\
$^{134}$$^{(a)}$Laborat\'orio de Instrumenta\c{c}\~ao e F\'isica Experimental de Part\'iculas - LIP, Lisboa;$^{(b)}$Departamento de F\'isica, Faculdade de Ci\^{e}ncias, Universidade de Lisboa, Lisboa;$^{(c)}$Departamento de F\'isica, Universidade de Coimbra, Coimbra;$^{(d)}$Centro de F\'isica Nuclear da Universidade de Lisboa, Lisboa;$^{(e)}$Departamento de F\'isica, Universidade do Minho, Braga;$^{(f)}$Departamento de F\'isica Te\'orica y del Cosmos, Universidad de Granada, Granada (Spain);$^{(g)}$Departamento de F\'{\i}sica, Instituto Superior T\'ecnico, Universidade de Lisboa, Lisboa; Portugal.\\
$^{135}$Institute of Physics of the Czech Academy of Sciences, Prague; Czech Republic.\\
$^{136}$Czech Technical University in Prague, Prague; Czech Republic.\\
$^{137}$Charles University, Faculty of Mathematics and Physics, Prague; Czech Republic.\\
$^{138}$Particle Physics Department, Rutherford Appleton Laboratory, Didcot; United Kingdom.\\
$^{139}$IRFU, CEA, Universit\'e Paris-Saclay, Gif-sur-Yvette; France.\\
$^{140}$Santa Cruz Institute for Particle Physics, University of California Santa Cruz, Santa Cruz CA; United States of America.\\
$^{141}$$^{(a)}$Departamento de F\'isica, Pontificia Universidad Cat\'olica de Chile, Santiago;$^{(b)}$Millennium Institute for Subatomic physics at high energy frontier (SAPHIR), Santiago;$^{(c)}$Instituto de Investigaci\'on Multidisciplinario en Ciencia y Tecnolog\'ia, y Departamento de F\'isica, Universidad de La Serena;$^{(d)}$Universidad Andres Bello, Department of Physics, Santiago;$^{(e)}$Instituto de Alta Investigaci\'on, Universidad de Tarapac\'a, Arica;$^{(f)}$Departamento de F\'isica, Universidad T\'ecnica Federico Santa Mar\'ia, Valpara\'iso; Chile.\\
$^{142}$Department of Physics, University of Washington, Seattle WA; United States of America.\\
$^{143}$Department of Physics and Astronomy, University of Sheffield, Sheffield; United Kingdom.\\
$^{144}$Department of Physics, Shinshu University, Nagano; Japan.\\
$^{145}$Department Physik, Universit\"{a}t Siegen, Siegen; Germany.\\
$^{146}$Department of Physics, Simon Fraser University, Burnaby BC; Canada.\\
$^{147}$SLAC National Accelerator Laboratory, Stanford CA; United States of America.\\
$^{148}$Department of Physics, Royal Institute of Technology, Stockholm; Sweden.\\
$^{149}$Departments of Physics and Astronomy, Stony Brook University, Stony Brook NY; United States of America.\\
$^{150}$Department of Physics and Astronomy, University of Sussex, Brighton; United Kingdom.\\
$^{151}$School of Physics, University of Sydney, Sydney; Australia.\\
$^{152}$Institute of Physics, Academia Sinica, Taipei; Taiwan.\\
$^{153}$$^{(a)}$E. Andronikashvili Institute of Physics, Iv. Javakhishvili Tbilisi State University, Tbilisi;$^{(b)}$High Energy Physics Institute, Tbilisi State University, Tbilisi;$^{(c)}$University of Georgia, Tbilisi; Georgia.\\
$^{154}$Department of Physics, Technion, Israel Institute of Technology, Haifa; Israel.\\
$^{155}$Raymond and Beverly Sackler School of Physics and Astronomy, Tel Aviv University, Tel Aviv; Israel.\\
$^{156}$Department of Physics, Aristotle University of Thessaloniki, Thessaloniki; Greece.\\
$^{157}$International Center for Elementary Particle Physics and Department of Physics, University of Tokyo, Tokyo; Japan.\\
$^{158}$Department of Physics, Tokyo Institute of Technology, Tokyo; Japan.\\
$^{159}$Department of Physics, University of Toronto, Toronto ON; Canada.\\
$^{160}$$^{(a)}$TRIUMF, Vancouver BC;$^{(b)}$Department of Physics and Astronomy, York University, Toronto ON; Canada.\\
$^{161}$Division of Physics and Tomonaga Center for the History of the Universe, Faculty of Pure and Applied Sciences, University of Tsukuba, Tsukuba; Japan.\\
$^{162}$Department of Physics and Astronomy, Tufts University, Medford MA; United States of America.\\
$^{163}$Department of Physics and Astronomy, University of California Irvine, Irvine CA; United States of America.\\
$^{164}$University of West Attica, Athens; Greece.\\
$^{165}$University of Sharjah, Sharjah; United Arab Emirates.\\
$^{166}$Department of Physics and Astronomy, University of Uppsala, Uppsala; Sweden.\\
$^{167}$Department of Physics, University of Illinois, Urbana IL; United States of America.\\
$^{168}$Instituto de F\'isica Corpuscular (IFIC), Centro Mixto Universidad de Valencia - CSIC, Valencia; Spain.\\
$^{169}$Department of Physics, University of British Columbia, Vancouver BC; Canada.\\
$^{170}$Department of Physics and Astronomy, University of Victoria, Victoria BC; Canada.\\
$^{171}$Fakult\"at f\"ur Physik und Astronomie, Julius-Maximilians-Universit\"at W\"urzburg, W\"urzburg; Germany.\\
$^{172}$Department of Physics, University of Warwick, Coventry; United Kingdom.\\
$^{173}$Waseda University, Tokyo; Japan.\\
$^{174}$Department of Particle Physics and Astrophysics, Weizmann Institute of Science, Rehovot; Israel.\\
$^{175}$Department of Physics, University of Wisconsin, Madison WI; United States of America.\\
$^{176}$Fakult{\"a}t f{\"u}r Mathematik und Naturwissenschaften, Fachgruppe Physik, Bergische Universit\"{a}t Wuppertal, Wuppertal; Germany.\\
$^{177}$Department of Physics, Yale University, New Haven CT; United States of America.\\
$^{178}$Yerevan Physics Institute, Yerevan; Armenia.\\

$^{a}$ Also Affiliated with an institute covered by a cooperation agreement with CERN.\\
$^{b}$ Also at An-Najah National University, Nablus; Palestine.\\
$^{c}$ Also at Borough of Manhattan Community College, City University of New York, New York NY; United States of America.\\
$^{d}$ Also at Center for Interdisciplinary Research and Innovation (CIRI-AUTH), Thessaloniki; Greece.\\
$^{e}$ Also at CERN, Geneva; Switzerland.\\
$^{f}$ Also at CMD-AC UNEC Research Center, Azerbaijan State University of Economics (UNEC); Azerbaijan.\\
$^{g}$ Also at D\'epartement de Physique Nucl\'eaire et Corpusculaire, Universit\'e de Gen\`eve, Gen\`eve; Switzerland.\\
$^{h}$ Also at Departament de Fisica de la Universitat Autonoma de Barcelona, Barcelona; Spain.\\
$^{i}$ Also at Department of Financial and Management Engineering, University of the Aegean, Chios; Greece.\\
$^{j}$ Also at Department of Physics, California State University, Sacramento; United States of America.\\
$^{k}$ Also at Department of Physics, King's College London, London; United Kingdom.\\
$^{l}$ Also at Department of Physics, Stanford University, Stanford CA; United States of America.\\
$^{m}$ Also at Department of Physics, Stellenbosch University; South Africa.\\
$^{n}$ Also at Department of Physics, University of Fribourg, Fribourg; Switzerland.\\
$^{o}$ Also at Department of Physics, University of Thessaly; Greece.\\
$^{p}$ Also at Department of Physics, Westmont College, Santa Barbara; United States of America.\\
$^{q}$ Also at Hellenic Open University, Patras; Greece.\\
$^{r}$ Also at Imam Mohammad Ibn Saud Islamic University; Saudi Arabia.\\
$^{s}$ Also at Institucio Catalana de Recerca i Estudis Avancats, ICREA, Barcelona; Spain.\\
$^{t}$ Also at Institut f\"{u}r Experimentalphysik, Universit\"{a}t Hamburg, Hamburg; Germany.\\
$^{u}$ Also at Institute for Nuclear Research and Nuclear Energy (INRNE) of the Bulgarian Academy of Sciences, Sofia; Bulgaria.\\
$^{v}$ Also at Institute of Applied Physics, Mohammed VI Polytechnic University, Ben Guerir; Morocco.\\
$^{w}$ Also at Institute of Particle Physics (IPP); Canada.\\
$^{x}$ Also at Institute of Physics, Azerbaijan Academy of Sciences, Baku; Azerbaijan.\\
$^{y}$ Also at National Institute of Physics, University of the Philippines Diliman (Philippines); Philippines.\\
$^{z}$ Also at Technical University of Munich, Munich; Germany.\\
$^{aa}$ Also at The Collaborative Innovation Center of Quantum Matter (CICQM), Beijing; China.\\
$^{ab}$ Also at TRIUMF, Vancouver BC; Canada.\\
$^{ac}$ Also at Universit\`a  di Napoli Parthenope, Napoli; Italy.\\
$^{ad}$ Also at University of Colorado Boulder, Department of Physics, Colorado; United States of America.\\
$^{ae}$ Also at Washington College, Chestertown, MD; United States of America.\\
$^{af}$ Also at Yeditepe University, Physics Department, Istanbul; Türkiye.\\
$^{*}$ Deceased

\end{flushleft}

% Created with Glance <Atlas.Glance@cern.ch>

%-------------------------------------------------------------------------------

%-------------------------------------------------------------------------------
% Auxiliary material - comment out the following line if you do not have any.
\include{atlas-document-auxmat}
%-------------------------------------------------------------------------------

%-------------------------------------------------------------------------------
% Extra tables etc. for HepData - comment in the following line if you have any.
% \include{atlas-document-hepdata}
%-------------------------------------------------------------------------------

\end{document}
