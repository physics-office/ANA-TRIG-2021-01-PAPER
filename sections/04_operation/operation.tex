\section{Online Deployment}%
\label{sec:operation}



This section reports the \rnn{} operation in both 2017
(Section~\ref{ssec:2017_ringer_operation}) and 2018
(Section~\ref{ssec:2018_ringer_operation}) data acquisition periods. The operation efficiencies
are computed with respect to offline electrons, as indicated in
Section~\ref{ssec:dataset}\footnote{For efficiency measurements in this section,
	fake electrons are selected always employing \veto\vloose{} offline likelihood
	working point.}. Firstly, the detail of the algorithm cycle used during Run~2
(Section~\ref{ssec:run2_rnn_cycle}) is presented.

\subsection{Run~2 \rnn{} Cycle}\label{ssec:run2_rnn_cycle}

During the Run 2, the actual \rnn{} implementation
%The \rnn cycle 
can be partitioned into four chronological stages:

\begin{enumerate}[i]
  \item A development stage up to early 2017, where the \rnn{}
      potential was estimated from trigger emulation and data reprocessing;
  \item The commissioning stage (\SI{5.4}{\per\femto\barn}) occurred in
      early 2017 runs, where all primary
      triggers were duplicated with either the \rnn{} or cut-based algorithms
      operating in the \fastcalo{} stage of the HLT;
  \item The operation of the method as the baseline trigger, which occurred after 2017 Technical Stop 1 (TS1). For
    monitoring purposes, and to allow precise statistical evaluation of eventual
    disagreements between the \fastcalo{} methods from an offline
    perspective (Section~\ref{sec:off_ana}), a duplicated trigger pair, i.e.
    with and without \rnn{}, was kept operating unprescaled during this period
    (\SI{39.0}{\per\femto\barn}). Here, the efficiencies of the
    duplicated triggers in 2017 (Section~\ref{ssec:2017_ringer_operation}) are compared;
  \item Finally, the duplicated trigger was removed for 2018 operation.
    Therefore, the evaluation of 2018 \rnn{} operation relying on a comparison with
    2017 efficiency is presented in Section~\ref{ssec:2018_ringer_operation}.
\end{enumerate}

\subsection{2017 Operation}\label{ssec:2017_ringer_operation}

% FIXME Don't forget that there are some of these plots that were already
% approved.

%A backup trigger ($\text{e28\_lhtight\_nod0\_ivarloose}$) was employed for
A backup $E_T > 28$ GeV isolated trigger with Tight selection and without the \rnn{} was employed for
monitoring purposes after the first technical stop (TS1). During the monitoring process, a similar
operation of both triggers was observed in terms of signal efficiency using the
integrated luminosity along the period, as shown in
Figure~\ref{fig:e28_triggers}. The trigger
turn-on curves exhibit similar profile. In $\eta$, the \rnn{} shows a reasonably symmetric
profile with respect to positive and negative $\eta$. In addition, a difference of about half to one percentage point may be observed in $\eta$ because of the transition region between the endcap and barrel. Besides
aforementioned points, overall efficiency fluctuations are smaller than a few
per-mille. Although a slightly more prominent efficiency loss with respect to
\avgmu{} is observed in Figure~\ref{fig:e28_comp_mu} for the \rnn{} trigger, the
electron efficiency was kept nearly the same. Such loss occurs after the
linear threshold correction limit of $\avgmu=40$ that was employed during 2017.

Other important triggers were assessed by comparing the trigger efficiencies on
2017 data collected before and after switching to the \rnn{} algorithms.  As it can be observed in Figure~\ref{fig:2017_ts1}, the electron efficiency was also kept nearly unchanged for relevant single electron triggers. Note that results in
this plot are computed with two different data taking periods: with or
without the \rnn{} algorithm. %The loosest likelihood trigger with 17 GeV energy threshold shows the efficiency in the lowest-energy-threshold 




\begin{figure}[h!tb]
  \begin{center}
  \begin{subfigure}[c]{.58\textwidth}
  \centering
  \includegraphics[width=\textwidth]{sections/04_operation/figures/efficiencies/eff_EGAM1_e28_ringer_and_noringer_2017_after_ts1_HLT_et.pdf}
  \caption{}%

  \end{subfigure} \\
  %\hfill
  %\hspace{0.01\textwidth}
  \begin{subfigure}[c]{.58\textwidth}
  \centering
  \includegraphics[width=\textwidth]{sections/04_operation/figures/efficiencies/eff_EGAM1_e28_ringer_and_noringer_2017_after_ts1_HLT_eta.pdf}
  \caption{}%
  %\label{fig:e28_comp_eta}
  \end{subfigure} \\
  \begin{subfigure}[c]{.58\textwidth}
  \centering
  \includegraphics[width=\textwidth]{sections/04_operation/figures/efficiencies/eff_EGAM1_e28_ringer_and_noringer_2017_after_ts1_HLT_mu.pdf}
  \caption{}%
  \label{fig:e28_comp_mu}
  \end{subfigure}
  \caption{\label{fig:e28_triggers}HLT electron efficiency as a function of \et{}
    (a), \eta{} (b) and \avgmu{} (c) for the single electron isolated trigger
    requiring $\et{} > \SI{28}{\GeV}$ and \tight{} selection with and without the
    \rnn{} algorithm. Efficiencies are measured employing 2017 data collected.}
  \end{center}
\end{figure}
  
\begin{figure}[h!tb]
  \begin{center}
  \begin{subfigure}[c]{.58\textwidth}
  \centering
  \includegraphics[width=\textwidth]{sections/04_operation/figures/efficiencies/eff_EGAM1_e17_e26_e60_2017_before_and_after_ts1_et.pdf}
  \caption{}%
  \end{subfigure}\\
  %\hfill
  %\hspace{0.01\textwidth}
  \begin{subfigure}[c]{.58\textwidth}
  \centering
  \includegraphics[width=\textwidth]{sections/04_operation/figures/efficiencies/eff_EGAM1_e17_e26_e60_2017_before_and_after_ts1_eta.pdf}
  \caption{}%
  \end{subfigure} \\
  \begin{subfigure}[c]{.58\textwidth}
  \centering
  \includegraphics[width=\textwidth]{sections/04_operation/figures/efficiencies/eff_EGAM1_e17_e26_e60_2017_before_and_after_ts1_mu.pdf}
  \caption{}%
  \end{subfigure}
  %\hfill
  \caption{Efficiency of three single electron triggers as a function of \et (a), \eta (b) and \avgmu (c). Open (closed) markers contain the efficiency measurements on runs before (after) the deployment of the \rnn{}, thus referring to triggers being executed without (with) the \rnn{} algorithm. For 2017 collision data, the higher \avgmu{} values were reached only after the deployment.
  }%
  \label{fig:2017_ts1}
  \end{center}
\end{figure}


By contrasting the behavior of the duplicated trigger using fake electron data,
it becomes clear the power of the \rnn{} algorithm. An overall reduction factor of
the fake rate by a factor of 13.75 is achieved. It can be seen in
Figure~\ref{fig:e28_triggers_fake} that the improvement is similar for all
regions in the evaluated variables, particularly interesting when
considering the low \et{} and the end-cap regions.




\begin{figure}[h!tb]
  \begin{center}
  \begin{subfigure}[c]{.58\textwidth}
  \centering
  \includegraphics[width=\textwidth]{sections/04_operation/figures/efficiencies/eff_EGAM7_e28_ringer_and_noringer_2017_after_ts1_L2Calo_et.pdf}
  %\label{fig:e28_comp_et_fake}
  \caption{}
  \end{subfigure}\\
  %\hfill
  %\hspace{0.01\textwidth}
  \begin{subfigure}[c]{.58\textwidth}
  \centering
  \includegraphics[width=\textwidth]{sections/04_operation/figures/efficiencies/eff_EGAM7_e28_ringer_and_noringer_2017_after_ts1_L2Calo_eta.pdf}
  %\label{fig:e28_comp_eta_fake}
  \caption{}
  \end{subfigure}\\
  \begin{subfigure}[c]{.58\textwidth}
  \centering
  \includegraphics[width=\textwidth]{sections/04_operation/figures/efficiencies/eff_EGAM7_e28_ringer_and_noringer_2017_after_ts1_L2Calo_mu.pdf}
  %\label{fig:e28_comp_mu_fake}
  \caption{}
  \end{subfigure}
  %\hfill
  \caption{\label{fig:e28_triggers_fake} \fastcalo %\footnotemark{} 
  fake electron efficiency as a function of \et (a), \eta (b) and \avgmu (c) for the single electron isolated trigger requiring $\et > \SI{28}{\GeV}$ and \tight selection with and without the \rnn{} algorithm. Efficiencies are measured employing 2017 data collected after the \rnn algorithm deployment.}%
  
  \end{center}
\end{figure}



Besides the capability of improving early fake rejection, the usage of the
\rnn{} also contributed to reduce the final fake rate
(Figure~\ref{fig:e28_triggers_fake_hlt}) by a factor of 2, mostly coming from
the transition regions in the calorimeter. The discussion about CPU and final rate reduction will be covered in the next section.

 %This fake rate reduction when measured with
%respect to the offline electron selection does not seem to have impacted in the
%output rate (see Appendix~\ref{ssec:primary_rate_wrt_luminosity})\footnote{This
%  can provide some indication that measuring background efficiencies with
%respect to the offline may not be the best approach.}, but may have
%contributed to reduce signal contamination in physics analyses.

\begin{figure}[h!tb]
\begin{center}
\begin{subfigure}[c]{.58\textwidth}
\centering
\includegraphics[width=\textwidth]{sections/04_operation/figures/efficiencies/eff_EGAM7_e28_ringer_and_noringer_2017_after_ts1_et.pdf}
\caption{}
\end{subfigure}\\
%\hfill
%\hspace{0.01\textwidth}
\begin{subfigure}[c]{.58\textwidth}
\centering
\includegraphics[width=\textwidth]{sections/04_operation/figures/efficiencies/eff_EGAM7_e28_ringer_and_noringer_2017_after_ts1_eta.pdf}
\caption{}
\end{subfigure} \\
\begin{subfigure}[c]{.58\textwidth}
\centering
\includegraphics[width=\textwidth]{sections/04_operation/figures/efficiencies/eff_EGAM7_e28_ringer_and_noringer_2017_after_ts1_mu.pdf}
\caption{}
\end{subfigure}
\caption{HLT fake electron efficiency as a function of \et (a), \eta (b) and
\avgmu (c) for the single electron isolated trigger requiring $\et >
\SI{28}{\GeV}$ and \tight selection with and without the \rnn{} algorithm.
Efficiencies are measured employing 2017 data collected after the \rnn algorithm deployment.}%
\label{fig:e28_triggers_fake_hlt}
\end{center}
\end{figure}


\FloatBarrier

\subsection{2018 Operation}\label{ssec:2018_ringer_operation}

In 2018, the \rnn{} operated with a new tune based on collision data
(Section~\ref{ssec:2018}). It was also the case for the final HLT
selection\footnote{One exception was the \medium{} selection, where the HLT
likelihood selection operated in 2018 with the same 2017 tune.}. For the
lowest-transverse energy-threshold unprescaled trigger, an efficiency
improvement of at least one percentage point in central value is observed when
comparing both periods, resulting from a better operation in all selection
steps, but, in particular, this is due to improvements from the likelihood tunes in the period.

Despite maintaining high electron efficiency, a small reduction in fake acceptance was observed at the  \fastcalo{} step for some unprescaled triggers in comparison with 2017 after TS1. For the lowest-transverse energy-threshold the fake acceptance was reduced by a factor by 1.19. On the other hand, for the highest-transverse energy trigger, a reduction factor by 1.69 was achieved.


