\chapter{Conclusion}\label{sec:conclusion}
%-------------------------------------------------------------------------------


% Comparison with standard strategies
The \rnn{} brings an innovative design for electron selection based on
calorimetry information. By alleviating the implicit requirement of conceiving
variables to exploit individual properties of the particle interaction process, it introduces 
more general variables exploiting the process natural symmetries. With the
ambition to comply with specific trigger system demands, the feature extraction
of concentric ring energy sums provides an efficient description, which is quickly derived
through simple operations. Neural network models are employed for exploiting
nonlinear correlations between rings. The algorithm shows
high agreement with the offline and previous trigger selection methods, both based
on a maximum likelihood approach. It is one of the motivations whose culminated
in the employment of an ensemble of specialist models per regions of
pseudorapidity and transverse energy. Likewise, the \rnn{} algorithm preserves
better the signal efficiency with respect to pile-up with respect to the
previous cut-based strategy it replaced.


In the second half of 2017, the \rnn{} algorithm replaced such a cut-based strategy
in the \fastcalo{} decision step and became the baseline algorithm
for triggering events containing electrons above \SI{15}{\GeV}, as part of the ATLAS online system CPU reduction campaign. The \fastcalo{} selection is crucial for CPU demands, as a more
efficient algorithm allows to avoid heavier computations in the
subsequent steps.  It was the first time a neural network method was employed as
a baseline selection algorithm for event selection in the ATLAS trigger system.
During 2017 operation, based on neural networks trained with simulated data,
reduction factors of 13 in the \fastcalo{} and of 2 in
the \hlt{} fake rates were achieved with respect to the offline likelihood, while
keeping high electron efficiency when the \rnn{} was set to operate in a
configuration similar to lowest energy-threshold unprescaled single electron
trigger. It resulted in a reduction of \SI{60}{\%} in the CPU demands for this particular trigger, when it was set to operate alone. On the trigger system level,
estimated results on reprocessed data show a reduction of \SI{8}{\%} in the CPU
demands of \egamma{} triggers. For 2018, the \rnn{} was trained with collision
data resulting in an estimated additional reduction 35\% for the
mentioned trigger with respect to 2017.

\begin{comment}
    


% Quadrant and agreement analysis
The \rnn{} was also studied through the offline likelihood perspective 
to bring additional insights on its behavior. By
taking advantage of the low dimensionality, low correlation and high
interpretability power of the variables employed in the likelihood, we derived
the profiles of the disjoint decision cases of the 2017 duplicate trigger pair.
The choice for the duplicated pair considered the typical scenario employed for
analysis with the setup that would maximize the disagreement between both
configurations: with and without the \rnn{}. The proposed quadrant analysis
allowed to observe high agreement between both triggers with slight shifts
towards the signal region in the disagreement profiles of some calorimetry-based
variables. By checking the agreement between the trigger pair for the derivation
of the likelihood pdfs, the alteration using the full 2017 data was much lower
than estimated statistical fluctuations. The residuals were found to be bounded
by \SI{0.2}{$\sigma$} for all variables. When considering only the
disagreement profiles, homogeneity hypothesis at a \SI{5}{\%} level was not
rejected for the calorimetry variables, except for \reta{} and \rhad{} despite
using the most populated regions. For these variables, a shift towards signal
region was also observed. Hence, the \rnn{} had negligible impact in the
sensitive variables employed for physics analysis and in the offline selection
operation.

Finally, the presented results show that it is possible to tighten the requirements for online operation while still resulting in high agreement with the offline
selection. More generally, it demonstrates that designing a solution from scratch using the application specificities can provide new effective
strategies. We hope that it can serve as motivation for other developments in
HEP, particularly when considering the trigger applications.

\section{Outlook}


Preliminary evaluations have shown the potential of the \rnn{} algorithm for
triggering on electrons with $\et<\SI{15}{\GeV}$. This is particularly motivating given that triggers in this kinematic region are very
CPU demanding. Additionally, they have high output rate and are usually
prescaled, thus \rnn{} might also allow to collect additional data for those
triggers. For the beginning of Run 3 (2022), \rnn{} has been extended to all triggers below $\et<\SI{15}{\GeV}$ where a reduction of at least half of the fake rate in \fastcalo was observed.


Photons are other physics objects that may be triggered using the Ringer approach. Triggers combining photons to other physics objects are quite CPU demanding, thus motivating studies to evaluate whether additional Ringer developments will contribute to increased performance.
When considering pure photon triggers, the ring sums with linear classifiers can be investigated in order to keep the analysis strategies employed in channels as $\text{H}\rightarrow Z\gamma$. Another possibility for those triggers is to
employ the \rnn{} for increasing the trigger efficiency. A possible
setup is to duplicate the photon triggers so that the \rnn{} recovers
interesting events.  In this hybrid menu configuration, the standard trigger
would be available for those analyses so that they may benefit from high
interpretability and customization, whereas the \rnn{} trigger might be employed
for those analyses demanding more statistics. It is expected that this
configuration will demand minor additional resources but may allow higher
statistical significance of the results of physics analysis.


% No-had
Another interesting possibility is to employ the \rnn{} accessing only
electromagnetic information for triggers resulting in high readout rate from
hadronic calorimeter cells. The particular setup may even consider employing a
two step decision on the \fastcalo{}, a preliminary one based solely on
electromagnetic information followed by the full \rnn{} decision.




% Boosted configurations
Physics beyond the Standard Model is getting more attention at the LHC and is expected to be part of the main focus in the data-taking phases to
come. We aim at improving the \rnn{} algorithm to comprise boosted
configurations, particularly limiting the effect of other contributions at
the edge of the feature extraction window, which can result from these physics
processes. Additionally, the \rnn{} is being evaluated as an alternative
strategy for a long-lived particle analysis. %($\text{a}\rightarrow\text{H}(\gamma\gamma)Z(l^+l^-)$).


Nevertheless, we expect that the \rnn{} efficiency can still be improved. Indeed, the feature extraction
algorithm has many limitations as being initially proposed for online
operation, but a more complex algorithm accounting for the cell sizes can
provide more precise information, which, eventually, can be helpful for the
selection task. Shower asymmetries may be captured if extracting ring segments,
i.e.\@ quarter rings delimited by $\eta\times\phi$ axis. Complementary
discriminant information may be obtained by fusing the ring sums with the shower shape variables,
which is under investigation. Deep learning models can be exploited instead of
single hidden-layer MLPs, in order to achieve better suited decision boundaries,
specially considering models designed to exploit the sequential structure
presented by the rings.


% Calibration
The shower description by ring sums goes beyond capturing discriminant
information and may be used for calibration of the energy of electrons and
photons. While preliminary results with ring sums are motivating, we expect the aforementioned
improvements in the ring description to be exceptionally important for
calibration.

% Offline developments
The \rnn{} software has been extended to the offline framework. Currently,
the offline ring description is available for all electrons with $\et>\SI{14}{\GeV}$. This is an on-going activity.


\end{comment}
